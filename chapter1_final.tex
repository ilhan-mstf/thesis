\chapter{INTRODUCTION}



\begin{singlespace}
\epigraph{One can even shout out through refuse \ldots}{\hfill---Kurt Schwitters, \textit{Kurt Schwitters}, 1985}
\end{singlespace}



According to dictionary Merriam-Webster, trash means anything useless, worthless or discarded. In parallel to this definition, the common perception is to get rid of them as soon as possible. People tend to think that trash is valueless because it is no longer needed or not wanted anymore. People want to get rid of them not thinking any alternative. Stav says that \quotes{[t]hey forget about it and don’t think about all the time and energy and money put into disposing of it} \citep[as cited in][]{navarro2015followingtrash}. On the other hand the issue of trash and transformation is not that simple. There are a lot of points to examine the trash.

% Change in the production and consumption practices
\citet[11]{banash2013collage} draws attentions to the advancements in the recent history of humankind by stating that \quotes{over the course of the twentieth century, the twin developments of mass production and mass media in the capitalist economies of the [Western countries] completed a total transformation of everyday life, reorienting almost every activity toward consumption.} He argues that \quotes{things once locally produced and often handmade were now mass produced and commodified, turning local, artisanal producers into deskilled laborers serving the assembly line.} As a result of it amount of goods boost and the cost of them significantly reduced. Things once reused again and again, now thrown away because it is more affordable to replace with new one. As the production increased its by-product material waste is also increased. \quotes{The phenomenon of waste comes clearly into focus not merely as a by-product of manufacturing processes, but rather as an integral element in cycles of production and consumption} \citep[ix]{pye2010trashculture}.

% Dominant approaches
During the last decades much attention has been devoted to the waste. In particularly environmental considerations dominates the perceptions of waste. \citet[41]{ibarra2015beautiful} writes that \quotes{despite a relatively increased awareness about consumption and its consequences, the pace at which we also acquire and dispose of material objects is exploding}.

The act of trashing objects and the process of transforming them is not limited with geography, nationality and modern ages. It is a very common notion from urban areas to rural ones, from modern societies to ancient ones \citep[33]{rathje1992rubbish}. Anyone can encounter with trash in the crowded urban areas \quotes{as well as the remotest corners of the world} \citep[16]{cerny1996recycled}.

Trash is everywhere and, produced every time. It is inseparable from people’s activities. \quotes{Every day, [people] put unwanted material in toilets and garbage bins, regularly flushing it away or taking it out in bags to be transported far away from} \citep[xxv]{zimring2012encyclopedia} their habitat. In other words, refuse is part of people’s daily activities. Production of it never stops. Various activities of life such as eating, drinking, working and traveling leave behind wastes. These activities are not only taking place every time, but also spread to different places such as homes, offices and picnic areas. It is hard to say that any place is waste free. Nearly every place require waste bin because people can easily access vary kinds of goods that are going to be thrown away after used. If people do not find a waste bin, they often leave their trash wherever possible.

Trash is in the streets, in people’s home, in the sea that people swim, rotating around the globe\footnote{Satellite discards are disposed to the atmosphere and they rotate around the globe like satellites.}. Even if trash is tried to move away from people’s habitat, it is as close as the nearest waste bin.

People do not think what happens after they throw trash away.  It exits from people’s life but not from the world. It is stacked to another place. A project conducted by Massachusetts Institute of Technology in America explores the journey of trash by placing trackers onto the trash \citep{chen2009mit}. According to their results trash spreads away across the country and this journey takes month. Further it is not limited with the border of America because they are exported to the other countries. In other words Americans trash is not only their trash.

The vast amount of discarded items spread through the landfills to oceans. As commodities spread to the every corner of the world, trash do also. It moves from homes to garbage trucks, from streets to landfills and also from landfills to sculptures inside museums. In other words, trash is not only found on the most disgusting and avoided places, but also can be found on the most sterile and frequently visited places.

% Different scales
The scale of trash production is not same for everybody or every country. \citet[2]{chen2015waste} writes that by referring to analysis of World Bank \quotes{the amount of solid waste generated in cities is growing faster than the rate of urbanization.} She argues that when the people's income level and the rate of urbanization increased, production of waste also increases. Chen notes that nearly half of the world’s waste are produced by OECD\footnote{OECD (Organization for Economic Co-operation and Development) is an international economic organization whose members are mostly western developed countries.} countries. On the other hand, Africa and South Asia generate least waste. Similar to these as stated by \citet[16]{cerny1996recycled} \quotes{a person’s wealth has become measured not only in how much he or she can afford to consume, but in how much he or she can afford to throw away.} Moreover from trash many unexpected results can be extracted because as \citet[xxv]{zimring2012encyclopedia} argues that:

\begin{quote}
Our trash is a testament; what we throw away says much about our values, our habits, and our lives. ... Our trash is part of us, whether or not we choose to acknowledge it. ... The absence of a waste stream aroused suspicion, just as the presence of particular items tell us about the habits of the consumers who generate a waste stream.
\end{quote}

% Different types
Electronic waste, construction and demolition waste, medical waste are some of the various types of waste. They are result of different production and consumption practices. As an end product their effect and existence in the nature changes. For example power generation through use of nuclear technology and nuclear fission produces not only energy but also radioactive waste which is very dangerous for living creatures and remains unchanged in the nature for thousands years.

% Different disciplines
Trash attracts the attention of different peoples and disciplines \quotes{ranging from economics to environmental studies, but most particularly by those studying consumerism or material culture} \citep[63]{emgin2012trashion}. For instance for archaeologist dumps near the residential area are one of the significant places to figure out what people consume in the ancient times \citep{rathje1992rubbish}. Moreover scholars argues that there is a relationship between technological and waste. Trash can be seen as a design and technological problem \citep{mcdonough2010cradle}. Further economical aspect of trash can not ignored. \citet[65]{emgin2012trashion} states that there is a huge business based on eco-friendly products and recyclable goods that is increasingly promoted.

% Different people
Various people including garbage collectors and artists interact with trash for different purposes. There are people who collects plastics, scraps and papers from waste bins and landfills. For the rag-pickers, collecting others excess and discard is significant source of economical income. They collect and classify recycle materials to sell them. By collecting and selling them they endure their life. On the other side for some of the people it is not considered as trash. As stated by the editor of the Garbage issue of ReVista, Christmas decorations at Chocó, a poor region on Colombia’s Pacific Coast, are \quotes{all crafted from used tin cans, old newspapers, discarded textiles and found wood objects} \citep{erlick2015editorsletter}. She realized that any of them called the practice as recycling. For them using trash again and again is very natural and it is part of their life. On the contrary for the developed countries trash considered as a thing that must be avoided. There is no place for trash in their life. It can be understood from this anecdote that approaches to the garbage are not same for all regions of the world. It can be said that \quotes{perceptions of waste and the value of material are neither static nor universally shared} \citep[xxvi]{zimring2012encyclopedia}.

As noted by \cite{pye2010trashculture} \quotes{at least since the early twentieth century, the concern with discarded things and materials has been a recurring theme in art}. In the beginning of the twentieth century century objects that are considered as art enter the scope of art making. Using non-art objects in the art opened new dimensions for the language of art and practice art making. This is a revolutionary change in the practice of art making. Production of art and the approach to the art changed dramatically. Picasso and Braque are the first artists that used non-art object in their works. Later many of artists such as Kurt Schwitters and Joseph Cornell followed them with their collages and assemblages. Later Duchamp challenged the nature of art by using, non-unique, fabricated object to express his ideas. His work changed the way of perceiving art dramatically. With these developments trash has found a place in the art and contemporary artists become interested in trash. They use trash for different purposes and these are explored in detail in the following chapters. Trash find places in many mediums such as sculptures, paintings, assemblages and photographs. Artist express themselves through trash. Portraits are build from trash. Consumption habits of society are criticized through trash. \citet[2]{pye2010trashculture} states that \quotes{trash have influenced, and are also influenced by, cultural products such as films, visual art, museum exhibits and literature}.



%****************************************
\section{Purpose of the Study}
As it can be seen clearly that there are different aspects and approaches to the trash. In this thesis study to understand them is a key element because main driving force is to explore the alternative perceptions of trash. It is not static, not fixed, not totally isolated from people and social life. Instead of creating new usages, combinations or alternatives, common approach is to ignore all the possibilities embedded to the trash. People often do not look beyond the primary function of objects. To establish a solid understanding of trash and transformation of it, we have to give more attention to the trash. In order to move away the common perception this study asks that is it possible to reuse discarded items rather than refusing them?

The aim of this study is to explore the theories and methods focused on transformation of trash in the context of art and artistic act. Particularly this thesis is interested in the relationship between art and waste. It investigates that how waste represented as a subject and how waste is used as an art material to represent or draw attention to the other subjects. It asks and seeks answers of these questions: Why and how is trash used by the artist? Are there any differences with the original (or untouched, or blank) items compared to discarded (or used) items? Has using discarded material or trash have specific (or special) meaning (or message)?

Thesis project focuses on disposable items and specifically paper trash. Its purpose is to transform trash and make it worth to reconsider. At this point I need to state that the process of transformation is not to be confused with the kind of industrial recycling that is applying automated procedures to the products such as paper cups, glass bottles, plastics caps and aluminum cans with the purpose of returning them to the industrial production as raw materials. Further many people are inside the business of transforming trash through recycling and reusing. People has been looking for aluminum cans, glass bottles and plastics on the streets. They collect and sell them to survive. There are companies that are collect waste materials to regain them to the industry. In thesis they are not in the scope. Artist that give attention to the trash and transform them is in the scope of this thesis.

% TODO relocation.
% MOVE TO Conclusion of art chapter, discussion of transformation.
% FROM Trashion: The Return of the Disposed by Bahar Emgin
%\ People are often eager to see objects that were once considered useless and tasteless when they have been invigorated with new life.



%****************************************
\section{Structure of the Study}
Theory and artworks are not strictly separated from each other in this thesis. Artworks are explained with the help of theories. This study is structured by many examples of artworks in different part of the thesis. When an argument is presented, immediately an examplory artwork is given and how artists respond this issue is illustrated. Thereby artworks are examined in the theoretical and cultural context. Artworks are used as supportive elements for the stated arguments. However, artworks are discussed with all of their aspects. Only basic information is given and how it is related with this topic explained. By the way, it is better to keep in mind that all works can be analyzed in different contexts in more detail fashion.

Following chapter, Trash in Culture and Theory, explores the place of trash in the cultural life and theoretical approaches on trash. The aim is to understand the trash in cultural and theoretical aspects. Purpose is to establish better realization the notion of trash. This chapter provides theoretical and cultural background of the project. It is explored the trash is being worked is whose trash and what type of society generate this type of trash. What type of approach generate it and what are the dynamics of it? What type of trash we are talking about? It is understood through the patters of consumption patterns. And to reflect this cultural phenomena what philosophers and scholars say. Artist how they are reflected this notion.and also looked how scholars are conceptualized trash and its movement on the different values system. How can the transient nature of trash be explained?

In the third chapter, Trash (in) Art, seeks the root of using discarded materials in the artworks and analyses the sample artworks and artists. As already artworks are mentioned in the different part of the thesis, in this chapter some of them analyzed deeply. Artists’ methods and approaches to the subject are stated.

Fourth chapter consists of the overview of the project. In this chapter, my aim is to state the approach to subject clearly. The development process of the work is also presented in detail. Finally I put the factors, decisions and considerations that shaped the work.

The last chapter is reserved for an overall conclusion and evaluation of this thesis and further suggestions on the subject and the project.



%****************************************
\section{A Note on Terminology}
Many scholars and authors have used broad range of terms to refer the subject. Garbage, trash, rubbish, debris, detritus, waste, scrap, junk, refuse, discard, disposal, litter are some of them. Sometimes these words are used synonymously in causal speech but in fact there are slight differences between them. This situation signifies that the vocabulary of trash is so wide and there is no consensus on terminology. The authors of the book \quotes{Rubbish: the Archeology of Garbage} give clearer definition of some these words:

\begin{quote}
\textit{Trash} refers specifically to discards that are at least theoretically dry ---newspapers, boxes, cans, and so on. \textit{Garbage} refers technically to wet discards ---food remains, yard waste, and offal. \textit{Refuse} is an inclusive term for both the wet discards and the dry. \textit{Rubbish} is even more inclusive: It refers to all refuse plus construction and demolition debris. The distinction between wet and dry garbage was important in the days when cities slopped garbage to pigs, and needed to have the wet material separated from the dry; it eventually became irrelevant, but may see a revival if the idea of composting food and yard waste catches on. \citep[9]{rathje1992rubbish}
\end{quote}

Moreover artist Alice Bradshaw has published a 36 page \quotes{Rubbish Newspaper}. It is based on her extensive research on this subject. The newspaper contains these words and illustrates their place in literature and artworks.

In this study like many scholars I also use these words interchangeably. However mainly in this thesis scope I prefer to trash which better defines dry paper discard.

In addition to these there is also a confusion of the terms: reuse and recycling which are the general methods of transforming trash to the another object or concept. Both of them are used widely by scholars and artists.

According to dictionary Merriam-Webster, the word reuse means \quotes{to use again especially in a different way or after reclaiming or reprocessing.} Reusing includes usage of the same product unaltered in structure. Reusing lengthens the life of the item or material in society. To reuse is to use something again instead of throwing it away. In this thesis, reuse is explored through the work of the artist.

On the other hand recycle means \quotes{to treat or process (used or waste materials) so as to make suitable for reuse.} Particularly, through the systematic process of recycling, an object is used as a raw material in the production of new product. Recycling occurs when waste in an unchanged chemical form is used in the same process that created the original product. Crushed glass containers used to make new glass containers, and scrap metal used in foundries can be given as examples.

Recycling is very similar the rotting, and reuse is something like dry tree branches used by birds for their nest. These are two agents of nature to regain their resources.

Recycling can be viewed as down-cycling. The object smashed to the small particles to be used later in the production of something else. Although reuse can be viewed as up-cycling that gives another (or more) value to the discarded products. Down-cycling does not generates new meanings it tries to convert the product to already known state to process. 

\citet[72]{strasser1999waste} by referring to the Oxford English Dictionary notes that \quotes{recycling originated in the oil industry during the 1920s.} Here the main purpose is to reduce the waste of petroleum therefore \quotes{partially refined petroleum sent through the refining cycle again.} She explains that \quotes{the world became familiar during the early 1970s, as the burgeoning environmental movement promoted the separate collection of certain kinds of trash to promote their reuse in manufacturing.}

Some of the scholars use recycling \citep{cerny1996recycled,herman1998trashformations} for the recreation and transformation of waste. Many scholar used the word recycling when mentioning works use trash. However they are not mentioning the meaning that is decomposing things to the their particles. What they actually referring is reusing and combining things, concepts to create new mixtures. It is hard to say that they use it wrongly, but what they refer is actually upcycling. Creative reuse, inventing new things from discard. The term upcycling is coined by architect and designer William McDonaugh and chemist Michael Braungart. It can be explained as \quotes{the process of converting an industrial nutrient (material) into something of similar or greater value, in its second life} \citep[as cited in][63]{emgin2012trashion}.  Upcycling approach resists the argument that an object is dead once it is disposed of \citep{emgin2012trashion}. It can be argued that art, in this scope, acts as an instrument of transformation and re-presents artifacts what was once regarded waste. In the light of this argument, methods of analyzed artists in this thesis when producing their artworks can be categorized as upcycling. Additionally, undertaking thesis project is about handmade notebooks that are reproduced from discarded papers through the procedure of upcycling.
