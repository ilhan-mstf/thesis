\documentclass{article}
\usepackage[english]{babel}
\usepackage[utf8x]{inputenc}
\usepackage{apacite}

\title{Thesis Draft}
\author{Mustafa İlhan}

\begin{document}
\maketitle

In this thesis usage of discarded materials in the process of art making and the artwork itself is explored (or researched). How is it used by the artist? Are there any differences with the original items? (In other words using discarded materials or trash has specific (or special) meaning? What is the importance of it?) This is a work to explore the re-usage of trash in artworks. (place or the role in the art practice.)

Actually (It is questioned that) working with trash dictates a life practice, it is a convergence of life and art making. The process effects how one's lives and lifestyle. (But what type of interaction between life and art making process?) (This claim is the main driving force of my artwork which is part of a thesis.)

\section{Reality of trash}
\subsection{What is wrong with trash?}
The vast amount of industrial discarded items spread through the landfills to oceans. They are the result of highly complex industrial production methods. They are not easily disposable items. They live in the nature thousands of years. Most of them packages that are used carry or protect other materials. After real material used these packages became valueless (or useless). (types of trash can be mentioned here, but currently in the artwork I'm using package trash, therefore, it is more important.) How manage the all this increasing trash that damaging nature?  This is the common approach to trash and the main problem. (actually the sustainability problem.) It is not the only problem, It can be thought that it is a losing the ability to transform new things, alternative behaviors etc. (Instead of creating new opportunities or alternatives, it is a consuming all them and producing huge pile of trash.) (reference Zizek idealization of nature, to love nature is to love trash. Live with trash. Do not see it as trash actually. Then the question is how to love trash? how to live with trash? can living with our trash enrich (our perceptions, abilities)? how not to see them as trash and useless? Can it be possible with art?)

All these produced materials how much harmonious with nature? How many animals and the plant can use these discarded items? (Because of complex production methods their recycling requires complex processes again. Some of them already produced to protect goods from natural factors (like decaying etc.). However how can we protect nature from them?) It is very hard that spontaneously they become harmonious with nature. However, some artists turned to trash into site-specific sculptures that are more than trash heap. not discarding but bracing our attitudes turned them to a something that worth it to watch and think about it. (Converting what we create harmonious with the existing system.) Because it is not possible to think that nature will live harmoniously what we created. The more likely idea will be we will live harmoniously with what we create.

The issue of trash is not limited with ecological and economic perspective, it has also other dimensions.(draw attention to multidimensionality of this topic, but why? and what are the other dimensions?)

Trash itself is not the only problem, the practice, lifestyle causing it is more important problem. The dynamics of market and flow of objects into it plays important role production of trash.

\subsection{Throw away culture}
Continuously consuming things and disposing of something. It is an important concept to understand why trash is trash? (or how it become trash?) Behavioral pattern of throw away culture results in the trash. (This pattern does not consider recycling of it.) Artworks that are trying to raise awareness is related with this concept.

\subsubsection{How does it become trash?}
Here the purpose is to understand the dynamics that turn objects to trash. By understanding them is provide a roadmap (or ideas) how to turn trash to something valuable? (The purpose of this thesis is to find a way(methodology, approach) to add value to object using artistic methods?)

\subsubsection{Comparison of trashes}
The complexity of produced trashes of societies is increasing. For example developed countries that have nuclear plant generates radioactive wastes which highly hazardous for the environment is never exist previous societies. Think batteries and so on. Every society generates different types of wastes. Differs from country to country, society to society, ages to ages.

It can be thought that when the complexity of trashed increased to the effort to repair, reuse and recycle is increase. Therefore for ones that have no such complex tools it is becoming harder to reuse them. In other words objects become more complex their re-usage becomes less likely. 

\subsubsection{Types of trashes}
Different production process generates different types of trash. 

\subsection{Collecting trash}
One of the most important parts of the using trash in the artwork (or expressing something, or representation) is to collect them. What are the dynamics(considerations) of collecting them? (easily accessible materials or unique items.) Where to store them? Does it mean that live with trash? In other words collecting trash and using them is live with them? (making them part of life.) After the being part of the are they still trash? Can be thought that it is something that affects the lifestyle. (possessions and trash.) Another question is that how differs collecting trash from collecting other things such as objects that have archival value. What is the driving force? You may collect it to prevent object being lost. For archival things what you collect is something that has some sort of social use and meaning which is going to disappear. However, trash is never disappearing, even its amount increasing rapidly. For archival things people have memories with them, but does some applies for the trash? Who wants to keep trash? or who wants to re-see(re-visit) trash again (in a museum for example)?

\subsection{What might be the meaning of using trash as a medium in the artworks? Questioning trash as a medium for artist}
\begin{itemize}
\item Some works try to raise awareness the problems that are the result of trash. (It treats environment and nature.)
\item Some of them reflect people's lifestyle especially throw away culture. As a mirror of current lifestyle.
\item Try to find a new value and meaning from the discarded material that are useless anymore. To explore a new approach, new way. Subvert people's ideas about trash and their attitudes by turning materials to the something meaningful (or valuable). Trash to treasure.
\item Using discarded item to represent other discarded things by the ruling ideology or approach. For example, trash can be used to represent refugees. The things that we are trying to discard does not mean that they have no value, instead it means that we have no ability to reveal its potential. In other words, refugees have potential but we see them as players that will change our current system. Therefore, it can be said that willing to transform trash to treasure is to require change of current lifestyle. Rejecting discarding something especially thing that you get value from it is a process and spread through to the ones life.
\item One way is not to produce trash. (Zero trash philosophy.) The other one is to transform trash into something else.
\item What type of experience is that collecting and working on objects that are generally discarded? Experiencing out of common practice, being open to new explorations.
\item Instead of a world that produce trash, how could it be a world created from trash?
\item Combining industrial goods with objects transformed from trash is another way to find a place to trash in the community. It also signifies that trash still has a good quality to used with new materials. Creating composite products from new and reused items. Using the valuable thing with the invaluable thing. It becomes more valuable or less valuable. Depends on the perception.
\item Aesthetics of trash. Revealing aesthetics value of discarded stuff. (Unique visual value. Trash portraits, sculptures etc.)
\end{itemize}

\section{Etymology}
\subsection{The difference between reuse and recycle}
According to the dictionary, the word “reuse” means “to employ for some purpose” or “to put into service.” Reusing involves usage of the same product unchanged in form. If any item is used again and again over time, it is said to be reused. The main purpose of reusing is to lengthen the life of the item or material. We give out used clothes for charity which results in reusing. Other examples are; buying some items and then selling them as used items, repairing some lawn equipment and reusing them, upgrading a computer, renting books, journals, periodicals, DVDs and others. The main purpose is to make the item last as long as it can. To reuse is to use something again instead of throwing it away or sending it off to a recycling company. Why throw something away when you can give it another life? Reusing is the second best way to conserve and be earth-friendly because it keeps items out of landfills and reduces the greenhouse emissions caused by purchasing a new product. Using something multiple times -- like using a disposable container more than once -- is not the only way to reuse; you can also give old items a new purpose. For example, use an empty coffee can to store small craft supplies or an old loofah as a scouring sponge for cleaning sinks.

According to the dictionary, “recycle” means “to treat or process (used or waste materials) so as to make suitable for reuse.” In recycling an item, it is processed into a totally new product. It is an energy consuming process. For example, if we put some plastic bottles, paper, or aluminium items in a recycling bin, these materials may be recycled into a totally different thing as clothing items, fabric, or maybe a quilt. In this process, energy is required which depends upon the stages of transformation.

\subsection{Origins of words: waste, trash, rubbish, scrap, junk, refuse, discard, litter}
The origin of these synonyms reveals a whole side of human activity: our history revealed by what we have thrown away through the ages. What were people throwing out when these words were coined? 

Garbage is giblets, refuse of a fowl, waste parts of an animal (head, feet, etc.) used for human food. Garbology is a study of waste as a social science. In modern American usage, garbage is generally restricted to mean kitchen and vegetable wastes.

Waste comes from the Latin vastus, meaning empty, desolate, desert, or wilderness, and it’s interesting how the Romans called desert any wilderness that wasn’t settled, including forests.  German has retained the original meaning in wüste (desert). Vastus, which also gave us vast, vain, and devastate, came to mean a waste of money and ultimately garbage.  It is tempting to see a relation with the word west – the ancients didn’t like the west, where the sun “dies”, and associated the west side with death (the Egyptian tombs and pyramids are always on the west bank of the Nile, for instance)\cite{paul2013garbage}.

\section{Rubbish Theory}
Objects have a lifetime and they don't remain same through that lifetime. Their value, usage, location change over time. During its lifetime objects may circulate different markets and values systems like economical value, social value, aesthetic value etc. Especially this cycle has picked up speed with the advent of consumer culture, our most recent technological gadgets becoming obsolete within 3 years. Objects function and value are transformed by relocation and revaluation of objects from one place to the other or one discipline to another. This flow(transition) and transformation theorized with Rubbish Theory by Thompson \cite{thompson1979rubbish}. Thompson looks at the creation and destruction of value in man-made objects, cultural artifacts, and ideas. He notes how an object’s economic and/or cultural value diminishes over time rendering the objects worthless or redundant. The theory looks at how some of these objects then regain value, such as antiques or historic homes. It claims that there are three types of objects; transient (normal state, decreasing value, circulating), durable (permanent, increasing value, removed from circulation) and rubbish(zero value, will be destroyed or reinvested for economic and social value). The transition from transient to durable is only possible firstly transient to rubbish and later rubbish to durable. Further, there is a common idea/argument/motto that is "trash to treasure" among artists who use trash as a medium. Rubbish theory presents a conceptual approach to this argument. 

Although Thompson is quite successful categorizing states of objects throughout their lifetime, claimed transitions between states in the theory have some problems. Thompson label some transitions as possible and the others as impossible. "He allows goods only to move from a transient to become rubbish, and from rubbish they can either be destroyed or become durable. Movement in the other direction, from durable to either transient or rubbish, is not allowed in this system" \cite{meadow2011relocation}. Further, he does not allow move from transient to durable. However, Duchamp's fountain breaks this rule. Because urine used as a fountain is still functional and have a place in the market. In another word, it is not rubbish. This urine with the approach of Duchamp turned to be an artwork. It is one of the most influential piece of modern art and one of the best examples of ready-made. After Duchamp's intervention to the urine, it becomes a durable object placed in a museum.

In rubbish theory beyond the objects states how it happens transition of objects in practice is missing and Parsons fills this gap by claiming that transition from rubbish to durable are possible with finding objects, displaying objects, re-using objects \cite{parsons2008thompsons}. (explain details) (It can also be thought that they are the way of value creation.) (turning trash to treasure or something else is a value problem. transforming them creates new a value system? or just finding place existing value system. by the way there is a value theory related with (or inside of) game theory.)

Further another conducted research examines the psychological, social, and aesthetic factors involved in found object and found that ... \cite{camic2010trashed}.

\section{Collage, Assemblage and the Found Object}
In this chapter root of using objects in the artworks is examined. Using objects on artworks beyond their intended purpose. Developing artworks not only painting but also using paper and other stuff by pasting them together.

\subsection{Collage}
Collage originates from the French \textit{coller} is an artistic technique of applying manufactured, printed, or “found” materials, such as bits of newspaper, fabric, wallpaper, etc., to a panel or canvas, frequently in combination with painting. In about 1912–13 Pablo Picasso and Georges Braque extended this technique, combining fragments of paper, wood, linoleum, and newspapers with oil paint on canvas to form compositions. Pasting paper is not a new technique but using this it in the art making is a revolutionary movement in the  language of art \cite{waldman1992collage}.

\subsection{Assemblage}
Assemblage work produced by the incorporation of everyday objects into a composition. It is similar to collage, but main difference is that assemblage is three dimensional rather collage is two-dimensional. Diverse range of things can be used production of work. In 1961, the exhibition "The Art of Assemblage" was featured at the New York Museum of Modern Art. William C Seitz, the curator of the exhibition, described assemblages as being made up of preformed natural or manufactured materials, objects, or fragments not intended as art materials \cite{seitz1961art}.

\subsection{the Found Object (Ready-mades)}
Found object originates from the French \textit{objet trouvé}, describing art created from undisguised, but often modified, objects or products that are not normally considered art, often because they already have a non-art function. Pablo Picasso first publicly utilized the idea when he pasted a printed image of chair caning onto his painting titled Still Life with Chair Caning (1912). Marcel Duchamp is thought to have perfected the concept several years later when he made a series of ready-mades, consisting of completely unaltered everyday objects selected by Duchamp and designated as art. The most famous example is Fountain (1917), a standard urinal purchased from a hardware store and displayed on a pedestal, resting on its side.

\subsection{\textit{Bricolage}}
Something constructed using whatever was available at the time.

\subsection{Discussion}
Are artworks made from trash just examples of collage and assemblage or more than from them? What about the experience and interaction with other people? Turning art making process to a life practice (or part of life) can be explained in the context of collage (which is mainly related with how a 2d canvas created). But all of them work in fragments, combine many objects together.

\section{Artwork, Project}
\subsection{Why (package) paper?}
Easy to collect. Easy to find. Thrown out even if it is good quality. Packaging materials are very widespread. Appropriate for painting and writing.

\bibliographystyle{apacite}
\bibliography{ref}

\end{document}