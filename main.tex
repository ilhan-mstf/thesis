\documentclass{article}
\usepackage[english]{babel}
\usepackage[utf8x]{inputenc}

\usepackage{apacite}

\title{Thesis Draft}
\author{Mustafa İlhan}

\begin{document}
\maketitle

In this thesis usage of discarded materials in the process of art making and artwork itself is explored. How it is used by the artist? Is there any differences with the original items? (In other words using discarded materials or trash has specific (or special) meaning? What are the importance of it?) This is a work to explore the re-usage of trash in artworks. (place or the role in the art practice.)

Actually (It is questioned that) working with trash dictates a life practice, it is a convergence of life and art making. The process effects how one's lives and lifestyle. (But what type of interaction between life and art making process?) (This claim is the main driving force of my artwork which is part of a thesis.)

\section{Reality of trash}
\subsection{What is wrong with trash?}
Huge amount of industrial discarded items spread through the landfills to oceans. They are the result of highly complex industrial production methods. They are not easily disposable items. They live in the nature thousands of years. Most of them packages that are used carry or protect other materials. After real material used these packages became valueless (or useless). (types of trash can be mentioned here, but currently in the artwork I'm using package trash therefore it is more important.) How manage the all this increasing trash that damaging nature?  This is the common approach to trash and the main problem. (actually the sustainability problem.) It is not the only problem, It can be thought that it is a loosing the ability to transform new things, alternative behaviors etc. (Instead of creating new opportunities or alternatives, it is a consuming all them and producing huge pile of trash.) (reference Zizek idealization of nature, to love nature is to love trash. live with trash. do not see it as trash actually. Then question is to how to love trash? how to live with trash? can living with our trash enrich (our perceptions, abilities)? how not to see them as trash and useless? Can it be possible with art?)

All these produced materials how much harmonious with the nature? How much animals and plant can be use these discarded items? (Because of complex production methods their recycling requires complex processes again. Some of them already produced to protect goods from natural factors (like decaying etc.). However how can we protect nature from them?) It is very hard that spontaneously they become harmonious with the nature. However some artists turned to trash into site specific sculptures that are more than trash heap. not discarding but bracing our attitudes turned them to a something that worth it to watch and think about it. (Converting what we create harmonious with the existing system.) Because it is not possible to think that nature will live harmoniously what we created. More likely idea will be we will live harmoniously with what we create.

The issue of trash is not limited with ecological and economic perspective, it has also other dimensions.(draw attention to multidimensionality of this topic, but why? and what are the other dimensions?)

Trash itself is not the only problem, the practice, lifestyle causing it is more important problem. The dynamics of market and flow of objects into it plays important role production of trash.

\subsection{Throw away culture}
Continuously consuming things and disposing something. It is important concept to understand why trash is trash? (or how it become trash?) Behavioral pattern of throw away culture result trash. (this pattern does not think about the recycling of it.) artworks that are trying to raise awareness is related with this concept. 

\subsubsection{How does it become trash?}
Here purpose is to understand the dynamics that turn objects to trash. By understanding them is provide a road map (or ideas) how to turn trash to something valuable? (The purpose of this thesis is to find a way(methodology, approach) to add value to object using artistic methods?)

\subsubsection{Comparison of trashes}
The complexity of produced trashes of societies is increasing. For example developed countries that have nuclear plant generates radioactive wastes which highly hazardous for environment is never exist previous societies. Think batteries and so on. Every society generates different types of wastes. Differs from country to country, society to society, ages to ages.

\subsubsection{Types of trashes}

\subsection{Collecting trash}
One of the most important part of the using trash in the artwork (or expressing something, or representation) is to collect them. What are the dynamics(considerations) of collecting them? (easily accessible materials or unique items.) Where to store them? Does it mean that live with trash? In other words collecting trash and using them is live with them? (making them part of life.) After the being part of the are they still trash? Can be thought that it is something that effects the lifestyle. (possessions and trash.) Another question is that how differs collecting trash from collecting other things such as objects that have archival value. What is the driving force? You may collecting it to prevent object being lost. For archival things what you collect is something that has some sort of social use and meaning which is going to disappear. However trash is never disappears, even its amount increasing rapidly. For archival things people have memories with them, but does some applies for the trash? Who wants to keep trash? or who wants to re-see(re-visit) trash again (in a museum for example)?

\subsection{What might be meaning of using trash in the artworks? Questioning trash in the context of art}
\begin{itemize}
\item Some works try to raise awareness the problem of trash. (its threat to environment and nature.)
\item Some of them reflect people's lifestyle especially throw away culture. mirroring of current lifestyle.
\item Try to find new combination, meaning from the discarded material that are useless anymore and reached end in the specific lifestyle. to explore new approach, new way. subvert peoples ideas about trash and their attitudes by turning materials to the something meaningful (or valuable).
\item Using discarded item to represent other discarded things by the ruling ideology or approach. For example trash can be used to represent refugees. The things that we are trying to discard does not mean that they have no value, instead it means that we have no ability to reveal its potential. In other words refugees have potential but we see them as a players that will change our current system. Therefore it can be said that willing to transform trash to treasure is to requires change of current lifestyle. Rejecting discarding something especially thing that you get value from it is a process and spread through to the ones life.
\item One way is to not producing trash. zero trash philosophy. The other one is to transform trash into something else.
\item How does it feel that collecting and working on objects that are generally discarded? (experiencing out of common practice, it is open to new explorations...)
\item Instead of a world that produce trash, how could it be a world created from trash?
\end{itemize}

\section{Rubbish Theory}

\section{Collage, Assemblage and Found Object}

\bibliographystyle{apacite}
\bibliography{ref}

\end{document}