\documentclass[a4paper]{article}
\usepackage[english]{babel}
\usepackage[utf8x]{inputenc}
\usepackage{apacite}

\providecommand{\keywords}[1]{\textbf{\textit{Keywords---}} #1}

\title{Thesis Draft}
\author{Mustafa İlhan}

\begin{document}
\maketitle

\begin{abstract}
In this thesis (re-)usage of discarded materials in the process of art making and the artwork itself is explored (or researched). Why and how are they used by the artist? Are there any differences with the original items compared to discarded items? (In other words using has using discarded materials or trash specific (or special) meaning and message?) This is a work to explore the re-usage of trash in artworks (place or the role in the art practice). (and also in which scope? medium, message, life practice\ldots) 

Actually (It is questioned that) working with trash dictates a life practice and, it is a convergence of behavioral patterns (attitudes toward to trash) and art making. The process effects how one's lives and lifestyle. (But what type of interaction between life and art making process?) (This claim is the main driving force of my artwork which is part of a thesis.)
\end{abstract}
\keywords{trash, art}

\section{Reality of trash (other side of production)}
\subsection{Trash is everywhere and, produced every time}
Modern (developed) societies are continuously generating trash and, pile them on landfills. It's a common object category that all people share its possession. During daily activities trash is generated and people get rid of them by throwing away. (Various objects become trash after their primary functions daily. Who defined the primary function? Primary function is the only function. People cares the package of the objects that they buy. They buy the coffee not the cup of it. After coffee finished the life of cup also finishes. There is a lifetime defined (or forecast) by the producer of objects. However it is not bound to the producer, also consumer play an important role. There are different choices. Throw it away. Keep it. Give it. Mostly wining choice is throwing away and by the result of it mountains of garbage are increasing.) The vast amount of industrial discarded items spread through the landfills to oceans. They are the result of highly complex industrial production methods. They are not easily disposable items. They live in the nature thousands of years. Most of them packages that are used to carry or protect other materials. After real material used these packages became valueless (or useless). (types of trash can be mentioned here, but currently in the artwork I'm using package trash, therefore, it is more important.) How manage the all this increasing trash that damaging nature?  This is the common approach to trash and the main problem. (actually the sustainability problem.) It is not the only problem, It can be thought that it is a losing the ability to transform new things, alternative behaviors etc. (Instead of creating new opportunities or alternatives, it is a consuming all them and producing huge pile of trash.) (reference Zizek idealization of nature, to love nature is to love trash. Live with trash. Do not see it as trash actually. Then the question is how to love trash? how to live with trash? can living with our trash enrich (our perceptions, abilities)? how not to see them as trash and useless? Can it be possible with art?)

(Zizek part: talks about ecology at garbage dump in London, and starts with the words "This is where we should start feeling at home." to the ecologist. how ecology turns to ideology and mentions the our wrong perception about the ecology. Draw attention to the event if trash disappears from our world but not world. He thinks that the way of approaching ecology is problematic, because accepting that nature as a balanced harmonous thing. He claims that it is ideological in the sense that wrong thinking important problems. nature contains unimaginable catastrophes. think oil and distinct animals and plants. we profit balance part of the nature but it is created from catastrophe. Are we aware of this catastrophe. he asserts that ecology will slowly turn to religion that "is a kind of an unquestionable highest authority." Ideology of ecology warns us like, "Don't do that. It would be too much." its voice is like "Don't mess with D.N.A. Don't mess with nature. Don't do it" etc. we should not forget that we are part of the ecology. we must more alienated from the nature. we must find poetry and spirituality in the dimension of trash. thats the true love of world. Love is not idealization. This part can be extended.)

How much harmonious all these produced materials with nature? How many animals and the plant can use these discarded items? (Because of complex production methods their recycling requires complex processes again. Some of them already produced to protect goods from natural factors (like decaying etc.). However how can we protect nature from them?) It is very hard that spontaneously they become harmonious with nature. However, some artists turned to trash into site-specific sculptures that are more than trash heap. not discarding but bracing our attitudes turned them to a something that worth it to watch and think about it. (Converting what we create harmonious with the existing system.) Because it is not possible to think that nature will live harmoniously what we created. The more likely idea will be we will live harmoniously with what we create.

Getting rid of it is not the significant reaction. It still waits us. Maybe the first thing to do is accept the waste, to accept that there are things out there that serve nothing. To break out of this eternal cycle of functioning.

The issue of trash is not limited with ecological and economic perspective, it has also other dimensions.(draw attention to multidimensionality of this topic, but why? and what are the other dimensions?)

Trash itself is not the only problem, the practice, lifestyle causing it is more important problem. The dynamics of market and flow of objects into it plays important role production of trash.

trash of developed societies has higher recycling period in the nature. higher damage to the nature. hard to reuse. hard to transform. sometimes not to safe to keep them. 

\subsubsection{Perspectives of trash from different disciplines}
Trash is ecological problems and it is treats the balance of it. Animals do not aware of plastics materials and eat them. Management problem of municipals. 

\subsection{Throw away culture}
Continuously consuming things and disposing of something. It is an important concept to understand why trash is trash? (or how it become trash?) Behavioral pattern of throw away culture results in the trash. (This pattern does not consider recycling of it.) Artworks that are trying to raise awareness is related with this concept.

(Our trash generating behavioral patterns reveals what it is wrong? You are walking outside and entered an coffee shop. buy a coffee inside a paper cup. carried with paper package.) 

\subsubsection{How does it become trash?}
Here the purpose is to understand the dynamics that turn objects to trash. By understanding them is provide a roadmap (or ideas) how to turn trash to something valuable? (The purpose of this thesis is to find a way(methodology, approach) to add value to object using artistic methods?)

\subsubsection{Comparison of trashes}
The complexity of produced trashes of societies is increasing. For example developed countries that have nuclear plant generates radioactive wastes which highly hazardous for the environment is never exist previous societies. Think batteries and so on. Every society generates different types of wastes. Differs from country to country, society to society, ages to ages.

It can be thought that when the complexity of trashed increased to the effort to repair, reuse and recycle is increase. Therefore for ones that have no such complex tools it is becoming harder to reuse them. In other words objects become more complex their re-usage becomes less likely. 

\subsubsection{Types of trashes}
Different production process generates different types of trash. 

\subsection{What is wrong with trash?}
Relationship between entropy (second law of thermodynamics) and waste. Resources of nature turns to waste that it can revert it. Creating that are reversible again is problematic through the nature of sustainability. What is produced after it is consumed become worthless. 

\subsection{Collecting trash}
One of the most important parts of the using trash in the artwork (or expressing something, or representation) is to collect them. What are the dynamics(considerations) of collecting them? (easily accessible materials or unique items.) Where to store them? Does it mean that live with trash? In other words collecting trash and using them is live with them? (making them part of life.) After the being part of the are they still trash? Can be thought that it is something that affects the lifestyle. (possessions and trash.) Another question is that how differs collecting trash from collecting other things such as objects that have archival value. What is the driving force? You may collect it to prevent object being lost. For archival things what you collect is something that has some sort of social use and meaning which is going to disappear. However, trash is never disappearing, even its amount increasing rapidly. For archival things people have memories with them, but does some applies for the trash? Who wants to keep trash? or who wants to re-see(re-visit) trash again (in a museum for example)?

Agnes varda, the gleaner and I

\subsection{What might be the meaning of using trash as a medium in the artworks? Questioning trash as a medium for artist}
\begin{itemize}
\item Some works try to raise awareness the problems that are the result of trash. (It treats environment and nature.)
\item Some of them reflect people's lifestyle especially throw away culture. As a mirror of current lifestyle.
\item Try to find a new value and meaning from the discarded material that are useless anymore. To explore a new approach, new way. Subvert people's ideas about trash and their attitudes by turning materials to the something meaningful (or valuable). Trash to treasure.
\item Using discarded item to represent other discarded things by the ruling ideology or approach. For example, trash can be used to represent refugees. The things that we are trying to discard does not mean that they have no value, instead it means that we have no ability to reveal its potential. In other words, refugees have potential but we see them as players that will change our current system. Therefore, it can be said that willing to transform trash to treasure is to require change of current lifestyle. Rejecting discarding something especially thing that you get value from it is a process and spread through to the ones life.
\item One way is not to produce trash. (Zero trash philosophy.) The other one is to transform trash into something else.
\item What type of experience is that collecting and working on objects that are generally discarded? Experiencing out of common practice, being open to new explorations.
\item Instead of a world that produce trash, how could it be a world created from trash?
\item Combining industrial goods with objects transformed from trash is another way to find a place to trash in the community. It also signifies that trash still has a good quality to used with new materials. Creating composite products from new and reused items. Using the valuable thing with the invaluable thing. It becomes more valuable or less valuable. Depends on the perception.
\item Aesthetics of trash. Revealing aesthetics value of discarded stuff. (Unique visual value. Trash portraits, sculptures etc.)
\end{itemize}

\section{Etymology}
\subsection{The difference between reuse and recycle}
According to the dictionary, the word “reuse” means “to employ for some purpose” or “to put into service.” Reusing involves usage of the same product unchanged in form. If any item is used again and again over time, it is said to be reused. The main purpose of reusing is to lengthen the life of the item or material. We give out used clothes for charity which results in reusing. Other examples are; buying some items and then selling them as used items, repairing some lawn equipment and reusing them, upgrading a computer, renting books, journals, periodicals, DVDs and others. The main purpose is to make the item last as long as it can. To reuse is to use something again instead of throwing it away or sending it off to a recycling company. Why throw something away when you can give it another life? Reusing is the second best way to conserve and be earth-friendly because it keeps items out of landfills and reduces the greenhouse emissions caused by purchasing a new product. Using something multiple times -- like using a disposable container more than once -- is not the only way to reuse; you can also give old items a new purpose. For example, use an empty coffee can to store small craft supplies or an old loofah as a scouring sponge for cleaning sinks.

Reuse occurs when waste in an unchanged chemical form is used in a process that did not create the original product. Examples include crushed glass containers (cullet) used to manufacture glass wool insulation or manufactured sand, and various forms of waste polypropylene used to make clothing.

According to the dictionary, “recycle” means “to treat or process (used or waste materials) so as to make suitable for reuse.” In recycling an item, it is processed into a totally new product. It is an energy consuming process. For example, if we put some plastic bottles, paper, or aluminium items in a recycling bin, these materials may be recycled into a totally different thing as clothing items, fabric, or maybe a quilt. In this process, energy is required which depends upon the stages of transformation.

Recycling occurs when waste in an unchanged chemical form is used in the same process that created the original product. Examples are crushed glass containers (cullet) used to make new glass containers, and scrap metal used in foundries. 

Reusing is possible with re seeing (rethinking). Reusing is possible meet the needs of the human itself. Using creativity and personal approach can change objects functions. It is possible to use objects for different purposes.  

\subsection{Origins of words: waste, trash, rubbish, scrap, junk, refuse, discard, litter}
The origin of these synonyms reveals a whole side of human activity: our history revealed by what we have thrown away through the ages. What were people throwing out when these words were coined? 

Garbage is giblets, refuse of a fowl, waste parts of an animal (head, feet, etc.) used for human food. Garbology is a study of waste as a social science. In modern American usage, garbage is generally restricted to mean kitchen and vegetable wastes.

Waste comes from the Latin vastus, meaning empty, desolate, desert, or wilderness, and it’s interesting how the Romans called desert any wilderness that wasn’t settled, including forests.  German has retained the original meaning in wüste (desert). Vastus, which also gave us vast, vain, and devastate, came to mean a waste of money and ultimately garbage.  It is tempting to see a relation with the word west – the ancients didn’t like the west, where the sun “dies”, and associated the west side with death (the Egyptian tombs and pyramids are always on the west bank of the Nile, for instance)\cite{paul2013garbage}.

\section{Rubbish Theory}
Objects have a lifetime and they don't remain same through that lifetime. Their value, usage, location change over time. During its lifetime objects may circulate different markets and values systems like economical value, social value, aesthetic value etc. Especially this cycle has picked up speed with the advent of consumer culture, our most recent technological gadgets becoming obsolete within 3 years. Objects function and value are transformed by relocation and revaluation of objects from one place to the other or one discipline to another. This flow(transition) and transformation theorized with Rubbish Theory by Thompson \cite{thompson1979rubbish}. Thompson looks at the creation and destruction of value in man-made objects, cultural artifacts, and ideas. He notes how an object’s economic and/or cultural value diminishes over time rendering the objects worthless or redundant. The theory looks at how some of these objects then regain value, such as antiques or historic homes. It claims that there are three types of objects; transient (normal state, decreasing value, circulating), durable (permanent, increasing value, removed from circulation) and rubbish(zero value, will be destroyed or reinvested for economic and social value). The transition from transient to durable is only possible firstly transient to rubbish and later rubbish to durable. Further, there is a common idea/argument/motto that is "trash to treasure" among artists who use trash as a medium. Rubbish theory presents a conceptual approach to this argument. 

Although Thompson is quite successful categorizing states of objects throughout their lifetime, claimed transitions between states in the theory have some problems. Thompson label some transitions as possible and the others as impossible. "He allows goods only to move from a transient to become rubbish, and from rubbish they can either be destroyed or become durable. Movement in the other direction, from durable to either transient or rubbish, is not allowed in this system" \cite{meadow2011relocation}. Further, he does not allow move from transient to durable. However, Duchamp's fountain breaks this rule. Because urine used as a fountain is still functional and have a place in the market. In another word, it is not rubbish. This urine with the approach of Duchamp turned to be an artwork. It is one of the most influential piece of modern art and one of the best examples of ready-made. After Duchamp's intervention to the urine, it becomes a durable object placed in a museum.

In rubbish theory beyond the objects states how it happens transition of objects in practice is missing and Parsons fills this gap by claiming that transition from rubbish to durable are possible with finding objects, displaying objects, re-using objects \cite{parsons2008thompsons}. (explain details) (It can also be thought that they are the way of value creation.) (turning trash to treasure or something else is a value problem. transforming them creates new a value system? or just finding place existing value system. by the way there is a value theory related with (or inside of) game theory.)

Further another conducted research examines the psychological, social, and aesthetic factors involved in found object and found that ... \cite{camic2010trashed}.

"Rubbish theory, a philosophy that attempts to address how value is placed on material objects." "It is a body of thought that addresses how the value of material objects is socially constructed and deconstructed." "An awareness of rubbish theory is important to the understanding of the sociology of consumption and waste because, while what is and is not considered garbage may seem obvious and natural, the value of objects is based on the perceptions of people." "The classic examples of these categories are the durable 18thcentury Queen Anne tall-boy chest and the transient used automobile." "What decides whether or not something is a durable or transient is often the perceptions of the powerful members of society, those with a vested interest in owning objects whose value will always increase, while the remainder of society owns objects whose value will eventually decrease to nothing."

\subsection{Implications of Rubbish Theory}

\section{Collage, Assemblage and the Found Object}
In this chapter root of using objects in the artworks is examined. Using objects on artworks beyond their intended purpose. Developing artworks not only painting but also using paper and other stuff by pasting them together.

\subsection{Collage}
Collage originates from the French \textit{coller} is an artistic technique of applying manufactured, printed, or “found” materials, such as bits of newspaper, fabric, wallpaper, etc., to a panel or canvas, frequently in combination with painting. In about 1912–13 Pablo Picasso and Georges Braque extended this technique, combining fragments of paper, wood, linoleum, and newspapers with oil paint on canvas to form compositions. Pasting paper is not a new technique but using this it in the art making is a revolutionary movement in the  language of art \cite{waldman1992collage}.

\subsection{Assemblage}
Assemblage work produced by the incorporation of everyday objects into a composition. It is similar to collage, but main difference is that assemblage is three dimensional rather collage is two-dimensional. Diverse range of things can be used production of work. In 1961, the exhibition "The Art of Assemblage" was featured at the New York Museum of Modern Art. William C Seitz, the curator of the exhibition, described assemblages as being made up of preformed natural or manufactured materials, objects, or fragments not intended as art materials \cite{seitz1961art}.

\subsection{the Found Object (Ready-mades)}
Found object originates from the French \textit{objet trouvé}, describing art created from undisguised, but often modified, objects or products that are not normally considered art, often because they already have a non-art function. Pablo Picasso first publicly utilized the idea when he pasted a printed image of chair caning onto his painting titled Still Life with Chair Caning (1912). Marcel Duchamp is thought to have perfected the concept several years later when he made a series of ready-mades, consisting of completely unaltered everyday objects selected by Duchamp and designated as art. The most famous example is Fountain (1917), a standard urinal purchased from a hardware store and displayed on a pedestal, resting on its side.

\subsection{\textit{Bricolage}}
Something constructed using whatever was available at the time.

Claude Levi-Strauss notes: the bricoleur works not from the principle of making things only if natural resources are available but makes things according to those things at hand, making do with what is available. It is an expression that, like the natural cycles of the Earth, attempts to make something new from something old. \cite{levi1966savage}

\subsection{Folk Art}

\subsection{History of Consumption and Waste}
"The use of trash as a fine art medium dates back at least to the work of early-20th-century artists such as Fortunato Depero and Kurt Schwitters. Use of found materials, including garbage, has been associated with assemblage art since the 1950s and has been practiced by other well-known artists, including graphic artist Christian Boltanski, sculptor Louise Bourgeois, and photographer Andres Serrano. Art made from garbage has since become much more common in fine arts venues such as museums, galleries, and high-profile installations, including H. A. Schuldt’s famous “Trash People,” which has traveled around the world since 1996." \cite{tauxe2012encyclopedia}

\subsection{Discussion}
Are artworks made from trash just examples of collage and assemblage or more than from them? What about the experience and interaction with other people? Turning art making process to a life practice (or part of life) can be explained in the context of collage (which is mainly related with how a 2d canvas created). But all of them work in fragments, combine many objects together. 

Garbage is often viewed as a form of society’s excess---as the unwanted things that are thrown out without regard. 

In the world of computer science, the term garbage also refers to situations of loss in which data or objects in memory go unused in computer operations.

\section{Artworks and trash}
Types of artworks are created from trash.

\section{Artwork, Project}

\subsection{Paper}
"Paper is an indispensable product throughout the world. Its primary use is as a medium for writing, essential for bureaucracy, education, communications, information storage, and in the spread of information. In addition, it is used for the packaging for transport and convenience of a wide range of items from food to industrial equipment. Paper also has specific technological uses, such as for filters and in art, home furnishings, and architecture, and it has a range of uses for hygiene purposes. Paper in several forms is consumed on a daily basis by each person in the Western world." \cite{trafford2012paper}

\subsubsection{Environmental Impact}
Paper is both biodegradable and a renewable resource, which means in consumption and waste terms, its environmental impact is relatively small compared to the many more-toxic and bulky waste products that are found in everyday garbage. However, the chemicals, water, and electricity used in its manufacture are considerable---and these are nonrenewable resources---and certain types of chemicals used in paper production are toxic. In addition, if waste paper is sent to a landfill, it releases carbon dioxide emissions. Further, forest resources are not always as renewable as one may like to think. These environmental impacts can be greatly reduced by recycling (paper being one of the most easily and cheaply recyclable products in everyday use) and by conscientious consumption practices.

Paper made exclusively from wood is called virgin paper, while paper produced out of used paper that is re-pulped is called recycled paper. Recycling paper can greatly diminish demand for virgin fiber from wood. However, there will always be a demand for virgin paper because, although paper is thought of as a renewable resource, it cannot be recycled indefinitely. It can only be recycled four to six times, as the fibers get shorter and weaker each time. In addition, some virgin pulp must be introduced into the process each time to maintain the strength and quality of the fiber, so no matter how much is recycled, paper will always need some virgin fiber.

\subsubsection{History}
The word paper comes from papyrus, the plant that was first used for making a medium for writing in ancient Egypt.

\subsubsection{Production}
All types and qualities of paper share the same basic method of manufacture, including newspaper paper, print paper, and carton used for boxes.

\subsubsection{Uses}
Paper has become the most ubiquitous product in the age of information. Such products often complete their journey from shop floor to garbage in a single day; for example, newspapers, print paper, packaging, lavatory paper, tea bags, transport tickets, price tags, shopping bags, flyers, leaflets, wrapping paper, napkins, and tissues. 

\subsection{Why (package) paper?}
Easy to collect. Easy to find. Thrown out even if it is good quality. Packaging materials are very widespread. Appropriate for painting and writing.

\section{Uncategorized ideas, references}
Trash art is not collage (assemblage or found object) or fragments. it is more than that. The carried messages through the medium have different meaning. It has relationship with activism, craftivism. It refuses consumption based life cycle. It suggests a life practice.

"Every day, we put unwanted material in toilets and garbage bins, regularly flushing it away or taking it out in bags to be transported far away from our homes by others. The names we give this material---waste, garbage, refuse, trash, rubbish--- have pejorative definitions. Worthless. Rejected and useless matter of any kind. Unimportant." "Our trash is a testament; what we throw away says much about our values, our habits, and our lives." "While dictionary definitions of garbage describe it as “filth” and “worthless,” scholars are careful to note that perceptions of waste and the value of material are neither static nor universally shared." "\ldots the question of who owns these discards is not trivial." "The absence of a waste stream aroused suspicion, just as the presence of particular items tell us about the habits of the consumers who generate a waste stream. Our trash is part of us, whether or not we choose to acknowledge it." \cite{zimring2012encyclopedia}

\subsection{Culture, Values, and Garbage}
"The Trash Talk project emphasizes the complex, yet overlooked, relationships that garbage and people share. In terms of their relationship to garbage, all people interact with it on two levels. One is a material connection, indicative of the physical and sensory contacts that people have with garbage. In some households, this connection begins with an individual removing an item from packaging, disposing of that item in the kitchen receptacle, placing that item and others into a larger bin, taking that bin to the curbside, and then the material connection ends. Others, including workers in sanitation plants and recycling centers, then continue a material connection with the garbage, but the material connection of the consumer and the garbage ends with the bin on the curbside. The second connection that people maintain with garbage is an ideational one. Unlike the material one, which is manifested in things that can be touched, moved, and sensed, the ideational connection operates on the level of cognition. The differentiation of an item of value from an item of trash, for example, has nothing to do with the material principles of the object. Instead, humans determine whether the object is of value or whether it is considered trash. The decision of whether an individual decides to dispose of a broken radio or to consider it an heirloom to be kept is highly subjective and rooted in the value systems of a culture." "After the item is eaten, the individual has to decide what to do with the remainder, such as the leftover package. The package might be reused, re-purposed, or recycled but, most likely, will be disposed of in the trash." \cite{lukas2012culture}

\subsection{Garbage Art}
"Garbage art (alternatively known as trash art or recycled art) is art created from materials including post-consumer and other waste, collected debris, or objects previously used for other purposes." "Creating art from garbage involves transforming the meaning of objects by placing them in new, aestheticized contexts. This practice is not new; tribal peoples have adapted bits of trash from industrialized societies into their traditional arts since coming into contact with products of the developed world." "Creating art from trash involves “consuming” garbage in the sense that artists appropriate and rearrange the materials in personal ways, transform their meanings, utilize them to their own ends, and represent them in new ways.It involves taking unwanted materials out of their “waste” context and recontextualizing them as “art.”" \cite{tauxe2012encyclopedia}

\subsection{Garbage in Modern Thought}
"Philosophers and intellectuals have expressed the need to focus on the centrality of garbage, but for everyday individuals, the understanding of garbage is often as something “out of sight, out of mind.”" "Modern humans, as part of their penchant for consumption and unsustainable living, often think very little about the waste that they produce." "Like many aspects of capitalist living, the person throwing away a piece of trash does not connect the various levels of production, consumption, and post-consumption involved in the trash. It becomes a secondary matter---an afterthought." "Martin O’Brien, among many thinkers, argues that the understanding of garbage should be a central concept, especially since garbage typically correlates with social change, social roles, and institutions. Thus, beyond the level of individuals and their relationship to garbage, there is an interest in understanding the central role that garbage plays in all of society’s roles, institutions, and forms of change." "Garbage is excess--- it is a part of society that society no longer desires." \cite{lukas2012garbage}

\subsubsection{Categorization and Value}
"Garbage is categorization, according to Susan Strasser." "In recycling programs and in places of refuse disposal, items of trash are categorized depending on their potential value, possible environmental harm, or time of decay. Consumers have become accustomed to the categories that are often applied to garbage. Many cities require people to dispose of their garbage in an orderly fashion---perhaps separating wet household waste from dry---and recycling programs ask individuals to divide their recyclable items into sets (such as plastic, glass, aluminum, and paper) and smaller subsets (such as PET or 01, PE-HD or 02, and PVC or 03). Garbage is an illustration of how humans use mental categories to order the material world." \cite{lukas2012garbage}

"According to John Scanlon, garbage is indicative of a separation of the world---the desirable from the unwanted. Michael Thompson uses the riddle of the rich and poor person’s approach to snot (one keeps his in a handkerchief, the other disposes of it with a tissue) to underscore the curious ways in which garbage is connected to the issue of value. While garbage is universal---all societies, extinct and extant, have produced or produce garbage--- the conditions under which garbage is understood are culturally determined. Many non-Western societies attach a much greater value to items after they are discarded. In the United States and many other nations, garbage often results not because something no longer has utilitarian value but because the item in question is defined as something of no value. Thus, garbage is not only an objective condition of material culture, but also a subjective one of mentalist culture. People define what is trash and what is valuable." \cite{lukas2012garbage}

\subsubsection{Semiotic Context}
"In popular writing (such as novels), in television, films, music, and other forms of mass expression, the term trash is used to signify work that is of especially low value." \cite{lukas2012garbage}

\section{Archaeology of Garbage}
\subsection{Garbology}
"Weberman infamously used techniques of what he deemed garbology to uncover what he saw as the essential nature of people. He once said, perhaps indirectly referencing Jean Brillat-Savarin’s quote about food, “You are what you throw away.”" \cite{lukas2012garbage}

"The field of garbology involves the study of refuse and waste. It enables researchers to document information on the nature and changing patterns of modern refuse, hence assisting in the study of contemporary human society or culture. According to the Oxford English Dictionary, the term was first used by waste collectors in the 1960s. A. J. Weberman popularized the term in describing his study of Bob Dylan’s garbage in 1970. It was pioneered as an academic discipline by William Rathje at the University of Arizona in 1973."

\subsection{Trash as History/Memory}
\cite{bullock2012trash}

\bibliographystyle{apacite}
\bibliography{ref}

\end{document}