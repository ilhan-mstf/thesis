\documentclass[12pt]{report}
\usepackage[hidelinks]{hyperref}
\usepackage[T1]{fontenc}
\usepackage[utf8]{inputenc}
\usepackage[turkish,english,shorthands=:!]{babel}
\usepackage{apacite}
\usepackage[a4paper,width=155mm,top=25mm,bottom=25mm]{geometry}
\usepackage{epigraph}
%\usepackage{setspace}
%\doublespacing
\usepackage{graphicx}
%\graphicspath{ {graphics/} }
\usepackage{float}
\usepackage{caption}
\usepackage{subcaption}
\usepackage{wrapfig}
\usepackage[titletoc]{appendix}
%\usepackage{chngcntr}

%\counterwithout{figure}{chapter}

\setlength{\epigraphwidth}{.6\textwidth}
\setlength{\epigraphrule}{0pt}

\hypersetup{
    pdftitle			= {Thesis Draft},
    pdfauthor			= {Mustafa ilhan},
    pdfsubject			= {Trash, Art},
    pdfkeywords			= {mfa thesis, trash, art},
    colorlinks			= false,
    pdfborder			= {0 0 0},
    pdfpagemode			= UseOutlines
}

% Quotation command
\providecommand{\quotes}[1]{``#1''}
\providecommand{\singlequotes}[1]{`#1'}

\newenvironment{blockquote}{%
  \par%
  \medskip
  \leftskip=4em\rightskip=2em%
  \noindent\ignorespaces}{%
  \par\medskip}
  
\renewcommand*\bibname{Bibliography}


%****************************************
% Title Ideas for Thesis 
% - Trash as an Artistic Medium
% - Trash Notebooks: Trash in the Context of Art
% - Trash Notebooks: Exploring the potential of trash through art practice
% - Notebooks from Trash
% - Transforming Trash as an Artistic Act
%........................................


\begin{document}
\selectlanguage{english}
\begin{titlepage}
    \begin{center}
        
        \Huge
        Transforming Trash as an Artistic Act
        
        \vfill
        \Large
        A Master’s Thesis\\
        by\\
        Mustafa İlhan
        
        \vfill
        Department of\\
        Communication and Design\\
        İhsan Doğramacı Bilkent University\\
        Ankara\\
        November 2015
        
    \end{center}
\end{titlepage}

%\title{Transforming Trash as an Artistic Act}
%\author{Mustafa İlhan}
%\maketitle

\pagenumbering{roman}


%****************************************
% Ideas for Abstract
% You can add short definition of trash. Any object can be art in 20th century. Society generated trash and a culture that do not reuse and throws away.
% Structure of this thesis: written part of it and the project support each other.
%........................................


%****************************************
% Structure of Abstract
% What is this study?
% What is questioned/examined/investigated?
% How is it investigated?
% What is the outcome?
%........................................


%****************************************
% ENGLISH
\begin{abstract}
The purpose of this study is to discuss transformation of trash in the context of (art and) artistic act. Approach of artists who uses discarded material in their works treated in detail. It is questioned that why and how they use discarded materials and investigated in terms of theoretical and cultural aspects.

Within this framework, thesis project argues that it is possible to see the process of collecting, transforming and placing trash as an artistic act.

\vfill\noindent\textbf{Keywords:} Transformation of Trash, Trash in Art, Rubbish Theory, Throw Away Culture, Handmade Notebooks.
\end{abstract}
%........................................

\selectlanguage{turkish}
%****************************************
% TURKISH
\begin{abstract}

% Ne, Nasıl, Sonuç.

% Sanatta çöpün yeri inceleniyor, çöpü dönüştürmenin bir sanatsal eylem olduğu iddia edilmektedir. Ya da en azından proje kapsamında üretilen işin. Ya da bu bağlamda ele alınabileceği iddia edilmektedir.

Bu çalışmanın amacı çöpü dönüştürmenin sanat ve sanatsal eylem bağlamında tartışmaktır. Çöpün kültürel ve kuramsal olarak nasıl ele alındığı araştırılmıştır. Çöpü dönüştürmenin nasıl ne ne şekilde mümkün olduğu incelenmiştir. Sanatsal üretim yöntemi olarak çöpü (tercih etmiş)kullanan sanatçıların yaklaşımları ele alınmıştır. Neden ve nasıl atıkları sanat işlerinde ve eylemlerinde kullandıkları sorgulanmıştır ve bunun toplumsal ve teorik olarak nereye oturduğu/ne anlama geldiği/nasıl anlaşılabileceği incelenmiştir.

% Tez dahilinde geliştirilen proje ile çöpten dönüştürülmüş defterler ile çöpe bakış açısının dönüştürülmesi amaçlanmıştır. Farklı zaman ve mekanlardan toplanan kağıtlar birbirinden farklı kompozisyonlar oluşturmakta ve endüstriyel defterlere alternatif oluşturmaktadır. Aynı zamanda bu defter hakim sistemin dışındaki mekanlarda bulunarak (kamusal ortak alan) 

% Sanat dışı objelerin sanata malzeme olması sanatın dilinde genişlemere olanak vermiştir. Beraberinde pek çok akım ve üretim biçimini tetiklemiştir. Hem sanatın dili genişledi hem de objelerin sanat için farklı anlamlarda ve amaçlarla kullanılmaya başladı. Bu çalışmda çöpün sanatsal bağlamda ele alınması yapılacaktır. Çöpü daha iyi anlayabilmek için throw away. Çöpün nasıl sanatta yer bulduğu incelenmiştir. ve aynı zamanda aslında çöpün ne olduğu sorgulanmıştır.

Çöp ve sanatsal eylem birlikte alınarak incelenmiştir. Çöp olgusu ve sanat dışı malzemelerin sanatta kullanılması sorgulanmıştır. Teoriler ve methodlar ele alınmıştır. Çöpün ne olduğu sorgulanmaktadır. 

Tez dahilinde geliştirilen proje için insanlar örgütlenerek farklı mekanlardan farklı tiplerde atık kağıtlar toplanmıştır. Bu atık maddeler dönüştürülerek defter yapılmıştır. Renk, doku, biçim ve üretim yöntemi ile endüstriyel defterlere alternatif bir yaklaşım olması amaçlanmıştır. Bu defterler insanların ulaşabileceği belirli mekanlara dağıtılmıştır (yerleştirilmiştir). Mekanlar da aynı şekilde eleştirel bakış açısına açık yerler arasından seçilmiştir. Alternatif bir ortam sunan mekanlar seçilmiştir. Bu proje sonunda dönüştürülen şeylerin aynı zamanda dönüşütüğü şeyi de başka bir şeye dönüştürdüğü anlaşılmaktadır. Çöpe olan bakış açısının dönüştürülmesi amaç edilmiştir.

\vfill\noindent\textbf{Anahtar Kelimeler:} Çöpün Dönüştürülmesi, Sanatta Çöp.
\end{abstract}
%........................................
\selectlanguage{english}

%****************************************
% SAMPLE ABSTRACTS

% Bu çalışmada, ne kadar itici bir kavram olsa da, çöplük olgusunun çirkinlikten güzelliğe ulaşarak estetik bir haz uyandırması amaç edinilmiştir. Iticici gözüken bir konunun resim diliyle anlatımı ile tersine çekici ilginç bir sanat konusu olabileceğinin somutlaştırılması amaçlanmıştır. Sanatçı bir anlamda doğada yaşamda bulunan küçük ayrıntıları, sanatsal bir gözle yaklaşarak gösterebilendir. Resim için konu denince hep bildik tanıdık nesneler, belli bir düzenleme anlayışı (Natürmort) ya da idealize edilmiş konular, temalar akla gelir. Oysa önemsiz gibi gözüken itici karmaşık konular bile sanatsal kaygılarla ele ahnabilmelidir. Önemli olan, bu konalara estetik yaklaşım ve izlenen anlatım yol ve teknikleridir. Özgün baskı dallarından kazı resim (gravür) tekniğinin ele alınan konunun görselleştirilmesinde, anlatımında en uygun teknik olmasıdır. Çünkü her sanatçı kendi doğasına en uygun olan teknik ve malzemelerin olanaklarıyla kendini en rahat biçimde ifade etme olanağı bulur. Kendine özgü bir biçim ve anlatım diliyle dile getirmek gerçekleştirmek çabalan özgünlüğü yakalamak içindir. Bu çalışmada özgünlüğü yakalamak, çöplük olgusunun oluşturduğu dinamik yapı, objelerin biçimlerine bağlı kalmadan plastik dille görselleştirilmiştir. Seçilen teknik doğası gereği raslantısal tadlarada olanak vermektedir. Oluşturulan resimlerde durağan bir dengeden çok dinamik biçim ilişkilerine dayanan gerilimli bir etkinin yakalanması amaçlanmıştır.

% ON THE RELATIONSHIP BETWEEN IMAGES AND WORDS AND THEIR RELATION TO BODY AND PERCEPTION
% Prehistoric man created a mark and throughout the history this mark evolved and bifurcated into two: a word and an image. While images were cherished, words were set apart from images. 
% This thesis attempts to look at the relationship between images and words through seeking their connection to perception and body. It investigates how image-word dichotomy occurred and how this dichotomy obscured the connection between writing and body. The thesis also examines different approaches to overcome this phenomenon in the context of Modern Art. 
% By examining my artwork within this framework, it argues that it is possible to embody the inseparable relationship between images and words through reconnecting with the body’s primordial existence.

% ALTERNATIVE PHOTOGRAPHY IN THE DIGITAL AGE: PERFECT PHOTOGRAPHS IN AN IMPERFECT WAY
% This thesis explores the possibility of an alternative future of photography from the union of digital and chemical domains of the photographic medium. The historical photographic processes known as Cyanotype, Salt print and Vandyke Brown are employed for this project in conjunction with the modern inkjet printer produced digital negatives. 
% As being highly sensitive to the variables during the process, each alternative photographic print exhibits a visual uniqueness. In this aspect, there is conceptual correlation with the visual uniqueness of alternative photographic processes and the visual uniqueness of albinism. Emphasizing the human element in subject, vision and craft of making photographs, this project aims to produce unique photographs of a visually unique subject.

% LAND ART ON THE BORDER BETWEEN TOPOLOGY AND ATOPOLOGY
% The purpose of this study is to discuss the Land Art movement from a topological and atopological perspective. In order to establish an extensive understanding of the matters of topology and atopology, Arkady Plotnitsky’s formalization of quasi- mathematical thinking, which is derived from Jacques Derrida’s philosophy, is treated in detail. The artistic stance, Robert Smithson, as a major figure of Land Art movement is analyzed both from the artistic and the theoretical perspectives. Thereafter, an algebraic reading of the Smithsonian conceptualization is executed in order to illuminate the liaison between the Land Art movement and the matters of topology and atopology. Finally, the thesis project, Nonlocalizable Displaced Mirrors depicts the whole attitude, which is taken throughout the study, towards the issue of Land Art on the Border between Topology and Atopology.

%........................................


\setcounter{secnumdepth}{3}
\setcounter{tocdepth}{3}
\tableofcontents
\listoffigures
%{%
%\let\oldnumberline\numberline%
%\renewcommand{\numberline}{\figurename~\oldnumberline}%
%\listoffigures%
%}
%\listoftables

\pagenumbering{arabic}
\chapter{Introduction}
Some text...


%****************************************
% Structure of this chapter
% Section: topic and its significance
% Section: the purpose of this study
% Section: overview of chapters


% This written portion was to evaluate the project in terms of a justificatory theoretical framework.

% This question “what the future of photography would be like” is the main motivator of this thesis and the visual project is based on.

% Burada önemli olan şey soru veya sorgulanan şey. Bu soru projeye yön verecek. Yazılı tez ise bunun doğrulamasını, üretilen işin kavramsal ve kuramsal çerçevesini belirleyecek. Tartışmalar ne üzerine olmalı bu durumda. 

% The project aims to what? and what needs to justify what it questions? 

% At this point, the question of what will be in the future of photography has to be asked. Bu çok önemli, bir şeyler hazırlayıp neyin sorgulanması gerektiğini sormak gerekli.


%****************************************
% About the thesis statement and focus:
% Is it too broad?
% Is it debatable? Is is fact? (Trash is not a end point of objects, there is a life waiting for them?)
% What is my side? throwing away is required for society to go further? or omitting the tons of possibilities... 
% Supporting claims?
% What is my thesis statement? Again same topic... It needs to be solved...


% Tipografi ve ideoloji arasında bir ilişki vardır. Fotoğrafın geleceği? Kelime ve imaj arasındaki ilişki, beden ve algı üzerinden incelenebilir? kelime ve imaj arasında bir problme olduğundan bahsediyor. Sanat işlerinde çöp kullanmak toplum tarafından atılan şeyi tekrar kullanılabileceğini gösterebilir. Toplum ve çöp nispeten bir problemli bir durum. Burada bir dert var. Atılan tüketilen manalar var. Sanat bunları provoke mi ediyor. Peki ya ben ne öneriyorum, yani aslında bir şey önermem mi gerekiyor? Sanat ve çöp arasında nasıl bir ilişki vardır? Çöp ile diğer nesneler arasında nasıl bir ilişki vardır? 

% Zaten hali hazırda sanatçılar bunu yapıyorlar, tez aslında bunları inceliyor olabilir ama tez aslında benim işimi inceliyor. iş neyi soruyor ise aslında tez de onu soruyor. Yapılan işlerle sorulan sorular arasında bir bağlantı var. Bu noktada aslında benim işin neyi sorguladığını bulmam gerekli. Çöp dönüştürülebilir, sanatsal bağlamda. (Çok geniş değil mi abi sanatsal bağlamda demek? Çöp de çok geniş bir konu.) Konuyu bir şeklide daraltmak gerekli. 

% çöpe yeni bir alternatif yaşam üretmek gerekli. Aslında hepimiz bir şeyleri dönüştürüyoruz. Agnes varda aslında neyi anlatıyor: Toplayıcılık hala devam ediyor, bu toplayıcılar arasında snatçıları da geziyor, çünkü onlar da topluyorlar. Onlar da o çöplerde farklı şeyler görüyorlar. Kendi de mesela çöpe atılmış bir şeye anlam yüklüyor. Kendisininde aslında bir toplayıcı olduğunun farkına varıyor.

% Çöp ve dönüşüm, dönüşümün aşamaları... Başka sanatçıların methodlarını incelemek pek mantılı değil, onun dışında başka bir şey bulmam gerekli.

% Negative images of production and consumption. 

%****************************************
% Examples from other thesis.
% The aim of this thesis is to investigate the theories and methodologies of type design within the context of ideology. // The purpose of this thesis is to explore the transformation of discarded items through the artistic methodologies (or in the context of art). It is questioned the role of art in transformation of discarded items. This thesis follows the discussions of what is waste and the states of waste or meaning of waste. Within this thesis the project transformed the collected discarded items and then later spreads them to the community again. 


% The discussions (mentioned discussion are also formed the thesis project entitled ... )
% Aim of this thesis is ...
% What does it discuss?


%****************************************
% Sample Phrases:
% There has been a recent spate of artistic work focusing on (over-)consumption using the lens of disposal and discard.
% as a reaction to the consumerist society.





\chapter{Theory of Trash}

% FROM Uncle Fernando’s Garbage Triptych, http://alphabet-city.org/issues/trash/articles/uncle-fernando-s-garbage-triptych
\epigraph{\textit{\quotes{Anger is nothing compared to garbage:\\ Garbage eats anger for breakfast.\\ It eats all of us in the end.}}}{\hfill --- Priscilla Uppal}

In this chapter trash is examined in theoretical (and conceptual) aspect. Theoretical foundations of trash are discussed. (Further benefited from historical and social aspect of trash.) How do scholars conceptualized the trash? What do they say about it? In this chapter it will presented. It draws attention of many scholars and people who are professional in different areas. Further trash is a topic that common for the all ages through the human history. In other word this topic is not a new one. It always part of the human practice. It can be thought that it is refused, discarded thing as soon as possible there are people also carefully observing the concept giving attention to the discarding things.    

%%%
%%%
%%%
\section{Etymology/Terminology}
(Garbage, trash, rubbish, debris, detritus, waste, scrap, junk, refuse, discard, disposal, litter: during my research scholars and authors all have different names for the stuff.) I realized that many scholars and resources have used different words (or terminology) to define trash. Although there are slight differences between these words, sometimes they are used to signify same concept. It is important to understand vocabulary of this topic before dive into the theory. It is very important to choose right terminology, because ... and also make more clearer the border of the work. As mentioned it very diverse vocabulary to define the practice and there is no agreed terminology. Some scholars uses junk art, other recycling art. Sometimes it makes difficult to research on this topic. And also I must point out that this topic is not limited with the English language. The topic is broad and it is not limited with english literature and vocabulary. 
% TODO reference for other languages and societies.

%%
%%
\subsection{Origins of Vocabulary}
The origin of these synonyms reveals a whole side of human activity: our history revealed by what we have thrown away through the ages. What were people throwing out when these words were coined? 

Garbage is giblets, refuse of a fowl, waste parts of an animal (head, feet, etc.) used for human food. Garbology is a study of waste as a social science. In modern American usage, garbage is generally restricted to mean kitchen and vegetable wastes.

Waste comes from the Latin vastus, meaning empty, desolate, desert, or wilderness, and it’s interesting how the Romans called desert any wilderness that wasn’t settled, including forests.  German has retained the original meaning in wüste (desert). Vastus, which also gave us vast, vain, and devastate, came to mean a waste of money and ultimately garbage.  It is tempting to see a relation with the word west – the ancients didn’t like the west, where the sun “dies”, and associated the west side with death (the Egyptian tombs and pyramids are always on the west bank of the Nile, for instance)\cite{paul2013garbage}.

%From http://dictionary.reference.com/
\textbf{Waste.}
\begin{itemize}
\item v. to consume, spend, or employ uselessly or without adequate return; use to no avail or profit; squander:to waste money; to waste words.
\item v. to fail or neglect to use: to waste an opportunity.
\item n. An unusable or unwanted substance or material, such as a waste product. See also hazardous waste, landfill. 
\item Antonyms: save
\end{itemize}

\textbf{Trash.}
\begin{itemize}
\item n. anything worthless, useless, or discarded; rubbish.
\item n. foolish or pointless ideas, talk, or writing; nonsense.
\item n. literary or artistic material of poor or inferior quality. a literary or artistic production of poor quality.
\item Garbage collector.
\end{itemize}

\textbf{Garbage.}
\begin{itemize}
\item n. discarded animal and vegetable matter, as from a kitchen; refuse.
\item n. any matter that is no longer wanted or needed; trash.
\item Synonyms: litter, refuse, junk, rubbish.
\item Origin. early 15c., "giblets of a fowl, waste parts of an animal," later confused with garble in its sense of "siftings, refuse." Perhaps some senses derive from Old French garbe "a bundle of sheaves, entrails," from Proto-Germanic *garba- (cf. Dutch garf, German garbe "sheaf"), from PIE *ghrebh- "a handful, a grasp." Sense of "refuse, filth" is first attested 1580s; used figuratively for "worthless stuff" from 1590s. Garbology "study of waste as a social science" is from 1976.
\end{itemize}

\textbf{Rubbish.}
\begin{itemize}
\item n. worthless, unwanted material that is rejected or thrown out; debris; litter; trash.
\item n. nonsense, as in writing or art.
\end{itemize}

\textbf{Junk.}
\begin{itemize}
\item n. any old or discarded material, as metal, paper, or rags.
\item n. anything that is regarded as worthless, meaningless, or contemptible; trash.
\item v. to cast aside as junk; discard as no longer of use; scrap.
\item adj. cheap, worthless, unwanted, or trashy.
\end{itemize}

\textbf{Refuse.}
\begin{itemize}
\item n. something that is discarded as worthless or useless; rubbish; trash; garbage.
\item Antonyms: accept, welcome.
\end{itemize}

% TODO other words, relation, comparasition etc.

%%
%%
\subsection{Reuse and Recycle}
One of the most encountered words in this topic (transforming trash) is reuse and recycle. Both of them are used to signify the concept. 

According to the dictionary, the word “reuse” means “to employ for some purpose” or “to put into service.” Reusing involves usage of the same product unchanged in form. If any item is used again and again over time, it is said to be reused. The main purpose of reusing is to lengthen the life of the item or material. We give out used clothes for charity which results in reusing. Other examples are; buying some items and then selling them as used items, repairing some lawn equipment and reusing them, upgrading a computer, renting books, journals, periodicals, DVDs and others. The main purpose is to make the item last as long as it can. To reuse is to use something again instead of throwing it away or sending it off to a recycling company. Why throw something away when you can give it another life? Reusing is the second best way to conserve and be earth-friendly because it keeps items out of landfills and reduces the greenhouse emissions caused by purchasing a new product. Using something multiple times -- like using a disposable container more than once -- is not the only way to reuse; you can also give old items a new purpose. For example, use an empty coffee can to store small craft supplies or an old loofah as a scouring sponge for cleaning sinks.

Reuse occurs when waste in an unchanged chemical form is used in a process that did not create the original product. Examples include crushed glass containers (cullet) used to manufacture glass wool insulation or manufactured sand, and various forms of waste polypropylene used to make clothing.

According to the dictionary, “recycle” means “to treat or process (used or waste materials) so as to make suitable for reuse.” In recycling an item, it is processed into a totally new product. It is an energy consuming process. For example, if we put some plastic bottles, paper, or aluminum items in a recycling bin, these materials may be recycled into a totally different thing as clothing items, fabric, or maybe a quilt. In this process, energy is required which depends upon the stages of transformation.

Recycling occurs when waste in an unchanged chemical form is used in the same process that created the original product. Examples are crushed glass containers (cullet) used to make new glass containers, and scrap metal used in foundries. 

Reusing is possible with re seeing (rethinking). Reusing is possible meet the needs of the human itself. Using creativity and personal approach can change objects functions. It is possible to use objects for different purposes. 

Recycling can be viewed as down-cycling. The object smashed to the small particles to be used later in the production of something else. Although reuse can be viewed as up-cycling that gives another (or more) value to the discarded products. Down-cycling does not generates new meanings it tries to convert the product to already known state to process. 

Recycling is very similar the rotting (decaying), reuse is something like dry tree branches used by birds for their nest. These are two agents of nature to regain their resources.

Many scholar used the word recycling when mentioning works uses trash and the concept of it. However they are not mentioning the meaning that is decomposing things to the their particles. What they is actually is reusing and combining things, concepts, creating new mixtures.

% TODO not sure for this section, maybe mentioned above...
I use word 'x' because \ldots (not sure for this.)

%%%
%%%
%%%
\section{What is trash?}
In this thesis work to understand the different aspects of trash is a key element. Because as scholars agreed on it is part of our life and daily practice and it is very common concepts from developed western societies to rural areas, from disaster areas to \ldots(something beautiful here). Sometimes it is accepted as a problem that is need to carefully and seriously manged. On the other hand it is accepted as a source of diversity. (To establish a solid understanding for trash, it is important to see its dilemmas (it is really a dilemma?)). Therefore we should ask these questions: What is trash? How does it became trash? Is it a end product or a source material? How much is it valuable? How much is it dangerous?

%%
%%
\subsection{Trash is everywhere and, produced every time}
Modern (developed) societies are continuously generating trash and, pile them on landfills. It's a common object category that all people share its possession. During daily activities trash is generated and people get rid of them by throwing away. (Various objects become trash after their primary functions daily. Who defined the primary function? Primary function is the only function. People cares the package of the objects that they buy. They buy the coffee not the cup of it. After coffee finished the life of cup also finishes. There is a lifetime defined (or forecast) by the producer of objects. However it is not bound to the producer, also consumer play an important role. There are different choices. Throw it away. Keep it. Give it. Mostly wining choice is throwing away and by the result of it mountains of garbage are increasing.) The vast amount of industrial discarded items spread through the landfills to oceans. They are the result of highly complex industrial production methods. They are not easily disposable items. They live in the nature thousands of years. Most of them packages that are used to carry or protect other materials. After real material used these packages became valueless (or useless). (types of trash can be mentioned here, but currently in the artwork I'm using paper packages, therefore, it is more important.) How manage the all this increasing trash that damaging nature?  This is the common approach to trash and the main problem. (actually the sustainability problem.) It is not the only problem, It can be thought that it is a losing the ability to transform new things, alternative behaviors etc. (Instead of creating new opportunities or alternatives, it is a consuming all them and producing huge pile of trash.) (reference Zizek idealization of nature, to love nature is to love trash. Live with trash. Do not see it as trash actually. Then the question is how to love trash? how to live with trash? can living with our trash enrich (our perceptions, abilities)? how not to see them as trash and useless? Can it be possible with art?)

% TODO PRAP. REF.
Waste pickers are an important part of the story in Latin America, as more is being thrown away than ever. The World Bank estimates that the amount of solid waste generated in cities is growing faster than the rate of urbanization. The higher the income level and the rate of urbanization, the greater the amount of solid waste produced. OECD countries produce almost half of the world’s waste. Africa and South Asia produce the least waste. High-income countries have the highest collection rates and are most likely to dispose of waste to landfills or incinerators. Low-income countries have the lowest collection rates and are most likely to dispose of their waste in open dumps. However, low-income countries also have the largest numbers of informal waste pickers who collect, sort, and reclaim recyclables---thus reducing costs to the city and to the environment. (FROM Trash as Treasure, BY WILLIAM L. FASH AND E. WYLLYS ANDREWS)

% TODO PRAP. REF.
Literature is recycled material, a pretext for making more art. I learned this distillation of lots of literary criticism in workshops with children. I also learned that creative and critical thinking are practically the same faculty, since both take a distance from found material and turn it into stuff for interpretation. For a teacher of literature over a long lifetime, these are embarrassingly basic lessons to be learning so late, but I report them here for anyone who wants to save time and stress. (FROM Recycle the Classics, BY DORIS SOMMER)

% From Beautiful Trash Art and Transformation BY PAOLA IBARRA ReVista
% TODO PRAP. REF.
We relate to garbage daily. We use it, produce it and dispose of it. Endlessly. The most obsessive of us get rid of it as fast as we can. The hoarder likes to salvage a few things for later use---the plastic and glass containers, the cardboard boxes. We know that capitalism’s escalating cycles of production, consumption and obsolescence keep worsening an already problematic relationship between humankind, waste and nature (not to mention social and economic relations). Despite a relatively increased awareness about consumption and its consequences, the pace at which we also acquire and dispose of material objects is exploding. Particularly in the connection between
garbage and the arts, I am interested in two questions. First, the issue of recycling as a general practice in the arts; and secondly, in the whole issue of representation---that is, representation of waste as subject, and representation (of waste or others subjects) through waste as material. (BY PAOLA IBARRA)

%%
%%
\subsection{Cycle of trash}
Trash moves, objects moves from place to place. From homes to garbage trucks. From streets to land fills. From garbage baskets to sculptures. [REF. MIT Garbage project] Lots of different people touches to trash. With the trash what moves?

% Here is important thing is moving nature of trash. Objects moves and gains different meanings through this movement. Sometimes it becomes artwork, sometimes it becomes archelogical part, waits in the dumpsite or wait to be recycle. It also signifies that it is relative and result of a classification issue. All of them are creates a harmony. supports each other with different words and conceptualization. 

% <From "Trash Moves On Landfills, Urban Litter and Art" by Maite Zubiaurre
Below part Explains journey of trash, its different steps. Object moves and also trash also moves. How is it life of trash? This part explains life of trash. How does it intersect with other people in which places? This part also can be understand by pointing out different part of it with detailed explanation. For example an artist collect from trash from streets and the other one goes to the (For example Vik Muniz) landfill. In other words there are different places to touch on trash. Every place generates different story? Or to understand it more deeply it covers different part of it. Or provides ideas about it. Lots of people touches it from philosophers to artist. Also in the later parts it draws attention to them. [PRAP, REF Maite Zubiaurre]

Trash moves, all the time. It becomes a steadily growing heap of clutter behind closed walls, accumulates and festers under tight lids, travels from a small trash can in the kitchen to a large one on the curbside, joins other people’s rubbish when the garbage truck arrives, drives to the transfer station, where it circles around on conveyer belts, bids farewell to recyclable or composable goods, is loaded (if declared useless: the ultimate trash) into yet another garbage truck, or barge, or even train, until it arrives at its final destination: a sanitary landfill. Even in the landfill, it does not remain still. Monster "waste handling dozers" move rubbish around, compact it and press it against the soil. More importantly, they incessantly “sculpt” refuse with their huge shovels and caterpillar wheels, making sure the garbage mound does not tip over to create a fetid avalanche. When night falls, and the trash load of the day finally disappears under a thick layer of mud, detritus still moves: once underground, it settles differently, and decomposes at a different speed, thus continuously altering landfill topography: where there was an even plateau, now there is an abruptly descending slope, and a valley; and where there was a perfectly smooth road, now there are deep crevices in the pavement. This is how trash moves. But\ldots who moves on trash? In the United States, it is mostly big-wheeled machines, an industrious army of giant yellow insects busying themselves on a heap of rubbish. In Latin America, it is mostly people. People who hand-pick garbage, who build their shacks on densely compacted trash layers, and who, day in and day out, eagerly throw themselves into the boisterous cascades of fresh debris falling from garbage trucks. In many of the garbage dumps around the world, scavenging becomes a steady job. \quotes{Garbage properly \quotes{stored} and put away brings peace of mind, as do corpses boxed and buried, or criminals confined to a cell.} And
thanks to Art: for Art shows how trash--even the one that stops moving, and particularly the one that lies squished, squashed, and weathered, almost fossilized, on the ground---has the potential to move: to move us, that is. (through the works of Filomena Cruz's photographic series “Road Kill”) [PRAP, REF Maite Zubiaurre]
% >

%%
%%
\subsection{Perspectives related with trash from different disciplines}
This is very important because the problem of trash is being tried to be handled by different disciplines. In other words there different approaches to the trash. Different problems, different solutions. Which perspective that I have. In what ways my project differs from them. The purpose is to participate people in this work, by collecting them etc. And also offer to transform it. Rescue it and than later transform it. In short, the difference between the other disciplines must be clear. 
\begin{itemize}
\item Ecological perspective: Trash causes ecological problems and it treats the balance of nature. Animals do not aware of plastics materials and they unconsciously eat them.
\item Management of it handled by the municipals generally.
\item Technological perspectives
\end{itemize}

% Sample sentence:
% Such a conceptualization of waste as “the degree zero of value” has been contested for some time in different disciplines, ranging from economics to environmental studies, but most particularly by those studying consumerism or material culture

\quotes{We live in a badly engineered world, because the vast amounts of waste (both material and energetic) are needless; and that waste could be virtually eliminated through better design} \cite{mcdonough2010cradle}. In other word the problem is our technology which is not perfect. (and I'm not sure that at some point that technology will reach to the perfection or not.)

\quotes{As I prepared this issue of ReVista, some have asked me if Bogotá’s garbage crisis inspired the theme. Yes and no.  After Christmas, I traveled with a group of friends to the Chocó, an isolated and impoverished region on Colombia’s Pacific Coast. Christmas decorations abounded, and I noticed they were almost all crafted from used tin cans, old newspapers, discarded textiles and found wood objects. No one called it recycling. Trash was to be used and used again.}\cite{} Already a group of people live with their trash, what is problematic is here global ruling consumerist is not live with their trash. They can not handle their trash. There is no place for trash in their life.

\quotes{There’s a relationship between graveyards and landfills, one that makes us uncomfortable, Zubiaurre explained. \quotes{What is happening to trash is what is going to happen to us. We’re all going to end up in a dump, and we’re going to decompose. That’s the ultimate destiny of humankind, and we don’t want to face that.} Trash is also regarded differently, depending on where you live. Last year, an undergrad in Zubiaurre’s honors collegium seminar went to a poor neighborhood and scavenged through people’s trash; no one cared, Zubiaurre said. But when the same student went to Beverly Hills to go through trash, the police were nearly called. \quotes{Who decides what is public and what is private? How come trash becomes highly private in a rich neighborhood, but truly disposable in a poor neighborhood?} Zubiaurre said.} \cite{}

WALL-E is a 2008 American computer-animated science-fiction comedy film produced by Pixar Animation Studios. A robot named WALL-E, who is designed to clean up an abandoned, waste-covered Earth far in the future. WALL-E resembles to giant dozers that moves and piles trash in the landfills. 

%%
%%
\subsection{Throw away culture}
Continuously consuming things and disposing of something. It is an important concept to understand why trash is trash? (or how it become trash?) Behavioral pattern of throw away culture results in the trash. (This pattern does not consider recycling of it.) Artworks that are trying to raise awareness is related with this concept.

(Our trash generating behavioral daily consumption patterns \ldots What are the results of them? Why are they throwing?) (There is a behavior that throws away and opposite of it there is a collecting behavior.) 

%%% Reference from Identity, mobility and the throwaway society
we live in a throwaway society (Barr, 2004; Cooper, 2003, 2005; Cooper and Mayers, 2000; Strasser, 1999, cf. O’Brien, 1999; Hawkins and Muecke, 2003); 
\quotes{In the two previous sections we have demonstrated the paucity of the thesis of the throwaway society. In this thesis the undeniable matter of waste, itself pressing, urgent and excessive, is used to infer the presence of a society defined by its generation; a society ceaselessly discarding and abandoning its surplus as excess, as part of an endless desire for the new. Morally corrupt and unequivocally environmentally damaging, the rhetoric of the throwaway society classifies discarding as intrinsically bad and commands us to assume control of our wasting, suggesting the adoption of heightened regulatory practices around disposal as the means to ensure that we clean-up our act. The thesis, however, lacks depth and provenance. It is, actually, glib. Indeed, to infer the presence of a throwaway society from contemporary levels of waste generation is problematic for at least four reasons.}
% In this paper criticized the perception of throw away culture and arguments. They show the weak points of the arguments. What I understand from trow away culture?

In the scope of this thesis increasing consumption is motivator (draws attention more) but it does not bind to this. Always needed to keep in my mind that the practice of discard is not belongs to only current society. 

%%
%%
\subsection{How does it become trash?}
Here the purpose is to understand the dynamics that turn objects to trash. By understanding them is provide a roadmap (or ideas) how to turn trash to something valuable? (The purpose of this thesis is to find (or explore) a way(methodology, approach) to add value to object using artistic methods? Therefore first question is why they are less-valued, and ignored. How to make them valuable? How to make them part of our life again?)

%%
%%
\subsection{Types and Comparison of trashes}
The complexity of produced trashes of societies is increasing. For example developed countries that have nuclear plant generates radioactive wastes which highly hazardous for the environment is never exist previous societies. Think batteries and so on. Every society generates different types of wastes. Differs from country to country, society to society, ages to ages.

It can be thought that when the complexity of trashed increased required effort to repair, reuse and recycle is also increase. Therefore for the ones that have no complex tools it is becoming harder to reuse objects. In other words objects become more complex their re-usage becomes less likely. 

Different production process generates different types of trash. According to production process, decomposition process\ldots

The approach to the different type of trash will be different. In other word if trash is a result of classification of objects, it can be easily extracted that there is classification inside of it. There are some trashes that are more close to the people. More easy to convert them. more easy to regain to the society.  

%%
%%
\subsection{What is wrong with trash?}
Relationship between entropy (second law of thermodynamics) and waste. Resources of nature turns to waste that it can revert it. Creating that are reversible again is problematic through the nature of sustainability. What is produced after it is consumed become worthless. 

From my point of view and approach in this thesis, trash is only one of the thing that is being discarded by humans and communities. There are lots of things that are being excluded such as homosexuals, trans, disabled peoples etc. Even if they are excluded, there is also life for them. 

% TODO Reference
John Scanlan's book, On Garbage shows how western progress always has cleared away and discarded what went before; not only material waste but also knowledge. He believes that by examining our garbage we can gain useful insight into the condition of contemporary life.

%%
%%
\subsection{Collecting trash}
One of the most important parts of the using trash in the artwork (or expressing something, or representation) is to collect them. What are the dynamics(considerations) of collecting them? (easily accessible materials or unique items.) Where to store them? Does it mean that live with trash? In other words collecting trash and using them is live with them? (making them part of life.) After the being part of the are they still trash? Can be thought that it is something that affects the lifestyle. (possessions and trash.) Another question is that how differs collecting trash from collecting other things such as objects that have archival value. What is the driving force? You may collect it to prevent object being lost. For archival things what you collect is something that has some sort of social use and meaning which is going to disappear. However, trash is never disappearing, even its amount increasing rapidly. For archival things people have memories with them, but does some applies for the trash? Who wants to keep trash? or who wants to re-see(re-visit) trash again (in a museum for example)?

% TODO: ragpickers from benjamin and archades project.

%%%
%%%
%%%
\section{Rubbish Theory}
Objects have a lifetime and they don't remain same through that lifetime. Their value, usage, location change over time. During its lifetime objects may circulate different markets and values systems like economical value, social value, aesthetic value etc. Especially this cycle has picked up speed with the advent of consumer culture, our most recent technological gadgets becoming obsolete within 3 years. Objects function and value are transformed by relocation and revaluation of objects from one place to the other or one discipline to another. This flow(transition) and transformation theorized with Rubbish Theory by Thompson \cite{thompson1979rubbish}. Thompson looks at the creation and destruction of value in man-made objects, cultural artifacts, and ideas. He notes how an object’s economic and/or cultural value diminishes over time rendering the objects worthless or redundant. The theory looks at how some of these objects then regain value, such as antiques or historic homes. It claims that there are three types of objects; transient (normal state, decreasing value, circulating), durable (permanent, increasing value, removed from circulation) and rubbish(zero value, will be destroyed or reinvested for economic and social value). The transition from transient to durable is only possible firstly transient to rubbish and later rubbish to durable. Further, there is a common idea/argument/motto that is "trash to treasure" among artists who use trash as a medium. Rubbish theory presents a conceptual approach to this argument. 

Although Thompson is quite successful categorizing states of objects throughout their lifetime, claimed transitions between states in the theory have some problems. Thompson label some transitions as possible and the others as impossible. "He allows goods only to move from a transient to become rubbish, and from rubbish they can either be destroyed or become durable. Movement in the other direction, from durable to either transient or rubbish, is not allowed in this system" \cite{meadow2011relocation}. Further, he does not allow move from transient to durable. However, Duchamp's fountain breaks this rule. Because urine used as a fountain is still functional and have a place in the market. In another word, it is not rubbish. This urine with the approach of Duchamp turned to be an artwork. It is one of the most influential piece of modern art and one of the best examples of ready-made. After Duchamp's intervention to the urine, it becomes a durable object placed in a museum.

In rubbish theory beyond the objects states how it happens transition of objects in practice is missing and Parsons fills this gap by claiming that transition from rubbish to durable are possible with finding objects, displaying objects, re-using objects \cite{parsons2008thompsons}. (explain details) (It can also be thought that they are the way of value creation.) (turning trash to treasure or something else is a value problem. transforming them creates new a value system? or just finding place existing value system. by the way there is a value theory related with (or inside of) game theory.)

Further another conducted research examines the psychological, social, and aesthetic factors involved in found object and found that ... \cite{camic2010trashed}.

"Rubbish theory, a philosophy that attempts to address how value is placed on material objects." "It is a body of thought that addresses how the value of material objects is socially constructed and deconstructed." "An awareness of rubbish theory is important to the understanding of the sociology of consumption and waste because, while what is and is not considered garbage may seem obvious and natural, the value of objects is based on the perceptions of people." "The classic examples of these categories are the durable 18th century Queen Anne tall-boy chest and the transient used automobile." "What decides whether or not something is a durable or transient is often the perceptions of the powerful members of society, those with a vested interest in owning objects whose value will always increase, while the remainder of society owns objects whose value will eventually decrease to nothing."

The paper suggests the Theory is useful in foregrounding the material dimensions of markets. It also highlights the importance of thinking in terms of movement, flow and circulation in markets. Finally the theory suggests that value emerges through our ways of seeing and placing objects. (From Liz Parsons)

%%
%%
\subsection{Rubbish Theory in Practice}
% TODO PRAP. REF.
A key critique of the rubbish theory is its neglect of the practices of value creation. Thus the paper draws from existing studies in consumer research in exploring three such sets of practices: finding objects, displaying objects, and transforming and re-using objects.

Plenty of work in consumer research explores the ways in which goods might act as symbolic resources for lifestyle and identity construction (i.e. Belk 1988), but there is less reflection on the actual practices and activities through which goods become meaningful and valued. McCracken’s (1988) work on ‘Meaning Manufacture and Movement in the World of Goods’ begins to address this gap. He views advertising and the fashion system as instruments of meaning transfer between the culturally constituted world and consumer goods. He then suggests that a series of consumer rituals operate to transfer meanings from consumer goods to the individual consumer. These rituals include those of possession, exchange, grooming and divestment. The strengths of his argument include a focus on the mobile quality of meaning and some exploration of the instruments though which meaning is transferred. However he is not clear as to the practices which constitute these rituals. In addition his contention that ‘meaning resides in three locations: the culturally constituted world, the consumer good, and the individual consumer’ (1988: 89) fails to completely capture the complexity and fluidity of meaning movement. There is a linearity to his conceptualisation which misses the constant flux and flow of meanings in markets.

In ‘The Social Life of Things’ Appadurai (1986) highlights the restlessness of objects arguing that ‘from a methodological point of view it is things-in-motion that illuminate their human and social context.’ (1986: 5). Appadurai usefully observes that ‘commodities, like persons, have social lives’ (1986: 3). He focuses on the ‘commodity potential’ of things. ‘things can move in and out of the commodity state, that such movements can be slow or fast, reversible or terminal, normative or deviant’ (1986: 13). However these movements do appear to be reduced to the opposites of commoditization and singularization, one might ask the question, does anything exist in-between? This is where Thompson’s Rubbish Theory comes in.

\subsubsection{Transients, Rubbish and Durables}
The category of objects and experiences of no value (rubbish objects and experiences) is largely invisible. The transient represents the usual state of commodities as objects which are declining in value and which have finite life spans. Whereas the durable increase in value over time and have (ideally) infinite life spans (1979: 7). Thompson uses the example of a used car as a transient and Queen Anne tallboy as a durable. He further observes that their category membership determines the way we act towards them. 

Thompson argues that rubbish represents an important possible ‘in-between’ category in a ‘region of flexibility’ which is not subject to the same control mechanisms of the valuable and socially significant categories of transient and durable. Therefore it ‘is able to provide the path for the seemingly impossible transfer of an object from transience to durability’ (1979: 9) he further suggests that ‘a transient object gradually declining in value and in expected life-span may slide across into rubbish’ (1979: 9) where it has the chance of being re-discovered, brought to light or cherished once gain. Figure one demonstrates the possible paths an object may take (from transient to rubbish and from rubbish to durable). 

Thompson comments that we only notice rubbish when it is in the wrong place, and highlights the embarrassment and anxiety that mis-placed rubbish, or rubbish which has found its way in to the wrong place can cause ‘Something which has been discarded, but never threatens to intrude, does not worry us at all.’ (1979: 92) but rubbish in the wrong place is ‘emphatically visible and extremely embarrassing’ (1972: 92). (Further there is similar analogy at Waste and Want. The author give the example of shoe and claim that thrash is relative. The shoe on the dinner table is something disgusting but in the foots is not at like that.) Rubbish objects are things that are no longer used or loved or cared for and often no longer seen. Rubbish objects linger on the periphery of our lives, in the back of the drawer, bottom of the wardrobe or cupboard, corner of the garage or garden shed gathering dust.

\quotes{In an ideal world\ldots an object would reach zero value and zero expected life-span at the same instant, and then... disappear into dust. But, in reality, it usually does not do this; it just continues to exist in a timeless and valueless limbo, where, at some later date (if it has not by that time turned, or been made, into dust) it has the chance of being discovered’ (1979: 8-9)}

\subsubsection{The Practices of Value Creation: From Rubbish to Durable}
Three such sets of practices are explored below, they include: finding objects, displaying objects and transforming and re-using objects. It is argued that each of these sets of practices changes the way we view an object moving it from being seen as a ‘rubbish object’ of no value to a ‘durable object’ of increasing value.

\textbf{Finding Objects:} One key way in which objects may slide from the category of rubbish to durable is through the act of finding. Indeed, Gabriel and Lang (1995) include the ‘Consumer as Explorer’ as one of many possible consumer identities.What, then might one mean by ‘the find’? Ultimately the find relates to discovery, and suggests that something has been otherwise overlooked, ignored or hidden away. The find may not involve objects which are new to us, it is possible to find some of ones own items if they have been hidden away long enough in an attic and thus made strange to us. The concept of the find also suggests that the found object has some qualities that others (or indeed ourselves) have in the past overlooked, as such it is closely related to ‘bringing to light’. The find may be extended to embrace features of objects as well as objects themselves. This directs us to their ‘potentialities’, objects may have been there all along but we’ve suddenly found them to be useful, likeable or beautiful. It might be that some aspect of them has simply been brought to our attention. Equally, as discussed below in relation to transforming objects, we may make alterations to objects which bring out their potential. The transition from thing of little or no value (rubbish), to thing of value (durable) can result form a relatively minor shift in the way we see something.

\textbf{Displaying Objects:} by displaying objects in home, fashion etc. (This is not a strong category.)

\textbf{Transforming and Re-using Objects:} These transformations may involve creating new uses for old things to fit in with contemporary lifestyles. Transformations may also involve the modification or updating of objects through painting, alteration or repair. Transformations may not only be based around creating new uses but also creating new looks. The re-use of objects also creates value for things that otherwise would be allowed to slip away (or slide terminally into the rubbish category).

%%
%%
\subsection{Theoretical Analysis of Found Object}
% TODO PRAP. REF.
\subsubsection{Seeking Objects}
As a species Homo sapiens has been gathering and collecting objects for thousands of years. Food, clothing, weapons, fuel, animals, and plants are the more obvious items, but visually pleasing objects, things that arouse curiosity, and shapes that stimulate the imagination have also been sought. The search for “things,” collecting them (Humphrey, 1979), and the need to embellish and make the ordinary special (Dissanayake, 1988) have been essential parts of the evolutionary process of human development (Bettinger, 1991; Dissanayake, 1992). For example, many early cave drawings occur around a natural feature within the cave such as a projection or indentation (Bahn, 1998; Lewis-Williams, 2004), making these natural features potentially comparable to found objects in modern and contemporary art (Read, 1930). When early prehistoric eople recognized a cave’s natural feature as a “found object” and incorporated it in a picture, it is possible that the natural feature took on a higher value than if left alone, thus adding to the value of the object on the cave wall. Once found and incorporated in paintings and drawings on a cave wall, it is possible that these protrusions and indentations became objects that functioned symbolically.

In contemporary societies, people seek objects to adorn bodies, decorate homes and gardens, and personalize places of work (Menzel, 1994). Seeking and finding objects, whether in vast Asian cities, remote African villages, or the high streets of Europe and North America, have become part of daily life for millions of people. Although most of these objects are purchased new in shops or online, many others are preused items, castoffs, or trash found in back alleys and country lanes, dumpsters, and rubbish bins or purchased cheaply in flea markets, garage and boot sales, and in shops catering in previously owned goods. Some individuals have chosen to salvage objects out of economic necessity, as depicted in Jean-Francois Millet’s 1857 painting The Gleaners and by Agne`s Verga’s 2002 documentary film Les Glaneurs et la glaneuse, whereas others have done it for a range of other reasons, including a desire to rescue them (Belk, Wallendorf, \& Sherry, 1989). These “other reasons” pose an opportunity for researchers to better understand why people voluntarily seek society’s discarded material objects and how they make use of them. A form of cultural reuse, the process
of salvaging and using found and second-hand objects has potential implications for \ldots 

% [TODO adapt for my case] The main aim of the current study was to produce an initial conceptual model that explains contemporary found and second-hand object use in blablabla.

% [TODO adapt for my case] Found objects first came to the attention of the general public through their use by artists in early part of the 20th century. To help contextualize the application of found objects, the following section provides an overview of their use by Western artists. This is followed by an examination of objects in human development and includes aspects of psychoanalytic theory, material objects and their relationship to identity, selected cognitive theories, and an outline of the social life of objects through rubbish theory.

\subsubsection{Found Objects in Art}
The term found object, as used in this article, refers to an existing object or artifact that is picked up (found) and generally not bought or originally intended as art, yet it is also considered to have some value (e.g., aesthetic, novelty, remembrance) to the finder. It is during the locating and finding process that the value of the object, once considered to be junk or rubbish, changes. The junk object becomes transformed into the valued found object. Objet trouve, translated from the French as “object found,” appears to have been first used by Marcel Duchamp in 1913 in reference to objects he made use of in his “readymade” art (Richter, 1965). His earliest known application for an objet trouve was seen in his well-known piece Bicycle Wheel, “where he had simply upturned a wheel on a stool” (Gale, 1997, p. 97) and labeled it a work of art. A more public and controversial introduction of the readymade occurred in 1917 when Duchamp attempted but failed to exhibit the highly contentious piece The Fountain, where a single white urinal became a readymade piece of art (Tomkins, 1996). These pieces were the beginning of Duchamp’s shift from an art striving for beauty and possessing a higher complex or hidden meaning beyond what was seen to an art form that made use of, and occasionally celebrated, the common materials and objects found in everyday living.

Gascoygne (1936), writing about the artist’s use of “the strange medley of materials” (p. 169), referred to as objets trouve in Surrealist art, suggesting that the artist “discovers a hidden symbolic significance in the [found] object which is preserved when the object is ‘framed’ as art” (p. 170). The finder discovers an unrealised significance in the object. A new boundary is formed around the object by the finder through removing it from its found environment and placing it in a new one, thus empowering the finder in the role of creating a new reality for the object. He argued that the found object, before it is found, approximates a zero value aesthetically; the zero value increases for the finder---beholder on discovery of the object and increases further if the object is placed in another context. \ldots the meaning of material objects was derived from their symbolic relation to another (e.g., person, time, place, experience) rather than through their physical attributes (Causey, 2003; Csikszentmihalyi \& Rochberg-Halton, 1981). 

How a region of flexibility develops, what social factors are involved in taking innovative and creative responses toward rubbish, and how an individual changes (and enhances) his or her creative responses toward society’s detritus are areas that require additional examination. (Important points a gap in the literature. Although Parsons article analyze practices, it is not very broad and detailed. Further what is the importance of it is missing.)

Camic's \quotes{project surveyed participants who currently use found objects in their daily lives in order to investigate and understand possible purposes, determining factors, and potential benefits of their use.} 

\textbf{Motivations and found object practice:}
\begin{itemize}
\item Discovery and engagement
  \begin{itemize}
  \item Seeking out: The adventure of hunting
  \item Finding the unexpected
  \item The location of discovery
  \item Metamorphosis and transformation: Imagining a use for the object
  \item Creative action
  \end{itemize}
\item History and time past
  \begin{itemize}
  \item Trigger of personal memories
  \item The object’s story: Reflection about earlier “life” of object and previous owner
  \item Capturing something elusive: Making sense of an unknown past
  \end{itemize}
\item The symbolic and functional object
  \begin{itemize}
  \item Aesthetic properties: Visual, tactile, original
  \item Cost: The thrill of a free find
  \item Evocative: More interesting than something new
  \item Useable solutions for practical problems
  \end{itemize}
\item Psychological processes
  \begin{itemize}
  \item Personal resourcefulness
  \item Impact on memory, emotion, cognition
  \item Meaningfulness of object
  \item Arouses interest: Motivates
  \item Social engagement
  \end{itemize}
\item Ecological affirmation
  \begin{itemize}
  \item Environmental concerns
  \item Against the cult of the new: Slowing down consumption
  \item Transforming rubbish
  \end{itemize}
\end{itemize}

The importance and enjoyment of found objects to those who participated in this study, using them in various ways and for different reasons, were strongly evident. The interaction between finder and object is an attempt to make meaning of an object that has been found, and by being found and desired becomes transformed.

Gascoygne (1936) recognized that finding an unrealized significance in a material (found) object was empowering partially because, through the creative agency of the finder, the object’s aesthetic value had increased from zero to something greater. The transformation of rubbish from a negative value to a positive value requires the finder to develop a symbolic meaning, and sometimes a functional use, for the object that goes beyond its present situation as culturally labeled detritus while simultaneously responding to its current physical and aesthetic elements. When the found object is seen by the finder as a symbol representing another entity (e.g., when an old blue bottle with foreign lettering comes to symbolize far away intrigue, mystery, and sophistication), support is given to what Dittmar (1992) described as socially constructing a material identity for the object. Expanding on Dittmar’s use of a social interactionist perspective, the results of the present study support the possibility that the entire found object process---finding, reclassifying, and reusing objects---becomes a symbol of identity for the finder. This supports Digby’s (2006) argument that individuals make use of salvaged objects as souvenirs, which are no longer part of the commodity cycle, to rework and construct individual and social identities.

An important difference, however, which also appears to contribute to the aesthetic experience, takes place when discovering an object unexpectedly in a nonpredetermined place and time. Unlike appreciating art in a museum, which is a boundaried activity occurring at a scheduled time and place with the anticipation of finding and looking at art objects, finding discarded objects can occur anywhere at anytime and thus, according to participants, can “trigger” a burst of sensory attention and a “surge” of cognitive activity on the part of the finder. 

Results of this study also support consideration of found objects as important artifacts that are signifiers of cultural meaning.

[TODO need to focus on dynamics of value creation, and to do that analyzes of values bound to materials. So what is value?]
\textbf{Value.}
\begin{itemize}
\item n. relative worth, merit, or importance: the value of a college education; the value of a queen in chess.
\item n. monetary or material worth, as in commerce or trade: This piece of land has greatly increased in value.
\item n. estimated or assigned worth; valuation: a painting with a current value of $500,000.
\item Value is that quality of anything which renders it desirable or useful: the value of sunlight or good books. Worth implies especially spiritual qualities of mind and character, or moral excellence: Few knew her true worth.
\item the desirability of a thing, often in respect of some property such as usefulness or exchangeability; worth, merit, or importance
\end{itemize}

%%%
%%%
%%%
\section{Literature review, discussions, ideas\ldots}
Trash art is not collage (assemblage or found object) or fragments. it is more than that. The carried messages through the medium have different meaning. It has relationship with activism, craftivism. It refuses consumption based life cycle. It suggests a life practice.

"Every day, we put unwanted material in toilets and garbage bins, regularly flushing it away or taking it out in bags to be transported far away from our homes by others. The names we give this material---waste, garbage, refuse, trash, rubbish--- have pejorative definitions. Worthless. Rejected and useless matter of any kind. Unimportant." "Our trash is a testament; what we throw away says much about our values, our habits, and our lives." "While dictionary definitions of garbage describe it as “filth” and “worthless,” scholars are careful to note that perceptions of waste and the value of material are neither static nor universally shared." "\ldots the question of who owns these discards is not trivial." "The absence of a waste stream aroused suspicion, just as the presence of particular items tell us about the habits of the consumers who generate a waste stream. Our trash is part of us, whether or not we choose to acknowledge it." \cite{zimring2012encyclopedia}

%%
%%
\subsection{Examined-life, Zizek}
(There is a curious fragment where Zizek, dressed as a sanitation man, stands in a waste deposit in a pile of garbage and ponders over garbage and human existence. In documentary film, Examined-life, Zizek talks about ecology in the middle of a garbage dump in London, and his part starts with these sentence: "This---garbage dump--- is where we should start feeling at home". He asserts his claim at first and go through explaining how ecology turns to ideology and mentions wrong perception about the ecology. Draw attention to notion of even if trash disappears from our world but not world. It seems as though the thrown out garbage disappears from
our world. However, it disappears only from the world of illusions, but still exists in reality. He thinks that the way of approaching ecology is problematic, because accepting that nature as a balanced harmonious thing. He claims that it is ideological in the sense that wrong thinking important problems. Nature contains unimaginable catastrophes. think oil and distinct animals and plants. we profit balance part of the nature but it is created from catastrophe. Are we aware of this catastrophe. He asserts that ecology will slowly turn to religion that "is a kind of an unquestionable highest authority." Ideology of ecology warns us like, "Don't do that. It would be too much." its voice is like "Don't mess with D.N.A. Don't mess with nature. Don't do it" etc. We should not forget that we are part of the ecology. We must more alienated from the nature. We must find poetry and spirituality in the dimension of trash. That's the true love of world. Love is not about idealization. This part will be extended later.)

\quotes{According to Zizek, modern understanding of ecology is the real false consciousness, connected with mystification of real problems. Postmodern mysticism arises when disasters begin to be rationalized, interpreted in strict logic terms of cause-effect relations. Such interpretation makes life easier. However, nature is not an absolute balance and total harmony (this aspect of Zizek’s thought makes him akin to classical conservatives). Nature is a series of unthinkable disasters. Zizek believes that ecology is transforming into a new western conservative ideology: “One should not play games with nature! Do not touch DNA! Do not develop new medicines! Do not invent new technologies!” How one should react to these reproaches? Zizek’s recipe is to reinforce alienation from nature, to become more artificial.} \cite{vafin2012zizek}

His basic argument is that the modern eco movement is a conservative ideology. It says don't do X from an authoritarian high ground and it idolizes and mystifies ecology. Basically he says the eco movement has it's head in the clouds. If it loved the environment it would recognize the rubbish we create and the chaotic nature of ecological change and try to further divorce itself from that process(nature) and try to turn the whole thing into art. 

\begin{itemize}
\item Our relation to our filth follows an “out of sight, out of mind” principle, but trash doesn’t disappear.
\item Ideology addresses real problems but mystifies them.
\item We search for meaning when a horrible event happens to make it easier to accept.
\item The ideology of ecology is that world is in the best possible state and that humans disturb nature.
\item Nature is not an organism in balance that humans exploit, but rather a series of great catastrophes.
\item Ecology is becoming more like religion with dogmas.
\item Even if we learn the potential catastrophes of nature, we ignore them as long as they don’t manifest near us.
\item The solution is not to worry about saving nature, but to figure out how to survive without it by becoming more artificial.
\item Learn to love our trash as a part of ecology.
\end{itemize}	

How much harmonious all these produced materials with nature? How many animals and the plant can use these discarded items? (Because of complex production methods their recycling requires complex processes again. Some of them already produced to protect goods from natural factors (like decaying etc.). However how can we protect nature from them?) It is very hard that spontaneously they become harmonious with nature. However, some artists turned to trash into site-specific sculptures that are more than trash heap. not discarding but bracing our attitudes turned them to a something that worth it to watch and think about it. (Converting what we create harmonious with the existing system.) Because it is not possible to think that nature will live harmoniously what we created. The more likely idea will be we will live harmoniously with what we create.

Getting rid of it is not the significant action. It still waits us at somewhere else or the next generations have to cope with it. (Establish relationship with the afterthought.) Maybe the first thing to do is accept the trash, to accept that there are things out there that serve nothing. To break out of this eternal cycle of functioning.

The issue of trash is not limited with ecological and economic perspective, it has also other dimensions.(draw attention to multidimensionality of this topic, but why? and what are the other dimensions?)

Trash itself is not the only problem, the practice, lifestyle causing it is more important problem. The dynamics of market and flow of objects into it plays important role production of trash.

Trash of developed societies has higher decomposing period in the nature. Therefore results in higher damage to the nature, hard to reuse, hard to transform, sometimes not to safe to keep them because toxic elements etc. 

\textbf{Summary.} Zizek mentions that ecology is something wrong point to approach this topic. Show the problematic understanding thought to the garbage. Suggest another thing, and point out the need for the new approach. Beyond the ecological problem there are different sides of it. Different readings can be added to the topic. Adding spirituality and new aesthetics dimension to the topic. 


%%
%%
\subsection{Culture, Values, and Garbage}
"The Trash Talk project emphasizes the complex, yet overlooked, relationships that garbage and people share. In terms of their relationship to garbage, all people interact with it on two levels. One is a material connection, indicative of the physical and sensory contacts that people have with garbage. In some households, this connection begins with an individual removing an item from packaging, disposing of that item in the kitchen receptacle, placing that item and others into a larger bin, taking that bin to the curbside, and then the material connection ends. Others, including workers in sanitation plants and recycling centers, then continue a material connection with the garbage, but the material connection of the consumer and the garbage ends with the bin on the curbside. The second connection that people maintain with garbage is an ideational one. Unlike the material one, which is manifested in things that can be touched, moved, and sensed, the ideational connection operates on the level of cognition. The differentiation of an item of value from an item of trash, for example, has nothing to do with the material principles of the object. Instead, humans determine whether the object is of value or whether it is considered trash. The decision of whether an individual decides to dispose of a broken radio or to consider it an heirloom to be kept is highly subjective and rooted in the value systems of a culture." "After the item is eaten, the individual has to decide what to do with the remainder, such as the leftover package. The package might be reused, re-purposed, or recycled but, most likely, will be disposed of in the trash." \cite{lukas2012culture}

%%
%%
\subsection{Garbage in Modern Thought}
"Philosophers and intellectuals have expressed the need to focus on the centrality of garbage, but for everyday individuals, the understanding of garbage is often as something “out of sight, out of mind.”" "Modern humans, as part of their penchant for consumption and unsustainable living, often think very little about the waste that they produce." "Like many aspects of capitalist living, the person throwing away a piece of trash does not connect the various levels of production, consumption, and post-consumption involved in the trash. It becomes a secondary matter---an afterthought." "Martin O’Brien, among many thinkers, argues that the understanding of garbage should be a central concept, especially since garbage typically correlates with social change, social roles, and institutions. Thus, beyond the level of individuals and their relationship to garbage, there is an interest in understanding the central role that garbage plays in all of society’s roles, institutions, and forms of change." "Garbage is excess--- it is a part of society that society no longer desires." \cite{lukas2012garbage}

%
\subsubsection{Categorization and Value}
"Garbage is categorization, according to Susan Strasser." "In recycling programs and in places of refuse disposal, items of trash are categorized depending on their potential value, possible environmental harm, or time of decay. Consumers have become accustomed to the categories that are often applied to garbage. Many cities require people to dispose of their garbage in an orderly fashion---perhaps separating wet household waste from dry---and recycling programs ask individuals to divide their recyclable items into sets (such as plastic, glass, aluminum, and paper) and smaller subsets (such as PET or 01, PE-HD or 02, and PVC or 03). Garbage is an illustration of how humans use mental categories to order the material world." \cite{lukas2012garbage}

"According to John Scanlon, garbage is indicative of a separation of the world---the desirable from the unwanted. Michael Thompson uses the riddle of the rich and poor person’s approach to snot (one keeps his in a handkerchief, the other disposes of it with a tissue) to underscore the curious ways in which garbage is connected to the issue of value. While garbage is universal---all societies, extinct and extant, have produced or produce garbage--- the conditions under which garbage is understood are culturally determined. Many non-Western societies attach a much greater value to items after they are discarded. In the United States and many other nations, garbage often results not because something no longer has utilitarian value but because the item in question is defined as something of no value. Thus, garbage is not only an objective condition of material culture, but also a subjective one of mentalist culture. People define what is trash and what is valuable." \cite{lukas2012garbage}

%
\subsubsection{Semiotic Context}
"In popular writing (such as novels), in television, films, music, and other forms of mass expression, the term trash is used to signify work that is of especially low value." \cite{lukas2012garbage}

%%
%%
\subsection{Garbology}
"Weberman infamously used techniques of what he deemed garbology to uncover what he saw as the essential nature of people. He once said, perhaps indirectly referencing Jean Brillat-Savarin’s quote about food, “You are what you throw away.”" \cite{lukas2012garbage}

"The field of garbology involves the study of refuse and waste. It enables researchers to document information on the nature and changing patterns of modern refuse, hence assisting in the study of contemporary human society or culture. According to the Oxford English Dictionary, the term was first used by waste collectors in the 1960s. A. J. Weberman popularized the term in describing his study of Bob Dylan’s garbage in 1970. It was pioneered as an academic discipline by William Rathje at the University of Arizona in 1973."

In his book “Garbology: Our Dirty Love Affair With Trash”, the Pulitzer prize-winning author Edward Humes notes that other wealthy countries with high living standards have rejected the disposable products that make up much of America's rubbish.

%%
%%
\subsection{Trash as History/Memory}
% TODO From encylopedia
\cite{bullock2012trash}

%%
%%
\subsection{Trash Aesthetics}
%Walter Benjamin's trash aesthetics and Adornos reflection. 
\quotes{Benjamin’s approach to history is through ‘trash’---through the spent and discarded materials that crowd the everyday}  \cite{highmore2002thrashaesthetics}. Benjamin’s importance as a theorist of the everyday is most evident in his attention to the everyday experiences of modernity. In the face of the endless proliferation of trash, Benjamin potentially suggests a ‘trash aesthetics’ that could be used radically and critically to attend to the everyday. The method might be thought of in terms of ‘recycling’ – an ecology of everyday experience.


%%%%%%%%%%%%%%%%%%%%%%%%%%%%%%%%%%%%%%%%%%%%%%%%%%%%%%%
%%%%%%%%%%%%%%%%%%%%%%%%%%%%%%%%%%%%%%%%%%%%%%%%%%%%%%%
%%%%%%%%%%%%%%%%%%%%%%%%%%%%%%%%%%%%%%%%%%%%%%%%%%%%%%%
% HERE Same sample phrases are listed you can use them:

% The work that follows is divided into three sections

% The artist thinks, acts, performs music, and writes outside the framework that society has created.

% this thesis takes a rather different approach to the resonant possibilities of discarded things. It looks to philosophical ideas and our entangled experiences of things, time and stories, which need to be traversed in order for a discarded object to be called ‘waste’.

% I also want to suggest a different way of considering trash. Maybe art maybe suggest an alternative way of seeing.

% I’d like to criticise a set of concepts or ways of thinking about discarded things that to me just don’t seem quite adequate.

% In an effort to expand art activism's capacity to create real social change, this article will (1) examine the theoretical framework behind art activism and art's efficacy in accessing emotional pathways; (2) explicate ways to strategically approach art activism through the use of specific case studies; and (3) explain one practical form of art activism-theater-based conflict resolution-that is transforming the ways communities are addressing social injustice.

% FROM The Ruin and the Ruined in the Work of Kurt Schwitters.
% The German avant-garde was working from ruins literally and metaphorically, and trash was both practically and freely available; to use it was an action that took the ruins of our society, its discarded, to question how meaning is constructed.
% Marx wrote that it was not the materiality of the object but the social relations that create value, the use of urban detritus in particular, the squalid results of mass-produced human relations, infuses the materiality of Schwitters’ work with an anthropological quality

\subsection{Discussions}
Objects moves around and their values change constantly. (The idea that objects lead social lives was elaborated and discussed in detail in Arjun Appadurai (ed.). The Social Life of Things: Commodities in Cultural Perspective)

Igor Kopytoff, a professor of anthropology, introduced the notion of commoditization “as a process of becoming rather than as an all-or-none state of being.” As such, Kopytoff wrote, the biography of an object was considerably similar to that of a person: occupying different positions, leading diverse careers in the course of different periods between a beginning and an end, being defined by different regimes of value that are both economically and culturally inscribed. (Igor Kopytoff, “The Cultural Biography of Things: Commoditization as Process)

% FROM Trashion: The Return of the Disposed by Bahar Emgin
In light of this argument, one could claim that the end of the life of an object corresponds to the moment in which it is disposed of. This disposal might take place in different forms and for different reasons; however, in the most literal and common sense, the life of an object ends in a trashcan in the form of waste. In this moment, the object is left valueless in all the possible meanings of the term value: It can no more serve a function, it can on no account be exchanged for anything else, and it can by no means engage in the processes of signification to connote and endow its user with specific social values.

Referring to the work of Susan Strasser, Hawkins argues that disposal was central to the logic of mass production and hence should not be assessed as only particular to consumerism in the twentieth century: “Mass production of objects and their consumption depends on widespread acceptance of, even pleasure in, exchangeability; replacing the old, the broken, the out of fashion with the new. The capacity for serial replacement is also the capacity to throw away without concern.”

% Prap. same source...
% On the contrary, with respect to the issue of disposability, waste was handled merely “as a technical problem, something to be administered by the most efficient and rational technologies of removal.” 9 Only through the rise of environmental movements in the 1960s did the disposal of waste come to be loaded with negative meanings and iewed through a moral framework. The enormous quantities of waste accumulating in urban centers, Hawkins writes in “Plastic Bags,” were not only taken as a threat to the environment, but also as a sign of an individualistic, insensitive, and hedonistic consumer society. 10 Waste now became evil. If the environment is to be saved from our destructive power, then waste should be “managed,” Hawkins asserts. 11 Consequently, recycling gained its contemporary prominence “as virtue-added disposal\ldots disposal in which the self is morally purified, disposal as an act of redemption.” 12 Disposal in the form of recycling is now a moralistic attitude through which we pay the debt we owe to the world. Upcycling... On the other side of the coin is the business stemming from these practices; recyclers not only ease their conscience through recycling; they also make a profit. Recycling, as “the huge tertiary sector devoted to getting rid of things, is central to the maintenance of capitalism; it doesn’t just allow economies to function by removing excess and waste—it is an economy, realizing commercial value in what’s discarded,” Hawkins and Muecke write in Culture and Waste. 16 In the same manner, upcycling has already been turned into a business: Certain designers labeled eco-friendly are earning money through upcycling, competitions are organized around trashion, numerous websites are devoted to promoting and selling upcycled objects, and online and print resources explain how to upcycle at home. In short, there is a whole sector of upcycling now.

% Design, as a conduit of disposal, reintroduces rubbish as objects of distinction, invaluable and potentially priceless. People are often eager to see objects that were once considered useless and tasteless when they have been invigorated with new life.

% There is commodity aspect of it and also the process of accommodation. In this process design plays a significant role. Trash is waiting to be discovered. At the same time forgotten styles are also used in works. Therefore actually trash and forgotten styles can be considered in the same status.

% This can be sample thesis statement sentence:
% This article is about those objects that are recreated from trash through the process of upcycling. Upcycling is a term used by architect and designer William McDonaugh and chemist Michael Braungart and refers to “the process of converting an industrial nutrient (material) into something of similar or greater value, in its second life.” 4 I argue that design, in this instance, acts as a tool of transformation and reintroduces into certain orders what was once deemed waste. This theory counters the argument that an object is dead once it is disposed of.

% From The Ruin and the Ruined in the Work of Kurt Schwitters by Gemma Carroll
% Karl Marx had already written that economic value does not inhere in the materiality of the object but emerges from the social relations and organisation of labour which produces it, and that the separation between consumer market and the sphere of culture had become indiscernible.
% Schwitters is able to use the ruined, the waste products, as an anthropological exploration of society from both its unpleasant outcomes and its decay. \ldots trash was both practically and freely available; to use it was an action that took the ruins of our society, its discarded, to question how meaning is constructed. As he wrote: ‘It grows more or less according to the principle of a metropolis.’ 13 The Merzbau was itself a city; and just as Marx wrote that it was not the materiality of the object but the social relations that create value, the use of urban detritus in particular, the squalid results of mass-produced human relations, infuses the materiality of Schwitters’ work with an anthropological quality. Material has transformed into information, and ‘how’ has surpassed ‘what’ we see. The grottos in the Merzbau that still reveal this detritus most clearly could not be re-created in Bissegger’s reconstruction because, arguably, they are an exploration of absence, an exploration of ruin. As Schwitters wrote: ‘One can even shout out through refuse.’ 14 His words still echo.

% FROM Waste by William Viney
Douglas argues that our classification of dirt lies not with what objects are but where those objects are. (Think that the transformation process. Previous argument support that the transformation of trash is possible by changing the place of them. In other words removing them from landfill and waste bins to the book selves accomplish to transform trash.) ‘Dirt’, writes Douglas, ‘is the by-product of a systematic ordering and classification of matter, in so far as ordering involves rejecting inappropriate elements’. For Douglas dirt is a spatial problem, a question of not what stuff is but where it is.

% “Dirt”, writes Douglas, “is the by-product of a systematic ordering and classification of matter, in so far as ordering involves rejecting inappropriate elements.” Dirt is only dirty in certain places, when it is out of its correct position. Just as faeces, for example, is considered dirty when it is in our kitchens but not when it is in our bodies, so it is that our classification of waste depends on the location of objects. 

%%% My experience here: For example when i collect the papers under the dishes or food packages, people often thinks that it is disgusting and call them as dirty. But it is very strange that the fat on the paper is previously what they are eat. 

% objects we call ‘waste’ have peculiar powers to make that temporality an explicit part of what they are and how we judge them.

% Here susan strasser ile douglasın fikirlerinde bir ortak nokta görülebilmekte. Özellikle trashın relative olması ve bu işin aslında bir ordering and classification olduğu konusunda. Remember the example given: shoes  on the dinner table. 

% King Lear, feeling of waste

% waste is “matter is out place”, a definition first given by Lord Palmerston in the mid-nineteenth century and incubated by the British anthropologist Mary Douglas, in her book Purity and Danger.

% I prefer to use the word waste to describe the things that have, for whatever reason, been leftover from use or for which use has been precluded.

% https://narratingwaste.wordpress.com/tag/king-lear/
% Not all waste is dirty, it not always dangerous, contagious or abject. \ldots waste might be quite useful in making time and in keeping time.

% Binding things together and separating things from each other. 

% Cornelia Parker's work, narrate an absent event of waste
% There is also strong relationship with the collecting discarded stuff and archeologies of waste.  


%%% Reference
% we live in a throwaway society (Barr, 2004; Cooper, 2003, 2005; Cooper and Mayers, 2000; Strasser, 1999, cf. O’Brien, 1999; Hawkins and Muecke, 2003); 
% In the two previous sections we have demonstrated the paucity of the thesis of the throwaway society. In this thesis the undeniable matter of waste, itself pressing, urgent and excessive, is used to infer the presence of a society defined by its generation; a society ceaselessly discarding and abandoning its surplus as excess, as part of an endless desire for the new. Morally corrupt and unequivocally environmentally damaging, the rhetoric of the throwaway society classifies discarding as intrinsically bad and commands us to assume control of our wasting, suggesting the adoption of heightened regulatory practices around disposal as the means to ensure that we clean-up our act. The thesis, however, lacks depth and provenance. It is, actually, glib. Indeed, to infer the presence of a throwaway society from contemporary levels of waste generation is problematic for at least four reasons.

%%%
%%%
%%%
\section{Summary}
\begin{itemize}
\item Shared practice through the different ages of humanity and different societies. 
\item Moving nature of trash/objects.
\item Classification is the main key, trash is not related with the object property. 
\item Value creation, shifts in the society. 
\item Trash has archaeological value.
\item Trash has potential to be used again.
\end{itemize}

Thompson thinks that the switch only occurs from losing all the value given the object. He refers that to gain new value previous systems ruins must be cleaned. Changing the context by first rubbishing them.


\chapter{Trash in Art}

\epigraph{\textit{\quotes{One day, in a rubbish heap, I found an old bicycle seat lying beside a rusted handlebar, and my mind instantly linked them together. I assembled these two objects, which everyone then recognized as a bull’s head. The metamorphosis was accomplished, and I wish another metamorphosis would occur in the reverse sense. If my bull’s head were thrown in a junk heap, perhaps one day some boy would say, \singlequotes{Here’s something that would make a good handlebar for my bicycle!} }}}{\hfill ---Pablo Picasso}

How does trash draw attention of artist(, and also mine)? In this chapter how artist approached to the trash and their techniques are discussed in the context of art. (Historical development, methodologies, tactics, motivations\ldots) Focused on visual (plastic) arts(, because there is also a music genre called as trash music?). How does it turn to medium?

% TODO motivations of artists. their ideas about them, because they also point what is wrong with it? or their perceptions.

%%%
%%%
%%%
\section{Root in the Art History}
In this chapter root of using objects in the artworks is examined. Using objects on artworks beyond their intended purpose. Developing artworks not only painting but also using paper and other stuff by pasting them together.

% TODO praphase
\textbf{[TODO praphrase]} "The use of trash as a fine art medium dates back at least to the work of early-20th-century artists such as Fortunato Depero and Kurt Schwitters. Use of found materials, including garbage, has been associated with assemblage art since the 1950s and has been practiced by other well-known artists, including graphic artist Christian Boltanski, sculptor Louise Bourgeois, and photographer Andres Serrano. Art made from garbage has since become much more common in fine arts venues such as museums, galleries, and high-profile installations, including H. A. Schuldt’s famous “Trash People,” which has traveled around the world since 1996." \cite{tauxe2012encyclopedia}

%%
%%
\subsection{Collage}
Collage originates from the French \textit{coller} is an artistic technique of applying manufactured, printed, or “found” materials, such as bits of newspaper, fabric, wallpaper, etc., to a panel or canvas, frequently in combination with painting. In about 1912–13 Pablo Picasso and Georges Braque extended this technique, combining fragments of paper, wood, linoleum, and newspapers with oil paint on canvas to form compositions. Pasting paper is not a new technique but using this it in the art making is a revolutionary movement in the  language of art \cite{waldman1992collage}.

% TODO reading...
\cite{greenberg1984collage}

%%
%%
\subsection{Assemblage}
Assemblage work produced by the incorporation of everyday objects into a composition. It is similar to collage, but main difference is that assemblage is three dimensional rather collage is two-dimensional. Diverse range of things can be used production of work. In 1961, the exhibition "The Art of Assemblage" was featured at the New York Museum of Modern Art. William C Seitz, the curator of the exhibition, described assemblages as being made up of preformed natural or manufactured materials, objects, or fragments not intended as art materials \cite{seitz1961art}.

%%
%%
\subsection{Found Object (Ready-mades)}
Found object originates from the French \textit{objet trouvé}, describing art created from undisguised, but often modified, objects or products that are not normally considered art, often because they already have a non-art function. Pablo Picasso first publicly utilized the idea when he pasted a printed image of chair caning onto his painting titled Still Life with Chair Caning (1912). Marcel Duchamp is thought to have perfected the concept several years later when he made a series of ready-mades, consisting of completely unaltered everyday objects selected by Duchamp and designated as art. The most famous example is Fountain (1917), a standard urinal purchased from a hardware store and displayed on a pedestal, resting on its side.

%%
%%
\subsection{Bricolage}
Something constructed using whatever was available at the time.

Claude Levi-Strauss notes: the bricoleur works not from the principle of making things only if natural resources are available but makes things according to those things at hand, making do with what is available. It is an expression that, like the natural cycles of the Earth, attempts to make something new from something old. \cite{levi1966savage}

% TODO reading...
\cite{strasser1999waste}

%%
%%
\subsection{Folk Art}
In contrast to fine art, folk art is primarily utilitarian(practical and functional, not just for show) and decorative rather than purely aesthetic. The nature of folk art is specific to its particular culture.

Some scholars think that the roots of collage is folk art. (Find resource.) It's methodologies used in this art. The same exist for craft. Some works(for example sculptures from trash) are highly requires craft skills.

%%
%% junkyard art
\subsection{Garbage Art, Recycling Art}
"Garbage art (alternatively known as trash art or recycled art) is art created from materials including post-consumer and other waste, collected debris, or objects previously used for other purposes." "Creating art from garbage involves transforming the meaning of objects by placing them in new, aestheticized contexts. This practice is not new; tribal peoples have adapted bits of trash from industrialized societies into their traditional arts since coming into contact with products of the developed world." "Creating art from trash involves “consuming” garbage in the sense that artists appropriate and rearrange the materials in personal ways, transform their meanings, utilize them to their own ends, and represent them in new ways.It involves taking unwanted materials out of their “waste” context and recontextualizing them as “art.”" \cite{tauxe2012encyclopedia}

% From Beautiful Trash Art and Transformation BY PAOLA IBARRA, ReVista
Recycling has always been a common practice in the arts at least at a non-material level. From creating a world of words in literature, to rhythm and images in poetry, sampling in hip hop music, representation in the visual arts, or editing the illusory continuity of a film, art implies taking disparate elements (ideas, images, references, objects, etc.) and putting them together to form a new whole. Take and put. De-contextualize and re-contextualize. In that sense, art, as a system, is an act of recycling. 

%%%
%%%
%%%
\section{The case of \quotes{The Gleaners and I}}
The Gleaners and I is a 2000 French documentary film by Agnès Varda that features various kinds of gleaning. The Gleaners and I is notable for its fragmented and free-form nature along with it being the first time Varda used digital cameras. This style of film making is often interpreted as a statement that great things like art can still be created through scraps, yet modern economies encourage people to only use the finest product.

It's a self-reflexive film because the director establish a relationship with the practice of gleaners and her film making practice. Some people gleans crops, the others discarded food, the other baby dolls and Agnès Varda gleans images.  

\quotes{Agnès Varda’s film, The Gleaners and I, documents the history and current practice of gleaning in France. Historically, gleaning is the act of collecting leftover crops from farmers’ fields after the harvest. However, in the film Varda expands this definition to include actions that are presently coined \quotes{dumpster diving} where people collect any types of rubbish or unwanted items to reuse them in their own way. Moreover, Varda includes her actions of collecting images with a video camera as gleaning. She is gleaning images. The Gleaners and I far more than document the lives of gleaners. It highlights the degree of global consumerism of the modern world and the ways art can exist within it in relation to gleaning.}

The official subject of this film is gleaning, the act of gathering remnants of crops from a field after the harvest. As Varda demonstrates, people can be discovered throughout the French countryside gleaning everything from potatoes to grapes, apples to oysters, much as they did hundreds of years ago (though no longer in organized groups). More figuratively, there are also urban gleaners who salvage scraps from bins, appliances from the side of the road, or vegetables from stalls after the markets have closed. And then there’s Varda herself, a gleaner of images, driving around France with a digital camera and a tiny crew (at times, she wields a smaller camera herself, permitting an even greater degree of intimacy).

Making use out of something that has been left behind and labeled as obsolete is not unique to farms and crops. There is so much discarded, yet still-viable food in dumpsters that many people live off it entirely. Seeing the value in what someone else has defined as trash is an art in itself.

Once as a common practice of gleaning throughout the years has evolved, but not disappeared. She keeps light to the modern life gleaners that are not visible every time. One of the interesting thing is here Agnès Varda feels that as a aging person will later become discarded person. In other word we can be subject of the refusal. At some point she become the subject of the thing that she track. 

There are many aspects of this documentary film. First draw attention the practice of gleaning. The discarded items that are not fit in the industrial standards because of their shape, color etc. Even if these items are discarded and we are not aware of them, there is also another life for them. Modern life gleaners feed from them. In another words someones trash becomes another trash. There are people live the boundaries of consumption societies. They create their life from the unwanted items. what the society refuses, they find a life or create a life from them. Their lifestyle is also can be seen as alternative to the life on the urban areas with deep relationship with the consumerism. (as also Zizek mentions that love is not idealization. But the industrial processes has some standards and beyond that standards there is not a place for everyone. Life is being lived in the cites is somehow idealized life or tried to be idealized life. beyond the border of the urban there is new life generated. What is not succeed in the cities succeed outside of it.)

She uses unconsciously recorded pieces in the film. \quotes{The last definition of gleaning, gleaning images, ties into what I found both the most amusing and perplexing scene in the film. It is the scene where Varda had forgotten to turn her camera off and accidentally films the ground and her lens cap bouncing along as she walks. She sets this scene to a jazz soundtrack. While watching the scene I was certainly confused. Why is this in here? What does this have to do with gleaning? And certainly why did the scene last so long? But there was also something lovely about it. Maybe it was the music, but for some reasons my senses found it audibly and aesthetically pleasing. I wasn’t able to make much sense out of the meaning behind this scene; I believe Varda referred to it as \quotes{the dance of the lens cap}.  Luckily, Ruth Cruickshank’s article, The Work of Art in the Age of Global Consumption: Agnes Varda’s Les Glaneurs et la glaneuse (The Gleaners and I), addressed this scene and clarified the potential deeper meaning behind it. Cruickshank likened the scene to something that is normally thrown away, or in this case edited out of the film, much like the way trash is thrown away or food is left behind after a harvest. Varda gleans the image of the dancing lens cap, \quotes{Where many documentary makers would leave such accidental footage on the cutting room floor, Varda draws attention to how what would habitually be perceived as waste may be viewed as supplement with its own intrinsic value. Rather than literally treating it like dirt, Varda retains and prompts reassessment of that which is normally left out of shot} \cite{cruickshank2007work}. Things that are often forgotten or discarded can easily be revamped to create something useful to someone. The scene was revamped using music and it became beautiful, much like the gleaners who found fish in the trash cooked it to make it edible, or the artist gleaner who piled discarded baby dolls into totem poles.} 

Refused is not only foods and trash. there are a lot of things that are pushed outside border of the society. She talks and investigates artist that are using discarded items in the artworks. What people not found anything the artist see countless possibilities from them. There is always alternative ways to see something different. Giving life or finding life from discarded life.

The clock found the garbage pile and taken by Agnès Varda. It does not work properly but it has important meaning for her. The time does not go on and it does not remind her that she is aging.

Here another point is that Agnes display images of people picking up things from the ground like their ancestors. Everyone somehow collecting things in their life but particularly she selects these people and their images in action. There should reason for this? For some individuals, gleaning is not a novelty or a clever way to save money, but a necessity of life. They require it to sustain their life. Combine the elements from different peoples that seems totally unrelated gains powerful theme for the documentary. What gleaning means become more open (or powerful). It draws a picture of body combination of different parts (connected, dependent to the each other). The reason of gleaning varies but the fact that gleaning is continues in different forms.

She discusses the importance found in the meaning and purpose of art forms like this: \quotes{Varda seeks to encourage viewers to consider what potential agency is demonstrated in the artfulness and contingency of gleaning by individuals excluded\ldots from the homogenizing systems of global consumption} \cite{cruickshank2007work}. One can find more treasure in trash than many of New York’s finest galleries and art exhibits; they bring a grassroots feel to what has always been seen as a stuffy and prude aspect of society.

The Gleaners And I is not an environmental film. Gleaning, Varda implies, can be understood more broadly as a form of resistance, a way of refusing to be boxed in by conventional expectations; as such, it demands that we re-learn age-old skills as well as supply individual creativity and initiative.

The film tracks a series of gleaners as they hunt for food, knicknacks, thrown away items, and personal connection. Varda travels the French countryside as well as the city to find and film not only field gleaners, but also urban gleaners and those connected to gleaners, including a wealthy restaurant owner whose ancestors were gleaners. The film spends time capturing the many aspects of gleaning and the many people who glean to survive. One such person is the teacher named Alain, an urban gleaner with a master's degree who teaches French to immigrants.

Varda's other subjects include artists who incorporate recycled materials into their work, symbols she discovers during her filming (including a clock without hands and a heart-shaped potato), and the French laws regarding gleaning versus abandoned property. Varda also spends time with Louis Pons, who explains how junk is a "cluster of possibilities." Louis Pons (born 1927) is a French collage artist. He specializes in reliefs and assemblages made entirely from discarded objects and junk. In Agnès Varda's documentary The Gleaners and I, Pons explains his artistic process and understanding of art; what others see as "a cluster of junk," he sees as "a cluster of possibilities;" and that the function of art is to tidy up one's inner and exterior worlds.

%%%
%%%
%%%
\section{Example Artworks}
Artist and artworks related with trash and their analysis. Stages of their works and methods.

% FROM ReVista
Trash is dirty. Trash is smelly. Trash can provide the raw materials for exquisite art---from sculpture to film and beyond.

% FROM Paola Ibarra ReVista
Whether in sculpture, photography or other media, art frequently deals, directly or by allusion, with daily challenges of life in Latin America and elsewhere. Is there a limit to recycling and representation? Or is there a point at which waste cannot become art (or anything else)?

\begin{itemize}
% FROM Maite Zubiaurre, ReVista Garbage
\item Filomena Cruz creates photographic series “Road Kill” painstakingly captures tiny “trash corpses” on the pavement. It is that particular type of trash meet in the public places. trash left behind, trash on the sidewalk, squished, squashed, and weathered. Filomena Cruz sees trash that we don’t want to see. A
piece of chewing gum with an “engraved” leaf; a flattened-out tube; a corroding paper napkin with a still intact heart; or a frog-green Crayola melting in the heat, all speak the language of “worthlessness” suddenly becoming meaningful (and moving).For one thing, trash corpses faithfully record city life. 
  \begin{figure}[ht]
      \centering
      \includegraphics[width=0.8\textwidth]{graphics/FilomenaCruz_RoadKill_ReVista.jpg}
      \caption{Road Kill by Filomena Cruz}
      \label{fig:FilomenaCruz_RoadKill_ReVista}
  \end{figure}

\item Vic Muniz creates monumental trash art in Jardim Gramacho with the help of \textit{catadores} (Waste Land, 2010)

% https://www.youtube.com/watch?v=sJxxdQox7n0
\item Favio Chávez and Nicolás Gómez decide to build musical instruments out of garbage and get 35 children from Cateura, Paraguay’s biggest trash dump, to travel the world with their “Recycled Orchestra,” or “Landfill Harmonic”.

% http://www.eloisacartonera.com.ar/ENGversion.html
\item “Eloísa Cartonera,” a work cooperative in Buenos Aires, proudly produces handmade books with cardboard covers: \quotes{We purchase [\ldots] cardboard from the
urban pickers (\textit{cartoneros}) who pick it from the streets. Our books are on Latin American literature, the most beautiful we had a chance to read in our lives.} \quotes{Some of them are preserved as art books at university libraries, while others circulate as literary pieces expected to disintegrate in time---something anticipated of the material they are made from.} [from PAOLA IBARRA, ReVista]
  \begin{figure}[ht]
      \centering
      \includegraphics[width=0.8\textwidth]{graphics/EloisaCartonera_Books.jpg}
      \caption{Books covered by Eloisa Cartonera}
      \label{fig:EloisaCartonera_Books}
  \end{figure}

% FROM Paola Ibarra, ReVista
%\item Forever; Blue Yonder by artist Kyle Huffman
%\item Too Too---Much Much by Thomas Hirschhorn
%\item Autoconstrucción by Abraham Cruzvillegas
%\item Pictures of Garbage by Vik Muniz

% FROM Daniel Lind-Ramos BY LOWELL FIET, ReVista
%\item Daniel Lind-Ramos

% FROM A Present from the Sea BY SONIA CABANILLAS, ReVista
% https://www.youtube.com/watch?v=v6IoEF_Tsrw
\item Nick Quijano has some rules to create assemblages. \quotes{There are certain self-imposed rules to this creative process: first, the assemblages or artefactos must all come from material washed ashore on this beach; second, it must be plastic and industrial refuse, result of the processing of fossil fuel; third, it must be polished by a long stay in deep waters, sometimes even encrusted with corals, shells or pebbles, or simply scraped by the ocean floor. As a sign of respect and sacralization, these pieces will be incorporated without any adjustment: no cutting or bending, seen as a mutilation of the object. Its identity cannot be veiled or masked but always must be recognizable amidst the other components; e.g, a comb must remain a comb even as one may see it as a mustache.} The sea returns this refuse; it is not biodegradable.

% FROM Burning Messages BY MICHAEL WELLEN, ReVista
%\item Antonio Berni

% FROM Haiti in the Time of Trash BY LINDA KHACHADURIAN, ReVista
\item Haiti case. When I ask him why he chooses to work in the medium of trash, he replies, \quotes{It gives respect to my city to use the garbage. It shows that everything can be used, and nothing was lost.} (TODO motivation: \quotes{I get more inspiration working with recycled materials because those pieces are unique and can’t be duplicated}) Eugène says that he’s partial to metal, which has become more and more difficult to find because of the clean-up initiative by the city. When I ask him if part of him wishes there were no such effort underway, he answers: \quotes{No. When you have clean streets you have good health, and that is the most important thing.} (This is very strange. It shows that working with trash and being clean healthy is not a contradiction. Both of them exist together.) \quotes{Other people come to Haiti and see junkyards, but we see magical playgrounds,} Jean explains as he watches them.

% FROM Thinking on Film and Trash BY ERNESTO LIVON-GROSMAN, ReVista
% By the 1950s a film like Tire Dié (Fernando Birri, Argentina, 1956) already portrays the collecting, classifying and recycling of trash not only as a source of informal income but as a commercial activity linked to the formal economy. In these films, trash is not the end of a process of consumption but the beginning of a cycle of production. These movies share the idea that trash could be a departure point to think about the modern condition as defined by consumption, class disparities, contamination and urban development. The poet Charles Baudelaire is one of the first to make the connection between the rag picker and the modern city. Walter Benjamin picks it up and from then on the fragmentary condition of trash will remain associated with contemporary art and ultimately with the Modern condition: the industrial refuse could be redeemed by art. It is in this sense that filmmaking becomes allegorical and mimics the process of recycling when it reappropiates archival materials and found footage to create new narratives from scraps, fragments, of films that were not in any way connected to these new narratives.

\item Joseph Cornell

\item \textbf{American Beauty.} The film American Beauty , which features a long, poetic clip of a plastic bag swirling on an eddy of air, snagged five Academy Awards, yet I for one still find it hard to think of plastic bags as things of beauty. 

\item \textbf{Aaron Kramer.} His motto: "Trash is the failure of imagination." \cite{meyer2007turning} In addition to concern for the environment, Kramer was drawn to recycled art because of one simple factor---the price. "Free is certainly great, and that was a driving force for me early on in my career," he said.

\item Nelson’s breakthrough work was The Coral Reef, which he mounted in 2000 at Matt’s Gallery from objects and debris gathered from alleys, trash bins, and car-trunk sales all over London. The title refers to Nelson’s aim of creating an intricate, reeflike network of lives “all existing under one sea, which is capitalism,” he says. (It is very corralate between Ages Varda's work. She shows at the side of human practice, Nelson looks the topic from object side.)

% From http://www.thisiscolossal.com/2014/06/historical-fine-oil-portraits-on-crumpled-trash-by-kim-alsbrooks/, http://www.thisiscolossal.com/2015/05/new-historical-portraits-on-flattened-cans-by-kim-alsbrooks/, http://kimalsbrookswhitetrash.blogspot.com.tr/
\item
\parbox[t]{
	\dimexpr\textwidth-\leftmargin}{%
      \vspace{-2.5mm}
      \begin{wrapfigure}[22]{r}{0.4\textwidth}
        \centering
        \vspace{-\baselineskip}
        \includegraphics[width=\linewidth]{graphics/Alsbrooks.jpg}
        \caption{The White Trash Series by Kim Alsbrooks}
      \end{wrapfigure}
Kim Alsbrooks, historical oil portraits on flattened beers cans and fast food containers. \quotes{The White Trash Series was developed while living in the South out of frustration with some of the prevailing ideologies, in particular, class distinction. This ideology seems to be based on a combination of myth, biased history and a bizarre sentimentality about old wars and social structures. With the juxtaposition of the portraits from museums, once painted on ivory, now on flattened trash like beer cans and fast food containers, the artist sets out to even the playing field, challenging the perception of the social elite in today’s society.} \quotes{On technique: The trash is found flat, on the street. One cannot flatten the trash. It just doesn't work. It must be found so that there are no wrinkles in the middle and the graphic should be well centered. Then the portraits are found that are complimentary to the particular trash. Generally I depict miniature portraits from the watercolor on ivory era (17th-18th century more or less). The trash is gessoed in the oval shape, image drawn in graphite, painted in oils and varnished.} I need to mention that this images are very subversing to the perception of us about these images. It is unexpected way of presenting images. There is a relationship with the image and painting. At least this work questioned this relation. With tihs work one can ask that with the painting trashed can become valueable thing or with this can this paintiing became valueless. Artsist combines two thing that can be never thought together. New combination, new meaning.}

\end{itemize}

Childs can enjoy with trash. I remember from my childhood, we collect crown cap and play with them. Some caps are found less and they worth more. We are looking everywhere for them. To make them flat we put them on the railways. After train passed we get perfect plat cap. At that time it is not trash for us. It has a value and part of our games and enjoy. To have fun a bunch of trash can be enough for us.

In the case of recycling, a dead objects, or an object that reached end of life will begin a new cycle of life. By the artist or other parts are give them a new life. They can reborn as a different things. Do people collect them aware of it? Discarded items unites together with the hands of a artist. 

Trash is global topic men. When you talk about trash, everyone have ideas about it. 

Think a city that has trash monument in the every corner. created from their trash. merged with the city life and gaining unique cityscapes and aesthetics. or think that a museum a trash museum. Exhibits works of art embracing the trash in all aspects. Maybe done by the artist or the visitors from all around world. A place for garbage other than a landfill. Waiting their creator to meet again. What a great idea isn't it? Meeting their creators again. But this time their creator can recognize their trash. They transformed to totally new thing. Reborn. Transformed (Kafka, Gregor Samsa).

%%
%% TODO sample statements here.
%\subsection{Artist statements}

\begin{itemize}
\item I am an artist living in New York City, and my experience of daily life here is both the catalyst for and the subject of my art. Working from the local, the personal and the ordinary - the paper take-out coffee cup in the palm of my hand, the view from my studio window in the Garment District, friends and their children, the community garden around the corner from my apartment in Hell's Kitchen - my drawings, paintings and installations are about what happens when the familiar suddenly undergoes a perspective shift and is revealed in all its wonder and infinite possibility. With this shift, mundane things become extraordinary, as nodes of rapidly expanding sets of connections, relationships and new artworks. 

My approach to art-making is both observational and process-oriented. A fascination with cultural, physical and temporal change combined with the possibility of infinite variation unify projects as diverse as drawing and painting on my used coffee cups each day, repeatedly recording the view from my studio in paint and photography over the course of six years, depicting in two and three dimensions the incursion of human activity on the fractal meanders of salt marshes along the New Jersey coast, and painting portraits of alternative families in my New York City neighborhood. (from http://gwynethleech.com/)
\end{itemize}


%%
%%
\subsection{What might be the meaning of using trash as a medium in the artworks? Questioning trash as a medium for artist}
\begin{itemize}
\item Some works try to raise awareness the problems that are the result of trash. (It treats environment and nature.)
\item Some of them reflect people's lifestyle especially throw away culture. As a mirror of current lifestyle.
\item Try to find a new value and meaning from the discarded material that are useless anymore. To explore a new approach, new way. Subvert people's ideas about trash and their attitudes by turning materials to the something meaningful (or valuable). Trash to treasure.
\item Using discarded item to represent other discarded things by the ruling ideology or approach. For example, trash can be used to represent refugees. The things that we are trying to discard does not mean that they have no value, instead it means that we have no ability to reveal its potential. In other words, refugees have potential but we see them as players that will change our current system. Therefore, it can be said that willing to transform trash to treasure is to require change of current lifestyle. Rejecting discarding something especially thing that you get value from it is a process and spread through to the ones life.
\item One way is not to produce trash. (Zero trash philosophy.) The other one is to transform trash into something else.
\item What type of experience is that collecting and working on objects that are generally discarded? Experiencing out of common practice, being open to new explorations.
\item Instead of a world that produce trash, how could it be a world created from trash?
\item Combining industrial goods with objects transformed from trash is another way to find a place to trash in the community. It also signifies that trash still has a good quality to used with new materials. Creating composite products from new and reused items. Using the valuable thing with the invaluable thing. It becomes more valuable or less valuable. Depends on the perception.
\item Aesthetics of trash. Revealing aesthetics value of discarded stuff. (Unique visual value. Trash portraits, sculptures etc.)
\end{itemize}

%*****************************************
% Maybe...
% Artistic tactics
% Topology of artworks
%*****************************************

%%%
%%%
%%%
\section{Discussions}
Are artworks made from trash just examples of collage and assemblage or more than from them? What about the experience and interaction with other people? Turning art making process to a life practice (or part of life) can be explained in the context of collage (which is mainly related with how a 2d canvas created). But all of them work in fragments, combine many objects together. 

Garbage is often viewed as a form of society’s excess---as the unwanted things that are thrown out without regard. 

In the world of computer science, the term garbage also refers to situations of loss in which data or objects in memory go unused in computer operations.

(Trash has a history. They have different stories. But same destiny?)

Lets make clear the motto: "trash to treasure". What is treasure here. What does it signify? Treasure---something very valuable. from trash. Then it is trash. Not once it is trash. What type of treasure. Who can get value from it. Who creates this treasure? How does it created? Trash is actually is opposite of treasure and then it is switched. In other words trash become treasure and treasure become trash?

Lets consider the practice of working with trash. Why someone believes that discarded objects of people \ldots Looking things that people are refused. Transforming is the one of the most common practice of whole history of man and nature. This practice is forgotten or it is still alive. Agnes Varda actually shows that it is still exist but in different forms. They use it what ruling lifestyle throw out. They are everywhere actually but you should look them carefully. They live in the borders. But live. Event if they are refused they live. Actually it is hard to say that they want to integrate to the system. They want to change it? 

ability of transform, well actually it is very easy to throw away. hard thing is to keep it. force the material. the dimension of trash.

% FROM drink UP, author: Werthan, Sarah, artist: Leech, Gwyneth. A Year in Cups. http://gwynethleech.com/
% trash and art collided. Paper and art, actually paper already medium of art, but is there anything different here. Itself is a part of a work, not the drawing, or painting.
% Documenting via a blog or a website. (what type of dimension it brings the work? maybe connect them, leave message.)
% Buying a beverage is a daily event for \ldots
% Creating art in public places can demystify the process for passers by, Leech says, making artistic expression more accessible and part of people’s everyday lives.
% “People see that an artist can make work anywhere, and make creative spaces anywhere,” she asserts. 

% Trash is as a material or a subject

\begin{itemize}
\item Where do you think this object came from?
\item Why do you think that someone labeled this object as waste?
\item How can you transform this object to tell the story that you thought of before?
\item What story does it tell you?
\item Does it remind you of an event, a specific time in history or in your life, a place, or a state of mind?
\item How can you bring that story to life? (Taylor, 2006, p. 9).
\end{itemize}

% TODO PRAP. from rethink, reimagine, reinvent
Recycling art approach to using reclaimed objects in artworks requires rethinking, or examining the affordances of a particular object to explore the possibilities for the object's inclusion in an artwork. Assemblage art involves the creation of new and innovative objects from what were once considered objects of waste; that is. through their use in assemblage pieces, reclaimed objects are endowed with a new. sometimes paradoxical meaning. The transformations of the objects used in assemblage pieces ask viewers to reconsider the notion of "valuable" as they are challenged to look at everyday objects with a new perspective (Taylor, 2006). Viewers confront issues relating to the functionality of objects during modern processes of production, consumption and distribution.

Artifact exploration promotes historical thinking, literacy investigation, and cultural expression (Higgs \& McNeal, 2006; Levstik \& Barton, 2001; Morris, 1998). Meanings are embedded within cultural artifacts and language (Vygotsky, 1978).

%%
%% ACTIVISM as a methodology 
%% TODO reading...
\section{Art and activism}
Using trash in the art has some critical messages to societies and authorities. Especially in the trash case there is a activist(political) act. 

%*************************************
% Phrases...
% What specific social issues are you trying to address through Labyrinth?
% In view of the diversity of online crowdsourced art projects, as illustrated by the examples cited so far, it is useful to map out this artistic trend by developing a comprehensive and multidimensional typology of online crowdsourced art. Table 1 organizes this classification according to a set of multiple criteria. (But this one can be used at introduction, but it is too long to fit on introduction. Maybe select works by giving prominence to some of the features. So the approach to the trash is going to be introduced and also it is reveals the what i am doing in this context.) (A Typology of Online Crowdsourced Art. Diye bir örnek var aslında benzer bir şekilde bu trash içinde yapılabilir.) 
\section{Crowdsourcing, Participation}
% TODO needs reference, FROM Crowdsourced Art and Collective Creativity
In the words of Jeff Howe (2006b), the Wired columnist who coined this term in June 2006, \quotes{crowdsourcing represents the act of a company or institution taking a function once performed by employees and outsourcing it to an undefined (and generally large) network of people in the form of an open call}. The vital elements that qualify an outreach strategy as crowdsourcing are, according to Howe, the use of the open call format and the reliance on a large network of potential workers. Although in some cases there is a material reward for the best contributions, the existence of financial incentives is not a required feature in crowdsourcing. Because of the diversity of its applications, crowdsourcing continues to be a disputed term in both the scholarly literature and the popular press; Howe’s original definition is, in this sense, a helpful delineation of its practical sphere.

The value of crowdsourcing lies in the collective intelligence of the contributors. Pierre Levy (1997) describes this concept as “a form of universally distributed intelligence, constantly enhanced, coordinated in real time, and resulting in the effective mobilization of skills” (p. 13). The question of collective intelligence—and its potential efficiency in various practical settings—has received much attention in both academia and journalism. Researchers studying team performance generally agree that, under the right circumstances and with appropriate motivation, large groups of people can work together and harness their collective intelligence to achieve efficient results (Benkler, 2006; Rheingold, 2002; Surowiecki, 2004). Nevertheless, artistic creativity is different from innovation and intelligence, and it requires a unique set of skills and sensibilities as well as a particular type of cultural capital; if we admit that crowds can have collective intelligence, do they also have collective creativity in an artistic sense?

All these pages are rescued and with their [hi]story they are separated from unused new produced blank sheets and notebooks. They are not bought, not gift. They are found. They are accepted. One of the most creative medium is paper and pencil. Chance given to the people through this work.

However, in view of its reliance on the artistic contribution of a large pool of usually anonymous participants, this type of art raises important questions about notions of collective creativity, authorship, collaboration, and the shifting structure of artistic production in the new digital environment. It is well studied area. Pick a method here. Transforming trash with collaboration.

As curator Andrea Grover notes, “having the audience become co-creators is not a new impulse”; the Internet simply offered a new platform to accomplish this goal (Strickland, 2011, para. 5).

Here this question can be raised about why this type collaboration. Another option can be working together with people on a table. Creating things at that time and transforming the objects here. Possibly the connection to the people will more realistic and intense. However same things can be succeed on the internet at some level. 

While crowdsourced art challenges the traditional role of the artist, it simultaneously redefines the conventional function of the public, turning them from passive receivers into engaged producers. (This totally a new area to discuss, I'm going to summarize and introduce concepts and debates. How are they support my work and how they are give way to me?)

% This article therefore aims to fill these critical gaps by analyzing the practice of online crowdsourced art within a framework of collective creativity and participation theories. Principally, my interest is in answering two key questions. What is crowdsourced art and how can it be classified? And how does the structure of the artwork determine the degree or significance of participation?

% why collaboration, is it really collaboration?

% TODO 
% Umberto Eco's Open Work

% Not all artist transform trash although some deconstruct them. Michael Landy is one of them. 
% Michael Landy's Break Down Inventory is a two week show / display of destruction process of his all possessions on a dissemble line with the help of 10 workers. Firstly they are classified and recorded for three years and the deconstructed in two weeks by separating every element to the smallest part. Reveal all his possessions. and loosing them while you are alive. Turning them to rubbish making them unusable. breaking down the all the meaning. breaking down the connections. 
% Michael Landy's Art Bin uses a art gallery to create his dust bin. Describing the work, simply called Art Bin, as 'about failure', Landy is inviting members of the public to bring their own artistic failures along to the gallery from 29 January, where their worthlessness will be assessed. Damien Hirst, Gillian Wearing, Tracey Emin and Mark Titchner have already contributed, offering sculptures, paintings and prints. "There's no hierarchy once they are in the bin" All of them are accepted as same. Ultimate equality. Tosing them to the dust bin makes them rubbish even if they are combined from failed attempts. Nothing's too good for the art bin. Everything can fit into the art bin. Michael Landy transforms the SLG into Art Bin, a container for the disposal of works of art. As people discard their art works the enormous 600m³ bin becomes, in Michael Landy’s words, “a monument to creative failure”. It is a very big glass bin and isolates people. There is a provocative approach to the art and art objects by saying that "Nothing's too good for the art bin". There is no limitation of throwing away even if art. 

% In this thesis project case the issue is not the language of art and its methods.


\chapter{My work}

%%%
%%%
%%%
\section{My Artwork/Project}
Here stages of my artwork are listed:
\begin{itemize}
\item \textbf{Memories.} As far as I know I collect things it is hard to say collecting started after an exact event or time but there are several memories that I remember. One of these (First of them) is from my primary school teacher. When I was third class my teacher give us a homework to bring colorful paper to make something(whatever it is I don't remember). The day after we bring some colorful paper but she did not. Instead of this she bring paper cut out from packages that are colorful, shinny and qualify. And I remember clearly that she suggest that same for us. Do not throw out packages, look for the useful parts and keep them to use later. 
\item \textbf{Motivation.} I can not throw them away. When I throw them I became sad for them. I have to find something useful for it. Even if I can not find it, I can pass it to another person who can make use of it for own purposes. We are humans that can produce, transform items. One of most developed species in the earth that produce and use tools. There is a lot of effort to produce thrown away and it seems that all this efforts are wasted. What I mention is not related about ultimate productivity. It is more close to being thoughtful, and taking responsibility of tools, items and objects that we are using. Rather than throwing out, creating a way that all are have chance to live together is much more close to my perception. 
\item \textbf{Collecting.} One of the most found trash in my habitat is paper. I work at METU Technopolis, live at 100. yil which is the nearest settlement to METU and study at Bilkent. People live there commonly use paper and needs paper. Paper is nearly everywhere. Reusing the paper is not limited with recycling of it. There is a another ways of it. When we recycle them actually we again send away them and use it as we all know. (industrial papers and notebooks.) There is no richness here. Same type of paper. Produced after a industrial process. I collected them from my friends (people that have communicate often). Sometimes I collect them from trash bins and roads. 
\item \textbf{Transforming.} I turned them to notebooks. Actually I use it for my self. And while I was using of it I am very proud of it. I am studying nearly more 20 twenty years and I have always need for notebooks and use them. It is some sort of passion for me notebooks. Because I always admire notebook beautifully design or uniquely designed. I collect them whenever I find and most of the time I only save them for later use. 
\end{itemize}

%%
%%
\subsection{Functionality and Art Debate}
From an art historical perspective, you could say that functional art is the inverse of Marcel Duchamp's famous readymades, where he transformed utilitarian objects---a urinal, a bottle rack, etc.---into conceptual artworks by fiat: it became art because he said it was. Duchamp's works kills the functionality. It works beyond the functional perception. Moves the debate to the conceptual frame. However it is not every time case. Today many functional art objects are as avidly acquired by collectors as their fine-art brethren, and are appreciated just as much for their beauty as their use value. Ancient Chinese vases, for example, while still capable of performing their originally intended function (displaying flowers), are prized for their historic and aesthetic value more than anything else. \quotes{In conceptual art,} Sol LeWitt writes, \quotes{the idea or concept is the most important aspect of the work\ldots The idea becomes the machine that makes the art} \cite{lewitt1967paragraphs}. Therefore anything can be turned to art with a good idea. 

% TODO PRAP.
Comparing art to craft is like comparing philosophy to engineering: they're two separate ways of looking at the same thing. To me art is communication of an idea or an emotion, while craft is the physical manipulation of material. An object can easily be both, either, or neither. A sculpture, for example, may communicate, but it was constructed using craft. Likewise a teapot can communicate an idea, but it was crafted. Function is misleading and no distinction. Functional objects can still communicate ideas, so art can be functional. One object could be viewed two ways: if you look at the way it was made and the materials used, you are looking at its craft, if you think about its ideas, you are viewing it as art. An object could have been crafted, but contain no art. Even a painting can be crafted but artless. A ready-made might be art with no craft. I very much like the idea of a spectrum. One last thought: skill doesn't enter into the definition of art, since a piece could succeed as art but be poorly crafted.

%%
%%
\subsection{Paper}
"Paper is an indispensable product throughout the world. Its primary use is as a medium for writing, essential for bureaucracy, education, communications, information storage, and in the spread of information. In addition, it is used for the packaging for transport and convenience of a wide range of items from food to industrial equipment. Paper also has specific technological uses, such as for filters and in art, home furnishings, and architecture, and it has a range of uses for hygiene purposes. Paper in several forms is consumed on a daily basis by each person in the Western world." \cite{trafford2012paper}

%
\subsubsection{Environmental Impact}
Paper is both biodegradable and a renewable resource, which means in consumption and waste terms, its environmental impact is relatively small compared to the many more-toxic and bulky waste products that are found in everyday garbage. However, the chemicals, water, and electricity used in its manufacture are considerable---and these are nonrenewable resources---and certain types of chemicals used in paper production are toxic. In addition, if waste paper is sent to a landfill, it releases carbon dioxide emissions. Further, forest resources are not always as renewable as one may like to think. These environmental impacts can be greatly reduced by recycling (paper being one of the most easily and cheaply recyclable products in everyday use) and by conscientious consumption practices.

Paper made exclusively from wood is called virgin paper, while paper produced out of used paper that is re-pulped is called recycled paper. Recycling paper can greatly diminish demand for virgin fiber from wood. However, there will always be a demand for virgin paper because, although paper is thought of as a renewable resource, it cannot be recycled indefinitely. It can only be recycled four to six times, as the fibers get shorter and weaker each time. In addition, some virgin pulp must be introduced into the process each time to maintain the strength and quality of the fiber, so no matter how much is recycled, paper will always need some virgin fiber.

%
\subsubsection{History}
The word paper comes from papyrus, the plant that was first used for making a medium for writing in ancient Egypt.

%
\subsubsection{Production}
All types and qualities of paper share the same basic method of manufacture, including newspaper paper, print paper, and carton used for boxes.

%
\subsubsection{Uses}
Paper has become the most ubiquitous product in the age of information. Such products often complete their journey from shop floor to garbage in a single day; for example, newspapers, print paper, packaging, lavatory paper, tea bags, transport tickets, price tags, shopping bags, flyers, leaflets, wrapping paper, napkins, and tissues. 

%%
%%
\subsection{Why (package) paper?}
Easy to collect. Easy to find. Thrown out even if it is good quality. Packaging materials are very widespread. Appropriate for painting and writing. Has a very short life time. Disposable, there are a lot of package outside. No need to carry it. Every place gives you package paper. 

%%
%%
\subsection{Why covering notebooks and books?}
They are all package paper, already used as carry things and this work it has used again for the same purpose (but in different connection, this time trash is bound to the notebook). Trash is used to cover the papers. Cover is the most visible part of the notebook and book. Therefore trash becomes most visible part of the produced item. What is the function of cover? It gives ideas about what is inside, distinguishes from other things, protect it.

%%
%%
\subsection{Why giving away?}
Aim is to spread the idea by making something useful from trash. Increase diversity, activate (or encourage) people to embrace the trash.

%%
%%
\subsection{Why in the public space?}
To reach more audience. Actually the audience is out of the art galleries. They are walking in the streets. Putting them to next to people is much more effective. It is not visible and nearly it is hided from society. It is dirty. Removed from the society. There is a effort to hide them away. However in this project it is again showed to the society. Because it is revisited and reclaimed. 

%%
%%
\subsection{Artistic tactics}
Here I followed some tactics to realize my purposes. I called them artistic tactics. Easy to carry while traveling. Small notebooks. Placing them to their routes. 

\subsection{Why is it art?}
Uses artistic methods? Does it represent anything? The idea is transforming things. Anything can be transformed and re-contextualized again. The limitation is our imagination and approach. It has a claim. In can be analyzed and examined in the context of art. It is hard to say that something is art or not. However it can be examined in the context of art. There are artworks reject art galleries. There are artworks also reject commodities. It has purpose that. spread out the idea in different manner. Subverts peoples ideas. 

The work aims to liberate the imagination and change the way people see the trash.

\chapter{Conclusion}





%
%
\begin{singlespace}
\epigraph{If I seem to be over-interested in junk, it is because I am, and I have a lot of it, too --- half a garage full of bits and broken pieces. I use these things for repairing other things\ldots But it can be seen that I do have a genuine and almost miserly interest in worthless objects. My excuse is that in this era of planned obsolescence, when a thing breaks I can usually find something in my collection to repair it --- a toilet, or a motor, or a lawn mower. But I guess the truth is that I simply like junk.}{\hfill---John Steinbeck, \textit{Travels with Charley, 1962}}
\end{singlespace}




Trash is everywhere. Part of our activities. Rubbish as an object state is not static. Its value, usage may vary according to the people. Artistic intentions are aimed to transform the material. Whatever the reason that people are discarding things there must be other type of action that can be able to reverse it. Aimed to create new possibilities from trash.





%
%
Through this thesis many artworks are analyzed. They have differences in terms of purpose and approached to the trash. Some of them only valued from its plastics effects (physical properties). Some of them turned to the performance. 

Their methodologies are different. Their works can be viewed in various dimensions. Their works are also show that the great diversity and the potential usages of trash. 
There are different motivations. 

Reconsidered the notion of trash through the examining the phenonemnons of current age disposable items and ecological concerns. 

There are lots of points to approach to the trash. This shows that our richness of this topic. 











%
%
\textbf{Evolution, Critics} During the process I have also generated lots of trashed papers. Cutting and pasting things also leave unused items. I do not know what to do with them. It is hard to use all of them. Saved for later use. On the other hand being aware of this trasnformation act is very tiny part of the big picture. I can say that my trash is reduced but is it same for the other people. Hard to collect all of them. 

There is also discard is being generated during the production process of papers. How to use them? Need to find a way because varda uses her discarded item inside of the film. Always(or in every proceses) new discard items are generated.






%
%
\textbf{Future Works.} Web site will expand in time. New notebooks from different papers (in terms of location, meaning, usage purpose)

Within the scope of thesis only words that are English is examined but in different languages and cultures there might be different naming for various stages of object and trash. Analysis of them may be subject of another thesis.



% TODO conclusion?
% \comment{ANALOGY Between landfill and graveyards. WHOSE TRASH?} \quotes{There’s a relationship between graveyards and landfills, one that makes us uncomfortable, Zubiaurre explained. \quotes{What is happening to trash is what is going to happen to us. We’re all going to end up in a dump, and we’re going to decompose. That’s the ultimate destiny of humankind, and we don’t want to face that.}

% Buradaki en önemli amaç değiştirmek dönüştürmek, sadece çöpü dönüştürmek yetmez, insanların fikirlerini de dönüştürmek gerekli ve bu bir süreç işi.

\chapter{Uncategorized}

%****************************************
% EPIGRAPH:
% How to cite epigraph http://blog.apastyle.org/apastyle/2013/10/how-to-format-an-epigraph.html

%%%
%%%
%%%
\section{In Theory}

% TODO PRAP. REF.
% Ne diyor?
Literature is recycled material, a pretext for making more art. I learned this distillation of lots of literary criticism in workshops with children. I also learned that creative and critical thinking are practically the same faculty, since both take a distance from found material and turn it into stuff for interpretation. For a teacher of literature over a long lifetime, these are embarrassingly basic lessons to be learning so late, but I report them here for anyone who wants to save time and stress. (FROM Recycle the Classics, BY DORIS SOMMER)

%%
%%
\subsection{Cycle of trash}

% Like a summary sentence
Here the important thing is moving nature of trash. Objects moves and gains different meanings through this movement. Sometimes it becomes artwork, sometimes it becomes archaeological part, waits in the dump-site or wait to be recycle. It also signifies that it is relative and result of a classification issue. All of them are creates a harmony. supports each other with different words and conceptualization. 

% <From "Trash Moves On Landfills, Urban Litter and Art" by Maite Zubiaurre
Below part Explains journey of trash, its different steps. Object moves and also trash also moves. How is it life of trash? This part explains life of trash. How does it intersect with other people in which places? This part also can be understand by pointing out different part of it with detailed explanation. For example an artist collect from trash from streets and the other one goes to the (For example Vik Muniz) landfill. In other words there are different places to touch on trash. Every place generates different story? Or to understand it more deeply it covers different part of it. Or provides ideas about it. Lots of people touches it from philosophers to artist. Also in the later parts it draws attention to them. [PRAP, REF Maite Zubiaurre]

Trash moves, all the time. It becomes a steadily growing heap of clutter behind closed walls, accumulates and festers under tight lids, travels from a small trash can in the kitchen to a large one on the curbside, joins other people’s rubbish when the garbage truck arrives, drives to the transfer station, where it circles around on conveyer belts, bids farewell to recyclable or composable goods, is loaded (if declared useless: the ultimate trash) into yet another garbage truck, or barge, or even train, until it arrives at its final destination: a sanitary landfill. Even in the landfill, it does not remain still. Monster "waste handling dozers" move rubbish around, compact it and press it against the soil. More importantly, they incessantly “sculpt” refuse with their huge shovels and caterpillar wheels, making sure the garbage mound does not tip over to create a fetid avalanche. When night falls, and the trash load of the day finally disappears under a thick layer of mud, detritus still moves: once underground, it settles differently, and decomposes at a different speed, thus continuously altering landfill topography: where there was an even plateau, now there is an abruptly descending slope, and a valley; and where there was a perfectly smooth road, now there are deep crevices in the pavement. This is how trash moves. But\ldots who moves on trash? In the United States, it is mostly big-wheeled machines, an industrious army of giant yellow insects busying themselves on a heap of rubbish. In Latin America, it is mostly people. People who hand-pick garbage, who build their shacks on densely compacted trash layers, and who, day in and day out, eagerly throw themselves into the boisterous cascades of fresh debris falling from garbage trucks. In many of the garbage dumps around the world, scavenging becomes a steady job. \quotes{Garbage properly \quotes{stored} and put away brings peace of mind, as do corpses boxed and buried, or criminals confined to a cell.} And
thanks to Art: for Art shows how trash--even the one that stops moving, and particularly the one that lies squished, squashed, and weathered, almost fossilized, on the ground---has the potential to move: to move us, that is. (through the works of Filomena Cruz's photographic series “Road Kill”) [PRAP, REF Maite Zubiaurre]
% >

%%
%%
\subsection{Perspectives related with trash from different disciplines}
This is very important because the problem of trash is being tried to be handled by different disciplines. In other words there different approaches to the trash. Different problems, different solutions. Which perspective that I have. In what ways my project differs from them. The purpose is to participate people in this work, by collecting them etc. And also offer to transform it. Rescue it and than later transform it. In short, the difference between the other disciplines must be clear. 
\begin{itemize}
\item Ecological perspective: Trash causes ecological problems and it treats the balance of nature. Animals do not aware of plastics materials and they unconsciously eat them.
\item Management of it (handled by the municipals generally).
\item Technological perspectives. (seeing trash as a design problem. developed technologies and societies generates lots of trash, so this situation cause to inquiry that are they really developed? development in what sense?)
\end{itemize}

% Sample sentence:
% Such a conceptualization of waste as “the degree zero of value” has been contested for some time in different disciplines, ranging from economics to environmental studies, but most particularly by those studying consumerism or material culture

\quotes{We live in a badly engineered world, because the vast amounts of waste (both material and energetic) are needless; and that waste could be virtually eliminated through better design} \cite{mcdonough2010cradle}. In other word the problem is our technology which is not perfect. (and I'm not sure that at some point that technology will reach to the perfection or not.)

\quotes{As I prepared this issue of ReVista, some have asked me if Bogotá’s garbage crisis inspired the theme. Yes and no.  After Christmas, I traveled with a group of friends to the Chocó, an isolated and impoverished region on Colombia’s Pacific Coast. Christmas decorations abounded, and I noticed they were almost all crafted from used tin cans, old newspapers, discarded textiles and found wood objects. No one called it recycling. Trash was to be used and used again.}\cite{} Already a group of people live with their trash, what is problematic is here global ruling consumerist is not live with their trash. They can not handle their trash. There is no place for trash in their life.

\quotes{There’s a relationship between graveyards and landfills, one that makes us uncomfortable, Zubiaurre explained. \quotes{What is happening to trash is what is going to happen to us. We’re all going to end up in a dump, and we’re going to decompose. That’s the ultimate destiny of humankind, and we don’t want to face that.} Trash is also regarded differently, depending on where you live. Last year, an undergrad in Zubiaurre’s honors collegium seminar went to a poor neighborhood and scavenged through people’s trash; no one cared, Zubiaurre said. But when the same student went to Beverly Hills to go through trash, the police were nearly called. \quotes{Who decides what is public and what is private? How come trash becomes highly private in a rich neighborhood, but truly disposable in a poor neighborhood?} Zubiaurre said.} \cite{zubiaurre2015trash}

%%
%%
\subsection{Types and Comparison of trashes}
The complexity of produced trashes of societies is increasing. For example developed countries that have nuclear plant generates radioactive wastes which highly hazardous for the environment is never exist previous societies. Think batteries and so on. Every society generates different types of wastes. Differs from country to country, society to society, ages to ages.

It can be thought that when the complexity of trashed increased required effort to repair, reuse and recycle is also increase. Therefore for the ones that have no complex tools it is becoming harder to reuse objects. In other words objects become more complex their re-usage becomes less likely. 

Different production process generates different types of trash. According to production process, decomposition process\ldots

The approach to the different type of trash will be different. In other word if trash is a result of classification of objects, it can be easily extracted that there is classification inside of it. There are some trashes that are more close to the people. More easy to convert them. more easy to regain to the society.  

%%
%%
\subsection{What is wrong with trash?}
Relationship between entropy (second law of thermodynamics) and waste. Resources of nature turns to waste that it can revert it. Creating that are reversible again is problematic through the nature of sustainability. What is produced after it is consumed become worthless. 

From my point of view and approach in this thesis, trash is only one of the thing that is being discarded by humans and communities. There are lots of things that are being excluded such as homosexuals, trans, disabled peoples etc. Even if they are excluded, there is also life for them. 

% TODO Reference
John Scanlan's book, On Garbage shows how western progress always has cleared away and discarded what went before; not only material waste but also knowledge. He believes that by examining our garbage we can gain useful insight into the condition of contemporary life.

%%
%%
\subsection{Collecting trash}
One of the most important parts of the using trash in the artwork (or expressing something, or representation) is to collect them. What are the dynamics(considerations) of collecting them? (easily accessible materials or unique items.) Where to store them? Does it mean that live with trash? In other words collecting trash and using them is live with them? (making them part of life.) After the being part of the are they still trash? Can be thought that it is something that affects the lifestyle. (possessions and trash.) Another question is that how differs collecting trash from collecting other things such as objects that have archival value. What is the driving force? You may collect it to prevent object being lost. For archival things what you collect is something that has some sort of social use and meaning which is going to disappear. However, trash is never disappearing, even its amount increasing rapidly. For archival things people have memories with them, but does some applies for the trash? Who wants to keep trash? or who wants to re-see(re-visit) trash again (in a museum for example)?

[TODO: ragpickers from benjamin and archades project.]


%%%
%%%
%%%
\subsection{Literature review, discussions, ideas\ldots}
Trash art is not collage (assemblage or found object) or fragments. it is more than that. The carried messages through the medium have different meaning. It has relationship with activism, craftivism. It refuses consumption based life cycle. It suggests a life practice.

"Every day, we put unwanted material in toilets and garbage bins, regularly flushing it away or taking it out in bags to be transported far away from our homes by others. The names we give this material---waste, garbage, refuse, trash, rubbish--- have pejorative definitions. Worthless. Rejected and useless matter of any kind. Unimportant." "Our trash is a testament; what we throw away says much about our values, our habits, and our lives." "While dictionary definitions of garbage describe it as “filth” and “worthless,” scholars are careful to note that perceptions of waste and the value of material are neither static nor universally shared." "\ldots the question of who owns these discards is not trivial." "The absence of a waste stream aroused suspicion, just as the presence of particular items tell us about the habits of the consumers who generate a waste stream. Our trash is part of us, whether or not we choose to acknowledge it." \cite{zimring2012encyclopedia}



%%
%%
\subsection{Culture, Values, and Garbage}
"The Trash Talk project emphasizes the complex, yet overlooked, relationships that garbage and people share. In terms of their relationship to garbage, all people interact with it on two levels. One is a material connection, indicative of the physical and sensory contacts that people have with garbage. In some households, this connection begins with an individual removing an item from packaging, disposing of that item in the kitchen receptacle, placing that item and others into a larger bin, taking that bin to the curbside, and then the material connection ends. Others, including workers in sanitation plants and recycling centers, then continue a material connection with the garbage, but the material connection of the consumer and the garbage ends with the bin on the curbside. The second connection that people maintain with garbage is an ideational one. Unlike the material one, which is manifested in things that can be touched, moved, and sensed, the ideational connection operates on the level of cognition. The differentiation of an item of value from an item of trash, for example, has nothing to do with the material principles of the object. Instead, humans determine whether the object is of value or whether it is considered trash. The decision of whether an individual decides to dispose of a broken radio or to consider it an heirloom to be kept is highly subjective and rooted in the value systems of a culture." "After the item is eaten, the individual has to decide what to do with the remainder, such as the leftover package. The package might be reused, re-purposed, or recycled but, most likely, will be disposed of in the trash." \cite{lukas2012culture}

%%
%%
\subsection{Garbage in Modern Thought}
"Philosophers and intellectuals have expressed the need to focus on the centrality of garbage, but for everyday individuals, the understanding of garbage is often as something “out of sight, out of mind.”" "Modern humans, as part of their penchant for consumption and unsustainable living, often think very little about the waste that they produce." "Like many aspects of capitalist living, the person throwing away a piece of trash does not connect the various levels of production, consumption, and post-consumption involved in the trash. It becomes a secondary matter---an afterthought." "Martin O’Brien, among many thinkers, argues that the understanding of garbage should be a central concept, especially since garbage typically correlates with social change, social roles, and institutions. Thus, beyond the level of individuals and their relationship to garbage, there is an interest in understanding the central role that garbage plays in all of society’s roles, institutions, and forms of change." "Garbage is excess--- it is a part of society that society no longer desires." \cite{lukas2012garbage}

%
\subsubsection{Categorization and Value}
"Garbage is categorization, according to Susan Strasser." "In recycling programs and in places of refuse disposal, items of trash are categorized depending on their potential value, possible environmental harm, or time of decay. Consumers have become accustomed to the categories that are often applied to garbage. Many cities require people to dispose of their garbage in an orderly fashion---perhaps separating wet household waste from dry---and recycling programs ask individuals to divide their recyclable items into sets (such as plastic, glass, aluminum, and paper) and smaller subsets (such as PET or 01, PE-HD or 02, and PVC or 03). Garbage is an illustration of how humans use mental categories to order the material world." \cite{lukas2012garbage}

"According to John Scanlon, garbage is indicative of a separation of the world---the desirable from the unwanted. Michael Thompson uses the riddle of the rich and poor person’s approach to snot (one keeps his in a handkerchief, the other disposes of it with a tissue) to underscore the curious ways in which garbage is connected to the issue of value. While garbage is universal---all societies, extinct and extant, have produced or produce garbage--- the conditions under which garbage is understood are culturally determined. Many non-Western societies attach a much greater value to items after they are discarded. In the United States and many other nations, garbage often results not because something no longer has utilitarian value but because the item in question is defined as something of no value. Thus, garbage is not only an objective condition of material culture, but also a subjective one of mentalist culture. People define what is trash and what is valuable." \cite{lukas2012garbage}

%
\subsubsection{Semiotic Context}
"In popular writing (such as novels), in television, films, music, and other forms of mass expression, the term trash is used to signify work that is of especially low value." \cite{lukas2012garbage}

%%
%%
\subsection{Garbology}
Garbology is a study of waste as a social science. applying methodologies of archeology to the human debris. 

"Weberman infamously used techniques of what he deemed garbology to uncover what he saw as the essential nature of people. He once said, perhaps indirectly referencing Jean Brillat-Savarin’s quote about food, “You are what you throw away.”" \cite{lukas2012garbage}

"The field of garbology involves the study of refuse and waste. It enables researchers to document information on the nature and changing patterns of modern refuse, hence assisting in the study of contemporary human society or culture. According to the Oxford English Dictionary, the term was first used by waste collectors in the 1960s. A. J. Weberman popularized the term in describing his study of Bob Dylan’s garbage in 1970. It was pioneered as an academic discipline by William Rathje at the University of Arizona in 1973."

In his book “Garbology: Our Dirty Love Affair With Trash”, the Pulitzer prize-winning author Edward Humes notes that other wealthy countries with high living standards have rejected the disposable products that make up much of America's rubbish.

% Rubbish: The Archeology of Garbage, p.24
As noted, the Garbage Project has now been sorting and evaluating garbage, with scientific rigor, for two decades. THe Project has proved durable because its findings have supplied a fresh perspective on what we know---and what we think we know---about certain aspects of our life. (example of Medical researches)

%%
%%
\subsection{Trash as History/Memory}
% TODO From encylopedia
\cite{bullock2012trash}

%%
%%
\subsection{Trash Aesthetics}
%Walter Benjamin's trash aesthetics and Adornos reflection. 
\quotes{Benjamin’s approach to history is through \singlequotes{trash}---through the spent and discarded materials that crowd the everyday}  \cite{highmore2002thrashaesthetics}. Benjamin’s importance as a theorist of the everyday is most evident in his attention to the everyday experiences of modernity. In the face of the endless proliferation of trash, Benjamin potentially suggests a \singlequotes{trash aesthetics} that could be used radically and critically to attend to the everyday. The method might be thought of in terms of \singlequotes{recycling} --- an ecology of everyday experience.


%**************************************
% HERE Same sample phrases are listed you can use them:

% The work that follows is divided into three sections

% The artist thinks, acts, performs music, and writes outside the framework that society has created.

% this thesis takes a rather different approach to the resonant possibilities of discarded things. It looks to philosophical ideas and our entangled experiences of things, time and stories, which need to be traversed in order for a discarded object to be called ‘waste’.

% I also want to suggest a different way of considering trash. Maybe art maybe suggest an alternative way of seeing.

% I’d like to criticise a set of concepts or ways of thinking about discarded things that to me just don’t seem quite adequate.

% In an effort to expand art activism's capacity to create real social change, this article will (1) examine the theoretical framework behind art activism and art's efficacy in accessing emotional pathways; (2) explicate ways to strategically approach art activism through the use of specific case studies; and (3) explain one practical form of art activism-theater-based conflict resolution-that is transforming the ways communities are addressing social injustice.

% This thesis is written as a study of the socio-economic change that is currently happening in Serbia, but it’s a study that critically engages with everyday materials that provide the basis for change, rather than the economic development philosophies that are practiced through policy. 

%*******************************************************************
% FROM The Ruin and the Ruined in the Work of Kurt Schwitters.
% The German avant-garde was working from ruins literally and metaphorically, and trash was both practically and freely available; to use it was an action that took the ruins of our society, its discarded, to question how meaning is constructed.
% Marx wrote that it was not the materiality of the object but the social relations that create value, the use of urban detritus in particular, the squalid results of mass-produced human relations, infuses the materiality of Schwitters’ work with an anthropological quality

\subsection{Discussions}
Objects moves around and their values change constantly. (The idea that objects lead social lives was elaborated and discussed in detail in Arjun Appadurai (ed.). The Social Life of Things: Commodities in Cultural Perspective)

Igor Kopytoff, a professor of anthropology, introduced the notion of commoditization “as a process of becoming rather than as an all-or-none state of being.” As such, Kopytoff wrote, the biography of an object was considerably similar to that of a person: occupying different positions, leading diverse careers in the course of different periods between a beginning and an end, being defined by different regimes of value that are both economically and culturally inscribed. (Igor Kopytoff, “The Cultural Biography of Things: Commoditization as Process)

% FROM Trashion: The Return of the Disposed by Bahar Emgin
In light of this argument, one could claim that the end of the life of an object corresponds to the moment in which it is disposed of. This disposal might take place in different forms and for different reasons; however, in the most literal and common sense, the life of an object ends in a trashcan in the form of waste. In this moment, the object is left valueless in all the possible meanings of the term value: It can no more serve a function, it can on no account be exchanged for anything else, and it can by no means engage in the processes of signification to connote and endow its user with specific social values.

Referring to the work of Susan Strasser, Hawkins argues that disposal was central to the logic of mass production and hence should not be assessed as only particular to consumerism in the twentieth century: “Mass production of objects and their consumption depends on widespread acceptance of, even pleasure in, exchangeability; replacing the old, the broken, the out of fashion with the new. The capacity for serial replacement is also the capacity to throw away without concern.”

% Prap. same source...
% On the contrary, with respect to the issue of disposability, waste was handled merely “as a technical problem, something to be administered by the most efficient and rational technologies of removal.” 9 Only through the rise of environmental movements in the 1960s did the disposal of waste come to be loaded with negative meanings and iewed through a moral framework. The enormous quantities of waste accumulating in urban centers, Hawkins writes in “Plastic Bags,” were not only taken as a threat to the environment, but also as a sign of an individualistic, insensitive, and hedonistic consumer society. 10 Waste now became evil. If the environment is to be saved from our destructive power, then waste should be “managed,” Hawkins asserts. 11 Consequently, recycling gained its contemporary prominence “as virtue-added disposal\ldots disposal in which the self is morally purified, disposal as an act of redemption.” 12 Disposal in the form of recycling is now a moralistic attitude through which we pay the debt we owe to the world. Upcycling... On the other side of the coin is the business stemming from these practices; recyclers not only ease their conscience through recycling; they also make a profit. Recycling, as “the huge tertiary sector devoted to getting rid of things, is central to the maintenance of capitalism; it doesn’t just allow economies to function by removing excess and waste—it is an economy, realizing commercial value in what’s discarded,” Hawkins and Muecke write in Culture and Waste. 16 In the same manner, upcycling has already been turned into a business: Certain designers labeled eco-friendly are earning money through upcycling, competitions are organized around trashion, numerous websites are devoted to promoting and selling upcycled objects, and online and print resources explain how to upcycle at home. In short, there is a whole sector of upcycling now.

% Design, as a conduit of disposal, reintroduces rubbish as objects of distinction, invaluable and potentially priceless. People are often eager to see objects that were once considered useless and tasteless when they have been invigorated with new life.

% There is commodity aspect of it and also the process of accommodation. In this process design plays a significant role. Trash is waiting to be discovered. At the same time forgotten styles are also used in works. Therefore actually trash and forgotten styles can be considered in the same status.

% This can be sample thesis statement sentence:
% This article is about those objects that are recreated from trash through the process of upcycling. Upcycling is a term used by architect and designer William McDonaugh and chemist Michael Braungart and refers to “the process of converting an industrial nutrient (material) into something of similar or greater value, in its second life.” 4 I argue that design, in this instance, acts as a tool of transformation and reintroduces into certain orders what was once deemed waste. This theory counters the argument that an object is dead once it is disposed of.

% From The Ruin and the Ruined in the Work of Kurt Schwitters by Gemma Carroll
% Karl Marx had already written that economic value does not inhere in the materiality of the object but emerges from the social relations and organisation of labour which produces it, and that the separation between consumer market and the sphere of culture had become indiscernible.

% Schwitters is able to use the ruined, the waste products, as an anthropological exploration of society from both its unpleasant outcomes and its decay. \ldots trash was both practically and freely available; to use it was an action that took the ruins of our society, its discarded, to question how meaning is constructed. As he wrote: ‘It grows more or less according to the principle of a metropolis.’ 13 The Merzbau was itself a city; and just as Marx wrote that it was not the materiality of the object but the social relations that create value, the use of urban detritus in particular, the squalid results of mass-produced human relations, infuses the materiality of Schwitters’ work with an anthropological quality. Material has transformed into information, and ‘how’ has surpassed ‘what’ we see. The grottos in the Merzbau that still reveal this detritus most clearly could not be re-created in Bissegger’s reconstruction because, arguably, they are an exploration of absence, an exploration of ruin. As Schwitters wrote: ‘One can even shout out through refuse.’ 14 His words still echo.

% FROM Waste by William Viney
Douglas argues that our classification of dirt lies not with what objects are but where those objects are. (Think that the transformation process. Previous argument support that the transformation of trash is possible by changing the place of them. In other words removing them from landfill and waste bins to the book selves accomplish to transform trash.) ‘Dirt’, writes Douglas, ‘is the by-product of a systematic ordering and classification of matter, in so far as ordering involves rejecting inappropriate elements’. For Douglas dirt is a spatial problem, a question of not what stuff is but where it is. (from William Viney)

% “Dirt”, writes Douglas, “is the by-product of a systematic ordering and classification of matter, in so far as ordering involves rejecting inappropriate elements.” Dirt is only dirty in certain places, when it is out of its correct position. Just as faeces, for example, is considered dirty when it is in our kitchens but not when it is in our bodies, so it is that our classification of waste depends on the location of objects. 

%%% My experience here: For example when i collect the papers under the dishes or food packages, people often thinks that it is disgusting and call them as dirty. But it is very strange that the fat on the paper is previously what they are eat. 

% objects we call ‘waste’ have peculiar powers to make that temporality an explicit part of what they are and how we judge them.

% Here susan strasser ile douglasın fikirlerinde bir ortak nokta görülebilmekte. Özellikle trashın relative olması ve bu işin aslında bir ordering and classification olduğu konusunda. Remember the example given: shoes  on the dinner table. 

% King Lear, feeling of waste

% waste is “matter is out place”, a definition first given by Lord Palmerston in the mid-nineteenth century and incubated by the British anthropologist Mary Douglas, in her book Purity and Danger.

% I prefer to use the word waste to describe the things that have, for whatever reason, been leftover from use or for which use has been precluded.

% https://narratingwaste.wordpress.com/tag/king-lear/
% Not all waste is dirty, it not always dangerous, contagious or abject. \ldots waste might be quite useful in making time and in keeping time.

% Cornelia Parker's work, narrate an absent event of waste
% There is also strong relationship with the collecting discarded stuff and archeologies of waste.  
%%% Reference
% we live in a throwaway society (Barr, 2004; Cooper, 2003, 2005; Cooper and Mayers, 2000; Strasser, 1999, cf. O’Brien, 1999; Hawkins and Muecke, 2003); 
% In the two previous sections we have demonstrated the paucity of the thesis of the throwaway society. In this thesis the undeniable matter of waste, itself pressing, urgent and excessive, is used to infer the presence of a society defined by its generation; a society ceaselessly discarding and abandoning its surplus as excess, as part of an endless desire for the new. Morally corrupt and unequivocally environmentally damaging, the rhetoric of the throwaway society classifies discarding as intrinsically bad and commands us to assume control of our wasting, suggesting the adoption of heightened regulatory practices around disposal as the means to ensure that we clean-up our act. The thesis, however, lacks depth and provenance. It is, actually, glib. Indeed, to infer the presence of a throwaway society from contemporary levels of waste generation is problematic for at least four reasons.

%%%
%%%
%%%
\section{In Art}

%*************************************
% Phrases...
% What specific social issues are you trying to address through Labyrinth?
% In view of the diversity of online crowdsourced art projects, as illustrated by the examples cited so far, it is useful to map out this artistic trend by developing a comprehensive and multidimensional typology of online crowdsourced art. Table 1 organizes this classification according to a set of multiple criteria. (But this one can be used at introduction, but it is too long to fit on introduction. Maybe select works by giving prominence to some of the features. So the approach to the trash is going to be introduced and also it is reveals the what i am doing in this context.) (A Typology of Online Crowdsourced Art. Diye bir örnek var aslında benzer bir şekilde bu trash içinde yapılabilir.) 
\subsection{Crowdsourcing, Participation, Open Work}
% TODO needs reference, FROM Crowdsourced Art and Collective Creativity
In the words of Jeff Howe (2006b), the Wired columnist who coined this term in June 2006, \quotes{crowdsourcing represents the act of a company or institution taking a function once performed by employees and outsourcing it to an undefined (and generally large) network of people in the form of an open call}. The vital elements that qualify an outreach strategy as crowdsourcing are, according to Howe, the use of the open call format and the reliance on a large network of potential workers. Although in some cases there is a material reward for the best contributions, the existence of financial incentives is not a required feature in crowdsourcing. Because of the diversity of its applications, crowdsourcing continues to be a disputed term in both the scholarly literature and the popular press; Howe’s original definition is, in this sense, a helpful delineation of its practical sphere.

The value of crowdsourcing lies in the collective intelligence of the contributors. Pierre Levy (1997) describes this concept as “a form of universally distributed intelligence, constantly enhanced, coordinated in real time, and resulting in the effective mobilization of skills” (p. 13). The question of collective intelligence—and its potential efficiency in various practical settings—has received much attention in both academia and journalism. Researchers studying team performance generally agree that, under the right circumstances and with appropriate motivation, large groups of people can work together and harness their collective intelligence to achieve efficient results (Benkler, 2006; Rheingold, 2002; Surowiecki, 2004). Nevertheless, artistic creativity is different from innovation and intelligence, and it requires a unique set of skills and sensibilities as well as a particular type of cultural capital; if we admit that crowds can have collective intelligence, do they also have collective creativity in an artistic sense?

All these pages are rescued and with their [hi]story they are separated from unused new produced blank sheets and notebooks. They are not bought, not gift. They are found. They are accepted. One of the most creative medium is paper and pencil. Chance given to the people through this work.

However, in view of its reliance on the artistic contribution of a large pool of usually anonymous participants, this type of art raises important questions about notions of collective creativity, authorship, collaboration, and the shifting structure of artistic production in the new digital environment. It is well studied area. Pick a method here. Transforming trash with collaboration.

As curator Andrea Grover notes, “having the audience become co-creators is not a new impulse”; the Internet simply offered a new platform to accomplish this goal (Strickland, 2011, para. 5).

Here this question can be raised about why this type collaboration. Another option can be working together with people on a table. Creating things at that time and transforming the objects here. Possibly the connection to the people will more realistic and intense. However same things can be succeed on the internet at some level. 

While crowdsourced art challenges the traditional role of the artist, it simultaneously redefines the conventional function of the public, turning them from passive receivers into engaged producers. (This totally a new area to discuss, I'm going to summarize and introduce concepts and debates. How are they support my work and how they are give way to me?)

% This article therefore aims to fill these critical gaps by analyzing the practice of online crowdsourced art within a framework of collective creativity and participation theories. Principally, my interest is in answering two key questions. What is crowdsourced art and how can it be classified? And how does the structure of the artwork determine the degree or significance of participation?

% why collaboration, is it really a collaboration? It is methodology used in my practice. 

% TODO: Umberto Eco's Open Work


%*************************************
% ARTWORKS:
% Not all artist transform trash although some deconstruct them. Michael Landy is one of them. Michael Landy's Break Down Inventory is a two week show / display of destruction process of his all possessions on a dissemble line with the help of 10 workers. Firstly they are classified and recorded for three years and the deconstructed in two weeks by separating every element to the smallest part. Reveal all his possessions. and loosing them while you are alive. Turning them to rubbish making them unusable. breaking down the all the meaning. breaking down the connections. 
%.....................................


%*************************************
% Solid claims about its cultural aspect.
\subsection{Book: Recycled, Re-Seen: Folk Art from the Global Scrap Heap}

All of them must be paraphrased. Establish connection with the other sections.

\comment{Notes from the book.} They also choose the word recycle. (It is hard to say that they use it wrongly, but what they refer is actually upcycling. Creative reuse, inventing new things from discard.)

They also states that the practice of tribal people artfully transform Coca-Cola bottle (given as example by the authors). From western trash to tribal treasure. \paraphrase{For the Kalahari Bushmen, the process appears to be one not so much of reusing but of creating anew; not so much transforming, a inventing (p.9).} It stated these found objects is accepted as a gift from gods. A disposable item becomes imminent desire and profound social consequence. (Intercultural recycling.) \paraphrase{Both presentations tell a story about an aesthetic and cross-cultural process --- as well as an economic and political one --- which is defined by the act of recovering and transforming the detritus of the industrial age into handmade objects of renewed meaning, utility, devotion, and sometimes arresting beauty (p.10).} 

They claim that \paraphrase{the end result is a category of hybrid objects that bear the mark of at least two distinct domains, each with its won material, meaning, makers, and users (p.10).} (some of them utilitarian, some of them ornamental.) Whatever their ultimate function, each of these objects contains within itself a visual, material, and conceptual reference to multiple tehcnologies, histories and temporalities. 

\quote{Like collage in art or quotation in literature, the recycled object carries a kind of "memory" of its prior existence. Recycling always implies a stance towrd time --- between the past nad the present --- and often a perspective on cultures --- one's own and others. (Jacknis 1992)}

\paraphrase{It is stated that same object can have one meaning for one community or culture and another meaning or series of meaninf for another. Different objects have different life spans --- different degrees of permanence or disposability --- and that these life spans are socially constituted is also an integrated part of the story \ldots geographic and socioeconomic boundaries of class, caste and culture throughout the world. It is story of people abour the women, men, children etc. one person's trash into another's treasure.}

\paraphrase{the process of refabrication explored here is not to be confused with the kind of industrial stregth recycling to which we in the West are most accustomed. When we think of recycling in America and other industrialized nations we imagine an autometed sequence begining with the curbside disposal of aluminum cans, plastic bottles, and old newspapers. returned to the industrial process. solid waste management, global greening, and ecological awareness are the buzzwords that guide and motivate consumers and industries to engage in this process of secondary and postconsumer waste recycling. }

There are different kinds of recycling that is small-scale, hand done, and local --- a kind of recycling in which yesterday's newspapers are transformed by hand into tin trunk liners;empty food cans become kerosene lanterns; and old tires are refashioned ... For the same people are not used the items that are scavenged by people who have little or no contact with those who first possessed them, and may neither know nor care about their originally intended function. It is case from the film and also the case depicted in the real life photographs taken by anthropologist michael leahy which document his first colonial contact with new guinean highlanders 

(buraya yerlinin fotoğrafını koymak gerekli.)
“Renowned early-twentieth century anthropologist Michael Leahy encountered a Wabag man from Papua New Guinea wearing an aluminum whole wheat biscuit tin on his head. In the symbol system of this culture, large, bright, and shiny ornaments are connote health, well-being, sexual attractiveness, and the approval of the ancestors. 

\paraphrase{We are dealing specifically with industrially produced consumer discards and their subsequent transformation, these essays are necessarily situated in a particular time, place, and sensibility: the consumer culture of the late twentieth century. In its most basic form, I refer to a consumer culture as one "in which the activities and ethics of a society are determined by patterns of consumption" rather than production (Mamiya 1992, 2)}

While consumption has provided a foundation for the transnatioal captitalist system adntheus has much longer history than the last fifty years this thesis focused on the more recent history. because it is during the history of consumption. comsumption can be though that as gloabal not to a western or first world countires.

It is not limited with geography, ethinicity, gender, nationaltiy. The process of of retrieving and trasnforming a consumer package or product that someone else has thrown away is a phenomenon that is taking place in the largest metropolises of urban america as well as the remotest corners of amazonian rain forest. 

To frame a discussion that incorporates an understanding of such diverse locaions, objects, and peoples is to claim links between material process that originate under vastly different socail, economic, environmental, and political circumstances. 

The most obvious link are the raw materials the raw materials themselves: the packaging, broken pieces. becomes raw materials for other things. these consumer items. and it is spreated globally. It can be seen remotest corner of the world. the indigenization of western objects. (sahlins, 16) Do these hybrid objects, as many western critics might assume, point to passive acceptance of a homogenized consumer culture in the service of western capitalist expansion, or, even worse, a sense of want and deprivation when confronted with "our riches"? Marshall Sahlins: "The first commercial impulse of the local people is not to become just like us, but more like themselves. They do not necessarily despise our commodities. But they are selective in their demands and transformative of their uses of such things. (sahlins, 13)" 

human manufactured never envisioned possibilieties seen by people. It suggests a self-confididence and intellectual authority that allows local peoples to encompass western goods in their own meanings "in their own scheme of things." 

This misuse the detritus of the industrial age has been described by western theorist as ironic. the irony is often embodied visually and conceptually. opposite of natural use, expected perception. for example making something from nothing or turning trash into treasure. By juxtaposing different materials changing context and place. 

\quote{It is clear that one man's rubbish can be another man's desirable object; that rubbish, like beauty, is in the eye of the beholder. (thompson, 1979, 97)}

when an object is discarded it is perceived as being no longer of value to the person or society that once possessed it. Once a newspaper is read, or a bottle of soca pop consumed, its initial function is fulfilled and it is intended to be thrown aout as trash. .

Rubbish is, by definition, an object that is not or is no longer, owned by anyone, that falls outside all categories of economics, culture, and social control. As one of many things on the garbage heap, a discarded object even tends to take on a negative value as something unsanitary, dangerous. The socially constructed value of the object has shifted over time from its finite life span of usefulness and meaning to a timeless and valueless of socially sanctioned rubbish. 

The remarkable thing about many of these objects - especially those produced in the last of the twentieth century - is that they were specifially designed to end up on the garbage heap. They were designed to decrease in value over time - to be used one, or twice and than to be thrown away. This applies not only applies packing - designed containers that protect and promote products - but to an ever expanding list of products. themselves. (Waste and wantta bununla ilgili bir bölüm olmalı.)

throwaway spirit (vance packard) Stephen jay gould has observed: "in our world of material wealth where so many broken items are thrown away "

The flip side of this paradigm is that a person's wealth has become measured not only in how much he or she can afford to consume, but in how much he or she can afford to throw away. America is one of the greatest country in terms of generating trash and exporting them to the other third world, fourt world countires. (bu ülkeler hangileri onları foot notte açıklayabilirsin.)

while most waste ends up in these unsanitary and unsightly landfills, some small percentage is reincorparated --- or recycled --- back into the economic and cultural system of the local population. Indeed, this is the power of rubbish as a category of possessable objects: it has within itself the potential of being discovered, retrieved, and trasnformed back into an object of greater or lesser durability. 

If it is ultimately romantic to speak of these toys (or any other modern-day recyclia) in the language of resistance (by which I mean self-conscious political opposition), I would agree with Marshal Sahlins's assertion that "whether or not it comes to this [resistance], the indigenous mode of response to imperialism is always culturally subversive, insofar as the people must need to interpret the experiencei and they can do so only according to their own principles of existence. (sahlins 1992, 16)"

each recycled object contains within it a reference to two or more distinct times. 

Whatever their ultimate design or destination, these recycled artifacts are, by definition, "impure", "inauthentic" products of past and present, here and there, "us" and "them" (clifford 1992)

recycling as an economic strategy of survival in develoing countries throughout the world. creative production that is motivated, primarily, by adverse conditions of economic necessity. (lanfill orchestra here. bookcovers here.) Economic survival and adaptation are influential factors for both the makers, who build informal business on the making and selling of recycled goods, and the local consumers, for whom the market for affordable, utilitarian goods is devised. (p25)

\subsection{Statistics:}
The United nations estimates that two percent of the people in cities in nonindustrialized countries make a living from the refuse discarded by the richest ten to twenty percent (germer 1991)


\bibliographystyle{apacite}
\bibliography{literature}
\begin{appendices}
\appendix
\chapter{Glossary}

%From http://dictionary.reference.com/, http://www.etymonline.com/
\noindent\textbf{Waste.}
\begin{itemize}
\item v. to consume, spend, or employ uselessly or without adequate return; use to no avail or profit; squander:to waste money; to waste words.
\item v. to fail or neglect to use: to waste an opportunity.
\item n. An unusable or unwanted substance or material, such as a waste product. See also hazardous waste, landfill. 
\item Waste comes from the Latin vastus, meaning empty, desolate, desert, or wilderness, and it’s interesting how the Romans called desert any wilderness that wasn’t settled, including forests.  German has retained the original meaning in wüste (desert). Vastus, which also gave us vast, vain, and devastate, came to mean a waste of money and ultimately garbage.  It is tempting to see a relation with the word west – the ancients didn’t like the west, where the sun “dies”, and associated the west side with death (the Egyptian tombs and pyramids are always on the west bank of the Nile, for instance)\cite{paul2013garbage}.
\item Antonyms: save
\end{itemize}

\noindent\textbf{Trash.}
\begin{itemize}
\item n. anything worthless, useless, or discarded; rubbish.
\item n. foolish or pointless ideas, talk, or writing; nonsense.
\item n. literary or artistic material of poor or inferior quality. a literary or artistic production of poor quality.
\item Garbage collector.
\end{itemize}

\noindent\textbf{Garbage.}
\begin{itemize}
\item n. discarded animal and vegetable matter, as from a kitchen; refuse.
\item n. any matter that is no longer wanted or needed; trash.
\item Synonyms: litter, refuse, junk, rubbish.
\item Origin. early 15c., "giblets of a fowl, waste parts of an animal," later confused with garble in its sense of "siftings, refuse." Perhaps some senses derive from Old French garbe "a bundle of sheaves, entrails," from Proto-Germanic *garba- (cf. Dutch garf, German garbe "sheaf"), from PIE *ghrebh- "a handful, a grasp." Sense of "refuse, filth" is first attested 1580s; used figuratively for "worthless stuff" from 1590s. (Garbage is giblets, refuse of a fowl, waste parts of an animal (head, feet, etc.) used for human food. In modern American usage, garbage is generally restricted to mean kitchen and vegetable wastes.)
\end{itemize}

\noindent\textbf{Rubbish.}
\begin{itemize}
\item n. worthless, unwanted material that is rejected or thrown out; debris; litter; trash.
\item n. nonsense, as in writing or art.
\end{itemize}

\noindent\textbf{Junk.}
\begin{itemize}
\item n. any old or discarded material, as metal, paper, or rags.
\item n. anything that is regarded as worthless, meaningless, or contemptible; trash.
\item v. to cast aside as junk; discard as no longer of use; scrap.
\item adj. cheap, worthless, unwanted, or trashy.
\end{itemize}

\noindent\textbf{Refuse.}
\begin{itemize}
\item n. something that is discarded as worthless or useless; rubbish; trash; garbage.
\item Antonyms: accept, welcome.
\end{itemize}

\end{appendices}

\end{document}