\chapter{CONCLUSION}





%
%
\begin{singlespace}
\epigraph{If I seem to be over-interested in junk, it is because I am, and I have a lot of it, too --- half a garage full of bits and broken pieces. I use these things for repairing other things\ldots But it can be seen that I do have a genuine and almost miserly interest in worthless objects. My excuse is that in this era of planned obsolescence, when a thing breaks I can usually find something in my collection to repair it --- a toilet, or a motor, or a lawn mower. But I guess the truth is that I simply like junk.}{\hfill---John Steinbeck, \textit{Travels with Charley, 1962}}
\end{singlespace}




% The need to the item is ended up and it is least profitable to save it. Now it is considered as excess.

% Toplama mekanları: Yollar, Bilkent labı. Sabahleyin geçtiğim yollarda genelde öğrencilerin bıraktıları simit pohaça gibi şeylerin paketleri bulunmakta. Bilkent labaratuvarı, Restraurants, Pazar yeri, Kendi çalıştığım ofis, 


Trash is everywhere and produced every time. It is inseparable part of our activities. Rubbish as an object state is not static. Its value, usage might vary according to the people. Artistic intentions are aimed to transform the material. Whatever the reason that people are discarding things there must be other type of action that can be able to reverse it. Aimed to create new possibilities from trash.

This thesis forms the basic principles and approaches in the act of transformation. This research provides a framework to look at trash differently. 

This thesis work attempts to understand the transformation of trash through the artistic practices. The usage of materials in these works provide audience new interpretation of the material. 
Through the analysis of selected works methodologies, approaches, placement trash, positing of trash ...  show that trash is fluid, and value indeterminate, and are often determined by context and personal subjectivities.

% Our evaluations of what is 'rubbish' and 'art' or 'useful/non-useful' and 'good/bad' is both content and context dependent and are not fixed values. Apparent polar opposites of 'rubbish' and 'art' are blurred notions that are explored in many art works using rubbish materials. Indeed, temporal value judgements of 'rubbish' and 'art' are often at the core of such art works.

% This research began through a desire to understand more about the subject of rubbish in art practice. (or especially turning discarded papers to notebooks from a hobby to framed project or research.)

% While the notebooks have potentials to be rubbish, records of them in the website are preserved.





% In the project notebooks are given away. They are transformed to trash and served to the flow again. They will continue to flow. There is a possibility that they can be trashed again. However vice verse is possible. In other words has potential be turn to durable. Decision belongs to people again.

% Through this research another dimension of trash is explored. This present more solid understanding of trash.

% Sanatçıların sorguladıkları şeylerin sonuçlarını görmek gerekli. Neden çöpü kullanıyorlar. İlkinde  geleneksel yöntemleri terk etmeleri, 

% This research present different uses of trash and approaches. This suggests that it provides to artists wide range of representation styles. (limitless possibilities)

% In the introduction part, questioned the why artist choose to use discarded material. examples, different approaches. what are they. 

% But it cannot be completely explained by trend of moving away from traditional methods, modern times throwaway spirit play significant role in this movement. 

%
%
Through this thesis many artworks are analyzed. They have differences in terms of purpose and approached to the trash. Some of them only valued from its plastics effects (physical properties). Some of them turned to the performance. 

Their methodologies are different. Their works can be viewed in various dimensions. Their works are also show that the great diversity and the potential usages of trash. There are different motivations. (Artist have different motivations and interpretations on trash.)

Reconsidered the notion of trash through the examining the phenonemnons of current age disposable items and ecological concerns. 

There are lots of points to approach to the trash. This shows that our richness of this topic. 











%
%
During the process I have also generated lots of trashed papers. Cutting and pasting things also leave unused items. I do not know what to do with them. It is hard to use all of them. Saved for later use. On the other hand being aware of this transformation act is very tiny part of the big picture. I can say that my trash is reduced but is it same for the other people. Hard to collect all of them. There is also discard is being generated during the production process of papers. How to use them? Need to find a way because varda uses her discarded item inside of the film. Always(or in every processes) new discard items are generated.






%
%
Web site will expand in time. New notebooks from different papers (in terms of location, meaning, usage purpose)

Within the scope of thesis only words that are English is examined but in different languages and cultures there might be different naming for various stages of object and trash. Analysis of them may be subject of another thesis.



% TODO conclusion?
% \comment{ANALOGY Between landfill and graveyards. WHOSE TRASH?} \quotes{There’s a relationship between graveyards and landfills, one that makes us uncomfortable, Zubiaurre explained. \quotes{What is happening to trash is what is going to happen to us. We’re all going to end up in a dump, and we’re going to decompose. That’s the ultimate destiny of humankind, and we don’t want to face that.}

% The purpose is here to find new possibilities and dimensions to see the objects. The physically transforming objects is not sufficient, at the same time it will change people's viewpoint.

% try to the capture the reality behind the consumption practices.

% transform to their daily trash, use them for their on artist purposes, surface for their art.

% how it is essential for some people.

% as a portrait of them.

% zim car, 

% Duchamp, beyond the objects, the ideas embeded them gain importance when considering artworks.

% many possibilities in terms of their shape, size and color. offer great diversity.

% narratives through the people's discard. the image of people, their 

% record of their trash, preserving the images of them, or the others. 

% destructive manner, break down

% form - structure - shape

% different forms of trash. modes, practices, type, kind. 

% photos, everyday objects, disposables, luxury possessions like car. 

% place of trash 

% different people touches on trash. different cultures, different manner 

% for us and artist it is a source of diversity. for their work, representation and expression.

% it is not only found in landfill but also can be found in museums. 

- çöpün değişik formları. çöpü kullanmanın değişik formları. çöpün hayattaki değişik formları.
- çöpe dokunan farklı insanlar.
- o yüzden bizim için bir çeşitliliğin kaynağı
- o yüzden kaynak.
- o yüzden hazine
- nasıl bir hazine, ifade aracı olarak birer hazine.
- sadece çöplüklerde değil aynı zamanda müzelerde, 




% The initial reduction of the research scope leaves potentially much more relevant work to be done. Certain major avenues have not been discussed in depth, not because they are irrelevant, but because there simply has not been time or space to explore them in sufficient detail and some reduction in scope has been necessary to be specific in such a wide field

% For example, virtual/digital rubbish-as-art was specifically omitted in this project due to the object/material-basis of the enquiry. However, through the project development, the informal language basis of exchange in understanding the value of rubbish-as-art has proven fundamental and a further area to develop, which will most likely include in the virtual/digital realm. The question of whether there still needs to be a material/object basis of this further investigation is yet to be answered.

% The focus on material waste in art in this study has also largely omitted waste of other non-material things such as time. Wasting time as art and the relationship to labour and value is another significant area of exploration. The relationship between wasting time and material waste as documentation that is evident in some studied art works has not been developed in this thesis. Again, there is much potential for further work in this area and also links in with notions of value of online exchange also as 'wasting time'.

% Future works, as mentioned the value and meaning objects changes people to people society to society. Therefore more focused research can be conducted to analyze the local practices of transformation. the history and the place of the peoples life.

% Different selection of works and their analysis and relation between them will form another thesis which identifies another dimension of trash.




% Paraphrase 
%This thesis takes a rather different approach to the discarded objects. It looks to philosophical ideas and our entangled experiences of things, time and stories, which need to be traversed in order for a discarded object to be called ‘waste’.