\chapter{Conclusion}





%
%
\begin{singlespace}
\epigraph{If I seem to be over-interested in junk, it is because I am, and I have a lot of it, too --- half a garage full of bits and broken pieces. I use these things for repairing other things\ldots But it can be seen that I do have a genuine and almost miserly interest in worthless objects. My excuse is that in this era of planned obsolescence, when a thing breaks I can usually find something in my collection to repair it --- a toilet, or a motor, or a lawn mower. But I guess the truth is that I simply like junk.}{\hfill---John Steinbeck, \textit{Travels with Charley, 1962}}
\end{singlespace}




Trash is everywhere. Part of our activities. Rubbish as an object state is not static. Its value, usage may vary according to the people. Artistic intentions are aimed to transform the material. Whatever the reason that people are discarding things there must be other type of action that can be able to reverse it. Aimed to create new possibilities from trash.





%
%
Through this thesis many artworks are analyzed. They have differences in terms of purpose and approached to the trash. Some of them only valued from its plastics effects (physical properties). Some of them turned to the performance. 

Their methodologies are different. Their works can be viewed in various dimensions. Their works are also show that the great diversity and the potential usages of trash. 
There are different motivations. 

Reconsidered the notion of trash through the examining the phenonemnons of current age disposable items and ecological concerns. 

There are lots of points to approach to the trash. This shows that our richness of this topic. 











%
%
\textbf{Evolution, Critics} During the process I have also generated lots of trashed papers. Cutting and pasting things also leave unused items. I do not know what to do with them. It is hard to use all of them. Saved for later use. On the other hand being aware of this trasnformation act is very tiny part of the big picture. I can say that my trash is reduced but is it same for the other people. Hard to collect all of them. 

There is also discard is being generated during the production process of papers. How to use them? Need to find a way because varda uses her discarded item inside of the film. Always(or in every proceses) new discard items are generated.






%
%
\textbf{Future Works.} Web site will expand in time. New notebooks from different papers (in terms of location, meaning, usage purpose)

Within the scope of thesis only words that are English is examined but in different languages and cultures there might be different naming for various stages of object and trash. Analysis of them may be subject of another thesis.



% TODO conclusion?
% \comment{ANALOGY Between landfill and graveyards. WHOSE TRASH?} \quotes{There’s a relationship between graveyards and landfills, one that makes us uncomfortable, Zubiaurre explained. \quotes{What is happening to trash is what is going to happen to us. We’re all going to end up in a dump, and we’re going to decompose. That’s the ultimate destiny of humankind, and we don’t want to face that.}

% Buradaki en önemli amaç değiştirmek dönüştürmek, sadece çöpü dönüştürmek yetmez, insanların fikirlerini de dönüştürmek gerekli ve bu bir süreç işi.
