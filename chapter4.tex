\chapter{My work}

%%%
%%%
%%%
\section{My Artwork/Project}
Here stages of my artwork are listed:
\begin{itemize}
\item \textbf{Memories.} As far as I know I collect things it is hard to say collecting started after an exact event or time but there are several memories that I remember. One of these (First of them) is from my primary school teacher. When I was third class my teacher give us a homework to bring colorful paper to make something(whatever it is I don't remember). The day after we bring some colorful paper but she did not. Instead of this she bring paper cut out from packages that are colorful, shinny and qualify. And I remember clearly that she suggest that same for us. Do not throw out packages, look for the useful parts and keep them to use later. 
\item \textbf{Motivation.} I can not throw them away. When I throw them I became sad for them. I have to find something useful for it. Even if I can not find it, I can pass it to another person who can make use of it for own purposes. We are humans that can produce, transform items. One of most developed species in the earth that produce and use tools. There is a lot of effort to produce thrown away and it seems that all this efforts are wasted. What I mention is not related about ultimate productivity. It is more close to being thoughtful, and taking responsibility of tools, items and objects that we are using. Rather than throwing out, creating a way that all are have chance to live together is much more close to my perception. 
\item \textbf{Collecting.} One of the most found trash in my habitat is paper. I work at METU Technopolis, live at 100. yil which is the nearest settlement to METU and study at Bilkent. People live there commonly use paper and needs paper. Paper is nearly everywhere. Reusing the paper is not limited with recycling of it. There is a another ways of it. When we recycle them actually we again send away them and use it as we all know. (industrial papers and notebooks.) There is no richness here. Same type of paper. Produced after a industrial process. I collected them from my friends (people that have communicate often). Sometimes I collect them from trash bins and roads. 
\item \textbf{Transforming.} I turned them to notebooks. Actually I use it for my self. And while I was using of it I am very proud of it. I am studying nearly more 20 twenty years and I have always need for notebooks and use them. It is some sort of passion for me notebooks. Because I always admire notebook beautifully design or uniquely designed. I collect them whenever I find and most of the time I only save them for later use. 
\end{itemize}

%%
%%
\subsection{Functionality and Art Debate}
From an art historical perspective, you could say that functional art is the inverse of Marcel Duchamp's famous readymades, where he transformed utilitarian objects---a urinal, a bottle rack, etc.---into conceptual artworks by fiat: it became art because he said it was. Duchamp's works kills the functionality. It works beyond the functional perception. Moves the debate to the conceptual frame. However it is not every time case. Today many functional art objects are as avidly acquired by collectors as their fine-art brethren, and are appreciated just as much for their beauty as their use value. Ancient Chinese vases, for example, while still capable of performing their originally intended function (displaying flowers), are prized for their historic and aesthetic value more than anything else. \quotes{In conceptual art,} Sol LeWitt writes, \quotes{the idea or concept is the most important aspect of the work\ldots The idea becomes the machine that makes the art} \cite{lewitt1967paragraphs}. Therefore anything can be turned to art with a good idea. 

% TODO PRAP.
Comparing art to craft is like comparing philosophy to engineering: they're two separate ways of looking at the same thing. To me art is communication of an idea or an emotion, while craft is the physical manipulation of material. An object can easily be both, either, or neither. A sculpture, for example, may communicate, but it was constructed using craft. Likewise a teapot can communicate an idea, but it was crafted. Function is misleading and no distinction. Functional objects can still communicate ideas, so art can be functional. One object could be viewed two ways: if you look at the way it was made and the materials used, you are looking at its craft, if you think about its ideas, you are viewing it as art. An object could have been crafted, but contain no art. Even a painting can be crafted but artless. A ready-made might be art with no craft. I very much like the idea of a spectrum. One last thought: skill doesn't enter into the definition of art, since a piece could succeed as art but be poorly crafted.

%%
%%
\subsection{Paper}
"Paper is an indispensable product throughout the world. Its primary use is as a medium for writing, essential for bureaucracy, education, communications, information storage, and in the spread of information. In addition, it is used for the packaging for transport and convenience of a wide range of items from food to industrial equipment. Paper also has specific technological uses, such as for filters and in art, home furnishings, and architecture, and it has a range of uses for hygiene purposes. Paper in several forms is consumed on a daily basis by each person in the Western world." \cite{trafford2012paper}

%
\subsubsection{Environmental Impact}
Paper is both biodegradable and a renewable resource, which means in consumption and waste terms, its environmental impact is relatively small compared to the many more-toxic and bulky waste products that are found in everyday garbage. However, the chemicals, water, and electricity used in its manufacture are considerable---and these are nonrenewable resources---and certain types of chemicals used in paper production are toxic. In addition, if waste paper is sent to a landfill, it releases carbon dioxide emissions. Further, forest resources are not always as renewable as one may like to think. These environmental impacts can be greatly reduced by recycling (paper being one of the most easily and cheaply recyclable products in everyday use) and by conscientious consumption practices.

Paper made exclusively from wood is called virgin paper, while paper produced out of used paper that is re-pulped is called recycled paper. Recycling paper can greatly diminish demand for virgin fiber from wood. However, there will always be a demand for virgin paper because, although paper is thought of as a renewable resource, it cannot be recycled indefinitely. It can only be recycled four to six times, as the fibers get shorter and weaker each time. In addition, some virgin pulp must be introduced into the process each time to maintain the strength and quality of the fiber, so no matter how much is recycled, paper will always need some virgin fiber.

%
\subsubsection{History}
The word paper comes from papyrus, the plant that was first used for making a medium for writing in ancient Egypt.

%
\subsubsection{Production}
All types and qualities of paper share the same basic method of manufacture, including newspaper paper, print paper, and carton used for boxes.

%
\subsubsection{Uses}
Paper has become the most ubiquitous product in the age of information. Such products often complete their journey from shop floor to garbage in a single day; for example, newspapers, print paper, packaging, lavatory paper, tea bags, transport tickets, price tags, shopping bags, flyers, leaflets, wrapping paper, napkins, and tissues. 

%%
%%
\subsection{Why (package) paper?}
Easy to collect. Easy to find. Thrown out even if it is good quality. Packaging materials are very widespread. Appropriate for painting and writing. Has a very short life time. Disposable, there are a lot of package outside. No need to carry it. Every place gives you package paper. 

%%
%%
\subsection{Why covering notebooks and books?}
They are all package paper, already used as carry things and this work it has used again for the same purpose (but in different connection, this time trash is bound to the notebook). Trash is used to cover the papers. Cover is the most visible part of the notebook and book. Therefore trash becomes most visible part of the produced item. What is the function of cover? It gives ideas about what is inside, distinguishes from other things, protect it.

%%
%%
\subsection{Why giving away?}
Aim is to spread the idea by making something useful from trash. Increase diversity, activate (or encourage) people to embrace the trash.

%%
%%
\subsection{Why in the public space?}
To reach more audience. Actually the audience is out of the art galleries. They are walking in the streets. Putting them to next to people is much more effective. It is not visible and nearly it is hided from society. It is dirty. Removed from the society. There is a effort to hide them away. However in this project it is again showed to the society. Because it is revisited and reclaimed. 

%%
%%
\subsection{Artistic tactics}
Here I followed some tactics to realize my purposes. I called them artistic tactics. Easy to carry while traveling. Small notebooks. Placing them to their routes. 

\subsection{Why is it art?}
Uses artistic methods? Does it represent anything? The idea is transforming things. Anything can be transformed and re-contextualized again. The limitation is our imagination and approach. It has a claim. In can be analyzed and examined in the context of art. It is hard to say that something is art or not. However it can be examined in the context of art. There are artworks reject art galleries. There are artworks also reject commodities. It has purpose that. spread out the idea in different manner. Subverts peoples ideas. 

The work aims to liberate the imagination and change the way people see the trash.
