\chapter{Trash (in) Art}





\epigraph{One day, in a rubbish heap, I found an old bicycle seat lying beside a rusted handlebar, and my mind instantly linked them together. I assembled these two objects, which everyone then recognized as a bull’s head. The metamorphosis was accomplished, and I wish another metamorphosis would occur in the reverse sense. If my bull’s head were thrown in a junk heap, perhaps one day some boy would say, \quotes{Here’s something that would make a good handlebar for my bicycle!}}{\hfill ---Pablo Picasso, \textit{Trashformations}, 1998}






How and why does trash draw attention of artists? In this chapter how artist approached to the trash and their techniques are discussed in the context of art. (Historical development, methodologies, tactics, motivations\ldots) Focused on visual (plastic) arts(, because there is also a music genre called as trash music?). How does it turn to medium?






%
%
\section{Root in the Art History}
Using trashed objects in making of art can be examined with the enterance of non-art objects to art\todo{sadece bu yeterli olmayabilir}. In other words, using objects in artworks beyond their intended purpose or intended meaning and function. Developing artworks not only painting but also using paper and other stuff by pasting (or putting) them together. (How using non-art objects affected art?) \comment{Using object apart from their proposed meaning is not a new thing, through the ages people used objects and tools for different purposes. However it is not discussed in the context of fine art and not accepted method for art making.} First one is a collage work of Picasso. One of the earliest examples is Picasso’s \quotes{Still Life with Chair Caning} (1912), in which a piece of oilcloth with an imitation chair caning design was pasted onto the painting. This work of Picasso also can be considered as first example of assemblage because of a rope was used to frame the picture. Using non-art objects in the production of art challenges existing conventions of painting. 
% FROM http://www.sparknotes.com/biography/picasso/section7.rhtml
\paraphrase{This was a new way of making art; instead of painting a thing, you could stick whatever it was right onto the canvas. The three letters above the scrap of cloth, "JOU," can be understood as both the beginning of the word "JOURNAL," alluding to the customary newspaper lying across the café table, and as the French verb meaning "to play." The new technique of collage allowed new possibilities of playfulness.} \comment{\textbf{Pablo Picasso} first publicly utilized the idea when he pasted a printed image of chair caning onto his painting titled Still Life with Chair Caning (1912).} 

\begin{figure}[h!]
  \centering
  \includegraphics[height=6cm]{graphics/picasso_chair.png}
  \caption{Pablo Picasso, Still Life with Chair Caning, 1912, oil on oil-cloth over canvas edged with rope, 29 x 37 cm (Musée Picasso)}
  \label{fig:Picasso_Chair}
\end{figure}

\textbf{Marcel Duchamp} is thought to have perfected the concept several years later when he made a series of ready-mades, consisting of completely unaltered everyday objects selected by Duchamp and designated as art. The most famous example is Fountain (1917), a standard urinal purchased from a hardware store and displayed on a pedestal, resting on its side.

\begin{figure}[h!]
  \centering
  \includegraphics[height=6cm]{graphics/duchamp-bicycle-wheel-1913.jpg}
  \caption{Marcel Duchamp, Bicycle Wheel, 1913 (authorised reproduction 1951, original lost)}
  \label{fig:Duchamp_BicycleWheel}
\end{figure}

% FROM The Ruin and the Ruined in the Work of Kurt Schwitters by Gemma Carroll
\todo{works of Kurt Schwitters} \paraphrase{The German avant-garde was working from ruins literally and metaphorically, and trash was both practically and freely available; to use it was an action that took the ruins of our society, its discarded, to question how meaning is constructed. Schwitters is able to use the ruined, the waste products, as an anthropological exploration of society from both its unpleasant outcomes and its decay. \ldots trash was both practically and freely available; to use it was an action that took the ruins of our society, its discarded, to question how meaning is constructed. As he wrote: ‘It grows more or less according to the principle of a metropolis.’ The Merzbau was itself a city; and just as Marx wrote that it was not the materiality of the object but the social relations that create value, the use of urban detritus in particular, the squalid results of mass-produced human relations, infuses the materiality of Schwitters’ work with an anthropological quality. Material has transformed into information, and ‘how’ has surpassed ‘what’ we see. The grottos in the Merzbau that still reveal this detritus most clearly could not be re-created in Bissegger’s reconstruction because, arguably, they are an exploration of absence, an exploration of ruin.}





% TODO paraphrase
\paraphrase{The use of trash as a fine art medium dates back at least to the work of early-20th-century artists such as Fortunato Depero and Kurt Schwitters. Use of found materials, including garbage, has been associated with assemblage art since the 1950s and has been practiced by other well-known artists, including graphic artist Christian Boltanski, sculptor Louise Bourgeois, and photographer Andres Serrano. Art made from garbage has since become much more common in fine arts venues such as museums, galleries, and high-profile installations, including H. A. Schuldt’s famous “Trash People,” which has traveled around the world since 1996 \cite{tauxe2012encyclopedia}.}





%
%
\summary{Collage}
Collage originates from the French \textit{coller} is an artistic technique of applying manufactured, printed, or found materials, such as bits of newspaper, fabric, wallpaper, etc., to a panel or canvas, frequently in combination with painting. In about 1912–13 Pablo Picasso and Georges Braque extended this technique, combining fragments of paper, wood, linoleum, and newspapers with oil paint on canvas to form compositions. Pasting paper is not a new technique but using this it in the art making is a revolutionary movement in the  language of art \cite{waldman1992collage}. \todo{read greenberg1984collage}

% Book: Recycled, Re-Seen: Folk Art from the Global Scrap Heap
\summary{Hybridity} 
\paraphrase{Recycled object contains within it a reference to two or more distinct times. Whatever their ultimate design or destination, these recycled artifacts are, by definition, "impure", "inauthentic" products of past and present, here and there, "us" and "them" (clifford 1992)} \todo{ref.}

% TODO MOVE
% Book: Recycled, Re-Seen: Folk Art from the Global Scrap Heap
%\comment{Different meanings for everybody.} 
%\paraphrase{It is stated that same object can have one meaning for one community or culture and another meaning or series of meaning for another. Different objects have different life spans --- different degrees of permanence or disposability --- and that these life spans are socially constituted is also an integrated part of the story \ldots geographic and socioeconomic boundaries of class, caste and culture throughout the world. It is story of people --- the women, men, children etc. one person's trash into another's treasure.}

% Book: Recycled, Re-Seen: Folk Art from the Global Scrap Heap 
It is claimed that \paraphrase{the end result is a category of hybrid objects that bear the mark of at least two distinct domains, each with its won material, meaning, makers, and users (p.10).} \paraphrase{Whatever their ultimate function, each of these objects contains within itself a visual, material, and conceptual reference to multiple technologies, histories and temporalities.}

% Book: Recycled, Re-Seen: Folk Art from the Global Scrap Heap
\paraphrase{Like collage in art or quotation in literature, the recycled object carries a kind of "memory" of its prior existence. Recycling always implies a stance toward time --- between the past and the present --- and often a perspective on cultures --- one's own and others. (Jacknis 1992)} \todo{ref.}

% TODO MOVE TO COMMON APPROACH
%\comment{Semiotic Context, juxtapositions} \paraphrase{"In popular writing (such as novels), in television, films, music, and other forms of mass expression, the term trash is used to signify work that is of especially low value." \cite{lukas2012garbage}}






%
%
\summary{Assemblage} 
Assemblage work produced by the incorporation of everyday objects into a composition. It is similar to collage, but main difference is that assemblage is three dimensional rather collage is two-dimensional. Diverse range of things can be used production of work. In 1961, the exhibition "The Art of Assemblage" was featured at the New York Museum of Modern Art. William C Seitz, the curator of the exhibition, described assemblages as being made up of preformed natural or manufactured materials, objects, or fragments not intended as art materials \cite{seitz1961art}. \todo{Joseph Cornell, assemblage artist.}





%
%
\summary{Bricolage} 
Dictionary meaning\todo{which dictionary}; something constructed using whatever was available at the time. \paraphrase{Claude Levi-Strauss notes that the \textit{bricoleur} works not from the principle of making things only if natural resources are available but makes things according to those things at hand, making do with what is available. It is an expression that, like the natural cycles of the Earth, attempts to make something new from something old. \cite{levi1966savage}} The \textit{bricoleur} is \ldots someone who works with his [or her] hands and uses devious means compared to those of a craftsman \ldots \cite{levi1966savage} 

\paraphrase{\ldots someone that creates bricolage is described as a bricoleur, an odd-job man who works with his hands, employing the bricoles, the scraps or odds and ends.}

%\begin{quote}
%[He or she] is adept at performing a large number of diverse tasks; but, unlike the engineer, he [or she] does not subordinate each of them to the availability of raw materials and tools conceived and procured for the purpose of the project. His [or her] universe of instruments is closed and the rules of his [or her] game are always to make do with “whatever is at hand,” that is to say with a set of tools and materials which is always finite and is also heterogeneous because what it contains bears no relation to the current project, or indeed to any particular project, but is the contingent result of all the occasions there have been to renew or enrich the stock or to maintain it with the remains of previous constructions or destructions.\cite{levi1966savage}
%\end{quote}

%\begin{quote}
%The bricoleur, says Levi-Strauss, is someone who uses 'the means at hand,' that is, the instruments he finds at his disposition around him, those which are already there, which had not been especially conceived with an eye to the operation for which they are to be used and to which one tries by trial and error to adapt them, not hesitating to change them whenever it appears necessary, or to try several of them at once, even if their form and their origin are heterogenous---and so forth. There is therefore a critique of language in the form of bricolage, and it has even been said that bricolage is critical language itself\ldots If one calls bricolage the necessity of borrowing one's concepts from the text of a heritage which is more or less coherent or ruined, it must be said that every discourse is bricoleur.\cite{derrida1993structure}
%\end{quote}

%\begin{quote}
%The engineer, whom Lévi-Strauss opposes to the bricoleur, should be the one to construct the totality of his language, syntax, and lexicon. In this sense the engineer is a myth. A subject who would supposedly be the absolute origin of his own discourse and would supposedly construct it 'out of nothing,' 'out of whole cloth,' would be the center of the verbe, the verbe itself. The notion of the engineer who had supposedly broken with all forms of bricolage is therefore a theological idea; and since Lévi-Strauss tells us elsewhere that bricolage is mythopoetic, the odds are that the engineer is a myth produced by the bricoleur.\cite{derrida1993structure}
%\end{quote}

%[TODO use this ref]\cite{strasser1999waste}



%
%
\summary{Found Object} 
Found object originates from the French \textit{objet trouvé}, describing art created from undisguised, but often modified, objects or products that are not normally considered art, often because they already have a non-art function.

% TODO new comer
% Book: Recycled, Re-Seen: Folk Art from the Global Scrap Heap
%\comment{GIFT FROM GOD, NOT KNOWING PREVIOUS STATE} They also states that the practice of tribal people artfully transform Coca-Cola bottle (given as example by the authors). From western trash to tribal treasure. \paraphrase{For the Kalahari Bushmen, the process appears to be one not so much of reusing but of creating anew; not so much transforming, a inventing (p.9).} It stated these found objects is accepted as a gift from gods. A disposable item becomes imminent desire and profound social consequence. (Intercultural recycling.) \paraphrase{Both presentations tell a story about an aesthetic and cross-cultural process --- as well as an economic and political one --- which is defined by the act of recovering and transforming the detritus of the industrial age into handmade objects of renewed meaning, utility, devotion, and sometimes arresting beauty (p.10).}

% TODO new comer
% Book: Recycled, Re-Seen: Folk Art from the Global Scrap Heap
%[Having different meanings] For the same people are not used the items that are scavenged by people who have little or no contact with those who first possessed them, and may neither know nor care about their originally intended function. It is case from the film and also the case depicted in the real life photographs taken by anthropologist michael leahy which document his first colonial contact with new guinean highlanders.

% TODO new comer
% Book: Recycled, Re-Seen: Folk Art from the Global Scrap Heap
%\todo{yerlinin fotoğrafı}
%\quotes{Renowned early-twentieth century anthropologist Michael Leahy encountered a Wabag man from Papua New Guinea wearing an aluminum whole wheat biscuit tin on his head. In the symbol system of this culture, large, bright, and shiny ornaments are connote health, well-being, sexual attractiveness, and the approval of the ancestors.} \todo{ref.}





%(Summary From Paul M Camic.) 
%The term found object, as used in this article, refers to an existing object or artifact that is picked up (found) and generally not bought or originally intended as art, yet it is also considered to have some value (e.g., aesthetic, novelty, remembrance) to the finder. It is during the locating and finding process that the value of the object, once considered to be junk or rubbish, changes. The junk object becomes transformed into the valued found object. Objet trouve, translated from the French as “object found,” appears to have been first used by Marcel Duchamp in 1913 in reference to objects he made use of in his “readymade” art (Richter, 1965). His earliest known application for an objet trouve was seen in his well-known piece Bicycle Wheel, “where he had simply upturned a wheel on a stool” (Gale, 1997, p. 97) and labeled it a work of art. A more public and controversial introduction of the readymade occurred in 1917 when Duchamp attempted but failed to exhibit the highly contentious piece The Fountain, where a single white urinal became a readymade piece of art (Tomkins, 1996). These pieces were the beginning of Duchamp’s shift from an art striving for beauty and possessing a higher complex or hidden meaning beyond what was seen to an art form that made use of, and occasionally celebrated, the common materials and objects found in everyday living.

%Gascoygne (1936), writing about the artist’s use of “the strange medley of materials” (p. 169), referred to as objets trouve in Surrealist art, suggesting that the artist “discovers a hidden symbolic significance in the [found] object which is preserved when the object is ‘framed’ as art” (p. 170). The finder discovers an unrealised significance in the object. A new boundary is formed around the object by the finder through removing it from its found environment and placing it in a new one, thus empowering the finder in the role of creating a new reality for the object. He argued that the found object, before it is found, approximates a zero value aesthetically; the zero value increases for the finder---beholder on discovery of the object and increases further if the object is placed in another context. \ldots the meaning of material objects was derived from their symbolic relation to another (e.g., person, time, place, experience) rather than through their physical attributes (Causey, 2003; Csikszentmihalyi \& Rochberg-Halton, 1981). 

%How a region of flexibility develops, what social factors are involved in taking innovative and creative responses toward rubbish, and how an individual changes (and enhances) his or her creative responses toward society’s detritus are areas that require additional examination. (Important points a gap in the literature. Although Parsons article analyze practices, it is not very broad and detailed. Further what is the importance of it is missing.)

%The importance and enjoyment of found objects to those who participated in this study, using them in various ways and for different reasons, were strongly evident. The interaction between finder and object is an attempt to make meaning of an object that has been found, and by being found and desired becomes transformed.

%Gascoygne (1936) recognized that finding an unrealized significance in a material (found) object was empowering partially because, through the creative agency of the finder, the object’s aesthetic value had increased from zero to something greater. The transformation of rubbish from a negative value to a positive value requires the finder to develop a symbolic meaning, and sometimes a functional use, for the object that goes beyond its present situation as culturally labeled detritus while simultaneously responding to its current physical and aesthetic elements. When the found object is seen by the finder as a symbol representing another entity (e.g., when an old blue bottle with foreign lettering comes to symbolize far away intrigue, mystery, and sophistication), support is given to what Dittmar (1992) described as socially constructing a material identity for the object. Expanding on Dittmar’s use of a social interactionist perspective, the results of the present study support the possibility that the entire found object process---finding, reclassifying, and reusing objects---becomes a symbol of identity for the finder. This supports Digby’s (2006) argument that individuals make use of salvaged objects as souvenirs, which are no longer part of the commodity cycle, to rework and construct individual and social identities.

%An important difference, however, which also appears to contribute to the aesthetic experience, takes place when discovering an object unexpectedly in a nonpredetermined place and time. Unlike appreciating art in a museum, which is a boundaried activity occurring at a scheduled time and place with the anticipation of finding and looking at art objects, finding discarded objects can occur anywhere at anytime and thus, according to participants, can “trigger” a burst of sensory attention and a “surge” of cognitive activity on the part of the finder. 

%Results of this study also support consideration of found objects as important artifacts that are signifier of cultural meaning.

% [TODO adapt for my case] Found objects first came to the attention of the general public through their use by artists in early part of the 20th century. To help contextualize the application of found objects, the following section provides an overview of their use by Western artists. This is followed by an examination of objects in human development and includes aspects of psychoanalytic theory, material objects and their relationship to identity, selected cognitive theories, and an outline of the social life of objects through rubbish theory.

% Amelie olabilir. Orada fotoğraf toplayan çocuk olabilir. Onları itinayla toplayan bir adam. Tren garındaki fotoğraf kulubesi. Belki de dünyanın dört bir yanından gelen insanlar, bir kesişim noktası. Bir birinden farklı bir sürü insan. Filmde neden böyle bir eyleme yer veriliyor? Film fotoğraflar üzerinden defter üzerinden ilerliyor. Onun çevresinden bir hikaye kurgulanıyor. İnsanların artıklarını topluyor. Onların değer vermedikleri şeyleri topluyor. Bunlar baya baya aslında o insanlar. Bazı fotoğraflar çok fazla küçük parçalara ayrılıyor. Çok iyi bir örnek mi bilmiyorum. Burada niyet nedir ki? Ben buna şöyle bir okuma getireceğim diyebilirim. Amelie mesela küçük ayrıntılara dikkat eden bir insan, bu fotoğraf taplama işi de aynı şekilde. Bu arada bu fotoğraf toplama işi ilk değil. Found photos diye başka bir paper var ve orada aslında bu işin ilk uygulaması gösterilmekte. Ama benim işime yarayan kısmı ne? Neden o fotoğrafları ya da bunun filmini yapıyor. Aslında puzzlelar üretiyor. Belirli tekrar olacaktır. insanların kapalı bir ortamda nasıl pozlar verdiklerini sonrasında ise onları attıkları şeylerden görülebilir. Tekrar bir kitap haline geliyor. Tanınmayan insanların fotoğrafları. Nerde oldukları, neden ve nasıl çekildikleri belli olmayan bu yüzden aslında ilişkilerin insanlara bırakıldığı bir koleksiyon. Aslında tam da amelie bunu yapmıyor mu? Hikayeler üretiyor bu fotoğraflardan. Yeni karşılaşmalar üretmesi. Her ne kadar sıradan portre fotoğraflar olsa da birbirlerinden oldukça farklılardır. Bin bir türlü şey vardır aralarında. Nereye yerleşmeli. Başkasının çöpünü toplama. O dönemin o eşyasını arşivini tutma. ve herkes sunuluyor sonrasında bu.






%
%
%[Ready-Mades]





%
%
\section{Examples from contemporary artists}
Beyond historical development of art with non-art objects and trash, for contemporary artist what it means has slight differences. Firstly their purpose do not want to express the language of art by using trash and non-art objects because they are already done and the way of making art dramatically changed. Therefore what are their purpose and how they can be viewed?

% not easily understandable with the entering non-art objects in art. artist are responding the current societies habits in a different ways. what they are doing here is to see other possibilities and show actually what is going on. what does it mean all of these things.

% FROM ReVista
%\todo{başka bir yerde kullan. Trash is dirty. Trash is smelly.} 
\paraphrase{Trash can provide the raw materials for exquisite art---from sculpture to film and beyond.}

% FROM Paola Ibarra ReVista
\paraphrase{Whether in sculpture, photography or other media, art frequently deals, directly or by allusion, with daily challenges of life in Latin America and elsewhere. Is there a limit to recycling and representation? Or is there a point at which waste cannot become art (or anything else)?}

% TODO Naive debate??
\comment{All that can be seen as naive act because of little impact and outside of the system(which system?). As it compared to the huge portion of the material the transformed material is so little therefore it can be questioned that it has no impact and worthless as it compared to the mountains of landfill. However this very argumentative that having no impact or little impact. It might be true that this practices solve all problems of landfill or etc. But changes individuals and communities life. It recreate and rebuilds the community. Even if it is one simple object worth look and see the opportunities of it.}





%
% Tags: "dikkate alınmayanı tekrar ele alıyor". Mekan: sokaklar, şehir, kamusal alan. Traces of our life.
\textbf{Road Kill.} \paraphrase{Filomena Cruz creates photographic series named as \quotes{Road Kill} painstakingly captures tiny \quotes{trash corpses} on the pavement. It is that particular type of trash meet in the public places. Trash left behind, trash on the sidewalk, squished, squashed, and weathered. Filomena Cruz sees trash that people don’t want to see. A piece of chewing gum with an \quotes{engraved} leaf; a flattened-out tube; a corroding paper napkin with a still intact heart; or a frog-green Crayola melting in the heat, all speak the language of \quotes{worthlessness} suddenly becoming meaningful (and moving). For one thing, trash corpses faithfully record city life.} Uses public spaces and takes photos of streets and roads in unfamiliar fashion, focussing onto trash. Captures the trace of city life from trash. She does not collect the trash physically, collects the images of them. They are still there and continously changing.

\begin{figure}[h!]
  \centering
  \includegraphics[height=6cm]{graphics/FilomenaCruz_RoadKill_ReVista.jpg}
  \caption{Road Kill by Filomena Cruz}
  \label{fig:FilomenaCruz_RoadKill_ReVista}
\end{figure}






%
%
%\textbf{Sad Chair.} can be add here. Very similar to road kill, captures the record of city life. Leaved alone chairs somewhere in the city. 





%
%
\textbf{Pictures of Garbage.} Vic Muniz creates monumental portraits from trash in Jardim Gramacho, an open-air dump just outside Rio, with the help of \textit{catadores} in the documentary film (Waste Land, 2010) Waste Land, directed by Lucy Walker. Called “Pictures of Garbage,” they were created by Mr. Muniz in collaboration with the garbage pickers of Jardim Gramacho, an open-air dump just outside Rio that is one of the largest landfills in Latin America. (Important point is here collaboration with the pickers. Listening their stories and capturing their portraits. It methods similar to the mosaics, every type of discard collected from dump used his photos.) Trash and people who live there is the subject of his portraits. None of the trash is belong to that people. However again these trash express them perfectly because they make a life from trash. Their home is dump.

\begin{wrapfigure}{r}{0.4\textwidth}
  \begin{center}
    \includegraphics[width=0.37\textwidth]{graphics/vik-muniz-picturesofgarbage0.jpg}
  \end{center}
  \caption{Vic Muniz, Pictures of Garbage}
  \label{fig:VicMuniz_PicturesOfGarbage}
\end{wrapfigure}





%
%
\textbf{The Recycled Orchestra.} Favio Chávez and Nicolás Gómez decide to build musical instruments out of garbage and get 35 children from Cateura, Paraguay’s biggest trash dump, to travel the world with their \quotes{Recycled Orchestra}. 

Cateura, Paraguay is a small city that has grown atop a massive dump. It is regarded as one of the poorest slums in Latin America, a village where people live among a sea of garbage. Incredibly, the landfill itself is the primary form of subsistence for many residents, who pick through waste for items that can be used or sold. Prospects for most of the children born in Cateura is bleak as gangs and drugs await many of them. But then one day, something amazing happened. A garbage picker named Nicolás Gómez (known as “Cola”) found a piece of trash that resembled a violin and brought it to musician Favio Chávez. Using other objects collected from the dump, the pair constructed a functional violin in a place where a real violin is worth more a house. Using items gleaned completely from the dump, the pair then built a cello, a flute, a drum, and suddenly had a wild idea: could a children’s orchestra be born in one of the most depressed areas in the world? As you can guess, the answer was yes.

To view scenes from the landfill slum of Cateura in Paraguay is to look into the depths of extreme poverty. But within the contents of the landfill are glimmers of hope in the form of cardboard, utensils and other discarded materials that can be crafted into imperfect but usable musical instruments. These makeshift violins, flutes and cellos, combined with instruction from a local music teacher, have given birth to the Recycled Orchestra of Cateura. Through the music of Mozart, Beethoven and Vivaldi, this orchestra allows the young musicians to transcend their identity as children of poverty.

"A violin is worth more than a house here," says Favio Chavez, the orchestra's director and founder. In the midst of such an existence, these musicians have created something both special and truly awe-inspiring. "My life would be worthless without music." says one girl in pigtails. A young man named Juan Manuel Chavez, nicknamed Bebi, has a cello fashioned out of an oil can and old cooking tools. For the camera, he plays the Prelude to Bach's Cello Suite No. 1 — beautifully.

"People realize that we shouldn't throw away trash carelessly," says Chavez at the end of the trailer. "Well, we shouldn't throw away people either." \quotes{The world send us garbage. We send back music}.

[Critics] This project can be seen as a bricolage practice. They build their musical instruments from what is available around. What they create is actually not expected thing from trash. This work also can be seen as an example of resistence because playing western classical music with their instruments is very costly. To enjoy and play such a music requires economical wealth and it is not possible for everyone. They create their opportunites or turn trash to oppornutiy for them. Here economical restirictions result in great production boost with creativity. After seeing this instruments and listen their music changes peoples mind to the perception of the classical music and the intruments. The original items are well designed and perfect. They have identical outlook(or view). Not only trash is here transformed, but also perception to the trash and instruments are transformed here. After that point trash that you know is no longer same trash. At the same time with music lives of children are changed. They work on unimaginable thing for them. Quality of their music is different than the original one. 

\comment{This is a proff of the transformation of the original concept.}

% TODO new comer.
[Reasons of recycling] \comment{Economic conditions forces to people recycle and reuse. but even if they have no economical struggle, they continue recycle. what is the reason of it?} Recycling as an economic strategy of survival in developing countries throughout the world. creative production that is motivated, primarily, by adverse conditions of economic necessity. (lanfill orchestra here. bookcovers here.) Economic survival and adaptation are influential factors for both the makers, who build informal business on the making and selling of recycled goods, and the local consumers, for whom the market for affordable, utilitarian goods is devised. (p25)

% Dönüşümün aslında burada görebiliyorsun. Yav bu sanat mı peki? Sadece maddelerin dönüşmesi değil onlarla birlikte insanlarında dönüştüğünü görebiliyorsun burada. 
% Beni bu projede etkileyen en önemli şeyler: ilk olarak oradaki insanların hayatlarını değiştirmesi, hiç hayal edemiyecekleri bir noktaya ulaşmaları, tümünüde kendi emek ve yetenekleriyle başarmaları. yokluk içinden bir şeyler üretmeleri. Her ne kadar üretilen müzik aletleri mükemmel olmasalarda onlar için bulunmayacak bir nimet. Üretilen ürünün kalitesi gerçekten de endüstriyel standartların altında. Müziğin kalitesi diğer kadar olmasa bile farklı bir yaklaşım sunuyor olması önemli. Nasıl oluyor da bu aletten bu ses çıkıyor. Amatör ruhun getirdiği bambaşka bir algı var. Müziğin ötesi, düşündürdüğü bambaşka şeyler var, olaya yeni bir boyut getiriyor. Başka bir alanla diğerinin buluştuğunun bir göstergesi.

\summary{Indigenization} 
The indigenization of western objects. (sahlins, 16) Do these hybrid objects, as many western critics might assume, point to passive acceptance of a homogenized consumer culture in the service of western capitalist expansion, or, even worse, a sense of want and deprivation when confronted with "our riches"? Marshall Sahlins: "The first commercial impulse of the local people is not to become just like us, but more like themselves. They do not necessarily despise our commodities. But they are selective in their demands and transformative of their uses of such things. (sahlins, 13)" 

\begin{figure}
    \centering
    \begin{subfigure}[b]{0.47\textwidth}
        \includegraphics[width=\textwidth]{graphics/landfill_harmonic-sax.jpg}
        \caption{Sax}
        \label{fig:landfill_harmonic-sax}
    \end{subfigure}
    \begin{subfigure}[b]{0.47\textwidth}
        \includegraphics[width=\textwidth]{graphics/landfill_harmonic-violin.jpg}
        \caption{Violin}
        \label{fig:landfill_harmonic-violin}
    \end{subfigure}
    \caption{Instruments of Recycled Orchestra}\label{fig:animals}
\end{figure}





% TODO: http://www.publicbooks.org/interviews/recycling-literary-culture-a-conversation-with-lucia-rosa
% http://www.eloisacartonera.com.ar/ENGversion.html
\textbf{Eloísa Cartonera.} a work cooperative in Buenos Aires, proudly produces handmade books with cardboard covers. \quotes{We purchase [\ldots] cardboard from the urban pickers (cartoneros) who pick it from the streets. Our books are on Latin American literature, the most beautiful we had a chance to read in our lives.} \quotes{Some of them are preserved as art books at university libraries, while others circulate as literary pieces expected to disintegrate in time---something anticipated of the material they are made from.} [from PAOLA IBARRA, ReVista]

It reminds me that litreary is also a combination of reused thing. Because to create meanings we reuse words and idioms again and again with different combination.

\begin{figure}[ht]
  \centering
  \includegraphics[height=6cm]{graphics/EloisaCartonera_Books2.jpg}
  \caption{Books covered by Eloisa Cartonera}
  \label{fig:EloisaCartonera_Books}
\end{figure}





%
%
\textbf{American Beauty.} The film American Beauty, which features a long, poetic clip of a plastic bag swirling on an eddy of air. It is hard to think of plastic bags as things of beauty. \todo{Bunun hakkında daha fazla bilgiye ihtiyaç var.}





%
% From http://www.thisiscolossal.com/2014/06/historical-fine-oil-portraits-on-crumpled-trash-by-kim-alsbrooks/, http://www.thisiscolossal.com/2015/05/new-historical-portraits-on-flattened-cans-by-kim-alsbrooks/, http://kimalsbrookswhitetrash.blogspot.com.tr/
\textbf{Paintings on beer can.} Kim Alsbrooks, historical oil portraits on flattened beers cans and fast food containers. \quotes{The White Trash Series was developed while living in the South out of frustration with some of the prevailing ideologies, in particular, class distinction. This ideology seems to be based on a combination of myth, biased history and a bizarre sentimentality about old wars and social structures. With the juxtaposition of the portraits from museums, once painted on ivory, now on flattened trash like beer cans and fast food containers, the artist sets out to even the playing field, challenging the perception of the social elite in today’s society.} \quotes{On technique: The trash is found flat, on the street. One cannot flatten the trash. It just doesn't work. It must be found so that there are no wrinkles in the middle and the graphic should be well centered. Then the portraits are found that are complimentary to the particular trash. Generally I depict miniature portraits from the watercolor on ivory era (17th-18th century more or less). The trash is gessoed in the oval shape, image drawn in graphite, painted in oils and varnished.} I need to mention that this images are very subversing to the perception of us about these images. It is unexpected way of presenting images. There is a relationship with the image and painting. At least this work questioned this relation. With tihs work one can ask that with the painting trashed can become valueable thing or with this can this paintiing became valueless. Artsist combines two thing that can be never thought together. New combination, new meaning.

\begin{wrapfigure}{r}{0.4\textwidth}
  \begin{center}
    \includegraphics[width=0.37\textwidth]{graphics/Alsbrooks.jpg}
  \end{center}
  \caption{Kim Alsbrooks, The White Trash Series}
  \label{fig:Alsbrooks}
\end{wrapfigure}





%
% ARTWORK http://timgaudreau.com/2012/trash/trash.html
[Self-Portrait as Revealed by Trash] Even if most of people pay little attention to the trash, everybody do not. An artist takes photos of his every trash, like a popular habit that is taking photos of face everyday. He focuses on the thing that paid little attention. At the end he exhibits all the photos by covering all the walls of room. It can seen as record of his activities, record of commodities. Most of them we are not aware of them. Draw the attention of is being thrown away and every time and everywhere. They are too much and can only fit covering all the walls. \todo[inline]{Tim Gaudreau Self-Portrait as Revealed by Trash: 365 days of photographing everything I threw out – Variation I, 2006} \paraphrase{After photographing my trash over the course of the year, I ended up with 5,000 images representing my waste stream. These collections of grids represent my first experiments with using those images. This initial work ultimately led to the full scale collages that have become my Self-portrait as Revealed by Trash: 365 days of photographing everything I threw out.}

% act of taking a picture of every single item that I threw out became something else that I can't quite put my finger on. It became more than a chore; it became part of my life (albeit, a bit awkward on first dates).

%  Did this project change your behavior? Yes! At the beginning of this project, on an average day, I would consume: (1) cup of iced coffee, (2) bottles of water, (1) bottle of iced tea and (1) bottle of a sports drink. After photographing so many plastic bottles, there came a point when I couldn't bear to admit throwing out another one. As I came to understand the flow of my plastic trash, I started by cutting back everywhere I could. I stopped using plastic and foil wrap in my kitchen; I started mixing my own iced tea from concentrate; Drinking water came from gallon jugs rather than pint bottles, then ultimately just tap water. Then I switched my morning coffee to a rather beautiful reusable ceramic mug and to mixing my own sports drink from powder. 

\begin{figure}[h!]
  \centering
  \includegraphics[height=6cm]{graphics/TimGaudreau_SelfPotraitRevealedByTrash.jpg}
  \caption{Tim Gaudreau, Self-Portrait as Revealed by Trash: 365 days of photographing everything I threw out – Variation I, 2006}
  \label{fig:TimGaudreau_SelfPotraitRevealedByTrash}
\end{figure}






[Chrish Jordan Break Down inventory] \paraphrase{Not all artist transform trash although some deconstruct them. Michael Landy is one of them. Michael Landy's Break Down Inventory is a two week show / display of destruction process of his all possessions on a dissemble line with the help of 10 workers. Firstly they are classified and recorded for three years and the deconstructed in two weeks by separating every element to the smallest part. Reveal all his possessions. and loosing them while you are alive. Turning them to rubbish making them unusable. breaking down the all the meaning. breaking down the connections.} It can be an example of downcycling process. It is preferred to decompose all the complex link and relationships between the objects. They are not just ordinary things they are possessions of artist. Break Down 2001, in which he systematically destroyed all his personal possessions. His work examines what we value and what we discard, consumerism and waste, and human labour and its worth. It happens in public space. People can see the process. This process last 2 weeks. Places where people buy things actually.

%\comment{Tüm sahip olduklarını yaşarken kendi isteği ile yok ediyor. Bu hepimizin paylaştığı bir şey.}

\begin{figure}[h!]
  \centering
  \includegraphics[height=6cm]{graphics/ChrisJordan_BreakDown.jpg}
  \caption{Chris Jordan, Break Down}
  \label{fig:ChrisJordan_BreakDown}
\end{figure}






%\textbf{Martin Kaltwasser \& Folke Föbberling: City As a Resource.} One Man's Trash Is Another Man's Treasure. Garbage dump or gold mine? The German artistic collaborators Martin Kaltwasser and Folke Köbberling see rubbish as a major resource. Their projects colonize public space in the name of recycling and design: Overnight, they can make pavilions, villages of huts and even whole houses appear. In installations, exhibitions and, most frequently, guerrilla architectural interventions, they question the conditions of urban life as determined by privatization. For example, in a 2004 project called "Hausbau," they built a house in front of West Berlin's infamous Gropius-designed, failed-utopia high-rise development, Gropiustadt, and moved their family into it for a one week stay; for 2005's "Kleinod," they built a bridge and a stairway between a local family's home and a nearby community garden. In City as a Resource: One Man's Trash Is Another Man's Treasure, Köbberling and Kaltwasser propose simple ways of livening up and re-appropriating the urban habitat with sly alternatives to conventional urban planning.

%Dump or rich source – the "free materials" discovered on the street, on wasteland and on building site rubbish dumps and recycled by Martin Kaltwasser and Folke Köbberling experience incredible metamorphoses. Pavilions, villages of huts and even whole houses are appearing over night. Since 2002 the artists have been using public space as their field of experiment in projects on the use of (free) resources. In installations, exhibitions and most frequently, in actual interventions they question the conditions of urban life determined by privatisation and economisation. Through realised projects the book reveals simple ways of livening up and re-appropriating the City as a Resource and thus offers new components in an alternative to conventional urban planning. Köbberling and Kaltwasser propose informal and self-organised structures rather than the usual methods emphasising control, security and as much marketing as possible. Their approach demonstrates that often – with the help of their own clever building logistics – a huge impact may be made with a minimum of financing.

% From Hold it! : the art & architecture of public-space-bricolage-resistance-resources-aesthetics of Folke Köbberling and Martin Kaltwasser
%Public space is under siege. The compulsion to consume, increased monitoring, and continuous traffic expansion will bring fundamental change to the appearance of cities. In 1998, Folke Kobberling and Martin Kaltwasser began implementing their concept of an artistic and architectural aesthetic of resistance to this appropriation. Using 'structural interventions' in streets, squares, bridges, parks and interior spaces they propose alternatives formed of urban 'waste': litter, trash, and other discarded or donated material. 

%Köbberling\&Kaltwasser’s large architectural structures attempt to make visible the value of movement and communication through the transformation of found materials into publicly used objects and spaces. Using the city as a field of artistic experimentation, addressing issues of public space and sustainability, their installations, exhibitions and interventions critically question privatisation and economic pressures. The work will be an act of resistance to occupy and reclaim a space and change its meaning. At the same time, the work mirrors the socio-economic aspect of the city - the city as a resource, the materiality of the city, the free material of a city. 

%For their projects they use the “city as a resource” and draw on materials they find in urban spaces, for example in containers or on building sites.

%Public space and transformed materials. 

% From: http://www.goethe.de/ins/cz/prj/art/kue/koe/enindex.htm
%Public space is under siege: pressure to consume, growing control and more and more traffic are threatening to change the picture of our cities dramatically. The couple Folke Köbberling and Martin Kaltwasser have been elaborating their idea of an artistic and architectonical aesthetics of resistance against this take-over since 1998.

%They confront consumerist ideologies with alternatives: structural intervention, artistic statements, actions and theories. In doing so, the artists make use of streets, squares, bridges, parks and interiors as operational spaces. And the material they use is always obtained from urban resources: things that have been thrown away, garbage, donated things.





%
% http://www.tate.org.uk/art/artworks/nelson-the-coral-reef-t12859/text-summary
% \textbf{The Coral Reef} Nelson’s breakthrough work was The Coral Reef, which he mounted in 2000 at Matt’s Gallery from objects and debris gathered from alleys, trash bins, and car-trunk sales all over London. The title refers to Nelson’s aim of creating an intricate, reeflike network of lives “all existing under one sea, which is capitalism,” he says. (It is very corralate between Ages Varda's work. She shows at the side of human practice, Nelson looks the topic from object side.)





% FROM Paola Ibarra, ReVista
% Forever; Blue Yonder by artist Kyle Huffman
% Too Too---Much Much by Thomas Hirschhorn
% Autoconstrucción by Abraham Cruzvillegas

% FROM Burning Messages BY MICHAEL WELLEN, ReVista
% Antonio Berni

% Bu adamın ilkeleri var. Benim de var aslında.
% FROM A Present from the Sea BY SONIA CABANILLAS, ReVista
% https://www.youtube.com/watch?v=v6IoEF_Tsrw
% Nick Quijano has some rules to create assemblages. \quotes{There are certain self-imposed rules to this creative process: first, the assemblages or artefactos must all come from material washed ashore on this beach; second, it must be plastic and industrial refuse, result of the processing of fossil fuel; third, it must be polished by a long stay in deep waters, sometimes even encrusted with corals, shells or pebbles, or simply scraped by the ocean floor. As a sign of respect and sacralization, these pieces will be incorporated without any adjustment: no cutting or bending, seen as a mutilation of the object. Its identity cannot be veiled or masked but always must be recognizable amidst the other components; e.g, a comb must remain a comb even as one may see it as a mustache.} The sea returns this refuse; it is not biodegradable.

% FROM Haiti in the Time of Trash BY LINDA KHACHADURIAN, ReVista
% Haiti. When I ask him why he chooses to work in the medium of trash, he replies, \quotes{It gives respect to my city to use the garbage. It shows that everything can be used, and nothing was lost.} (TODO motivation: \quotes{I get more inspiration working with recycled materials because those pieces are unique and can’t be duplicated}) Eugène says that he’s partial to metal, which has become more and more difficult to find because of the clean-up initiative by the city. When I ask him if part of him wishes there were no such effort underway, he answers: \quotes{No. When you have clean streets you have good health, and that is the most important thing.} (This is very strange. It shows that working with trash and being clean healthy is not a contradiction. Both of them exist together.) \quotes{Other people come to Haiti and see junkyards, but we see magical playgrounds,} Jean explains as he watches them.

% FROM Thinking on Film and Trash BY ERNESTO LIVON-GROSMAN, ReVista
% By the 1950s a film like Tire Dié (Fernando Birri, Argentina, 1956) already portrays the collecting, classifying and recycling of trash not only as a source of informal income but as a commercial activity linked to the formal economy. In these films, trash is not the end of a process of consumption but the beginning of a cycle of production. These movies share the idea that trash could be a departure point to think about the modern condition as defined by consumption, class disparities, contamination and urban development. The poet Charles Baudelaire is one of the first to make the connection between the rag picker and the modern city. Walter Benjamin picks it up and from then on the fragmentary condition of trash will remain associated with contemporary art and ultimately with the Modern condition: the industrial refuse could be redeemed by art. It is in this sense that filmmaking becomes allegorical and mimics the process of recycling when it reappropiates archival materials and found footage to create new narratives from scraps, fragments, of films that were not in any way connected to these new narratives.

% \textbf{Aaron Kramer.} His motto: "Trash is the failure of imagination." \cite{meyer2007turning} In addition to concern for the environment, Kramer was drawn to recycled art because of one simple factor---the price. "Free is certainly great, and that was a driving force for me early on in my career," he said.

% \textbf{SOUP} Mandy Barker is an international award winning photographer whose work involving marine plastic debris has received global recognition. The motivation for her work is to raise awareness about plastic pollution in the world's oceans whilst highlighting the harmful affect on marine life and ultimately ourselves.

% http://www.keaggy.com/chairs/sad/12/
% \textbf{Sad chair.} Photos of lonely chairs that are no longer wanted. Artist captures the continuously left alone chairs.}

% \textbf{ArtBin} Chris Jordan [Trashing Things]
%****************************************
% EVERYTHING CAN BE TRASH. Act of trashing not act of transforming. Trash is like kara delik gibi ışığın odan kaçamaması gibi hiç bir şeyin ondan kaçılmayacağını hissediyor aslında. 
% Michael Landy's Art Bin uses a art gallery to create his dust bin. Describing the work, simply called Art Bin, as 'about failure', Landy is inviting members of the public to bring their own artistic failures along to the gallery from 29 January, where their worthlessness will be assessed. Damien Hirst, Gillian Wearing, Tracey Emin and Mark Titchner have already contributed, offering sculptures, paintings and prints. "There's no hierarchy once they are in the bin" All of them are accepted as same. Ultimate equality. Tosing them to the dust bin makes them rubbish even if they are combined from failed attempts. Nothing's too good for the art bin. Everything can fit into the art bin. Michael Landy transforms the SLG into Art Bin, a container for the disposal of works of art. As people discard their art works the enormous 600m³ bin becomes, in Michael Landy’s words, “a monument to creative failure”. It is a very big glass bin and isolates people. There is a provocative approach to the art and art objects by saying that "Nothing's too good for the art bin". There is no limitation of throwing away even if art. 

% Herkes çöpü dönüştürmüyor, bazıları ise çöpe dönüştürüyor. Ama ben bununla ilgilenmiyorum aslında.
%........................................

% \textbf{Squeezing} 

%http://www.smithsonianmag.com/arts-culture/photographer-creates-fine-art-out-trash-we-throw-environment-180951615/?no-ist
% \textbf{Found in Nature} Barry Rosenthal

% \textbf{Compression} César Baldaccini [Trashing Things]

% \textbf{Found Photos}





%
%
\section{The case of \quotes{The Gleaners and I}}
The Gleaners and I is a 2000 French documentary film by Agnès Varda that features various kinds of gleaning. The Gleaners and I is notable for its fragmented and free-form nature along with it being the first time Varda used digital cameras. This style of film making is often interpreted as a statement that great things like art can still be created through scraps, yet modern economies encourage people to only use the finest product.

It's a self-reflexive film because the director establish a relationship with the practice of gleaners and her film making practice. Some people gleans crops, the others discarded food, the other baby dolls and Agnès Varda gleans images.  

\quotes{Agnès Varda’s film, The Gleaners and I, documents the history and current practice of gleaning in France. Historically, gleaning is the act of collecting leftover crops from farmers’ fields after the harvest. However, in the film Varda expands this definition to include actions that are presently coined \quotes{dumpster diving} where people collect any types of rubbish or unwanted items to reuse them in their own way. Moreover, Varda includes her actions of collecting images with a video camera as gleaning. She is gleaning images. The Gleaners and I far more than document the lives of gleaners. It highlights the degree of global consumerism of the modern world and the ways art can exist within it in relation to gleaning.}

The official subject of this film is gleaning, the act of gathering remnants of crops from a field after the harvest. As Varda demonstrates, people can be discovered throughout the French countryside gleaning everything from potatoes to grapes, apples to oysters, much as they did hundreds of years ago (though no longer in organized groups). More figuratively, there are also urban gleaners who salvage scraps from bins, appliances from the side of the road, or vegetables from stalls after the markets have closed. And then there’s Varda herself, a gleaner of images, driving around France with a digital camera and a tiny crew (at times, she wields a smaller camera herself, permitting an even greater degree of intimacy).

Making use out of something that has been left behind and labeled as obsolete is not unique to farms and crops. There is so much discarded, yet still-viable food in dumpsters that many people live off it entirely. Seeing the value in what someone else has defined as trash is an art in itself.

Once as a common practice of gleaning throughout the years has evolved, but not disappeared. She keeps light to the modern life gleaners that are not visible every time. One of the interesting thing is here Agnès Varda feels that as a aging person will later become discarded person. In other word we can be subject of the refusal. At some point she become the subject of the thing that she track. 

There are many aspects of this documentary film. First draw attention the practice of gleaning. The discarded items that are not fit in the industrial standards because of their shape, color etc. Even if these items are discarded and we are not aware of them, there is also another life for them. Modern life gleaners feed from them. In another words someones trash becomes another trash. There are people live the boundaries of consumption societies. They create their life from the unwanted items. what the society refuses, they find a life or create a life from them. Their lifestyle is also can be seen as alternative to the life on the urban areas with deep relationship with the consumerism. (as also Zizek mentions that love is not idealization. But the industrial processes has some standards and beyond that standards there is not a place for everyone. Life is being lived in the cites is somehow idealized life or tried to be idealized life. beyond the border of the urban there is new life generated. What is not succeed in the cities succeed outside of it.)

She uses unconsciously recorded pieces in the film. \quotes{The last definition of gleaning, gleaning images, ties into what I found both the most amusing and perplexing scene in the film. It is the scene where Varda had forgotten to turn her camera off and accidentally films the ground and her lens cap bouncing along as she walks. She sets this scene to a jazz soundtrack. While watching the scene I was certainly confused. Why is this in here? What does this have to do with gleaning? And certainly why did the scene last so long? But there was also something lovely about it. Maybe it was the music, but for some reasons my senses found it audibly and aesthetically pleasing. I wasn’t able to make much sense out of the meaning behind this scene; I believe Varda referred to it as \quotes{the dance of the lens cap}.  Luckily, Ruth Cruickshank’s article, The Work of Art in the Age of Global Consumption: Agnes Varda’s Les Glaneurs et la glaneuse (The Gleaners and I), addressed this scene and clarified the potential deeper meaning behind it. Cruickshank likened the scene to something that is normally thrown away, or in this case edited out of the film, much like the way trash is thrown away or food is left behind after a harvest. Varda gleans the image of the dancing lens cap, \quotes{Where many documentary makers would leave such accidental footage on the cutting room floor, Varda draws attention to how what would habitually be perceived as waste may be viewed as supplement with its own intrinsic value. Rather than literally treating it like dirt, Varda retains and prompts reassessment of that which is normally left out of shot} \cite{cruickshank2007work}. Things that are often forgotten or discarded can easily be revamped to create something useful to someone. The scene was revamped using music and it became beautiful, much like the gleaners who found fish in the trash cooked it to make it edible, or the artist gleaner who piled discarded baby dolls into totem poles.} 

Refused is not only foods and trash. there are a lot of things that are pushed outside border of the society. She talks and investigates artist that are using discarded items in the artworks. What people not found anything the artist see countless possibilities from them. There is always alternative ways to see something different. Giving life or finding life from discarded life.

The clock found the garbage pile and taken by Agnès Varda. It does not work properly but it has important meaning for her. The time does not go on and it does not remind her that she is aging.

Here another point is that Agnes display images of people picking up things from the ground like their ancestors. Everyone somehow collecting things in their life but particularly she selects these people and their images in action. There should reason for this? For some individuals, gleaning is not a novelty or a clever way to save money, but a necessity of life. They require it to sustain their life. Combine the elements from different peoples that seems totally unrelated gains powerful theme for the documentary. What gleaning means become more open (or powerful). It draws a picture of body combination of different parts (connected, dependent to the each other). The reason of gleaning varies but the fact that gleaning is continues in different forms.

She discusses the importance found in the meaning and purpose of art forms like this: \quotes{Varda seeks to encourage viewers to consider what potential agency is demonstrated in the artfulness and contingency of gleaning by individuals excluded\ldots from the homogenizing systems of global consumption} \cite{cruickshank2007work}. One can find more treasure in trash than many of New York’s finest galleries and art exhibits; they bring a grassroots feel to what has always been seen as a stuffy and prude aspect of society.

The Gleaners And I is not an environmental film. Gleaning, Varda implies, can be understood more broadly as a form of resistance, a way of refusing to be boxed in by conventional expectations; as such, it demands that we re-learn age-old skills as well as supply individual creativity and initiative.

The film tracks a series of gleaners as they hunt for food, knicknacks, thrown away items, and personal connection. Varda travels the French countryside as well as the city to find and film not only field gleaners, but also urban gleaners and those connected to gleaners, including a wealthy restaurant owner whose ancestors were gleaners. The film spends time capturing the many aspects of gleaning and the many people who glean to survive. One such person is the teacher named Alain, an urban gleaner with a master's degree who teaches French to immigrants.

Varda's other subjects include artists who incorporate recycled materials into their work, symbols she discovers during her filming (including a clock without hands and a heart-shaped potato), and the French laws regarding gleaning versus abandoned property. Varda also spends time with Louis Pons, who explains how junk is a "cluster of possibilities." Louis Pons (born 1927) is a French collage artist. He specializes in reliefs and assemblages made entirely from discarded objects and junk. In Agnès Varda's documentary The Gleaners and I, Pons explains his artistic process and understanding of art; what others see as "a cluster of junk," he sees as "a cluster of possibilities;" and that the function of art is to tidy up one's inner and exterior worlds.






%%%
%%%
%%%
\section{Summary}
% Turning art making process to a life practice\ldots

% TODO MOVE another chapter.
% Garbage is often viewed as a form of society’s excess---as the unwanted things that are thrown out without regard. 


Lets make clear the motto: "trash to treasure". What is treasure here. What does it signify? Treasure---something very valuable. from trash. Then it is trash. Not once it is trash. What type of treasure. Who can get value from it. Who creates this treasure? How does it created? Trash is actually is opposite of treasure and then it is switched. In other words trash become treasure and treasure become trash?

Lets consider the practice of working with trash. Why someone believes that discarded objects of people \ldots Looking things that people are refused. Transforming is the one of the most common practice of whole history of man and nature. This practice is forgotten or it is still alive. Agnes Varda actually shows that it is still exist but in different forms. They use it what ruling lifestyle throw out. They are everywhere actually but you should look them carefully. They live in the borders. But live. Event if they are refused they live. Actually it is hard to say that they want to integrate to the system. They want to change it? 

ability of transform, well actually it is very easy to throw away. hard thing is to keep it. force the material. the dimension of trash.

% FROM drink UP, author: Werthan, Sarah, artist: Leech, Gwyneth. A Year in Cups. http://gwynethleech.com/
% trash and art collided. Paper and art, actually paper already medium of art, but is there anything different here. Itself is a part of a work, not the drawing, or painting.
% Documenting via a blog or a website. (what type of dimension it brings the work? maybe connect them, leave message.)
% Buying a beverage is a daily event for \ldots
% Creating art in public places can demystify the process for passers by, Leech says, making artistic expression more accessible and part of people’s everyday lives. The reason of website.
% “People see that an artist can make work anywhere, and make creative spaces anywhere,” she asserts. 

% Trash is as a material or a subject. Both of them. Can be traced examples of artworks.

\begin{itemize}
\item Where do you think this object came from?
\item Why do you think that someone labeled this object as waste?
\item How can you transform this object to tell the story that you thought of before?
\item What story does it tell you?
\item Does it remind you of an event, a specific time in history or in your life, a place, or a state of mind?
\item How can you bring that story to life? (Taylor, 2006, p. 9).
\end{itemize}

% TODO PRAP. from rethink, reimagine, reinvent
\paraphrase{Recycling art approach to using reclaimed objects in artworks requires rethinking, or examining the affordances of a particular object to explore the possibilities for the object's inclusion in an artwork. Assemblage art involves the creation of new and innovative objects from what were once considered objects of waste; that is. through their use in assemblage pieces, reclaimed objects are endowed with a new. sometimes paradoxical meaning. The transformations of the objects used in assemblage pieces ask viewers to reconsider the notion of "valuable" as they are challenged to look at everyday objects with a new perspective (Taylor, 2006). Viewers confront issues relating to the functionality of objects during modern processes of production, consumption and distribution.}

\todo[inline]{Artifact exploration promotes historical thinking, literacy investigation, and cultural expression (Higgs \& McNeal, 2006; Levstik \& Barton, 2001; Morris, 1998). Meanings are embedded within cultural artifacts and language (Vygotsky, 1978).}





"Garbage art (alternatively known as trash art or recycled art) is art created from materials including post-consumer and other waste, collected debris, or objects previously used for other purposes." "Creating art from garbage involves transforming the meaning of objects by placing them in new, aestheticized contexts. This practice is not new; tribal peoples have adapted bits of trash from industrialized societies into their traditional arts since coming into contact with products of the developed world." "Creating art from trash involves “consuming” garbage in the sense that artists appropriate and rearrange the materials in personal ways, transform their meanings, utilize them to their own ends, and represent them in new ways.It involves taking unwanted materials out of their “waste” context and recontextualizing them as “art.”" \cite{tauxe2012encyclopedia}





%
%
\textbf{What might be the meaning of using trash as a medium in the artworks? Questioning trash as a medium for artist}
\begin{itemize}
\item Some works try to raise awareness the problems that are the result of trash. (It treats environment and nature.)
\item Some of them reflect people's lifestyle especially throw away culture. As a mirror of current lifestyle.
\item Try to find a new value and meaning from the discarded material that are useless anymore. To explore a new approach, new way. Subvert people's ideas about trash and their attitudes by turning materials to the something meaningful (or valuable). Trash to treasure.
\item Using discarded item to represent other discarded things by the ruling ideology or approach. For example, trash can be used to represent refugees. The things that we are trying to discard does not mean that they have no value, instead it means that we have no ability to reveal its potential. In other words, refugees have potential but we see them as players that will change our current system. Therefore, it can be said that willing to transform trash to treasure is to require change of current lifestyle. Rejecting discarding something especially thing that you get value from it is a process and spread through to the ones life.
\item One way is not to produce trash. (Zero trash philosophy.) The other one is to transform trash into something else.
\item What type of experience is that collecting and working on objects that are generally discarded? Experiencing out of common practice, being open to new explorations.
\item Instead of a world that produce trash, how could it be a world created from trash?
\item Combining industrial goods with objects transformed from trash is another way to find a place to trash in the community. It also signifies that trash still has a good quality to used with new materials. Creating composite products from new and reused items. Using the valuable thing with the invaluable thing. It becomes more valuable or less valuable. Depends on the perception.
\item Aesthetics of trash. Revealing aesthetics value of discarded stuff. (Unique visual value. Trash portraits, sculptures etc.)
\end{itemize}

Artists works and the place of the trash in the art can be summarized as follow:
\begin{itemize}
\item Artist see them as potential. But what type potential it haves. In which areas. They are already thrown away, if it was value or has a place in the current system they are not thrown away. There is an alternative life and outside of the ruling system.
\item Visual uniqueness, aesthetics dimension.
\end{itemize}

% Ben bölümde aslında çöpün sanatta nasıl yer aldığına bakıyor olacağım. İlk kim kullanmış nasıl kullanmış. Sonrasında hangi anlamlarda kullanılmış. 

% Aslında ben uygun her yerde çöpe olan yaklaşımlarla ilgili örnekler verdim. Peki ya bu bölümde ne yapacağım. Belki gene değineceğim. Ya da ne anlamlara geldiklerini özetleyeceğim. 

% Hangi sonuçlar çıkacak aslında bu bölümden benim için kıymetli olan o. Özellikle caselerden. Bu nasıl bir artistict act oluyor.

% Her sanatçı bu arada çöpü dönüştürmüyor. Bazıları ise onu oldukları gibi kullanıyorlar. Yani farklı kullanım seneryoları var. Ama ben dönüştürmeye dair ne çıkaracağım bu bölümde. 

% aslında dönüştüğünü nereden iddia edebilirim. Rubbish theory? Durable state. Ya da onun tekrar hayata geçtiği. Benimkiler de belli noktalarda fösterilecekler. İnsanlar o defterleri saklıyorlar ne de olsa. Zaten dönüşme işlemi direk müzeye çıkmasıyla oluyor. İllaha onun materyal olarak dönüşmesi gerekmiyor. Bu yüzden her iş aslında bir şekilde dönüştürüyor. Sanatsal pratikler çöpü dönüştürüyorlar. 

% Sanatsal bir eylem olduğuna nasıl varacağım peki? Çünkü topluyorsun, belli bir yaklaşım geliştiriyorsun. Bu eyleme insanları davet ediyorsun. Sanatın eylem olma olayı... Buna burda değinip, kendi işime geçmeliyim.

% Ne topladım buradan. Bu işin sanatsal bağlamdaki yerini öğrendik, yaklaşımları öğrendik. Yeni ihtimaller görme, hayatta var olma, alternatif oluşturma gibi. Ben de bu tür şeyleri kendi işimde kullanacağım.

Childs can enjoy with trash. I remember from my childhood, we collect crown cap and play with them. Some caps are found less and they worth more. We are looking everywhere for them. To make them flat we put them on the railways. After train passed we get perfect plat cap. At that time it is not trash for us. It has a value and part of our games and enjoy. To have fun a bunch of trash can be enough for us.

Think a city that has trash monument in the every corner. created from their trash. merged with the city life and gaining unique cityscapes and aesthetics. or think that a museum a trash museum. Exhibits works of art embracing the trash in all aspects. Maybe done by the artist or the visitors from all around world. A place for garbage other than a landfill. Waiting their creator to meet again. What a great idea isn't it? Meeting their creators again. But this time their creator can recognize their trash. They transformed to totally new thing. Reborn. Transformed (Kafka, Gregor Samsa).

\todo[inline]{Overview of all artworks. How do they differ from each other in terms of transformation(conceptually, physically), methodology, collecting, composition and so on.}

\todo{a chart here.}

\summary{As a canvas}
Bazı işler çöpleri kendi işleri için bir kanvasa dönüştürmüşler. Çay poşetlerini boyayan kadın, Kitapları boyayanlar, teneke kutuları boyayanlar. Hepsi aslında kullandığı ürünle farklı bir ilişki kuruyor. Onları farklı bir hale getiriyor.  

Toplumun davranışlarına karşı bir cevap, görülmeyen, veya farkedilemeyen unutulup giden bir hale getiriyor.

\summary{Trashing}
Not all artist transform, some of them use them as they are. further some of them trashes things. [squeezing, artbin.] As we can think that trash is valuess. But the works of cesar claims that opposite of it. In his work he sequuzes objects and commodities. Makes them useless and functionless. But they earn great value in different value systems. In the literature generally focused topic is to analysis of transformation of trash to treasure (trash->treasure) but at the same time the opposite way is also exist.

% TODO From Beautiful Trash Art and Transformation BY PAOLA IBARRA, ReVista
Recycling has always been a common practice in the arts at least at a non-material level. From creating a world of words in literature, to rhythm and images in poetry, sampling in hip hop music, representation in the visual arts, or editing the illusory continuity of a film, art implies taking disparate elements (ideas, images, references, objects, etc.) and putting them together to form a new whole. Take and put. De-contextualize and re-contextualize. In that sense, art, as a system, is an act of recycling [from PAOLA IBARRA]. 

% TODO new comer
% FROM Book: Recycled, Re-Seen: Folk Art from the Global Scrap Heap
\summary{Resistance} 
\comment{Resistance, tactile, agnes varda?} If it is ultimately romantic to speak of these toys (or any other modern-day recyclia) in the language of resistance (by which I mean self-conscious political opposition), I would agree with Marshal Sahlins's assertion that "whether or not it comes to this [resistance], the indigenous mode of response to imperialism is always culturally subversive, insofar as the people must need to interpret the experience and they can do so only according to their own principles of existence. (sahlins 1992, 16)"

% TODO new comer
% FROM Book: Recycled, Re-Seen: Folk Art from the Global Scrap Heap
\summary{Ironic}
\comment{IRONIC, maybe added to results of transformation, reusing trashed items in different contexts.} This misuse the detritus of the industrial age has been described by western theorist as ironic. the irony is often embodied visually and conceptually. opposite of natural use, expected perception. for example making something from nothing or turning trash into treasure. By juxtaposing different materials changing context and place. 

% TODO new comer
% FROM Book: Recycled, Re-Seen: Folk Art from the Global Scrap Heap
\comment{People applies their own perception to interpreted the items.} Human manufactured never envisioned possibilities seen by people. It suggests a self-confididence and intellectual authority that allows local peoples to encompass western goods in their own meanings "in their own scheme of things." 


% Günlük rutinleri çöp üzerinden yakalayanlar var. Her gün ürettiği çöpün fotoğrafını çeken insan... Ürettikleri çöpleri ifade etme araçları olarak kullananlar.

