\chapter{CONCLUSION}





%
%
\begin{singlespace}
\epigraph{If I seem to be over-interested in junk, it is because I am, and I have a lot of it, too --- half a garage full of bits and broken pieces. I use these things for repairing other things\ldots But it can be seen that I do have a genuine and almost miserly interest in worthless objects. My excuse is that in this era of planned obsolescence, when a thing breaks I can usually find something in my collection to repair it --- a toilet, or a motor, or a lawn mower. But I guess the truth is that I simply like junk.}{\hfill---John Steinbeck, \textit{Travels with Charley, 1962}}
\end{singlespace}






This research began through a desire to understand more about the subject of rubbish in art practice. It attempts to figure out the transformation of trash through the artistic practices. It presents different uses of trash and approaches in the context of art. Example artworks suggest that it provides to artists wide range of possibilities. In other words it can be viewed as a source of diversity.

This thesis takes a rather different approach to the discarded objects. It looks to philosophical ideas, theories and people's experiences of things in throw away culture.

People's evaluations of what is trash and treasure are both content and context dependent and are not fixed values. The line between them are blurred in the explored practices and works of art. Indeed, temporal value judgements of 'rubbish' and 'art' are often at the core of such art works. The analysis of selected works methodologies and approaches show that trash is fluid, and value indeterminate, and are often determined by context and personal subjectivities. 

This thesis analyzes the basic principles and approaches in the act of transformation of trash. It provides examples of artworks that inquiry the issue of trash in various dimensions. César shows that trash is not only found in dumps but also can be found in museums. Dong shows that how trash is essential for some people to hold on the life.

Whatever the reason of discarding objects the alternative if it showed through the works of artist and the thesis project. In the project aimed to create new possibilities from trash. The usage of discarded paper bags, packages in the project provide audience new interpretations of the materials and meanings. On the other hand transforming discarded papers to notebooks evolved from a hobby to research grounded framed project.

The project developed in the scope of the thesis open to further developments. Content of the web site will expand in time. New notebooks from various discarded papers (in terms of location, meaning and purpose) will be compiled. Upcycling of trash will be extended to the other waste materials in addition to the paper. However research, practices, methods followed in this thesis form the basis of all these.

There are lots of points to approach to the trash. This shows that the richness of this topic. The scope of the research leaves many more areas untouched. There is much potential for further work in this subject. Certain major avenues have not been discussed in depth, not because they are irrelevant, but because there simply has not been time or space to explore them in sufficient detail and some reduction in scope has been necessary to be specific in such a wide field.

For instance, research can be extended to the notion of waste in the digital realm. Both digital notions like spam emails and  effects of digital devices on e-waste need further investigation.

The focus on waste material in art in this study has also largely disregarded waste of other non-material concepts such as time. Wasting time is another significant area of exploration.

As mentioned the value and meaning of objects changes from people to people, from society to society. Therefore more focused research can be conducted to analyze the local practices of transformation. This thesis work is formed in the English language but local languages have potentials to reveal the various concepts and practices of trash.

Different selection of works and their analysis and relation between them will form another thesis which identifies another dimension of trash.
