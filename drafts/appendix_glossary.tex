\chapter{GLOSSARY}

%From http://dictionary.reference.com/, http://www.etymonline.com/
\noindent\textbf{Waste.}
\begin{itemize}
\item v. to consume, spend, or employ uselessly or without adequate return; use to no avail or profit; squander:to waste money; to waste words.
\item v. to fail or neglect to use: to waste an opportunity.
\item n. An unusable or unwanted substance or material, such as a waste product. See also hazardous waste, landfill. 
\item Waste comes from the Latin vastus, meaning empty, desolate, desert, or wilderness, and it’s interesting how the Romans called desert any wilderness that wasn’t settled, including forests.  German has retained the original meaning in wüste (desert). Vastus, which also gave us vast, vain, and devastate, came to mean a waste of money and ultimately garbage.  It is tempting to see a relation with the word west – the ancients didn’t like the west, where the sun “dies”, and associated the west side with death (the Egyptian tombs and pyramids are always on the west bank of the Nile, for instance)\cite{paul2013garbage}.
\item Antonyms: save
\end{itemize}

\noindent\textbf{Trash.}
\begin{itemize}
\item n. anything worthless, useless, or discarded; rubbish.
\item n. foolish or pointless ideas, talk, or writing; nonsense.
\item n. literary or artistic material of poor or inferior quality. a literary or artistic production of poor quality.
\item Garbage collector.
\end{itemize}

\noindent\textbf{Garbage.}
\begin{itemize}
\item n. discarded animal and vegetable matter, as from a kitchen; refuse.
\item n. any matter that is no longer wanted or needed; trash.
\item Synonyms: litter, refuse, junk, rubbish.
\item Origin. early 15c., "giblets of a fowl, waste parts of an animal," later confused with garble in its sense of "siftings, refuse." Perhaps some senses derive from Old French garbe "a bundle of sheaves, entrails," from Proto-Germanic *garba- (cf. Dutch garf, German garbe "sheaf"), from PIE *ghrebh- "a handful, a grasp." Sense of "refuse, filth" is first attested 1580s; used figuratively for "worthless stuff" from 1590s. (Garbage is giblets, refuse of a fowl, waste parts of an animal (head, feet, etc.) used for human food. In modern American usage, garbage is generally restricted to mean kitchen and vegetable wastes.)
\end{itemize}

\noindent\textbf{Rubbish.}
\begin{itemize}
\item n. worthless, unwanted material that is rejected or thrown out; debris; litter; trash.
\item n. nonsense, as in writing or art.
\end{itemize}

\noindent\textbf{Junk.}
\begin{itemize}
\item n. any old or discarded material, as metal, paper, or rags.
\item n. anything that is regarded as worthless, meaningless, or contemptible; trash.
\item v. to cast aside as junk; discard as no longer of use; scrap.
\item adj. cheap, worthless, unwanted, or trashy.
\end{itemize}

\noindent\textbf{Refuse.}
\begin{itemize}
\item n. something that is discarded as worthless or useless; rubbish; trash; garbage.
\item Antonyms: accept, welcome.
\end{itemize}
