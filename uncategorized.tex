\chapter{Uncategorized}

%****************************************
% EPIGRAPH:
% How to cite epigraph http://blog.apastyle.org/apastyle/2013/10/how-to-format-an-epigraph.html

% FROM http://stanforddailyarchive.com/cgi-bin/stanford?a=d&d=stanford19920130-01.2.5&e=-------en-20--1--txt-txIN-------#
% The thing to do is to find some way --- and it can be done, I think, through art --- to enjoy life in its vast multiplicity. --- John Cage

% FROM Wild Art
% Bu epigraphta aslında şöyle bir sıkıntı var, bir şeyin miktarı artınca daha fazla insanın onun üzerinde düşeceğini varsaymak biraz sıkıntı olabilir.
% We are going to make a lot more junk, and there is going to be a lot more junk art. I think people are going to have a lot to say about it, and more people need to think about it. --- Jeremy Mayer, quote taken from American Craft Magazine, December/January 2011.

%%%
%%%
%%%
\section{In Theory}

%%
%%
\subsection{What is trash?}
In this thesis work to understand the different aspects of trash is a key element. Because as scholars agreed on (ref required.) it is part of our life and daily practice and it is very common concepts from developed western societies to rural areas, from disaster areas to \ldots(something beautiful here). Sometimes it is accepted as a problem that is need to carefully and seriously manged. On the other hand it is accepted as a source of diversity. (To establish a solid understanding for trash, it is important to see its dilemmas (it is really a dilemma?)). Therefore we should ask these questions: What is trash? How does it became trash? Is it a end product or a source material? How much is it valuable? How much is it dangerous?

%%
%%
\subsection{Trash is everywhere and, produced every time}
Modern (developed) societies are continuously generating trash and, pile them on landfills. It's a common object category that all people share its possession. During daily activities trash is generated and people get rid of them by throwing away. (Various objects become trash after their primary functions daily. Who defined the primary function? Primary function is the only function. People cares the package of the objects that they buy. They buy the coffee not the cup of it. After coffee finished the life of cup also finishes. There is a lifetime defined (or forecast) by the producer of objects. However it is not bound to the producer, also consumer play an important role. There are different choices. Throw it away. Keep it. Give it. Mostly wining choice is throwing away and by the result of it mountains of garbage are increasing.) The vast amount of industrial discarded items spread through the landfills to oceans. They are the result of highly complex industrial production methods. They are not easily disposable items. They live in the nature thousands of years. Most of them packages that are used to carry or protect other materials. After real material used these packages became valueless (or useless). (types of trash can be mentioned here, but currently in the artwork I'm using paper packages, therefore, it is more important.) How manage the all this increasing trash that damaging nature?  This is the common approach to trash and the main problem. (actually the sustainability problem.) It is not the only problem, It can be thought that it is a losing the ability to transform new things, alternative behaviors etc. (Instead of creating new opportunities or alternatives, it is a consuming all them and producing huge pile of trash.) (reference Zizek idealization of nature, to love nature is to love trash. Live with trash. Do not see it as trash actually. Then the question is how to love trash? how to live with trash? can living with our trash enrich (our perceptions, abilities)? how not to see them as trash and useless? Can it be possible with art?)

% TODO PRAP. REF.
Waste pickers are an important part of the story in Latin America, as more is being thrown away than ever. The World Bank estimates that the amount of solid waste generated in cities is growing faster than the rate of urbanization. The higher the income level and the rate of urbanization, the greater the amount of solid waste produced. OECD countries produce almost half of the world’s waste. Africa and South Asia produce the least waste. High-income countries have the highest collection rates and are most likely to dispose of waste to landfills or incinerators. Low-income countries have the lowest collection rates and are most likely to dispose of their waste in open dumps. However, low-income countries also have the largest numbers of informal waste pickers who collect, sort, and reclaim recyclables---thus reducing costs to the city and to the environment. (FROM Trash as Treasure, BY WILLIAM L. FASH AND E. WYLLYS ANDREWS)

% TODO PRAP. REF.
Literature is recycled material, a pretext for making more art. I learned this distillation of lots of literary criticism in workshops with children. I also learned that creative and critical thinking are practically the same faculty, since both take a distance from found material and turn it into stuff for interpretation. For a teacher of literature over a long lifetime, these are embarrassingly basic lessons to be learning so late, but I report them here for anyone who wants to save time and stress. (FROM Recycle the Classics, BY DORIS SOMMER)

% From Beautiful Trash Art and Transformation BY PAOLA IBARRA ReVista
% TODO PRAP. REF.
We relate to garbage daily. We use it, produce it and dispose of it. Endlessly. The most obsessive of us get rid of it as fast as we can. The hoarder likes to salvage a few things for later use---the plastic and glass containers, the cardboard boxes. We know that capitalism’s escalating cycles of production, consumption and obsolescence keep worsening an already problematic relationship between humankind, waste and nature (not to mention social and economic relations). Despite a relatively increased awareness about consumption and its consequences, the pace at which we also acquire and dispose of material objects is exploding. Particularly in the connection between garbage and the arts, I am interested in two questions. First, the issue of recycling as a general practice in the arts; and secondly, in the whole issue of representation---that is, representation of waste as subject, and representation (of waste or others subjects) through waste as material. (BY PAOLA IBARRA)

%%
%%
\subsection{Cycle of trash}
Trash moves, objects moves from place to place. From homes to garbage trucks. From streets to land fills. From garbage baskets to sculptures. [REF. MIT Garbage project] Lots of different people touches to trash. With the trash what moves?

% Like a summary sentence
Here the important thing is moving nature of trash. Objects moves and gains different meanings through this movement. Sometimes it becomes artwork, sometimes it becomes archaeological part, waits in the dump-site or wait to be recycle. It also signifies that it is relative and result of a classification issue. All of them are creates a harmony. supports each other with different words and conceptualization. 

% <From "Trash Moves On Landfills, Urban Litter and Art" by Maite Zubiaurre
Below part Explains journey of trash, its different steps. Object moves and also trash also moves. How is it life of trash? This part explains life of trash. How does it intersect with other people in which places? This part also can be understand by pointing out different part of it with detailed explanation. For example an artist collect from trash from streets and the other one goes to the (For example Vik Muniz) landfill. In other words there are different places to touch on trash. Every place generates different story? Or to understand it more deeply it covers different part of it. Or provides ideas about it. Lots of people touches it from philosophers to artist. Also in the later parts it draws attention to them. [PRAP, REF Maite Zubiaurre]

Trash moves, all the time. It becomes a steadily growing heap of clutter behind closed walls, accumulates and festers under tight lids, travels from a small trash can in the kitchen to a large one on the curbside, joins other people’s rubbish when the garbage truck arrives, drives to the transfer station, where it circles around on conveyer belts, bids farewell to recyclable or composable goods, is loaded (if declared useless: the ultimate trash) into yet another garbage truck, or barge, or even train, until it arrives at its final destination: a sanitary landfill. Even in the landfill, it does not remain still. Monster "waste handling dozers" move rubbish around, compact it and press it against the soil. More importantly, they incessantly “sculpt” refuse with their huge shovels and caterpillar wheels, making sure the garbage mound does not tip over to create a fetid avalanche. When night falls, and the trash load of the day finally disappears under a thick layer of mud, detritus still moves: once underground, it settles differently, and decomposes at a different speed, thus continuously altering landfill topography: where there was an even plateau, now there is an abruptly descending slope, and a valley; and where there was a perfectly smooth road, now there are deep crevices in the pavement. This is how trash moves. But\ldots who moves on trash? In the United States, it is mostly big-wheeled machines, an industrious army of giant yellow insects busying themselves on a heap of rubbish. In Latin America, it is mostly people. People who hand-pick garbage, who build their shacks on densely compacted trash layers, and who, day in and day out, eagerly throw themselves into the boisterous cascades of fresh debris falling from garbage trucks. In many of the garbage dumps around the world, scavenging becomes a steady job. \quotes{Garbage properly \quotes{stored} and put away brings peace of mind, as do corpses boxed and buried, or criminals confined to a cell.} And
thanks to Art: for Art shows how trash--even the one that stops moving, and particularly the one that lies squished, squashed, and weathered, almost fossilized, on the ground---has the potential to move: to move us, that is. (through the works of Filomena Cruz's photographic series “Road Kill”) [PRAP, REF Maite Zubiaurre]
% >

%%
%%
\subsection{Perspectives related with trash from different disciplines}
This is very important because the problem of trash is being tried to be handled by different disciplines. In other words there different approaches to the trash. Different problems, different solutions. Which perspective that I have. In what ways my project differs from them. The purpose is to participate people in this work, by collecting them etc. And also offer to transform it. Rescue it and than later transform it. In short, the difference between the other disciplines must be clear. 
\begin{itemize}
\item Ecological perspective: Trash causes ecological problems and it treats the balance of nature. Animals do not aware of plastics materials and they unconsciously eat them.
\item Management of it (handled by the municipals generally).
\item Technological perspectives. (seeing trash as a design problem. developed technologies and societies generates lots of trash, so this situation cause to inquiry that are they really developed? development in what sense?)
\end{itemize}

% Sample sentence:
% Such a conceptualization of waste as “the degree zero of value” has been contested for some time in different disciplines, ranging from economics to environmental studies, but most particularly by those studying consumerism or material culture

\quotes{We live in a badly engineered world, because the vast amounts of waste (both material and energetic) are needless; and that waste could be virtually eliminated through better design} \cite{mcdonough2010cradle}. In other word the problem is our technology which is not perfect. (and I'm not sure that at some point that technology will reach to the perfection or not.)

\quotes{As I prepared this issue of ReVista, some have asked me if Bogotá’s garbage crisis inspired the theme. Yes and no.  After Christmas, I traveled with a group of friends to the Chocó, an isolated and impoverished region on Colombia’s Pacific Coast. Christmas decorations abounded, and I noticed they were almost all crafted from used tin cans, old newspapers, discarded textiles and found wood objects. No one called it recycling. Trash was to be used and used again.}\cite{} Already a group of people live with their trash, what is problematic is here global ruling consumerist is not live with their trash. They can not handle their trash. There is no place for trash in their life.

\quotes{There’s a relationship between graveyards and landfills, one that makes us uncomfortable, Zubiaurre explained. \quotes{What is happening to trash is what is going to happen to us. We’re all going to end up in a dump, and we’re going to decompose. That’s the ultimate destiny of humankind, and we don’t want to face that.} Trash is also regarded differently, depending on where you live. Last year, an undergrad in Zubiaurre’s honors collegium seminar went to a poor neighborhood and scavenged through people’s trash; no one cared, Zubiaurre said. But when the same student went to Beverly Hills to go through trash, the police were nearly called. \quotes{Who decides what is public and what is private? How come trash becomes highly private in a rich neighborhood, but truly disposable in a poor neighborhood?} Zubiaurre said.} \cite{zubiaurre2015trash}

%%
%%
\subsection{How does it become trash?}
Here the purpose is to understand the dynamics that turn objects to trash. By understanding them is provide a roadmap (or ideas) how to turn trash to something valuable? (The purpose of this thesis is to find (or explore) a way(methodology, approach) to add value to object using artistic methods? Therefore first question is why they are less-valued, and ignored. How to make them valuable? How to make them part of our life again?)

%%
%%
\subsection{Types and Comparison of trashes}
The complexity of produced trashes of societies is increasing. For example developed countries that have nuclear plant generates radioactive wastes which highly hazardous for the environment is never exist previous societies. Think batteries and so on. Every society generates different types of wastes. Differs from country to country, society to society, ages to ages.

It can be thought that when the complexity of trashed increased required effort to repair, reuse and recycle is also increase. Therefore for the ones that have no complex tools it is becoming harder to reuse objects. In other words objects become more complex their re-usage becomes less likely. 

Different production process generates different types of trash. According to production process, decomposition process\ldots

The approach to the different type of trash will be different. In other word if trash is a result of classification of objects, it can be easily extracted that there is classification inside of it. There are some trashes that are more close to the people. More easy to convert them. more easy to regain to the society.  

%%
%%
\subsection{What is wrong with trash?}
Relationship between entropy (second law of thermodynamics) and waste. Resources of nature turns to waste that it can revert it. Creating that are reversible again is problematic through the nature of sustainability. What is produced after it is consumed become worthless. 

From my point of view and approach in this thesis, trash is only one of the thing that is being discarded by humans and communities. There are lots of things that are being excluded such as homosexuals, trans, disabled peoples etc. Even if they are excluded, there is also life for them. 

% TODO Reference
John Scanlan's book, On Garbage shows how western progress always has cleared away and discarded what went before; not only material waste but also knowledge. He believes that by examining our garbage we can gain useful insight into the condition of contemporary life.

%%
%%
\subsection{Collecting trash}
One of the most important parts of the using trash in the artwork (or expressing something, or representation) is to collect them. What are the dynamics(considerations) of collecting them? (easily accessible materials or unique items.) Where to store them? Does it mean that live with trash? In other words collecting trash and using them is live with them? (making them part of life.) After the being part of the are they still trash? Can be thought that it is something that affects the lifestyle. (possessions and trash.) Another question is that how differs collecting trash from collecting other things such as objects that have archival value. What is the driving force? You may collect it to prevent object being lost. For archival things what you collect is something that has some sort of social use and meaning which is going to disappear. However, trash is never disappearing, even its amount increasing rapidly. For archival things people have memories with them, but does some applies for the trash? Who wants to keep trash? or who wants to re-see(re-visit) trash again (in a museum for example)?

[TODO: ragpickers from benjamin and archades project.]


%%%
%%%
%%%
\subsection{Literature review, discussions, ideas\ldots}
Trash art is not collage (assemblage or found object) or fragments. it is more than that. The carried messages through the medium have different meaning. It has relationship with activism, craftivism. It refuses consumption based life cycle. It suggests a life practice.

"Every day, we put unwanted material in toilets and garbage bins, regularly flushing it away or taking it out in bags to be transported far away from our homes by others. The names we give this material---waste, garbage, refuse, trash, rubbish--- have pejorative definitions. Worthless. Rejected and useless matter of any kind. Unimportant." "Our trash is a testament; what we throw away says much about our values, our habits, and our lives." "While dictionary definitions of garbage describe it as “filth” and “worthless,” scholars are careful to note that perceptions of waste and the value of material are neither static nor universally shared." "\ldots the question of who owns these discards is not trivial." "The absence of a waste stream aroused suspicion, just as the presence of particular items tell us about the habits of the consumers who generate a waste stream. Our trash is part of us, whether or not we choose to acknowledge it." \cite{zimring2012encyclopedia}



%%
%%
\subsection{Culture, Values, and Garbage}
"The Trash Talk project emphasizes the complex, yet overlooked, relationships that garbage and people share. In terms of their relationship to garbage, all people interact with it on two levels. One is a material connection, indicative of the physical and sensory contacts that people have with garbage. In some households, this connection begins with an individual removing an item from packaging, disposing of that item in the kitchen receptacle, placing that item and others into a larger bin, taking that bin to the curbside, and then the material connection ends. Others, including workers in sanitation plants and recycling centers, then continue a material connection with the garbage, but the material connection of the consumer and the garbage ends with the bin on the curbside. The second connection that people maintain with garbage is an ideational one. Unlike the material one, which is manifested in things that can be touched, moved, and sensed, the ideational connection operates on the level of cognition. The differentiation of an item of value from an item of trash, for example, has nothing to do with the material principles of the object. Instead, humans determine whether the object is of value or whether it is considered trash. The decision of whether an individual decides to dispose of a broken radio or to consider it an heirloom to be kept is highly subjective and rooted in the value systems of a culture." "After the item is eaten, the individual has to decide what to do with the remainder, such as the leftover package. The package might be reused, re-purposed, or recycled but, most likely, will be disposed of in the trash." \cite{lukas2012culture}

%%
%%
\subsection{Garbage in Modern Thought}
"Philosophers and intellectuals have expressed the need to focus on the centrality of garbage, but for everyday individuals, the understanding of garbage is often as something “out of sight, out of mind.”" "Modern humans, as part of their penchant for consumption and unsustainable living, often think very little about the waste that they produce." "Like many aspects of capitalist living, the person throwing away a piece of trash does not connect the various levels of production, consumption, and post-consumption involved in the trash. It becomes a secondary matter---an afterthought." "Martin O’Brien, among many thinkers, argues that the understanding of garbage should be a central concept, especially since garbage typically correlates with social change, social roles, and institutions. Thus, beyond the level of individuals and their relationship to garbage, there is an interest in understanding the central role that garbage plays in all of society’s roles, institutions, and forms of change." "Garbage is excess--- it is a part of society that society no longer desires." \cite{lukas2012garbage}

%
\subsubsection{Categorization and Value}
"Garbage is categorization, according to Susan Strasser." "In recycling programs and in places of refuse disposal, items of trash are categorized depending on their potential value, possible environmental harm, or time of decay. Consumers have become accustomed to the categories that are often applied to garbage. Many cities require people to dispose of their garbage in an orderly fashion---perhaps separating wet household waste from dry---and recycling programs ask individuals to divide their recyclable items into sets (such as plastic, glass, aluminum, and paper) and smaller subsets (such as PET or 01, PE-HD or 02, and PVC or 03). Garbage is an illustration of how humans use mental categories to order the material world." \cite{lukas2012garbage}

"According to John Scanlon, garbage is indicative of a separation of the world---the desirable from the unwanted. Michael Thompson uses the riddle of the rich and poor person’s approach to snot (one keeps his in a handkerchief, the other disposes of it with a tissue) to underscore the curious ways in which garbage is connected to the issue of value. While garbage is universal---all societies, extinct and extant, have produced or produce garbage--- the conditions under which garbage is understood are culturally determined. Many non-Western societies attach a much greater value to items after they are discarded. In the United States and many other nations, garbage often results not because something no longer has utilitarian value but because the item in question is defined as something of no value. Thus, garbage is not only an objective condition of material culture, but also a subjective one of mentalist culture. People define what is trash and what is valuable." \cite{lukas2012garbage}

%
\subsubsection{Semiotic Context}
"In popular writing (such as novels), in television, films, music, and other forms of mass expression, the term trash is used to signify work that is of especially low value." \cite{lukas2012garbage}

%%
%%
\subsection{Garbology}
Garbology is a study of waste as a social science.

"Weberman infamously used techniques of what he deemed garbology to uncover what he saw as the essential nature of people. He once said, perhaps indirectly referencing Jean Brillat-Savarin’s quote about food, “You are what you throw away.”" \cite{lukas2012garbage}

"The field of garbology involves the study of refuse and waste. It enables researchers to document information on the nature and changing patterns of modern refuse, hence assisting in the study of contemporary human society or culture. According to the Oxford English Dictionary, the term was first used by waste collectors in the 1960s. A. J. Weberman popularized the term in describing his study of Bob Dylan’s garbage in 1970. It was pioneered as an academic discipline by William Rathje at the University of Arizona in 1973."

In his book “Garbology: Our Dirty Love Affair With Trash”, the Pulitzer prize-winning author Edward Humes notes that other wealthy countries with high living standards have rejected the disposable products that make up much of America's rubbish.

% Rubbish: The Archeology of Garbage, p.24
As noted, the Garbage Project has now been sorting and evaluating garbage, with scientific rigor, for two decades. THe Project has proved durable because its findings have supplied a fresh perspective on what we know---and what we think we know---about certain aspects of our life. (example of Medical researches)

%%
%%
\subsection{Trash as History/Memory}
% TODO From encylopedia
\cite{bullock2012trash}

%%
%%
\subsection{Trash Aesthetics}
%Walter Benjamin's trash aesthetics and Adornos reflection. 
\quotes{Benjamin’s approach to history is through \singlequotes{trash}---through the spent and discarded materials that crowd the everyday}  \cite{highmore2002thrashaesthetics}. Benjamin’s importance as a theorist of the everyday is most evident in his attention to the everyday experiences of modernity. In the face of the endless proliferation of trash, Benjamin potentially suggests a \singlequotes{trash aesthetics} that could be used radically and critically to attend to the everyday. The method might be thought of in terms of \singlequotes{recycling} --- an ecology of everyday experience.


%**************************************
% HERE Same sample phrases are listed you can use them:

% The work that follows is divided into three sections

% The artist thinks, acts, performs music, and writes outside the framework that society has created.

% this thesis takes a rather different approach to the resonant possibilities of discarded things. It looks to philosophical ideas and our entangled experiences of things, time and stories, which need to be traversed in order for a discarded object to be called ‘waste’.

% I also want to suggest a different way of considering trash. Maybe art maybe suggest an alternative way of seeing.

% I’d like to criticise a set of concepts or ways of thinking about discarded things that to me just don’t seem quite adequate.

% In an effort to expand art activism's capacity to create real social change, this article will (1) examine the theoretical framework behind art activism and art's efficacy in accessing emotional pathways; (2) explicate ways to strategically approach art activism through the use of specific case studies; and (3) explain one practical form of art activism-theater-based conflict resolution-that is transforming the ways communities are addressing social injustice.

% This thesis is written as a study of the socio-economic change that is currently happening in Serbia, but it’s a study that critically engages with everyday materials that provide the basis for change, rather than the economic development philosophies that are practiced through policy. 

%*******************************************************************
% FROM The Ruin and the Ruined in the Work of Kurt Schwitters.
% The German avant-garde was working from ruins literally and metaphorically, and trash was both practically and freely available; to use it was an action that took the ruins of our society, its discarded, to question how meaning is constructed.
% Marx wrote that it was not the materiality of the object but the social relations that create value, the use of urban detritus in particular, the squalid results of mass-produced human relations, infuses the materiality of Schwitters’ work with an anthropological quality

\subsection{Discussions}
Objects moves around and their values change constantly. (The idea that objects lead social lives was elaborated and discussed in detail in Arjun Appadurai (ed.). The Social Life of Things: Commodities in Cultural Perspective)

Igor Kopytoff, a professor of anthropology, introduced the notion of commoditization “as a process of becoming rather than as an all-or-none state of being.” As such, Kopytoff wrote, the biography of an object was considerably similar to that of a person: occupying different positions, leading diverse careers in the course of different periods between a beginning and an end, being defined by different regimes of value that are both economically and culturally inscribed. (Igor Kopytoff, “The Cultural Biography of Things: Commoditization as Process)

% FROM Trashion: The Return of the Disposed by Bahar Emgin
In light of this argument, one could claim that the end of the life of an object corresponds to the moment in which it is disposed of. This disposal might take place in different forms and for different reasons; however, in the most literal and common sense, the life of an object ends in a trashcan in the form of waste. In this moment, the object is left valueless in all the possible meanings of the term value: It can no more serve a function, it can on no account be exchanged for anything else, and it can by no means engage in the processes of signification to connote and endow its user with specific social values.

Referring to the work of Susan Strasser, Hawkins argues that disposal was central to the logic of mass production and hence should not be assessed as only particular to consumerism in the twentieth century: “Mass production of objects and their consumption depends on widespread acceptance of, even pleasure in, exchangeability; replacing the old, the broken, the out of fashion with the new. The capacity for serial replacement is also the capacity to throw away without concern.”

% Prap. same source...
% On the contrary, with respect to the issue of disposability, waste was handled merely “as a technical problem, something to be administered by the most efficient and rational technologies of removal.” 9 Only through the rise of environmental movements in the 1960s did the disposal of waste come to be loaded with negative meanings and iewed through a moral framework. The enormous quantities of waste accumulating in urban centers, Hawkins writes in “Plastic Bags,” were not only taken as a threat to the environment, but also as a sign of an individualistic, insensitive, and hedonistic consumer society. 10 Waste now became evil. If the environment is to be saved from our destructive power, then waste should be “managed,” Hawkins asserts. 11 Consequently, recycling gained its contemporary prominence “as virtue-added disposal\ldots disposal in which the self is morally purified, disposal as an act of redemption.” 12 Disposal in the form of recycling is now a moralistic attitude through which we pay the debt we owe to the world. Upcycling... On the other side of the coin is the business stemming from these practices; recyclers not only ease their conscience through recycling; they also make a profit. Recycling, as “the huge tertiary sector devoted to getting rid of things, is central to the maintenance of capitalism; it doesn’t just allow economies to function by removing excess and waste—it is an economy, realizing commercial value in what’s discarded,” Hawkins and Muecke write in Culture and Waste. 16 In the same manner, upcycling has already been turned into a business: Certain designers labeled eco-friendly are earning money through upcycling, competitions are organized around trashion, numerous websites are devoted to promoting and selling upcycled objects, and online and print resources explain how to upcycle at home. In short, there is a whole sector of upcycling now.

% Design, as a conduit of disposal, reintroduces rubbish as objects of distinction, invaluable and potentially priceless. People are often eager to see objects that were once considered useless and tasteless when they have been invigorated with new life.

% There is commodity aspect of it and also the process of accommodation. In this process design plays a significant role. Trash is waiting to be discovered. At the same time forgotten styles are also used in works. Therefore actually trash and forgotten styles can be considered in the same status.

% This can be sample thesis statement sentence:
% This article is about those objects that are recreated from trash through the process of upcycling. Upcycling is a term used by architect and designer William McDonaugh and chemist Michael Braungart and refers to “the process of converting an industrial nutrient (material) into something of similar or greater value, in its second life.” 4 I argue that design, in this instance, acts as a tool of transformation and reintroduces into certain orders what was once deemed waste. This theory counters the argument that an object is dead once it is disposed of.

% From The Ruin and the Ruined in the Work of Kurt Schwitters by Gemma Carroll
% Karl Marx had already written that economic value does not inhere in the materiality of the object but emerges from the social relations and organisation of labour which produces it, and that the separation between consumer market and the sphere of culture had become indiscernible.

% Schwitters is able to use the ruined, the waste products, as an anthropological exploration of society from both its unpleasant outcomes and its decay. \ldots trash was both practically and freely available; to use it was an action that took the ruins of our society, its discarded, to question how meaning is constructed. As he wrote: ‘It grows more or less according to the principle of a metropolis.’ 13 The Merzbau was itself a city; and just as Marx wrote that it was not the materiality of the object but the social relations that create value, the use of urban detritus in particular, the squalid results of mass-produced human relations, infuses the materiality of Schwitters’ work with an anthropological quality. Material has transformed into information, and ‘how’ has surpassed ‘what’ we see. The grottos in the Merzbau that still reveal this detritus most clearly could not be re-created in Bissegger’s reconstruction because, arguably, they are an exploration of absence, an exploration of ruin. As Schwitters wrote: ‘One can even shout out through refuse.’ 14 His words still echo.

% FROM Waste by William Viney
Douglas argues that our classification of dirt lies not with what objects are but where those objects are. (Think that the transformation process. Previous argument support that the transformation of trash is possible by changing the place of them. In other words removing them from landfill and waste bins to the book selves accomplish to transform trash.) ‘Dirt’, writes Douglas, ‘is the by-product of a systematic ordering and classification of matter, in so far as ordering involves rejecting inappropriate elements’. For Douglas dirt is a spatial problem, a question of not what stuff is but where it is. (from William Viney)

% “Dirt”, writes Douglas, “is the by-product of a systematic ordering and classification of matter, in so far as ordering involves rejecting inappropriate elements.” Dirt is only dirty in certain places, when it is out of its correct position. Just as faeces, for example, is considered dirty when it is in our kitchens but not when it is in our bodies, so it is that our classification of waste depends on the location of objects. 

%%% My experience here: For example when i collect the papers under the dishes or food packages, people often thinks that it is disgusting and call them as dirty. But it is very strange that the fat on the paper is previously what they are eat. 

% objects we call ‘waste’ have peculiar powers to make that temporality an explicit part of what they are and how we judge them.

% Here susan strasser ile douglasın fikirlerinde bir ortak nokta görülebilmekte. Özellikle trashın relative olması ve bu işin aslında bir ordering and classification olduğu konusunda. Remember the example given: shoes  on the dinner table. 

% King Lear, feeling of waste

% waste is “matter is out place”, a definition first given by Lord Palmerston in the mid-nineteenth century and incubated by the British anthropologist Mary Douglas, in her book Purity and Danger.

% I prefer to use the word waste to describe the things that have, for whatever reason, been leftover from use or for which use has been precluded.

% https://narratingwaste.wordpress.com/tag/king-lear/
% Not all waste is dirty, it not always dangerous, contagious or abject. \ldots waste might be quite useful in making time and in keeping time.

% Binding things together and separating things from each other. 

% Cornelia Parker's work, narrate an absent event of waste
% There is also strong relationship with the collecting discarded stuff and archeologies of waste.  
%%% Reference
% we live in a throwaway society (Barr, 2004; Cooper, 2003, 2005; Cooper and Mayers, 2000; Strasser, 1999, cf. O’Brien, 1999; Hawkins and Muecke, 2003); 
% In the two previous sections we have demonstrated the paucity of the thesis of the throwaway society. In this thesis the undeniable matter of waste, itself pressing, urgent and excessive, is used to infer the presence of a society defined by its generation; a society ceaselessly discarding and abandoning its surplus as excess, as part of an endless desire for the new. Morally corrupt and unequivocally environmentally damaging, the rhetoric of the throwaway society classifies discarding as intrinsically bad and commands us to assume control of our wasting, suggesting the adoption of heightened regulatory practices around disposal as the means to ensure that we clean-up our act. The thesis, however, lacks depth and provenance. It is, actually, glib. Indeed, to infer the presence of a throwaway society from contemporary levels of waste generation is problematic for at least four reasons.

%%%
%%%
%%%
\section{In Art}


%%
%% TODO sample statements here.
\subsection{Artist statements}
\begin{itemize}
\item I am an artist living in New York City, and my experience of daily life here is both the catalyst for and the subject of my art. Working from the local, the personal and the ordinary - the paper take-out coffee cup in the palm of my hand, the view from my studio window in the Garment District, friends and their children, the community garden around the corner from my apartment in Hell's Kitchen - my drawings, paintings and installations are about what happens when the familiar suddenly undergoes a perspective shift and is revealed in all its wonder and infinite possibility. With this shift, mundane things become extraordinary, as nodes of rapidly expanding sets of connections, relationships and new artworks. 

My approach to art-making is both observational and process-oriented. A fascination with cultural, physical and temporal change combined with the possibility of infinite variation unify projects as diverse as drawing and painting on my used coffee cups each day, repeatedly recording the view from my studio in paint and photography over the course of six years, depicting in two and three dimensions the incursion of human activity on the fractal meanders of salt marshes along the New Jersey coast, and painting portraits of alternative families in my New York City neighborhood. (from http://gwynethleech.com/)
\end{itemize}


%%
%% ACTIVISM as a methodology 
%% TODO reading...
\subsection{Art and activism}
Using trash in the art has some critical messages to societies and authorities. Especially in the trash case there is a activist(political) act. 

%*************************************
% Phrases...
% What specific social issues are you trying to address through Labyrinth?
% In view of the diversity of online crowdsourced art projects, as illustrated by the examples cited so far, it is useful to map out this artistic trend by developing a comprehensive and multidimensional typology of online crowdsourced art. Table 1 organizes this classification according to a set of multiple criteria. (But this one can be used at introduction, but it is too long to fit on introduction. Maybe select works by giving prominence to some of the features. So the approach to the trash is going to be introduced and also it is reveals the what i am doing in this context.) (A Typology of Online Crowdsourced Art. Diye bir örnek var aslında benzer bir şekilde bu trash içinde yapılabilir.) 
\subsection{Crowdsourcing, Participation}
% TODO needs reference, FROM Crowdsourced Art and Collective Creativity
In the words of Jeff Howe (2006b), the Wired columnist who coined this term in June 2006, \quotes{crowdsourcing represents the act of a company or institution taking a function once performed by employees and outsourcing it to an undefined (and generally large) network of people in the form of an open call}. The vital elements that qualify an outreach strategy as crowdsourcing are, according to Howe, the use of the open call format and the reliance on a large network of potential workers. Although in some cases there is a material reward for the best contributions, the existence of financial incentives is not a required feature in crowdsourcing. Because of the diversity of its applications, crowdsourcing continues to be a disputed term in both the scholarly literature and the popular press; Howe’s original definition is, in this sense, a helpful delineation of its practical sphere.

The value of crowdsourcing lies in the collective intelligence of the contributors. Pierre Levy (1997) describes this concept as “a form of universally distributed intelligence, constantly enhanced, coordinated in real time, and resulting in the effective mobilization of skills” (p. 13). The question of collective intelligence—and its potential efficiency in various practical settings—has received much attention in both academia and journalism. Researchers studying team performance generally agree that, under the right circumstances and with appropriate motivation, large groups of people can work together and harness their collective intelligence to achieve efficient results (Benkler, 2006; Rheingold, 2002; Surowiecki, 2004). Nevertheless, artistic creativity is different from innovation and intelligence, and it requires a unique set of skills and sensibilities as well as a particular type of cultural capital; if we admit that crowds can have collective intelligence, do they also have collective creativity in an artistic sense?

All these pages are rescued and with their [hi]story they are separated from unused new produced blank sheets and notebooks. They are not bought, not gift. They are found. They are accepted. One of the most creative medium is paper and pencil. Chance given to the people through this work.

However, in view of its reliance on the artistic contribution of a large pool of usually anonymous participants, this type of art raises important questions about notions of collective creativity, authorship, collaboration, and the shifting structure of artistic production in the new digital environment. It is well studied area. Pick a method here. Transforming trash with collaboration.

As curator Andrea Grover notes, “having the audience become co-creators is not a new impulse”; the Internet simply offered a new platform to accomplish this goal (Strickland, 2011, para. 5).

Here this question can be raised about why this type collaboration. Another option can be working together with people on a table. Creating things at that time and transforming the objects here. Possibly the connection to the people will more realistic and intense. However same things can be succeed on the internet at some level. 

While crowdsourced art challenges the traditional role of the artist, it simultaneously redefines the conventional function of the public, turning them from passive receivers into engaged producers. (This totally a new area to discuss, I'm going to summarize and introduce concepts and debates. How are they support my work and how they are give way to me?)

% This article therefore aims to fill these critical gaps by analyzing the practice of online crowdsourced art within a framework of collective creativity and participation theories. Principally, my interest is in answering two key questions. What is crowdsourced art and how can it be classified? And how does the structure of the artwork determine the degree or significance of participation?

% why collaboration, is it really a collaboration?

% TODO 
% Umberto Eco's Open Work

% Not all artist transform trash although some deconstruct them. Michael Landy is one of them. 
% Michael Landy's Break Down Inventory is a two week show / display of destruction process of his all possessions on a dissemble line with the help of 10 workers. Firstly they are classified and recorded for three years and the deconstructed in two weeks by separating every element to the smallest part. Reveal all his possessions. and loosing them while you are alive. Turning them to rubbish making them unusable. breaking down the all the meaning. breaking down the connections. 
% Michael Landy's Art Bin uses a art gallery to create his dust bin. Describing the work, simply called Art Bin, as 'about failure', Landy is inviting members of the public to bring their own artistic failures along to the gallery from 29 January, where their worthlessness will be assessed. Damien Hirst, Gillian Wearing, Tracey Emin and Mark Titchner have already contributed, offering sculptures, paintings and prints. "There's no hierarchy once they are in the bin" All of them are accepted as same. Ultimate equality. Tosing them to the dust bin makes them rubbish even if they are combined from failed attempts. Nothing's too good for the art bin. Everything can fit into the art bin. Michael Landy transforms the SLG into Art Bin, a container for the disposal of works of art. As people discard their art works the enormous 600m³ bin becomes, in Michael Landy’s words, “a monument to creative failure”. It is a very big glass bin and isolates people. There is a provocative approach to the art and art objects by saying that "Nothing's too good for the art bin". There is no limitation of throwing away even if art. 

% In this thesis project case the issue is not the language of art and its methods.


%*************************************
% Sample Artist Statement:
% ARTIST STATEMENT

% I'm the type of person who is always drawing. No matter where I am or what the circumstances, drawing and the movement of my hand help to keep me focused in the present. My fascination with paper coffee cups began when I found myself drawing on them during meetings. I soon realized how suitable the surfaces of these cups were for a variety of painting and drawing materials. In addition, the idea of recycling an inexhaustible supply appealed to me, as did the fact that I could work in a full-circle format, allowing me to create continuous compositions with essentially no beginnings and no ends. At that point, I began to save all of my coffee cups to use as "canvases" for making art.

% The subject matter of these cups is often inspired by specific locations, particularly patterns of light, shadow and ever-changing weather. They can also be spontaneous, intricate iterations of designs, motifs and organic forms. Though each cup is a finished artwork in itself, I assemble them into arrangements, either suspended in strands, aligned on surfaces in series, or displayed in unity with a painting. Regardless of the presentation, I am able to continually rearrange the individual elements to reveal new relationships. Like kaleidoscopes, the compositional possibilities are limitless, in a sense, depicting multi-faceted realtities where nothing ever stays the same for long.









