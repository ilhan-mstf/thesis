\chapter{Uncategorized}


%****************************************
% RESOURCES:
%
% Plagiarism check: http://mashable.com/2012/08/29/plagiarism-online-services/
% What is Plagiarism: http://www.plagiarism.org/plagiarism-101/what-is-plagiarism
%
% How to cite epigraph: http://blog.apastyle.org/apastyle/2013/10/how-to-format-an-epigraph.html
%
% Artist ve artist statementlar için kaynak:
% - http://www.amandadonahue.com/artist-statement/
% - http://www.smfa.edu/cyclo-show
% - http://www.gwynethleech.com/statement
%........................................


%**************************************
% HERE Same sample phrases are listed you can use them:

% The work that follows is divided into three sections

% The artist thinks, acts, performs music, and writes outside the framework that society has created.

% This thesis takes a rather different approach to the resonant possibilities of discarded things. It looks to philosophical ideas and our entangled experiences of things, time and stories, which need to be traversed in order for a discarded object to be called ‘waste’.

% I also want to suggest a different way of considering trash. Maybe art maybe suggest an alternative way of seeing.

% I’d like to criticize a set of concepts or ways of thinking about discarded things that to me just don’t seem quite adequate.

% In an effort to expand art activism's capacity to create real social change, this article will (1) examine the theoretical framework behind art activism and art's efficacy in accessing emotional pathways; (2) explicate ways to strategically approach art activism through the use of specific case studies; and (3) explain one practical form of art activism-theater-based conflict resolution-that is transforming the ways communities are addressing social injustice.

% This thesis is written as a study of the socio-economic change that is currently happening in Serbia, but it’s a study that critically engages with everyday materials that provide the basis for change, rather than the economic development philosophies that are practiced through policy. 
%........................................


%%%
%%%
%%%
\section{In Theory}

% TODO PRAP. REF.
% FROM Recycle the Classics, BY DORIS SOMMER
% Her alanda recycling var olabilir, literatürde de vardır?
\todo{WHAT, WHERE?} Literature is recycled material, a pretext for making more art. I learned this distillation of lots of literary criticism in workshops with children. I also learned that creative and critical thinking are practically the same faculty, since both take a distance from found material and turn it into stuff for interpretation. For a teacher of literature over a long lifetime, these are embarrassingly basic lessons to be learning so late, but I report them here for anyone who wants to save time and stress.

\comment{MOVES} \paraphrase{Igor Kopytoff, a professor of anthropology, introduced the notion of commoditization “as a process of becoming rather than as an all-or-none state of being.” As such, Kopytoff wrote, the biography of an object was considerably similar to that of a person: occupying different positions, leading diverse careers in the course of different periods between a beginning and an end, being defined by different regimes of value that are both economically and culturally inscribed. (Igor Kopytoff, “The Cultural Biography of Things: Commoditization as Process)} \todo{reference}

\comment{MOVES} \paraphrase{Objects moves around and their values change constantly. (The idea that objects lead social lives was elaborated and discussed in detail in Arjun Appadurai (ed.). The Social Life of Things: Commodities in Cultural Perspective)}\todo{reference, resource?}

% Like a summary sentence
\comment{TRASH MOVES.} Here the important thing is moving nature of trash. Objects moves and gains different meanings through this movement. Sometimes it becomes artwork, sometimes it becomes archaeological part, waits in the dump-site or wait to be recycle. It also signifies that it is relative and result of a classification issue. All of them are creates a harmony. supports each other with different words and conceptualization. 

% <From "Trash Moves On Landfills, Urban Litter and Art" by Maite Zubiaurre
\comment{TRASH MOVES.} \paraphrase{Below part Explains journey of trash, its different steps. Object moves and also trash also moves. How is it life of trash? This part explains life of trash. How does it intersect with other people in which places? This part also can be understand by pointing out different part of it with detailed explanation. For example an artist collect from trash from streets and the other one goes to the (For example Vik Muniz) landfill. In other words there are different places to touch on trash. Every place generates different story? Or to understand it more deeply it covers different part of it. Or provides ideas about it. Lots of people touches it from philosophers to artist. Also in the later parts it draws attention to them.}

% Ama çöp olarak zaten kalmıyor bu trash bir şekilde hayatımıza tekrar giriyor. Ne kadar itise de... Trashion: the return of disposed. trash moves. Peki geri dönüşü nasıl bir değişiklik yaratıyor. Ya da gerçekten geri geliyor mu? Kullandığımız ürünlerin ne kadarı tek kullanımlık. Tek kullanımlıklarla nerede karşılaşıyoruz. Evet belki hayatımızda çok kullanılan şeyler var fakat. daha tek kullanımlıklar da var denilebilir.
\comment{TRASH MOVES.} \paraphrase{Trash moves, all the time. It becomes a steadily growing heap of clutter behind closed walls, accumulates and festers under tight lids, travels from a small trash can in the kitchen to a large one on the curbside, joins other people’s rubbish when the garbage truck arrives, drives to the transfer station, where it circles around on conveyer belts, bids farewell to recyclable or composable goods, is loaded (if declared useless: the ultimate trash) into yet another garbage truck, or barge, or even train, until it arrives at its final destination: a sanitary landfill. Even in the landfill, it does not remain still. Monster "waste handling dozers" move rubbish around, compact it and press it against the soil. More importantly, they incessantly “sculpt” refuse with their huge shovels and caterpillar wheels, making sure the garbage mound does not tip over to create a fetid avalanche. When night falls, and the trash load of the day finally disappears under a thick layer of mud, detritus still moves: once underground, it settles differently, and decomposes at a different speed, thus continuously altering landfill topography: where there was an even plateau, now there is an abruptly descending slope, and a valley; and where there was a perfectly smooth road, now there are deep crevices in the pavement. This is how trash moves. But\ldots who moves on trash? In the United States, it is mostly big-wheeled machines, an industrious army of giant yellow insects busying themselves on a heap of rubbish. In Latin America, it is mostly people. People who hand-pick garbage, who build their shacks on densely compacted trash layers, and who, day in and day out, eagerly throw themselves into the boisterous cascades of fresh debris falling from garbage trucks. In many of the garbage dumps around the world, scavenging becomes a steady job. \quotes{Garbage properly \quotes{stored} and put away brings peace of mind, as do corpses boxed and buried, or criminals confined to a cell.} And thanks to Art: for Art shows how trash--even the one that stops moving, and particularly the one that lies squished, squashed, and weathered, almost fossilized, on the ground---has the potential to move: to move us, that is. (through the works of Filomena Cruz's photographic series “Road Kill”)}
% >

\todo{WASTE AND TECHNOLOGY RELATIONSHIP} \quotes{We live in a badly engineered world, because the vast amounts of waste (both material and energetic) are needless; and that waste could be virtually eliminated through better design} \cite{mcdonough2010cradle}. In other word the problem is our technology which is not perfect. (and I'm not sure that at some point that technology will reach to the perfection or not.)

% WHO OWNS TRASH? kısmı benim işimde sorgulanabilir, ya da ünlülerin trashlerini inceleyen adam kısmında ele alınabilir. Orada aslında başkalarının trashlerinin insanlar hakkında neler anlattığını çok güzel anlıyoruz.
\todo{ANALOGY Between landfill and graveyards. WHOSE TRASH?} \quotes{There’s a relationship between graveyards and landfills, one that makes us uncomfortable, Zubiaurre explained. \quotes{What is happening to trash is what is going to happen to us. We’re all going to end up in a dump, and we’re going to decompose. That’s the ultimate destiny of humankind, and we don’t want to face that.} Trash is also regarded differently, depending on where you live. Last year, an undergrad in Zubiaurre’s honors collegium seminar went to a poor neighborhood and scavenged through people’s trash; no one cared, Zubiaurre said. But when the same student went to Beverly Hills to go through trash, the police were nearly called. \quotes{Who decides what is public and what is private? How come trash becomes highly private in a rich neighborhood, but truly disposable in a poor neighborhood?} Zubiaurre said.} \cite{zubiaurre2015trash}

\todo{TRASH AND OTHER DISCARDED NOTIONS} From my point of view and approach in this thesis, trash is only one of the thing that is being discarded by humans and communities. There are lots of things that are being excluded such as homosexuals, trans, disabled peoples etc. Even if they are excluded, there is also life for them. 

% TODO Reference
\todo{WHAT?} John Scanlan's book, On Garbage shows how western progress always has cleared away and discarded what went before; not only material waste but also knowledge. He believes that by examining our garbage we can gain useful insight into the condition of contemporary life.

% FROM Culture, Values, and Garbage, Encyclopedia 
\comment{INTERACTION WITH GARBAGE. VALUE SYSTEM, DECIDING TO GARBAGE.} \paraphrase{"The Trash Talk project emphasizes the complex, yet overlooked, relationships that garbage and people share. In terms of their relationship to garbage, all people interact with it on two levels. One is a material connection, indicative of the physical and sensory contacts that people have with garbage. In some households, this connection begins with an individual removing an item from packaging, disposing of that item in the kitchen receptacle, placing that item and others into a larger bin, taking that bin to the curbside, and then the material connection ends. Others, including workers in sanitation plants and recycling centers, then continue a material connection with the garbage, but the material connection of the consumer and the garbage ends with the bin on the curbside. The second connection that people maintain with garbage is an ideational one. Unlike the material one, which is manifested in things that can be touched, moved, and sensed, the ideational connection operates on the level of cognition. The differentiation of an item of value from an item of trash, for example, has nothing to do with the material principles of the object. Instead, humans determine whether the object is of value or whether it is considered trash. The decision of whether an individual decides to dispose of a broken radio or to consider it an heirloom to be kept is highly subjective and rooted in the value systems of a culture." "After the item is eaten, the individual has to decide what to do with the remainder, such as the leftover package. The package might be reused, re-purposed, or recycled but, most likely, will be disposed of in the trash." \cite{lukas2012culture}}

% FROM Garbage in Modern Thought, Encyclopedia
\comment{THROW AWAY CULTURE.} \paraphrase{"Philosophers and intellectuals have expressed the need to focus on the centrality of garbage, but for everyday individuals, the understanding of garbage is often as something “out of sight, out of mind.”" "Modern humans, as part of their penchant for consumption and unsustainable living, often think very little about the waste that they produce." "Like many aspects of capitalist living, the person throwing away a piece of trash does not connect the various levels of production, consumption, and post-consumption involved in the trash. It becomes a secondary matter---an afterthought." "Martin O’Brien, among many thinkers, argues that the understanding of garbage should be a central concept, especially since garbage typically correlates with social change, social roles, and institutions. Thus, beyond the level of individuals and their relationship to garbage, there is an interest in understanding the central role that garbage plays in all of society’s roles, institutions, and forms of change." "Garbage is excess--- it is a part of society that society no longer desires." \cite{lukas2012garbage}}

% FROM Garbage in Modern Thought, Encyclopedia
\comment{CATEGORIZATION.} \paraphrase{"Garbage is categorization, according to Susan Strasser." "In recycling programs and in places of refuse disposal, items of trash are categorized depending on their potential value, possible environmental harm, or time of decay. Consumers have become accustomed to the categories that are often applied to garbage. Many cities require people to dispose of their garbage in an orderly fashion---perhaps separating wet household waste from dry---and recycling programs ask individuals to divide their recyclable items into sets (such as plastic, glass, aluminum, and paper) and smaller subsets (such as PET or 01, PE-HD or 02, and PVC or 03). Garbage is an illustration of how humans use mental categories to order the material world." \cite{lukas2012garbage}}

% Garbage is universal, reference
\comment{NO VALUE, UNIVERSAL} \paraphrase{"According to John Scanlon, garbage is indicative of a separation of the world---the desirable from the unwanted. Michael Thompson uses the riddle of the rich and poor person’s approach to snot (one keeps his in a handkerchief, the other disposes of it with a tissue) to underscore the curious ways in which garbage is connected to the issue of value. While garbage is universal---all societies, extinct and extant, have produced or produce garbage--- the conditions under which garbage is understood are culturally determined. Many non-Western societies attach a much greater value to items after they are discarded. In the United States and many other nations, garbage often results not because something no longer has utilitarian value but because the item in question is defined as something of no value. Thus, garbage is not only an objective condition of material culture, but also a subjective one of mentalist culture. People define what is trash and what is valuable." \cite{lukas2012garbage}}

% Hepsinin ayrı bir manası olması falan.
\comment{Semiotic Context, juxtapositions} \paraphrase{"In popular writing (such as novels), in television, films, music, and other forms of mass expression, the term trash is used to signify work that is of especially low value." \cite{lukas2012garbage}}

% Related with tracy emin, photos of every cast off/discard/reject, understanding human through their discard.
\comment{GARBOLOGY} Garbology is a study of waste as a social science. Applying methodologies of archeology to the human debris. \todo{reference rubbish archeology of garbage} \paraphrase{"Weberman infamously used techniques of what he deemed garbology to uncover what he saw as the essential nature of people. He once said, perhaps indirectly referencing Jean Brillat-Savarin’s quote about food, “You are what you throw away.”" \cite{lukas2012garbage}}

% Garbology, from Encyclopedia, Abhijit Roy
\paraphrase{"The field of garbology involves the study of refuse and waste. It enables researchers to document information on the nature and changing patterns of modern refuse, hence assisting in the study of contemporary human society or culture. According to the Oxford English Dictionary, the term was first used by waste collectors in the 1960s. A. J. Weberman popularized the term in describing his study of Bob Dylan’s garbage in 1970. It was pioneered as an academic discipline by William Rathje at the University of Arizona in 1973."} \todo{reference}

% Rubbish: The Archeology of Garbage, p.24
\paraphrase{As noted, the Garbage Project has now been sorting and evaluating garbage, with scientific rigor, for two decades. The Project has proved durable because its findings have supplied a fresh perspective on what we know---and what we think we know---about certain aspects of our life. (example of Medical researches)} \todo{reference}

\todo{read resource (Trash as History/Memory) \cite{bullock2012trash}}

% Walter Benjamin's trash aesthetics and Adornos reflection? Bu konudan ben hiç bir şey anlamadım açıkcası. Çünkü ne dedikleri ne yaptıkları hiç belli değil.
\todo{BU NE OLACAK? NEREYE BAGLANACAK, GARBOLOGY??}\paraphrase{\quotes{Benjamin’s approach to history is through \singlequotes{trash}---through the spent and discarded materials that crowd the everyday}  \cite{highmore2002thrashaesthetics}. Benjamin’s importance as a theorist of the everyday is most evident in his attention to the everyday experiences of modernity. In the face of the endless proliferation of trash, Benjamin potentially suggests a \singlequotes{trash aesthetics} that could be used radically and critically to attend to the everyday. The method might be thought of in terms of \singlequotes{recycling} --- an ecology of everyday experience.}

% FROM Trashion: The Return of the Disposed by Bahar Emgin
\comment{LIFE AN OBJECT} \paraphrase{In light of this argument, one could claim that the end of the life of an object corresponds to the moment in which it is disposed of. This disposal might take place in different forms and for different reasons; however, in the most literal and common sense, the life of an object ends in a trash can in the form of waste. In this moment, the object is left valueless in all the possible meanings of the term value: It can no more serve a function, it can on no account be exchanged for anything else, and it can by no means engage in the processes of signification to connote and endow its user with specific social values.} \todo{reference, (Trashion: The Return of the Disposed by Bahar Emgin)}

% FROM Trashion: The Return of the Disposed by Bahar Emgin
\comment{REFLOW IN THE MARKET} \paraphrase{Referring to the work of Susan Strasser, Hawkins argues that disposal was central to the logic of mass production and hence should not be assessed as only particular to consumerism in the twentieth century: “Mass production of objects and their consumption depends on widespread acceptance of, even pleasure in, exchangeability; replacing the old, the broken, the out of fashion with the new. The capacity for serial replacement is also the capacity to throw away without concern.”} \todo{reference, (Trashion: The Return of the Disposed by Bahar Emgin)}

\comment{TECHNICAL PROBLEM, NEED TO MANAGED, RECYCLING (CAPITALISM), UPCYCLING} \paraphrase{On the contrary, with respect to the issue of disposability, waste was handled merely “as a technical problem, something to be administered by the most efficient and rational technologies of removal.” Only through the rise of environmental movements in the 1960s did the disposal of waste come to be loaded with negative meanings and iewed through a moral framework. The enormous quantities of waste accumulating in urban centers, Hawkins writes in “Plastic Bags,” were not only taken as a threat to the environment, but also as a sign of an individualistic, insensitive, and hedonistic consumer society. Waste now became evil. If the environment is to be saved from our destructive power, then waste should be “managed,” Hawkins asserts. Consequently, recycling gained its contemporary prominence “as virtue-added disposal\ldots disposal in which the self is morally purified, disposal as an act of redemption.” Disposal in the form of recycling is now a moralistic attitude through which we pay the debt we owe to the world. Upcycling... On the other side of the coin is the business stemming from these practices; recyclers not only ease their conscience through recycling; they also make a profit. Recycling, as “the huge tertiary sector devoted to getting rid of things, is central to the maintenance of capitalism; it doesn’t just allow economies to function by removing excess and waste—it is an economy, realizing commercial value in what’s discarded,” Hawkins and Muecke write in Culture and Waste. In the same manner, upcycling has already been turned into a business: Certain designers labeled eco-friendly are earning money through upcycling, competitions are organized around trashion, numerous websites are devoted to promoting and selling upcycled objects, and online and print resources explain how to upcycle at home. In short, there is a whole sector of upcycling now.} \todo{reference, (Trashion: The Return of the Disposed by Bahar Emgin)}

\comment{DESIGN, ALTERNATIVE DICIPLINE.} \paraphrase{Design, as a conduit of disposal, reintroduces rubbish as objects of distinction, invaluable and potentially priceless. People are often eager to see objects that were once considered useless and tasteless when they have been invigorated with new life.} \todo{reference, (Trashion: The Return of the Disposed by Bahar Emgin)}

% Trash is waiting to be discovered. At the same time forgotten styles are also used in works. Therefore actually trash and forgotten styles can be considered in the same status.

\comment{UPCYCLING, sample thesis statement sentence} \paraphrase{This article is about those objects that are recreated from trash through the process of upcycling. Upcycling is a term used by architect and designer William McDonaugh and chemist Michael Braungart and refers to “the process of converting an industrial nutrient (material) into something of similar or greater value, in its second life.” I argue that design, in this instance, acts as a tool of transformation and reintroduces into certain orders what was once deemed waste. This theory counters the argument that an object is dead once it is disposed of.} \todo{Reference, (Trashion: The Return of the Disposed by Bahar Emgin).}

\comment{CATEGORIZATION OF WASTE} \paraphrase{Douglas argues that our classification of dirt lies not with what objects are but where those objects are. (Think that the transformation process. Previous argument support that the transformation of trash is possible by changing the place of them. In other words removing them from landfill and waste bins to the book selves accomplish to transform trash.) ‘Dirt’, writes Douglas, ‘is the by-product of a systematic ordering and classification of matter, in so far as ordering involves rejecting inappropriate elements’. For Douglas dirt is a spatial problem, a question of not what stuff is but where it is.} \todo{Reference, (Waste by William Viney).} 

\comment{CATEGORIZATION OF WASTE} \paraphrase{“Dirt”, writes Douglas, “is the by-product of a systematic ordering and classification of matter, in so far as ordering involves rejecting inappropriate elements.” Dirt is only dirty in certain places, when it is out of its correct position. Just as faces, for example, is considered dirty when it is in our kitchens but not when it is in our bodies, so it is that our classification of waste depends on the location of objects.}  \todo{Reference, (Waste by William Viney).}

\comment{CATEGORIZATION OF WASTE} \comment{susan strasser ile douglasın fikirlerinde bir ortak nokta görülebilmekte. Özellikle trashın relative olması ve bu işin aslında bir ordering and classification olduğu konusunda. Remember the example given: shoes  on the dinner table.}

\comment{CATEGORIZATION OF WASTE} \paraphrase{waste is “matter is out place”, a definition first given by Lord Palmerston in the mid-nineteenth century and incubated by the British anthropologist Mary Douglas, in her book Purity and Danger.}

% https://narratingwaste.wordpress.com/tag/king-lear/
% Not all waste is dirty, it not always dangerous, contagious or abject. \ldots waste might be quite useful in making time and in keeping time.

% Cornelia Parker's work, narrate an absent event of waste
% There is also strong relationship with the collecting discarded stuff and archeologies of waste.  

% King Lear, feeling of waste ???

%%%
%%%
%%%
\section{In Art}

% Crowdsourcing artworkümde neden insanlardan topladığımı, neden onları kullanıdığımı anlatacağım yer ile ilgili olabilir. 

%*************************************
% Phrases...
% What specific social issues are you trying to address through Labyrinth?
% In view of the diversity of online crowdsourced art projects, as illustrated by the examples cited so far, it is useful to map out this artistic trend by developing a comprehensive and multidimensional typology of online crowdsourced art. Table 1 organizes this classification according to a set of multiple criteria. (But this one can be used at introduction, but it is too long to fit on introduction. Maybe select works by giving prominence to some of the features. So the approach to the trash is going to be introduced and also it is reveals the what i am doing in this context.) (A Typology of Online Crowdsourced Art. Diye bir örnek var aslında benzer bir şekilde bu trash içinde yapılabilir.) 
\subsection{Crowdsourcing, Participation, Open Work}
% TODO needs reference, FROM Crowdsourced Art and Collective Creativity
In the words of Jeff Howe (2006b), the Wired columnist who coined this term in June 2006, \quotes{crowdsourcing represents the act of a company or institution taking a function once performed by employees and outsourcing it to an undefined (and generally large) network of people in the form of an open call}. The vital elements that qualify an outreach strategy as crowdsourcing are, according to Howe, the use of the open call format and the reliance on a large network of potential workers. Although in some cases there is a material reward for the best contributions, the existence of financial incentives is not a required feature in crowdsourcing. Because of the diversity of its applications, crowdsourcing continues to be a disputed term in both the scholarly literature and the popular press; Howe’s original definition is, in this sense, a helpful delineation of its practical sphere.

The value of crowdsourcing lies in the collective intelligence of the contributors. Pierre Levy (1997) describes this concept as “a form of universally distributed intelligence, constantly enhanced, coordinated in real time, and resulting in the effective mobilization of skills” (p. 13). The question of collective intelligence—and its potential efficiency in various practical settings—has received much attention in both academia and journalism. Researchers studying team performance generally agree that, under the right circumstances and with appropriate motivation, large groups of people can work together and harness their collective intelligence to achieve efficient results (Benkler, 2006; Rheingold, 2002; Surowiecki, 2004). Nevertheless, artistic creativity is different from innovation and intelligence, and it requires a unique set of skills and sensibilities as well as a particular type of cultural capital; if we admit that crowds can have collective intelligence, do they also have collective creativity in an artistic sense?

All these pages are rescued and with their [hi]story they are separated from unused new produced blank sheets and notebooks. They are not bought, not gift. They are found. They are accepted. One of the most creative medium is paper and pencil. Chance given to the people through this work.

However, in view of its reliance on the artistic contribution of a large pool of usually anonymous participants, this type of art raises important questions about notions of collective creativity, authorship, collaboration, and the shifting structure of artistic production in the new digital environment. It is well studied area. Pick a method here. Transforming trash with collaboration.

As curator Andrea Grover notes, “having the audience become co-creators is not a new impulse”; the Internet simply offered a new platform to accomplish this goal (Strickland, 2011, para. 5).

Here this question can be raised about why this type collaboration. Another option can be working together with people on a table. Creating things at that time and transforming the objects here. Possibly the connection to the people will more realistic and intense. However same things can be succeed on the internet at some level. 

While crowdsourced art challenges the traditional role of the artist, it simultaneously redefines the conventional function of the public, turning them from passive receivers into engaged producers. (This totally a new area to discuss, I'm going to summarize and introduce concepts and debates. How are they support my work and how they are give way to me?)

% This article therefore aims to fill these critical gaps by analyzing the practice of online crowdsourced art within a framework of collective creativity and participation theories. Principally, my interest is in answering two key questions. What is crowdsourced art and how can it be classified? And how does the structure of the artwork determine the degree or significance of participation?

% why collaboration, is it really a collaboration? It is methodology used in my practice. 

% TODO: Umberto Eco's Open Work


%*************************************
% ARTWORKS:



% Dark Matter: Explosive View.

%.....................................


%*************************************
% Solid claims about its cultural aspect.
\subsection{Book: Recycled, Re-Seen: Folk Art from the Global Scrap Heap}
\todo{Reference.}
\comment{GIFT FROM GOD, NOT KNOWING PREVIOUS STATE} They also states that the practice of tribal people artfully transform Coca-Cola bottle (given as example by the authors). From western trash to tribal treasure. \paraphrase{For the Kalahari Bushmen, the process appears to be one not so much of reusing but of creating anew; not so much transforming, a inventing (p.9).} It stated these found objects is accepted as a gift from gods. A disposable item becomes imminent desire and profound social consequence. (Intercultural recycling.) \paraphrase{Both presentations tell a story about an aesthetic and cross-cultural process --- as well as an economic and political one --- which is defined by the act of recovering and transforming the detritus of the industrial age into handmade objects of renewed meaning, utility, devotion, and sometimes arresting beauty (p.10).} 

\comment{HYBRIDITY} They claim that \paraphrase{the end result is a category of hybrid objects that bear the mark of at least two distinct domains, each with its won material, meaning, makers, and users (p.10).} (some of them utilitarian, some of them ornamental.) \paraphrase{Whatever their ultimate function, each of these objects contains within itself a visual, material, and conceptual reference to multiple technologies, histories and temporalities.}

\comment{SEMIOTIC CONTEXT} \quotes{Like collage in art or quotation in literature, the recycled object carries a kind of "memory" of its prior existence. Recycling always implies a stance toward time --- between the past and the present --- and often a perspective on cultures --- one's own and others. (Jacknis 1992)} \todo{reference}

\comment{HYBRIDITY} \paraphrase{It is stated that same object can have one meaning for one community or culture and another meaning or series of meaning for another. Different objects have different life spans --- different degrees of permanence or disposability --- and that these life spans are socially constituted is also an integrated part of the story \ldots geographic and socioeconomic boundaries of class, caste and culture throughout the world. It is story of people --- the women, men, children etc. one person's trash into another's treasure.}

\comment{OTHER DICIPLINE, INDUSTRY} \paraphrase{the process of re fabrication explored here is not to be confused with the kind of industrial strength recycling to which we in the West are most accustomed. When we think of recycling in America and other industrialized nations we imagine an automated sequence beginning with the curbside disposal of aluminum cans, plastic bottles, and old newspapers. returned to the industrial process. solid waste management, global greening, and ecological awareness are the buzzwords that guide and motivate consumers and industries to engage in this process of secondary and post consumer waste recycling.}

\comment{DIFFERENT SCALES OF RECYCLING} There are different kinds of recycling that is small-scale, hand done, and local --- a kind of recycling in which yesterday's newspapers are transformed by hand into tin trunk liners;empty food cans become kerosene lanterns; and old tires are refashioned ... For the same people are not used the items that are scavenged by people who have little or no contact with those who first possessed them, and may neither know nor care about their originally intended function. It is case from the film and also the case depicted in the real life photographs taken by anthropologist michael leahy which document his first colonial contact with new guinean highlanders 

\todo{yerlinin fotoğrafı}
\quotes{Renowned early-twentieth century anthropologist Michael Leahy encountered a Wabag man from Papua New Guinea wearing an aluminum whole wheat biscuit tin on his head. In the symbol system of this culture, large, bright, and shiny ornaments are connote health, well-being, sexual attractiveness, and the approval of the ancestors.} \todo{reference}

\comment{CONSUMER CULTURE} \paraphrase{We are dealing specifically with industrially produced consumer discards and their subsequent transformation, these essays are necessarily situated in a particular time, place, and sensibility: the consumer culture of the late twentieth century. In its most basic form, I refer to a consumer culture as one "in which the activities and ethics of a society are determined by patterns of consumption" rather than production (Mamiya 1992, 2)}

\comment{CONSUMER CULTURE, SORUCE OF TRASH} While consumption has provided a foundation for the transnational capitalist system adntheus has much longer history than the last fifty years this thesis focused on the more recent history. because it is during the history of consumption. consumption can be though that as global not to a western or first world countries.

\comment{GLOBAL} It is not limited with geography, ethnicity, gender, nationality. The process of of retrieving and transforming a consumer package or product that someone else has thrown away is a phenomenon that is taking place in the largest metropolises of urban America as well as the remotest corners of amazonian rain forest. 

\comment{GLOBAL, SELECTING COMMODOTIES, } To frame a discussion that incorporates an understanding of such diverse locations, objects, and peoples is to claim links between material process that originate under vastly different social, economic, environmental, and political circumstances. \ldots The most obvious link are the raw materials the raw materials themselves: the packaging, broken pieces. becomes raw materials for other things. these consumer items. and it is spreated globally. It can be seen remotest corner of the world. the indigenization of western objects. (sahlins, 16) Do these hybrid objects, as many western critics might assume, point to passive acceptance of a homogenized consumer culture in the service of western capitalist expansion, or, even worse, a sense of want and deprivation when confronted with "our riches"? Marshall Sahlins: "The first commercial impulse of the local people is not to become just like us, but more like themselves. They do not necessarily despise our commodities. But they are selective in their demands and transformative of their uses of such things. (sahlins, 13)" 

\comment{People applies their own perception to interpreted the items.} Human manufactured never envisioned possibilities seen by people. It suggests a self-confididence and intellectual authority that allows local peoples to encompass western goods in their own meanings "in their own scheme of things." 

\comment{IRONIC, maybe added to results of transformation, reusing trashed items in different contexts.} This misuse the detritus of the industrial age has been described by western theorist as ironic. the irony is often embodied visually and conceptually. opposite of natural use, expected perception. for example making something from nothing or turning trash into treasure. By juxtaposing different materials changing context and place. 

\quotes{It is clear that one man's rubbish can be another man's desirable object; that rubbish, like beauty, is in the eye of the beholder. (thompson, 1979, 97)}

\comment{Karşı çıktığım şeylerden birisi objelerin ilk manaları, ilk kullanım alanları. many scholar stated the general attitude in the society. It is key ingredients of throw away society. initial function is fulfilled by }

\comment{How does it become trash?, initial function.} when an object is discarded it is perceived as being no longer of value to the person or society that once possessed it. Once a newspaper is read, or a bottle of soca pop consumed, its initial function is fulfilled and it is intended to be thrown out as trash.

\comment{DEFINITION} Rubbish is, by definition, an object that is not or is no longer, owned by anyone, that falls outside all categories of economics, culture, and social control. As one of many things on the garbage heap, a discarded object even tends to take on a negative value as something unsanitary, dangerous. The socially constructed value of the object has shifted over time from its finite life span of usefulness and meaning to a timeless and valueless of socially sanctioned rubbish. 

\comment{Capitalism, market.} The remarkable thing about many of these objects - especially those produced in the last of the twentieth century - is that they were specifically designed to end up on the garbage heap. They were designed to decrease in value over time - to be used one, or twice and than to be thrown away. This applies not only applies packing - designed containers that protect and promote products - but to an ever expanding list of products. themselves. \todo{Waste and wantta bununla ilgili bir bölüm olmalı.}

\comment{throwaway culture} throwaway spirit (vance packard) Stephen jay gould has observed: "in our world of material wealth where so many broken items are thrown away "

\comment{Social classification, throw away culture, place in society.} The flip side of this paradigm is that a person's wealth has become measured not only in how much he or she can afford to consume, but in how much he or she can afford to throw away. America is one of the greatest country in terms of generating trash and exporting them to the other third world, fourth world countries. (bu ülkeler hangileri onları foot notte açıklayabilirsin.)

\comment{small amount of it} while most waste ends up in these unsanitary and unsightly landfills, some small percentage is reincorparated --- or recycled --- back into the economic and cultural system of the local population. Indeed, this is the power of rubbish as a category of possessable objects: it has within itself the potential of being discovered, retrieved, and transformed back into an object of greater or lesser durability. 

\comment{Resistance, tactile, agnes varda?} If it is ultimately romantic to speak of these toys (or any other modern-day recyclia) in the language of resistance (by which I mean self-conscious political opposition), I would agree with Marshal Sahlins's assertion that "whether or not it comes to this [resistance], the indigenous mode of response to imperialism is always culturally subversive, insofar as the people must need to interpret the experience and they can do so only according to their own principles of existence. (sahlins 1992, 16)"

\comment{hybrid} each recycled object contains within it a reference to two or more distinct times. \comment{not every time actually, if you transfor ok. it applicable for my case.} Whatever their ultimate design or destination, these recycled artifacts are, by definition, "impure", "inauthentic" products of past and present, here and there, "us" and "them" (clifford 1992) \todo{reference}

\comment{Economic conditions forces to people recycle and reuse. but even if they have no economical struggle, they continue recycle. what is the reason of it.} Recycling as an economic strategy of survival in develoing countries throughout the world. creative production that is motivated, primarily, by adverse conditions of economic necessity. (lanfill orchestra here. bookcovers here.) Economic survival and adaptation are influential factors for both the makers, who build informal business on the making and selling of recycled goods, and the local consumers, for whom the market for affordable, utilitarian goods is devised. (p25)

\subsection{Statistics:}
The United nations estimates that two percent of the people in cities in nonindustrialized countries make a living from the refuse discarded by the richest ten to twenty percent (germer 1991)

