\chapter{Introduction}

\epigraph{One can even shout out through refuse \ldots}{\hfill---Kurt Schwitters, \textit{Kurt Schwitters}, 1985}





%
%
\comment{[Every.]}
\paraphrase{Every day, [people] put unwanted material in toilets and garbage bins, regularly flushing it away or taking it out in bags to be transported far away from our homes \cite{zimring2012encyclopedia}.} 

Trash exists in every age and every society. Refuse is part of people's consumption practices and it is a very common concept from developed cities to rural areas, from modern societies to ancient societies\cite[p.33]{rathje1992rubbish} \todo{more ref.}. For archaeologist dumps are one the significant places to figure how people consume in the ancient times \todo{ref}.

Trash is everywhere and, produced every time. During daily activities such as drinking coffee, eating beverages, waiting in the bank trash is generated. Different aspects of our life, eating, drinking, working, traveling, all of them are leave behind wastes and that become invisible. We are surrounded by all kinds of consumer goods. We are consuming everywhere while walking, working and traveling. The pattern is spread to all day and different places. It is hard to say that any (place) is waste free. Nearly every place people require waste bin. If people do not find a waste bin, people often leave it (to the ground?). Production of trash never stops. It is inseparable from people's activities. 

Trash is everywhere. Trash is in the streets, in your home, in the sea that you swam, rotating around the globe \footnote{Satellite discards leaved the atmosphere and they rotate around the globe like satellites.}. Even if trash is tried to be moved away from people life, it is close as nearest waste bin.

\comment{[Global.]}
\paraphrase{It is not limited with geography, ethnicity, gender, nationality. The process of retrieving and transforming a consumer package or product that someone else has thrown away is a phenomenon that is taking place in the largest metropolises of urban America as well as the remotest corners of the world.}





%
% FROM Trash as Treasure BY WILLIAM L. FASH AND E. WYLLYS ANDREWS, ReVista
\comment{[Facts.]}
\paraphrase{The World Bank estimates that the amount of solid waste generated in cities is growing faster than the rate of urbanization. The higher the income level and the rate of urbanization, the greater the amount of solid waste produced. OECD\footnote{OECD(Organization for Economic Co-operation and Development) is an international economic organization of 34 countries. Turkey is a member of this organization.} countries produce almost half of the world’s waste. Africa and South Asia produce the least waste. High-income countries have the highest collection rates and are most likely to dispose of waste to landfills or incinerators. Low-income countries have the lowest collection rates and are most likely to dispose of their waste in open dumps. However, low-income countries also have the largest numbers of informal waste pickers who collect, sort, and reclaim recyclables---thus reducing costs to the city and to the environment \cite{fash2015trash}.} 

% Üstteki kısmı neden diyorsun? Teze nasıl bir katkı sağlıyor? Throw away culture a taşınabilir.
% From recycled book.
\paraphrase{A person's wealth has become measured not only in how much he or she can afford to consume, but in how much he or she can afford to throw away. America is one of the greatest country in terms of generating trash and exporting them to the other third world, fourth world countries.\cite{cerny1996recycled}}





%
%
\comment{[Common.]}
(In the dictionary \todo{Which dictionary?} trash means anything useless, worthless or discarded. In parallel to this definition, common perception is to get rid of them as soon as possible. People tend to think that trash is valueless because it is trashed no longer needed or not wanted anymore. People want to refuse them from their lives most of the time not thinking beyond (or alternative). However to establish solid understanding of trash (or trashing things), we have to give attention to the trash (and ask why it is trash). \comment{(To understand the beyond common perception we have to ask...)} How does it become trash? \comment{Is it possible to do something other than refusing them?}

Instead of creating new usages, combinations or alternatives, common approach is to ignore all the possibilities embedded to the trash. People do not see beyond the primary function.





%
%
[Moves.] The vast amount of industrial discarded items spread through the landfills to oceans.

Trash moves, objects moves from place to place. From homes to garbage trucks. From streets to landfills. From waste bins to sculptures. Lots of different people touches to trash. With the trash what moves? Travels like humans. Travels like commodities. When we are telling about the travel of commodities, then it is easy to say that trashes also moves around. \todo{summarize here. ve bir yere bağla.}

People do not think what happens after they throw trash away. \paraphrase{They forget about it and don’t think about all the time and energy and money put into disposing of it.} It exits from people's life but not from the world. It is stacked to another place. 

A project conducted by MIT researches journey of trash by placing trackers onto the trash \cite{chen2009mit}. The results are surprising. Trashes spread away across the country and this journey takes month. Further it is not limited with the border of America because it is the one of the countries export their garbage to the other countries --- fourth world countries\todo[inline]{footnote what are these countries?}. In other words American's trash is not only their trash. As commodities spread to the every corner of the world, trash also. What happens when it is traveled to the other countries, how they approaches these items. \comment{They find new uses and meanings on them.} \todo{consider African native}





%
%
\comment{[Aspects. Approaches.]}
In this thesis study to understand the different aspects of trash is a key element. Sometimes it is accepted as a problem (for examples keeping streets clean, handling the smell) that is need to carefully and seriously managed. On the other hand it is accepted as a source of diversity. Further for the recyclers it is a source of income. The economical aspect of trash can not ignored. There are people who collects plastics, scraps and papers from waste bins and landfills. By collecting and selling them they endure their life. Further some of the countries \todo{ref.} imports garbage to recycle or use their own purposes. As it can be seen clearly there are different aspects and approaches to the trash. For some people it is disgusting thing and moved away from living space and for the others source of their life. \paraphrase{While dictionary definitions of garbage describe it as \quotes{filth} and \quotes{worthless}, scholars are careful to note that perceptions of waste and the value of material are neither static nor universally shared \cite{zimring2012encyclopedia}.}

\comment{[Relative.]}
As stated by the editor of Garbage issue of ReVista, Christmas decorations at Chocó, a poor region on Colombia’s Pacific Coast, are \quotes{all crafted from used tin cans, old newspapers, discarded textiles and found wood objects} \cite{erlick2015editorsletter}. She realized that any of them called the practice as recycling. For them using trash again and again is very natural and it is part of their life. On the contrary for the developed countries trash considered as a thing that must be avoided. There is no place for trash in their life. It can be understood that approach to the trash is not same for the all regions of world \todo{ref}.

% FROM Beautiful Trash Art and Transformation BY PAOLA IBARRA, ReVista
\paraphrase{We use it, produce it and dispose of it \cite{ibarra2015beautiful}. The hoarder likes to salvage a few things for later use---the plastic and glass containers, the cardboard boxes \cite{ibarra2015beautiful}.}

\comment{[Production of trash and removal of trash.]}
For example a project conducted by MIT Sensible City Lab. reveals the \quotes{removal-chain} of objects by tracking trash with the mobile trackers places onto the trash \cite{chen2009mit}.





%
%
\comment{[Parts of us.]}
\paraphrase{Our trash is a testament; what we throw away says much about our values, our habits, and our lives \cite{zimring2012encyclopedia}. Our trash is part of us, whether or not we choose to acknowledge it \cite{zimring2012encyclopedia}. The absence of a waste stream aroused suspicion, just as the presence of particular items tell us about the habits of the consumers who generate a waste stream \cite{zimring2012encyclopedia}.}






%
%
\comment{[Change in the production and consumption practices.]}
\paraphrase{Over the course of the twentieth century, the twin developments of mass production and mass media in the capitalist economies of the Western Countries completed a total transformation of everyday life, reorienting almost every activity toward consumption. Things once locally produced and often handmade were now mass produced and commodified, turning local, artisanal producers into deskilled laborers serving the assembly line.} As the result of it amount of goods boost and the cost of them significantly reduced. Things once reused again and again, now thrown away because it is more affordable to replace \todo{daha uygun bir kelime bul} with new one. \paraphrase{Mass production and commodification liquidated what remained of folkways tied to local production, demanding people construct the meaning of their lives through purchases rather than production. (FROM Collage Culture)} As the production increased its by-product material waste is also increased. \paraphrase{The phenomenon of waste comes clearly into focus not merely as a by-product of manufacturing processes, but rather as an integral element in cycles of production and consumption. (Trash Culture)} \todo{Ref: Trash Culture}






%
%
[Current] During the last decades much attention has been devoted to the waste. In particularly environmental considerations dominates the perceptions of waste.

% FROM Beautiful Trash Art and Transformation BY PAOLA IBARRA, ReVista
\paraphrase{Despite a relatively increased awareness about consumption and its consequences, the pace at which we also acquire and dispose of material objects is exploding \cite{ibarra2015beautiful}.}





%
%
\comment{[Art]}
\paraphrase{At least since the early twentieth century, the concern with discarded things and materials has been a recurring theme in art.} \todo{Bu nereden?}

% Şöyle bir şey var, sanata non-art objelerin girmesinden bahsediyoruz ama artık orda o seviyede değiliz. O zamandan bu zamana çok şey değişti. 
\comment{In the beginning of the 20th century non-art objects enter the scope of art making. This is a revolutionary change in the making of art. Production of art and the approach to the art changed dramatically. This changes actually started at the end of the 19th century with impressionist. (But not related with non-art objects.) Picasso is the first artist that used non-art object in his works. Later many of them followed him. First works are collage which gluing different papers together. Yes using non art objects in the art is introduced and opened new dimensions for the language of art. But some of them treated as sculptures from trash. Reflects our world. But it is not limited with this. Some of the works are provokes the consumption habits of society and offers an alternative perception.}





%
%****************************************
\section{Purpose of the Study}
% FROM Beautiful Trash Art and Transformation BY PAOLA IBARRA, ReVista
\paraphrase{Particularly in the connection between garbage and the arts, I am interested in two questions. First, the issue of recycling as a general practice in the arts; and secondly, in the whole issue of representation---that is, representation of waste as subject, and representation (of waste or others subjects) through waste as material \cite{ibarra2015beautiful}.}

Some of the artist also interested in others trash. Why people are interested in rubbish?

\todo[inline]{what do you mean when saying ``transforming trash"? Needs to be opened. I claimed that such a thing exist? when it is exist? by physically or conceptually? actually both of them exist.}

The aim of this study is to explore the theories and methods in transformation of trash in the context of art and artistic act.

In this thesis (re-)usage of discarded materials in the process of art making and the artwork itself is explored. Why and how are they used by the artist? Are there any differences with the original items compared to discarded items? Has using discarded material or trash specific (or special) meaning and message? 

\paraphrase{Trash have influenced, and are also influenced by, cultural products such as films, visual art, museum exhibits and literature.} \todo{ref missing.}

% TODO new comer
\paraphrase{This thesis seeks to answer the following questions: How did art and literature respond to this age of consumption? What do the productions and practices of artists and writers reveal about the meaning of mass production, consumption, reification (depersonalization), mechanical reproduction, and meaning? (FROM Collage Culture)}

Do people want to take back (or revisit) the trash once thrown away? The purpose is to turn them to a things that worth to reconsider. 

Questioned that how do artists reflect garbage and what are their tactics?





%
%
[Different types of trash] There are different types of trashes result of different production and consumption practices. For example \paraphrase{radioactive waste is usually a by-product of nuclear power generation and other applications of nuclear fission or nuclear technology}. Electronic waste (e-waste), Construction and demolition waste, Medical waste are one of them. As it can be seen that there are lots of different types and as various as activities of human. In this scope of thesis focused on disposable items and paper trash.





%
%
[Different disciplines of trash] Trash draws attention different peoples and disciplines. 

% From Trashion: The Return of Disposed, Bahar Emgin.
\paraphrase{Such a conceptualization of waste as “the degree zero of value” has been contested for some time in different disciplines, ranging from economics to environmental studies, but most particularly by those studying consumerism or material culture.} \todo{ref.}

Technological perspectives. (seeing trash as a design problem. developed technologies and societies generates lots of trash, so this situation cause to inquiry that are they really developed? development in what sense?)

Management of it (handled by the municipals generally). Removal from the life spaces of human.

As many people are inside the business of transforming trash through craft and so on. People on the streets looking for aluminum cans, glass bottles, plastic. They collect and sell them to survive. There are companies that are collect waste materials to regain them to the industry. In thesis they are not in the scope. People that give attention to the trash and behave artistic act is in the scope of this thesis.

% TODO new comer
% PERCEPTION TO THE TRASH, DICIPLINES
\comment{WASTE AND TECHNOLOGY RELATIONSHIP} \quotes{We live in a badly engineered world, because the vast amounts of waste (both material and energetic) are needless; and that waste could be virtually eliminated through better design} \cite{mcdonough2010cradle}. In other word the problem is our technology which is not perfect. (and I'm not sure that at some point that technology will reach to the perfection or not.)

% TODO new comer
\comment{TECHNICAL PROBLEM, NEED TO MANAGED, RECYCLING (CAPITALISM), UPCYCLING} \paraphrase{On the contrary, with respect to the issue of disposability, waste was handled merely “as a technical problem, something to be administered by the most efficient and rational technologies of removal.” Only through the rise of environmental movements in the 1960s did the disposal of waste come to be loaded with negative meanings and iewed through a moral framework. The enormous quantities of waste accumulating in urban centers, Hawkins writes in “Plastic Bags,” were not only taken as a threat to the environment, but also as a sign of an individualistic, insensitive, and hedonistic consumer society. Waste now became evil. If the environment is to be saved from our destructive power, then waste should be “managed,” Hawkins asserts. Consequently, recycling gained its contemporary prominence “as virtue-added disposal\ldots disposal in which the self is morally purified, disposal as an act of redemption.” Disposal in the form of recycling is now a moralistic attitude through which we pay the debt we owe to the world. Upcycling... On the other side of the coin is the business stemming from these practices; recyclers not only ease their conscience through recycling; they also make a profit. Recycling, as “the huge tertiary sector devoted to getting rid of things, is central to the maintenance of capitalism; it doesn’t just allow economies to function by removing excess and waste—it is an economy, realizing commercial value in what’s discarded,” Hawkins and Muecke write in Culture and Waste. In the same manner, upcycling has already been turned into a business: Certain designers labeled eco-friendly are earning money through upcycling, competitions are organized around trashion, numerous websites are devoted to promoting and selling upcycled objects, and online and print resources explain how to upcycle at home. In short, there is a whole sector of upcycling now.} \todo{reference, (Trashion: The Return of the Disposed by Bahar Emgin)}

% TODO new comer
\comment{DESIGN, ALTERNATIVE DICIPLINE.} \paraphrase{Design, as a conduit of disposal, reintroduces rubbish as objects of distinction, invaluable and potentially priceless. People are often eager to see objects that were once considered useless and tasteless when they have been invigorated with new life.} \todo[inline]{reference, (Trashion: The Return of the Disposed by Bahar Emgin)}

% TODO new comer
\comment{UPCYCLING, sample thesis statement sentence} \paraphrase{This article is about those objects that are recreated from trash through the process of upcycling. Upcycling is a term used by architect and designer William McDonaugh and chemist Michael Braungart and refers to “the process of converting an industrial nutrient (material) into something of similar or greater value, in its second life.” I argue that design, in this instance, acts as a tool of transformation and reintroduces into certain orders what was once deemed waste. This theory counters the argument that an object is dead once it is disposed of.} \todo[inline]{Reference, (Trashion: The Return of the Disposed by Bahar Emgin).}

% TODO new comer
% FROM Recycled, Re-Seen: Folk Art from the Global Scrap Heap
\comment{OTHER DICIPLINE, INDUSTRY} \paraphrase{the process of re fabrication explored here is not to be confused with the kind of industrial strength recycling to which we in the West are most accustomed. When we think of recycling in America and other industrialized nations we imagine an automated sequence beginning with the curbside disposal of aluminum cans, plastic bottles, and old newspapers. returned to the industrial process. solid waste management, global greening, and ecological awareness are the buzzwords that guide and motivate consumers and industries to engage in this process of secondary and post consumer waste recycling.}





%
%****************************************
\section{Structure of the Study}
Theory and artworks are not strictly separated from each other. Artworks are explained with the help of theories. This study is structured by many examples of artworks in different part of the thesis. When a topic or argument is presented, immediately example artwork is given at this moment and how artist replies this issue is illustrated. Thereby artworks are examined in the theoretical and cultural context. Artworks are used as a supportive elements for the stated arguments.  However, not every time artworks are discussed in detail. Only basic information is given and how it is related with this topic explained. By the way it is better to keep in mind that all works can be analyzed in different contexts in more detail fashion.


Following chapter, Trash in Culture and Theory, explores the place of trash in the cultural life and theoretical approaches on(of?) trash. The aim is to understand the trash in cultural and theoretical aspects. (Better realization the notion of trash) Chapter 2 provides theoretical and cultural background. It is explored the trash is being worked is whose trash. Or what type of society generate this trash. What type of approach generate it and what are the dynamics of it? What type of trash we are talking about? It is understood through the patters of consumption patterns. And to reflect this cultural phenomena what philosophers and scholars say. Artist how they are reflected this notion.and also looked how scholars are conceptualized trash and its movement on the different values system. How explained the transient nature of trash?

Third chapter, Trash (in) Art, seeks the root of using discarded materials in the artworks and looks through the sample artworks and artists. As already artworks are mentioned in the different part of the thesis, in this chapter some of them analyzed deeply. Artists methods and approaches to the subject are stated.

Fourth chapter consists of the overview of the project. Stated the approach to topic clearly. Development process of the work is presented in detail. It will start with the development process of the work. I believe that the way an artist works and the obstacles that she encounters have a huge impact on how the final work is embodied. Development process of my work shaped the way I think on the subject of my thesis. Therefore, I will explain my process as detailed as possible. Then, I will continue with different aspects of my work; formal decisions and usage of light that I believe carry my work to another level. At the end of this chapter, we will see how my work functions and what it proposes on the image and word relationship. \todo{rewrite, copy paste from another thesis.}

Finally last chapter is reserved for an overall conclusion of this thesis and further suggestions on the subject and the project.





%
%****************************************
\section{A Note on Terminology}
Many scholars and authors have used different words (or terminology) to define trash. Garbage, trash, rubbish, debris, detritus, waste, scrap, junk, refuse, discard, disposal, litter are some of them. Although there are slight differences between these words, sometimes they are used to signify same concept (or used with the same purposes). It signifies that the topic is so broad and there is no consensus on terminology. Words are used interchangeably. From the book \quotes{Rubbish: the Archeology of Garbage} gives clearer definition of some these words:

% FROM Rubbish! The Archaeology of Garbage, p.9, rathje1992rubbish
\begin{quote}
Several words for the things we throw away---"garbage", "trash", "refuse", "rubbish"--- are used synonymously in causal speech but in fact have different meanings. \textit{Trash} refers specifically to discards that are at least theoretically "dry"---newspapers, boxes, cans, and so on. \textit{Garbage} refers technically to wet discards--- food remains, yard waste, and offal. Refuse is an inclusive term for both the wet discards and the dry. Rubbish is even more inclusive: It refers to all refuse plus construction and demolition debris. The distinction between wet and dry garbage was important in the days when cities slopped garbage to pigs, and needed to have the wet material separated from the dry; it eventually became irrelevant, but may see a revival if the idea of composting food and yard waste catches on. We will frequently use "garbage" in this book to refer to the totality of human discards because it is the word used naturally in ordinary speech. The word is etymologically obscure, though it probably derives from Anglo-French, and its earliest associations have to do with working in the kitchen.
\end{quote}

From another book (Trash Culture):
\paraphrase{Literature of trash feature a broad range of terms, reflecting the multifaceted and conceptually complex nature of the issue at hand. Many of these terms --- such as ‘trash’, ‘garbage’ and ‘rubbish’ --- are frequently used interchangeably. This is not least because the variety of terms reflects the variations in English and US vocabulary and thus all usages are retained as individual authors intended in order to reflect the international nature of this volume. Other related terms, such as ‘ruins’, ‘obsolete’, ‘waste’, ‘discards’ are used in a variety of contexts in each chapter. Again, these usages have been retained to reflect the focus of individual studies.}

In addition to these there is also confusion of the terms: reuse and recycling which are the (general) methods transforming trash to the another thing. Both of them are used widely by scholars. 

\paraphrase{According to the dictionary, the word “reuse” means “to employ for some purpose” or “to put into service.” Reusing involves usage of the same product unchanged in form. Reusing lengthens the life of the item or material. Other examples are; buying some items and then selling them as used items, repairing some lawn equipment and reusing them, upgrading a computer, renting books, journals, periodicals, DVDs and others. The main purpose is to make the item last as long as it can. To reuse is to use something again instead of throwing it away or sending it off to a recycling company. \textbf{Why throw something away when you can give it another life?} Using something multiple times -- like using a disposable container more than once -- is not the only way to reuse; you can also give old items a new purpose. For example, use an empty coffee can to store small craft supplies or an old loofah as a scouring sponge for cleaning sinks.} Reusing is possible with re seeing (rethinking). Reusing is possible meet the needs of the human itself. Using creativity and personal approach can change objects functions. It is possible to use objects for different purposes. 


According to the dictionary, “recycle” means “to treat or process (used or waste materials) so as to make suitable for reuse.” In recycling an item, it is processed into a totally new product. It is an energy consuming process. For example, if we put some plastic bottles, paper, or aluminum items in a recycling bin, these materials may be recycled into a totally different thing as clothing items, fabric, or maybe a quilt. In this process, energy is required which depends upon the stages of transformation. Recycling occurs when waste in an unchanged chemical form is used in the same process that created the original product. Examples are crushed glass containers (cullet) used to make new glass containers, and scrap metal used in foundries. 

Recycling is very similar the rotting (decaying), reuse is something like dry tree branches used by birds for their nest. These are two agents of nature to regain their resources.

Recycling can be viewed as down-cycling. The object smashed to the small particles to be used later in the production of something else. Although reuse can be viewed as up-cycling that gives another (or more) value to the discarded products. Down-cycling does not generates new meanings it tries to convert the product to already known state to process. 

Many scholar used the word recycling when mentioning works use trash and the concept of it. However they are not mentioning the meaning that is decomposing things to the their particles. What they is actually is reusing and combining things, concepts, creating new mixtures. It is hard to say that they use it wrongly, but what they refer is actually upcycling. Creative reuse, inventing new things from discard. Some of the scholar uses recycling \cite{cerny1996recycled,herman1998trashformations} for the recreation and transformation of it. Some of the scholars and artist uses upcycling for their process.

\comment{What were people throwing out when these words were coined?}

