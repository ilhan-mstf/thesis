\chapter{Introduction}

% How to cite epigraph http://blog.apastyle.org/apastyle/2013/10/how-to-format-an-epigraph.html
% Epigraph:
% FROM Wild Art
% The first question I ask myself when something doesn't seem to be beautiful is why do I think it's not beautiful? And very shortly you discover that there is no reason. If we can conquer that dislike, or begin to like what we did dislike, then the world is more open. That path --- of increasing one's enjoyment of life --- is the path, I think, we all best take: to use art not as self-expression, but as self-alteration; to become more open. --- John Cage, quote taken from the televison series `American Masters', Season 5, Episode 8, John Cage: I have nothing to say and I am saying it (aired on 17 September 1990).

% FROM http://stanforddailyarchive.com/cgi-bin/stanford?a=d&d=stanford19920130-01.2.5&e=-------en-20--1--txt-txIN-------#
% The thing to do is to find some way --- and it can be done, I think, through art --- to enjoy life in its vast multiplicity. --- John Cage

% FROM Wild Art
% Bu epigraphta aslında şöyle bir sıkıntı var, bir şeyin miktarı artınca daha fazla insanın onun üzerinde düşeceğini varsaymak biraz sıkıntı olabilir.
% We are going to make a lot more junk, and there is going to be a lot more junk art. I think people are going to have a lot to say about it, and more people need to think about it. --- Jeremy Mayer, quote taken from American Craft Magazine, December/January 2011.


%****************************************
% Structure of this chapter
% Section: topic and its significance
% Section: the purpose of this study
% Section: overview of chapters

% Konunun dikkat çekici ilginç kısımları: ?

% Çöp ve dönüşümden bahsetmek gerekli. Çöp nedir? Dönüşüm nedir? Çöpün bir dönüşümü var mıdır? Var ise bu nasıl bir şeydir?
% Meaning of words here. move reuse and recycle here.

% Belki direk bir sanat işiyle girebilirsin. Etkileyici olması açısından. Hangi iş olabilir? Çöpün ne olduğunu ele alan hem de ona karşı bir yaklaşım geliştiren bir örnek belki de benim işine çok benzeyen bir şey olabilir. Starbucks bardaklarından bir şeyler yapan insan olabilir. 

% Amelie olabilir. Orada fotoğraf toplayan çocuk olabilir. Onları itinayla toplayan bir adam. Tren garındaki fotoğraf kulubesi. Belki de dünyanın dört bir yanından gelen insanlar, bir kesişim noktası. Bir birinden farklı bir sürü insan. Filmde neden böyle bir eyleme yer veriliyor? Film fotoğraflar üzerinden defter üzerinden ilerliyor. Onun çevresinden bir hikaye kurgulanıyor. İnsanların artıklarını topluyor. Bunlar baya baya aslında o insanlar. Bazı fotoğraflar çok fazla küçük parçalara ayrılıyor. Çok iyi bir örnek mi bilmiyorum. Burada niyet nedir ki? Ben buna şöyle bir okuma getireceğim diyebilirim.

% Wall-E belki olabilir. Dünyanın aslında bir çöp yığını haline gelmesini gösteriyor olabilir. Çöp dağları, çöp denizleri. Atılmış binbir türlü nesne. ve wall-e ise bunları ne yapıyor sadece pressleyip bir kenera bırakıyor. Bundan basetmişken aslında pressleme işi yapan şu cesar. Bu şekilde olmak zorunda mı? Bunun alternatifi yok mu? Aslında bu tezde bunlara cevap bulunabilir. Çöpe olan bir sürü yaklaşıma ek olarak bir de bunlar var. Ya da başka neler var? Bunları araştıran inceleyen bir yazı. Wall-E'nin çizdiği portreyi tekrar ele alırsak ne görürürüz. İnsanlık buna mı dönüşücek? Başka bir ihtimal yok mu? Gerçekten de dünyayı bekleyen şey bu mu? Bir çöp yığını olması mı? Bu bir abartı mı yoksa gerçeklik payı yadsınacak cinsten mi?

% Bu trash in every sıralamasını neden yapıyorum? Genel bir çerçeve çizmekten çok aslında evrensel bir konu olduğuna değinmiş oluyorum. Peki bu çok açık bir şey mi? Yani aslında çok da açık olmayan bilgilerek vererek yaparsam olmayabilir. Evrensel olmasının benim için ne önemi var. Yaşamımızın bir parçası demek. Tamam da buna karşı bir şey mi var. Ben göremiyorum açıkcası. Herkes bunu biliyor. Farkında mıyız. Ama farkında olmak istemediğimiz bir şey ki sürekli kendimizden uzaklaştırıyoruz. O zaman bu gerçekleri aslında ne kadar biliyoruz. Ama bunları bilmek böyle davranmamız gerektiği anlamına mı gelmektedir? Aslında yabancı olmadığımız ama uzak durduğumuz bir konu. Belki de sırf bu yüzden kaçırdığımız bir sürü şey olabilir.
Trash in every age. In other words it is part of human activity. It exists early ages of human to now. Trash is part of the early days of human production / or activity. Firstly the word garbage is actually used for the parts of animal that cannot be consumed by the human. Legs, bones etc. [TODO etymology reference, see appendix] Archaeologist find things from people throw away.

Trash in every society. Some of them called as trash the others don't. [TODO introduction of revista]


% TODO Life time of objects.
% Nesnelerin yaşam süreleri, Bu yaşam sürelerinin ne kadar olduğu. Burada ilginç bilgiler olabilir. Aslında ne kadar çok üretip ne kadar kısa sürede tükettiğimi görüyoruz. Ama eskiden ürünler daha uzun sürelerde kullanılıyordı. Sonrasında ise bu kısa sürede tüketilen ürünlerden nasıl dağlar oluşuyor. Çöpçü kaç günde bir ziyaret ediyor. Çöp kutuları ne kadar sürede bir doluyor. Çöp kovalarının içlerinde neler var.


% Background information about the thesis. Inspirational points. The fact of trash.

% Firstly as a part of art making. Changes in art making? // Sanatın köklerine inme abi. Sen sanat bağlamında anlatacaksın bunları. Zaten ilerisinde değineceğin şeyleri neden şimdi tekrardan anlatıyorsun.
In the beginning of the 20th century non-art objects enter the scope of art making. This is a revolutionary change in the making of art. Production of art and the approach to the art changed dramatically. This changes actually started at the end of the 19th century with impressionist. (But not related with non-art objects.) Picasso is the first artist that used non-art object in his works. Later many of them followed him. First works are collage and then collage.  


% Aga çöpe farklı yaklaşımlar var. O yüzden background information olarak burada verilebilir ve scope belirlenebilir. Tam da bu noktada ise purpose kısmına geçilebilir. Çöpe nasıl yaklaşıyorlar falan filan. İşte ben tam da bu noktada sanatta nasıl yaklaşılmıştır dicem. Hatta basitçe sanatta bu tür işlere 


%%%
%%%
%%%
\section{Purpose of this study}

% Bu söylemeden önce insana böyle bir şeyin mümkün olduğunu anlatmak gerekli. Yani aslında transformation of trash diye bir şey var aga demeli. Kimler bunu nasıl yapmış. Neler var ortalıkta bunlara bakmak gerekli. Pres makineleri falan. Ama bunu aynı zamanda sanatsal açıdan inceleyenler, sanatsal amaçlarla yapanlar bulunmakta demek gerekli.
The aim of this study is to explore the theories and methods in transformation of trash in the context of art and artistic act.


%%%
%%%
%%% Bu kısım aslında yukarda bahsedilen purposın nasıl yapılacağını anlatacak.
\section{Overview of Chapters}
Following chapter ---Trash in Culture and Theory--- explores the place of trash in the cultural life and theoretical approaches on trash. The aim is (better) to understand the trash in cultural and theoretical aspects. (to better realization the notion of trash.)

Third chapter seeks the root of using trash in the artworks and looks through the sample artworks.

Fourth chapter explains the project that supports this thesis. Its stages and approaches to the topic mentioned here.

Finally last chapter offers an overall conclusion of this thesis and project.


%****************************************
% About the thesis statement and focus:
% Is it too broad?
% Is it debatable? Is is fact? (Trash is not a end point of objects, there is a life waiting for them?)
% What is my side? throwing away is required for society to go further? or omitting the tons of possibilities... 
% Supporting claims?
% What is my thesis statement? Again same topic... It needs to be solved...


%****************************************
% Examples from other thesis.
% The aim of this thesis is to investigate the theories and methodologies of type design within the context of ideology. // The purpose of this thesis is to explore the transformation of discarded items through the artistic methodologies (or in the context of art). It is questioned the role of art in transformation of discarded items. This thesis follows the discussions of what is waste and the states of waste or meaning of waste. Within this thesis the project transformed the collected discarded items and then later spreads them to the community again. 


%****************************************
% Sample phrases:
% The discussions (mentioned discussion are also formed the thesis project entitled ... )
% Aim of this thesis is ...
% There has been a recent spate of artistic work focusing on (over-)consumption using the lens of disposal and discard.
% as a reaction to the consumerist society.


%****************************************
% Some comments
% This written portion was to evaluate the project in terms of a justificatory theoretical framework.
% This question “what the future of photography would be like” is the main motivator of this thesis and the visual project is based on.
% Burada önemli olan şey soru veya sorgulanan şey. Bu soru projeye yön verecek. Yazılı tez ise bunun doğrulamasını, üretilen işin kavramsal ve kuramsal çerçevesini belirleyecek. Tartışmalar ne üzerine olmalı bu durumda? 
% The project aims to what? and what needs to justify what it questions? 
% At this point, the question of what will be in the future of photography has to be asked. Bu çok önemli, bir şeyler hazırlayıp neyin sorgulanması gerektiğini sormak gerekli.

% Tipografi ve ideoloji arasında bir ilişki vardır. Fotoğrafın geleceği? Kelime ve imaj arasındaki ilişki, beden ve algı üzerinden incelenebilir? kelime ve imaj arasında bir problem olduğundan bahsediyor. Sanat işlerinde çöp kullanmak toplum tarafından atılan şeyi tekrar kullanılabileceğini gösterebilir. Toplum ve çöp nispeten bir problemli bir durum. Burada bir dert var. Atılan tüketilen manalar var. Sanat bunları provoke mi ediyor. Peki ya ben ne öneriyorum, yani aslında bir şey önermem mi gerekiyor? Sanat ve çöp arasında nasıl bir ilişki vardır? Çöp ile diğer nesneler arasında nasıl bir ilişki vardır? 

% Zaten hali hazırda sanatçılar bunu yapıyorlar, tez aslında bunları inceliyor olabilir ama tez aslında benim işimi inceliyor. İş neyi soruyor ise aslında tez de onu soruyor. Yapılan işlerle sorulan sorular arasında bir bağlantı var. Bu noktada aslında benim işin neyi sorguladığını bulmam gerekli. Çöp dönüştürülebilir, sanatsal bağlamda. (Çok geniş değil mi abi sanatsal bağlamda demek? Çöp de çok geniş bir konu.) Konuyu bir şeklide daraltmak gerekli. Çöp sanat işlerine girmiştir. Çöp dönüştürmek aslında bir artistic acte dönüşüyor. Tezde bunu anlatıyor. Aga bu artisctic act dediğin şey de ne oluyor. Çöp dönüştürmek ne oluyor?

% Çöpe yeni bir alternatif yaşam üretmek gerekli! Neden böyle düşünüyorsun? Çünkü sürekli çöp üretiyorsun ama bir yandan da bundan dert yanıyorsun. Zizek işte bu konuya el basıyor. 

% Aslında hepimiz bir şeyleri dönüştürüyoruz! Neden böyle düşüyorsun? Örnekleri nelerdir? Bricologe bu iddayı destekler mi? Nesnleri üretim amaçlarından farklı amaçlar için kullanmak yeni bir şey değil. (Ama zaten ben sanat amacıyla kullanılmasından bahsedeceğim.) Belki bunu belirtmek için mantıklı olabilir. Ama burada bir farklılık var olamalı. Diğerinden ayıran diyeceğim ama öyle bir fark aramamak gerekli bence. Sanat o farkı yıkmak için uğraşıp duruyor. Belki de bu uğraşa katkısı bu olabilir bu işin. Bir şeyleri dönüştürme potansiyellerimiz olduğunu düşünebiliriz. Dönüştürme dediğimiz şey aslında hangi noktada açığa çıkıyor. Bir ihtiyaç anında, kaçış anında, ya da yeni alternatifler ararken mi?

% Çöpün dönüştürülmesi ne olaki? Buna bir şekilde girişte yer vermek gerekli. Çöpün market içindeki hareketi mi? Ya da aslında çöpün yaşamı. Belki de bu olabilir. Çöpün nasıl oluştuğu vs. Fabrikada üretilmesi. İnsanlara dağıtılması. Ve sonra çöp olması. Çöp dağları oluşması. Çöp kutusunda yer alması. Neler yok ki çöp kutusunda. Bir sanatçı çöp kovasına attığı sanat işleri. Sanat işleri çöplükleri vs. Çöp kavramı aslında o kadar çok şey için geçerli ki? İşte bu çok çeşitli çöp kavramını ele almak gerekli. Yani objeler bir şekilde farklı konumlarda mekanlarda yer değiştiriyorlar. Bazen çöp oluyorlar bazen değerli vs.

% Bir bir çeşit çöp var. Bunlar binbir çeşit davranış sonucunda oluşuyorlar. Bir işlemin sonunda ortaya çıkan son ürün. Çöp konumuna gelmesi vs. Kime göre çöp neye göre çöp. Ama işte ortada bir konsensus yok. Başkaları da farklı şekilde görüyor. Birine göre çöp diğerine göre ise bir hazine.

% Dönüşmek ne oluyor peki bu durumda. Bir durumdan başka bir duruma geçmesi. Farklı bir anlam, amaç kazanması olabilir mi? Mesela pisuvar dönüşmüş müydü? Gazeteyi kolajda kullandığım zaman gazeteyi dönüştürmüş mü oluyorum? Hangisi dönüştü? Zaten buradaki açık bir şekilde açığa çıkıyor aslında. Her ikisi de dönüşmüş oluyor. Farklı bir kullanım alanı bulmak. Farklı niyetlerle kullanmak olabilir belki de. Bir şeyin ne zaman dönüştüğünü iddia edebilirsin.

% Çöpün dönüşmesi diye bir şeyin olduğunu aslında kullanıcıya aktarmak gerekli introductionda.

% Agnes varda neyi anlatıyor: Toplayıcılık hala devam ediyor, bu toplayıcılar arasında sanatçıları da geziyor, çünkü onlar da topluyorlar. Onlar da o çöplerde farklı şeyler görüyorlar. Kendi de mesela çöpe atılmış bir şeye anlam yüklüyor. Kendisininde aslında bir toplayıcı olduğunun farkına varıyor.


%****************************************
% JUSTIFICATION (Gerekçesi)
% Eğer bir şeyin justificationından bahsediyorsak ben neleri iddia ediyorum:
% - Çöpü dönüştürdüğümü. Öncelikli olarak o gerçekten çöp mü? Sonrasında ise gerçekten dönüştü mü? (throw away culture çöpü anlatsa. rubbish theoryde dönüşümü anlatsa)
% - Peki neden bu işi yapıyorsun. Alternatifi aramak. Değer bulmak, değer bulunabilceğine inanmam aslında. Belki de değer üretmek için o değerleri yıkmak gerekli. Çöpe atıldığında bu yüzden bazı değerler yıkılıyordur. Artık yeni değerler vermek için gerekli yapı oluşuyordur o zaman. O yüzden çöpe atılması önemli. Bir yerin sonuna gelmiş olması aslında belki de yeni bir başlangıç olması için önemli olabilir. Çöplüğün bir başlangıç olması. Batırdığımız bitridiğimiz yerden yeni bir başlangıç. Çöpü üreten üretilmiş şeyleri bitiren zihniyetle tekar düşünmek. Aslında çöpe baktığında üretilen şeyin ne olduğunu tekrar sorguluyor olabilirsin. 
% - Çöp sanat işlerinde kullanılabilir. Bunlarla ilgili örnekler var.
% - Çöp bu sanat işlerinde dönüşmüştür. 
% - Elimdeki literatürden belli başlı argümanlar çıkarmalı, bu argümanlar benim yaptığım işi destekliyor demeli:


%****************************************
% ARGUMENTS:
% 


%****************************************
% SANATTAKİ KÖKLERİ ÜZERİNE:
% picasso falan gibi adamların derdi sanata farklı objeleri sokarken ki amaçları ya da değerlendirildikleri nokta farklı bir şeyler yapmaları. O dönemki anlayışı yıkmaları. Onun yerine daha geniş bir alan imkanı sunmaları. Biz şu anda onların açtıkları alandan top koşturuyoruz bir nebze. Onların buna yaklaşımları ile bizimkiler arasındaki bazı farklılıklar olacak. En azından biz onların yıkları şeyi yıktığımız iddia edemeyiz. Çünkü o kalıplar, yaklaşımlar açıldı ve yeni bir üretim alanı insanlara sunulmuş oldu. Sorulacak yeni sorular, yapılacak yeni tartışmalar vardı. 
% Yani farklı dönemlerde farklı amaçlarla kullanılmıştır. Farklı şekillerde okumak mümkündür. Ama benim işime yarayanları seçmek gerekli bunları. Bir kısmı için bu sanatın diline yeni bir şey sokması, sanatın alanını genişletmesi, yeni tartışma alanları açması şekilde işime yarayacak. Diğerleri ise çöpe ele almaları 


%****************************************
% It is a justification of the work. 
% What is trash? Who does call it trash? 
% Is it possible to transform it? How? Which methods?
% 








