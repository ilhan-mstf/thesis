\chapter{Introduction}


%****************************************
% Structure of this chapter
% Section: explain the topic and its significance, raise curiosity, give some facts about it.
% Section: state the the purpose of this study, specify the focus of it.
% Section: overview of chapters and structure of thesis.
% Section: a note on terminology, clarification about the comprehensive terminology.
%........................................


%****************************************
% Sample phrases:
% The discussions (mentioned discussion are also formed the thesis project entitled ... )
% Aim of this thesis is ...
% There has been a recent spate of artistic work focusing on (over-)consumption using the lens of disposal and discard.
% as a reaction to the consumerist society.
%........................................


\epigraph{One can even shout out through refuse \ldots}{\hfill---Kurt Schwitters, \textit{Kurt Schwitters}, 1985}


%****************************************
% Rubbish: the archaeology of garbage. rathje1992rubbish
% Garbage project's findings have supplied a fresh perspective on what we know --- and what we think we know --- about certain aspects of our lives. p.24

% The basic methods of garbage disposal are four: dumping it, burning it, turning it into something that can be useful (recycling), and minimizing the volume of material goods.(p.33)

% ancient peoples p.33

% fast food packaging. p97

% human behavior p.55 can you tell a lot about the customers from their garbage?

%p191-192 people tend to think of "recycling" as a relatively modern conceit that has only recently gained broad public acceptance, and whose practical benefits have only just began to be realized. Recycling itself is probably as old as --- indeed, seems to be a fundamental characteristic of --- human species.

%p.209 The reuse of paper, for example, involves processes that generated a considerable amount of hazardous waste. In order to recycle newspapers, magazines, and, indeed, any printed paper, the paper must first be de-inked. At the end of the de-inking process one is left with essentially two products: on the one hand, de-inked fiber that will be turned into new paper; and on th other, a large quantity of toxic sludge.
%........................................


%****************************************
% trasformations 

% during great depression, I learned firsthand about frugality as a child during ww2. my family carefully folded the wrapping paper from gifts to reuse on other occasions. p.8

% p11. for decades we've taken for granted used cars and old houses. Is there anyone alive who has always lived in new houses or driven only new cars?

% p.11 Economics has always been a factor in recycling. 

% p.12 recycling is not limited to folk art. 

% Bicycle wheel duchamp 1913.

% Use of recycled materials is not new, not is it exclusive to the united states. p.18

% p.22 Material, meaning, and memory --- these are artists' reasons for using found objects.

% p.48 recycling may be the most wasteful activity in modern America --- a waste of time and money, a waste of human and anutral resources. john tierney, "What a Waste," seattle postintelligencer...
%........................................

\paraphrase{The phenomenon of waste comes clearly into focus not merely as a by-product of manufacturing processes, but rather as an integral element in cycles of production and consumption. (Trash Culture)}

\comment{Common approach on trash is dominated by the environmentalist.}

\comment{Last decades much attention has been devoted to the waste.}

\comment{At least since the early twentieth century, the concern with discarded things and materials has been a recurring theme in art.} \todo{Bu nereden?}

\paraphrase{Over the course of the twentieth century, the twin developments of mass production and mass media in the capitalist economies of the Global North completed a total transformation of everyday life, reorienting almost every activity toward consumption. Things once locally produced and often handmade were now mass produced and commodified, turning local, artisanal producers into deskilled laborers serving the assembly line. This world of mass production radically altered the meaning of objects in an unprecedented and profound reification that reached into every sphere of life. This was not just the fate of objects, but also of the practices with which they are always enmeshed. Mass production and commodification liquidated what rema ined of folkways tied to local production, demanding people construct the meaning of their lives through purchases rather than production. This transition to a lifeworld of consumption affected not only concrete objects, but even more forcefully altered symbolic and aesthetic practices. Technologies of mechanical reproduction redefine d story, image, and music, altering the traditions of both fine and fol k art with mass produced and distributed forms in newspapers, advertising, radio, film, and television that shattered the aura of fine art and liquidated folk art almost completely. Rather than making their stories, images, and music, ever more urbanized workers and manage rs consumed new massproduced art forms that, epitomized b y the Hollywood studio system, developed into the walltowall media scape of twentyfour hour broadband that now blankets the city, the suburb, and the country alike. (FROM Collage Culture)}

\paraphrase{Given these developments, this book seeks to answer the following questions: How did art and literature respond to this age of consumption? What do the productions and practices of artists and writers reveal about the meaning of mass production, consumption, reification, mechanical reproduction, and meaning? (FROM Collage Culture)}

[COMMON PERCEPTION] (In the dictionary trash means anything useless, worthless or discarded. In parallel to this definition) Common perception to the trash is to get rid of it as soon as possible. People want to depose trash from their lives most of time not thinking beyond. However to establish solid understanding of trash, we have to give attention to the trash and ask why it is trash? How does it become trash? People tend to think that trash is valueless because it is trashed no longer needed or not wanted anymore. However is it really values? What does make it valueless? 

\paraphrase{Every day, we put unwanted material in toilets and garbage bins, regularly flushing it away or taking it out in bags to be transported far away from our homes by others. The names we give this material ---waste, garbage, refuse, trash, rubbish--- have pejorative definitions. Worthless. Rejected and useless matter of any kind. Unimportant. Our trash is a testament; what we throw away says much about our values, our habits, and our lives. While dictionary definitions of garbage describe it as “filth” and “worthless,” scholars are careful to note that perceptions of waste and the value of material are neither static nor universally shared. \ldots the question of who owns these discards is not trivial. The absence of a waste stream aroused suspicion, just as the presence of particular items tell us about the habits of the consumers who generate a waste stream. Our trash is part of us, whether or not we choose to acknowledge it \cite{zimring2012encyclopedia}}. \todo{ref for tracey emin et al.}

It is part of our consumption activities (/practices) and it is very common concepts from developed cities to rural areas, from modern societies to ancient societies.

[DILEMMA] In this thesis work to understand the different aspects of trash is a key element. \todo{These aspects are \ldots} Sometimes it is accepted as a problem (waste crises, ecological problems.) that is need to carefully and seriously managed. On the other hand it is accepted as a source of diversity. (To establish a solid understanding for trash, it is important to see its dilemmas (it is really a dilemma?) or aspects). Therefore we should ask these questions: What is trash? How does it became trash? Is it a end product or a source material? How much is it valuable? How much is it dangerous?

%\comment{In previous ages (which age? pre industrial ages?) objects and resources are used again and again. Production of objects are hard and laborious. Böyle çok fazla kullanılan nesneye örnek bulsak ne kadar güzel olur. Mendillere bunlara örnek olabilir. Aslında tam benim konumla ilgili çünkü tek kullanımlık. Hijyen kavramıyla ilgili giderek artan da bir ilgi necesiyle tek kullanımlık peçeteler gittikçe önem kazanıyor. Peki burda illaha geçmişe falan girmek gerek var mı? çok mu derdimde benim?}

% Trash is everywhere and, produced every time
[LIFE OF OBJECTS] Modern (developed) societies are continuously generating trash and, pile them on landfills. During daily activities (drinking, eating, waiting in the bank?) trash is generated and people get rid of them as quickly as possible. (Various objects become trash after their primary functions consumed. People do not care the package of the objects that they buy. They buy the coffee not the cup of it. After coffee finished the life of cup also finishes. Very small life. But is that really so? Is that really life of an object ends after tossed on the waste bin? \todo{Rubbish Theory, Zizek} (Objects life time in the nature is more than human being. People use them for just 5minutes, however they will exist years on the nature.) The vast amount of industrial discarded items spread through the landfills to oceans. They are the result of highly complex industrial production methods. \comment{They are not easily disposable items in the nature.} They live in the nature thousands of years. They are durable(in terms of resistant to natural affects) products. Most of them packages that are used to carry or protect other materials. After real material used these packages became valueless (or useless). (types of trash can be mentioned here, but currently in the artwork I'm using paper packages, therefore, it is more important.) How manage the all this increasing trash that damaging nature?  This is the common approach to trash and the main problem. (actually the sustainability problem.) It is not the only problem, It can be thought that it is a losing the ability to transform new things, alternative behaviors etc. (Instead of creating new opportunities or alternatives, it is a consuming all of them (which?) and producing huge pile of trash.) 

\textbf{Trash is global}, and shared concept for all societies throughout the ages. 
\textbf{Trash in every age.} In other words it is part of human activity of every historic age. It exists early ages of human to now. Trash is part of the early days of human production / or activity. Archaeologist find things from people throw away. 
\textbf{Trash in every society.} Some of them called as trash the others don't. Somehow all actions of people generate trash. It is common. 
\textbf{Trash is everywhere.} in the streets, in your home, in the sea that you swam, around the globe \todo{satellite discard covers the atmosphere}. Even if trash is isolated by people, it is close as closest waste bin.

[RELATIVE] As stated by the editor of Garbage issue of ReVista, Christmas decorations at Chocó, a poor region on Colombia’s Pacific Coast, are \quotes{all crafted from used tin cans, old newspapers, discarded textiles and found wood objects} \cite{erlick2015editorsletter}. She realized that any of them called the practice as recycling. For them using trash again and again is very natural and it is part of their life. On the contrary for the developed countries trash considered as a thing that must be avoided. There is no place for trash in their life. It can be understood that approach to the trash is not same for the all regions of world \todo{ref}.

% FROM Trash as Treasure BY WILLIAM L. FASH AND E. WYLLYS ANDREWS, ReVista
% TODO PRAP.
[FACTS] \paraphrase{The World Bank estimates that the amount of solid waste generated in cities is growing faster than the rate of urbanization. The higher the income level and the rate of urbanization, the greater the amount of solid waste produced. OECD\footnote{OECD(Organization for Economic Co-operation and Development) is an international economic organization of 34 countries. Turkey is member of this organization.} countries produce almost half of the world’s waste. Africa and South Asia produce the least waste. High-income countries have the highest collection rates and are most likely to dispose of waste to landfills or incinerators. Low-income countries have the lowest collection rates and are most likely to dispose of their waste in open dumps. However, low-income countries also have the largest numbers of informal waste pickers who collect, sort, and reclaim recyclables---thus reducing costs to the city and to the environment \cite{fash2015trash}.} \todo{Zengin adamın harcama lüksü}

% FROM Beautiful Trash Art and Transformation BY PAOLA IBARRA, ReVista
% TODO PRAP.
\paraphrase{\textbf{We relate to garbage daily. We use it, produce it and dispose of it. Endlessly. The most obsessive of us get rid of it as fast as we can.} The hoarder likes to salvage a few things for later use---the plastic and glass containers, the cardboard boxes. We know that capitalism’s escalating cycles of production, consumption and obsolescence keep worsening an already problematic relationship between humankind, waste and nature (not to mention social and economic relations). Despite a relatively increased awareness about consumption and its consequences, the pace at which we also acquire and dispose of material objects is exploding. Particularly in the connection between garbage and the arts, \textbf{I am interested in two questions. First, the issue of recycling as a general practice in the arts; and secondly, in the whole issue of representation---that is, representation of waste as subject, and representation (of waste or others subjects) through waste as material} \cite{ibarra2015beautiful}.}

[MIT, COMMON APPROACH] People do not think what happens after they throw it away. It exits from people's life but not the world. What happens when you throw it away? It is stacked from another place. \comment{Value of objects are not static.} Trash is not static \todo{ref. first paragraph.} also it has a long journey. A project conducted by MIT researches journey of trash by placing trackers onto the trash \cite{chen2009mit}. The results are surprising. Trashes spread away across the country and this journey takes month. Further it is not limited with the border of America because it is the one of the countries export their garbage to the other countries --- fourth world countries\todo[inline]{footnote what are these countries?}. In other words American's trash is not only their trash. As commodities spread to the every corner of the world, trash also. What happens when it is traveled to the other countries, how they approaches these items. \comment{They find new uses and meanings on them.} \todo[inline]{consider African native}

\begin{quote}
\paraphrase{Nobody wonders where, each day, they carry their load of refuse. Outside the city, surely; but each year the city expands, and the street cleaners have to fall farther back. The bulk of the outflow increases and the piles rise higher, become stratified, extend over a wider perimeter (by Italo Calvino, Invisible Citie). People just take their trash and put it on the curb and they forget about it and don’t think about all the time and energy and money put into disposing of it.}
\end{quote}

\comment{MIT project can be seen as a different approach to understand the trash and its \quotes{removal-chain} from people's life.}

\comment{Biodegradable matter is material capable of being decomposed by bacteria or other biological means. Biodegradable  matter usually consists of organic materials such as plant, animal, and other substance matter originating from living  organisms. It has the ability to be broken down into smaller, harmless products by way of the action of living organisms.  The term is often used in relation to ecology, garbology, and waste management. Biodegradable products are often associated with perishables (products, food, and waste materials subject to death or decay). The term biodegradability of a product  refers to its disposition to disintegrate as the result of natural processes.}

% Cycle of trash
[MOVES] Trash moves, objects moves from place to place. From homes to garbage trucks. From streets to land fills. From garbage baskets to sculptures. Lots of different people touches to trash. With the trash what moves? Trash moves around. Travels like humans. Travels like commodities. When we are telling about the travel of commodities, then it is easy to say that trashes also moves around.

[IMPORTANT, QUESTINING] Why people are interested in rubbish. Some of the artist also interested in others trash. \comment{O çöplerde ne bulmayı umuyorlar ki? Aslında çöp bizim davranışlarımız, yaşayız tarzımız hakkında bize yön veriyor. ve bu her zaman fiziksel dönüştürme ile değil de aynı zamanda anlamsal ve mekansal değiştirme sorulamalarla mümkün oluyor. Bazen de olduğu gibi kullanıyorlar. bu açıdan önemli. Çöpü çöp şekliden sunan sanatçılar. Ama bu nasıl mümkün ki? fotoğrafını çekenler aslında gerçekten ona müdahele etmemiş mi oluyorlar. Bence değil? bir şekilde müdahele var gene orada. o fotoğraflarda artık dikkatin objesi haline geliyor. çöpün dönüşmesi aslında bir diğer yandan çöpe bakışın değişmesi olarak görülebilir.}

% Şöyle bir şey var, sanata non-art objelerin girmesinden bahsediyoruz ama artık orda o seviyede değiliz. O zamandan bu zamana çok şey değişti. 
\comment{In the beginning of the 20th century non-art objects enter the scope of art making. This is a revolutionary change in the making of art. Production of art and the approach to the art changed dramatically. This changes actually started at the end of the 19th century with impressionist. (But not related with non-art objects.) Picasso is the first artist that used non-art object in his works. Later many of them followed him. First works are collage which gluing different papers together. Yes using non art objects in the art is introduced and opened new dimensions for the language of art. But some of them treated as sculptures from trash. Reflects our world. But it is not limited with this. Some of the works are provokes the consumption habits of society and offers an alternative perception.}

This trash phenomena catches the attention of artist. It is very human activity. In this context which trash? What type of discard. Human discard. trashes of human activity. items that belongs to human. commodities etc. 


%****************************************
% Different types of trash
[Different types of trash] The complexity of produced trashes of societies is increasing. For example developed countries that have nuclear plant generates radioactive wastes which highly hazardous for the environment is never exist previous societies. Think batteries and so on. Every society generates different types of wastes. Differs from country to country, society to society, ages to ages. It can be thought that when the complexity of trashed increased required effort to repair, reuse and recycle is also increase. Therefore for the ones that have no complex tools it is becoming harder to reuse objects. In other words objects become more complex their re-usage becomes less likely.  Different production process generates different types of trash. According to production process, decomposition process\ldots The approach to the different type of trash will be different. In other word if trash is a result of classification of objects, it can be easily extracted that there is classification inside of it. There are some trashes that are more close to the people. More easy to convert them. more easy to regain to the society. 

\comment{Biodegradable waste, Human waste, Organic waste, Food waste, Biomedical waste, Clinical waste, Medical waste, Household waste, Radioactive waste, Marine debris, Electronic waste (e-waste), Construction and demolition waste\ldots}

\comment{Liquid type, Solid type, Hazardous type, Organic type, Recyclable type}

\comment{Radioactive waste is waste that contains radioactive material. Radioactive waste is usually a by-product of nuclear power generation and other applications of nuclear fission or nuclear technology, such as research and medicine. Radioactive waste is hazardous to most forms of life and the environment, and is regulated by government agencies in order to protect human health and the environment.}
%........................................


%****************************************
% Different disciplines, Perspectives related with trash from different disciplines
\comment{Trash draws attention different peoples and disciplines. In other words there are different studies related with trash. Different approaches to the trash and son on.}

trash have influenced, and are also influenced by, cultural products such as films, visual art, museum exhibits and literature. 

% From Trashion: The Return of Disposed, Bahar Emgin.
\paraphrase{Such a conceptualization of waste as “the degree zero of value” has been contested for some time in different disciplines, ranging from economics to environmental studies, but most particularly by those studying consumerism or material culture.}

This is very important because the problem of trash is being tried to be handled by different disciplines. In other words there different approaches to the trash. Different problems, different solutions. Which perspective that I have. In what ways my project differs from them. The purpose is to participate people in this work, by collecting them etc. And also offer to transform it. Rescue it and than later transform it. In short, the difference between the other disciplines must be clear. 
\begin{itemize}
\item Ecological perspective: Trash causes ecological problems and it treats the balance of nature. Animals do not aware of plastics materials and they unconsciously eat them.
\item Management of it (handled by the municipals generally).
\item Technological perspectives. (seeing trash as a design problem. developed technologies and societies generates lots of trash, so this situation cause to inquiry that are they really developed? development in what sense?)
\end{itemize}

\comment{Trash draws attention of many scholars and people who are professional in different areas. Further trash is a common subject of the all ages through the human history. In other words this topic is not introduced in recent years (or ages). It always part of the human practice and life. Even if trash is refused and discarded thing, there are people giving attention to them.}
%........................................


%%%
%%%
%%%
\section{Purpose of the Study}
\todo[inline]{what do you mean when saying ``transforming trash"? Needs to be opened. I claimed that such a thing exist? when it is exist? by physically or conceptually? actually both of them exist.}

The aim of this study is to explore the theories and methods in transformation of trash in the context of art and artistic act.

In this thesis (re-)usage of discarded materials in the process of art making and the artwork itself is explored. Why and how are they used by the artist? Are there any differences with the original items compared to discarded items? Has using discarded material or trash specific (or special) meaning and message? This is a work to explore the re-usage of trash in artworks (place or the role in the art practice). (and also in which scope? medium, message, life practice\ldots) 

[Different approaches] As many people are inside the business of transforming trash through craft and so on. People on the streets looking for aluminum cans, glass bottles, plastic. They collect and sell them to survive. There are companies that are collect waste materials to regain them again the industry. In thesis they are not in the scope. People that give attention to the trash and behave artistic act is in the scope of this thesis.

In the scope of trash not biodegradable trash is considered. Produced, industrial commodities are considered.

%%%
%%%
%%% 
\section{Structure of the Study}
\comment{Theory and artworks are not strictly separated from each other. Artworks are explained with the help of theories.}

\comment{Research and the found samples of works helped a lot to shape my work. They provide me deeper understanding of the subject and what I am doing and they lead me during the development of the work.}

\comment{the thesis is separated into three part.}

This study is structured by many examples of artworks in different part of the thesis. When a topic or argument is presented, immediately example artwork is given at this moment and how artist replies this issue is illustrated. Thereby artworks are examined in the theoretical and cultural context. Artworks are used as a supportive elements for the stated arguments.  However, not every time artworks are discussed in detail. Only basic information is given and how it is related with this topic explained. \comment{By the way} it is better to keep in mind that all works can be analyzed in different contexts in more detail fashion.

Following chapter, Trash in Culture and Theory, explores the place of trash in the cultural life and theoretical approaches on trash. The aim is (better) to understand the trash in cultural and theoretical aspects. (to better realization the notion of trash.) Provides theoretical and cultural background. It is explored the trash is being worked is whose trash. Or what type of society generate this trash. Also in the project I collect the paper trash. What type of approach generate it and what are the dynamics of it? What type of trash we are talking about? It is understood through the patters of consumption patterns. And to reflect this cultural phenomena what philosophers and scholars say. Artist how they are reflected this notion.and also looked how scholars are conceptualized trash and its movement on the different values system. How explained the transient nature of trash?

Third chapter, Trash (in) Art, seeks the root of using discarded materials in the artworks and looks through the sample artworks and artists. As already artworks are mentioned in the different part of the thesis, in this chapter some of them analyzed deeply. Artists methods and approaches to the subject are stated.

\comment{Fourth chapter consists of the overview of the project. Stated the approach to topic clearly. Development process of the work is presented in detail. I will start with the development process of the work. I believe that the way an artist works and the obstacles that she encounters have a huge impact on how the final work is embodied. Development process of my work shaped the way I think on the subject of my thesis. Therefore, I will explain my process as detailed as possible. Then, I will continue with different aspects of my work; formal decisions and usage of light that I believe carry my work to another level. At the end of this chapter, we will see how my work functions and what it proposes on the image and word relationship.}

\comment{Finally last chapter is reserved for an overall conclusion of this thesis and further suggestions on the subject and the project.}

%%%
%%%
%%%
\section{A Note on Terminology}
\comment{using words interchangeably.}

Within the scope of thesis only words that are English is examined but in different languages and cultures there might be different naming for various stages of object and trash. Analysis of them may be subject of another thesis. (Garbage, trash, rubbish, debris, detritus, waste, scrap, junk, refuse, discard, disposal, litter: during my research scholars and authors have different names for the stuff.) I realized that many scholars and resources have used different words (or terminology) to define trash. Although there are slight differences between these words, sometimes they are used to signify same concept (or used with same purposes). Understanding diverse vocabulary of this topic before dive into the theory makes more clearer the borders of the work. There is no consensus on terminology. Some scholars uses junk art, others recycling art. The origin of vocabulary of this topic reveals a whole side of human activity: our history revealed by what we have thrown away through the ages. What were people throwing out when these words were coined? From the rubbish: the archeology of garbage gives clearer definition of some these words:

% FROM Rubbish! The Archaeology of Garbage, p.9, rathje1992rubbish
\begin{quote}
Several words for the things we throw away---"garbage", "trash", "refuse", "rubbish"--- are used synonymously in causal speech but in fact have different meanings. \textit{Trash} refers specifically to discards that are at least theoretically "dry"---newspapers, boxes, cans, and so on. \textit{Garbage} refers technically to wet discards--- food remains, yard waste, and offal. Refuse is an inclusive term for both the wet discards and the dry. Rubbish is even more inclusive: It refers to all refuse plus construction and demolition debris. The distinction between wet and dry garbage was important in the days when cities slopped garbage to pigs, and needed to have the wet material separated from the dry; it eventually became irrelevant, but may see a revival if the idea of composting food and yard waste catches on. We will frequently use "garbage" in this book to refer to the totality of human discards because it is the word used naturally in ordinary speech. The word is etymologically obscure, though it probably derives from Anglo-French, and its earliest associations have to do with working in the kitchen.
\end{quote}

\paraphrase{A note on terminology: The chapters in this book feature a broad range of   terms, reflecting the multifaceted and conceptually complex nature of   the issue at hand. Many of   these terms – such as ‘trash’, ‘garbage’ and ‘rubbish’ – are frequently used interchangeably. This is not least because the variety of   terms reflects the variations in English and US vocabulary and thus all usages are retained as individual authors intended in order to reflect the international nature of   this volume. Other related terms, such as ‘ruins’, ‘obsolete’, ‘waste’, ‘discards’ are used in a variety of contexts in each chapter. Again, these usages have been retained to reflect the focus of individual studies. (Trash Culture)}

Some notes on methods: 

%****************************************
% Reuse and Recycle
One of the most encountered words in this topic (transforming trash) is reuse and recycle. Both of them are used widely by scholars. \todo{ref}

According to the dictionary, the word “reuse” means “to employ for some purpose” or “to put into service.” Reusing involves usage of the same product unchanged in form. Reusing lengthens the life of the item or material. Other examples are; buying some items and then selling them as used items, repairing some lawn equipment and reusing them, upgrading a computer, renting books, journals, periodicals, DVDs and others. The main purpose is to make the item last as long as it can. To reuse is to use something again instead of throwing it away or sending it off to a recycling company. \textbf{Why throw something away when you can give it another life?} Reusing is significant way to conserve and be earth-friendly because it keeps items out of landfills and reduces the greenhouse emissions caused by purchasing a new product. Using something multiple times -- like using a disposable container more than once -- is not the only way to reuse; you can also give old items a new purpose. For example, use an empty coffee can to store small craft supplies or an old loofah as a scouring sponge for cleaning sinks.

\comment{There is a difference between garbage and waste, although garbage is a part and an aspect of waste.}

According to the dictionary, “recycle” means “to treat or process (used or waste materials) so as to make suitable for reuse.” In recycling an item, it is processed into a totally new product. It is an energy consuming process. For example, if we put some plastic bottles, paper, or aluminum items in a recycling bin, these materials may be recycled into a totally different thing as clothing items, fabric, or maybe a quilt. In this process, energy is required which depends upon the stages of transformation.

Recycling occurs when waste in an unchanged chemical form is used in the same process that created the original product. Examples are crushed glass containers (cullet) used to make new glass containers, and scrap metal used in foundries. 

Reusing is possible with re seeing (rethinking). Reusing is possible meet the needs of the human itself. Using creativity and personal approach can change objects functions. It is possible to use objects for different purposes. 

Recycling can be viewed as down-cycling. The object smashed to the small particles to be used later in the production of something else. Although reuse can be viewed as up-cycling that gives another (or more) value to the discarded products. Down-cycling does not generates new meanings it tries to convert the product to already known state to process. 

Recycling is very similar the rotting (decaying), reuse is something like dry tree branches used by birds for their nest. These are two agents of nature to regain their resources.

Many scholar used the word recycling when mentioning works use trash and the concept of it. However they are not mentioning the meaning that is decomposing things to the their particles. What they is actually is reusing and combining things, concepts, creating new mixtures.

It is hard to say that they use it wrongly, but what they refer is actually upcycling. Creative reuse, inventing new things from discard.

Some of the scholar uses recycling \cite{cerny1996recycled,herman1998trashformations} for the recreation and transformation of it. Some of the scholars and artist uses upcycling for their process. \todo{move to the more appropriate place}

% I prefer to use the word waste to describe the things that have, for whatever reason, been leftover from use or for which use has been precluded.

\begin{figure}[h!]
  \centering
  \includegraphics[height=6cm]{graphics/ChrisJordan_BreakDown.jpg}
  \caption{Chris Jordan, Break Down}
  \label{fig:ChrisJordan_BreakDown}
\end{figure}

\paraphrase{Not all artist transform trash although some deconstruct them. Michael Landy is one of them. Michael Landy's Break Down Inventory is a two week show / display of destruction process of his all possessions on a dissemble line with the help of 10 workers. Firstly they are classified and recorded for three years and the deconstructed in two weeks by separating every element to the smallest part. Reveal all his possessions. and loosing them while you are alive. Turning them to rubbish making them unusable. breaking down the all the meaning. breaking down the connections.} It can be an example of downcycling process. It is preferred to decompose all the complex link and relationships between the objects. They are not just ordinary things they are possessions of artist. Break Down 2001, in which he systematically destroyed all his personal possessions. His work examines what we value and what we discard, consumerism and waste, and human labour and its worth. It happens in public space. People can see the process. This process last 2 weeks. Places where people buy things actually.

\comment{Yıktığı şeyler aslında bir noktada hepimizin yaşayacağı şeyler. Bunu yaşarken yapıyor. Tüm sahip olduklarını yaşarken kendi isteği ile bunların hepsini yok ediyor.}

%****************************************
% UPCYCLING:
% Burada da upcycling kullanıyor aslında. http://www.gwynethleech.com/cups/suspended
% Raw+Matrial=Art burda da upcycling deniliyor.
%
% RECYCLING:
% recycled, re-seen book.
%........................................
