\chapter{Introduction}
Some text...

% This written portion was to evaluate the project in terms of a justificatory theoretical framework.

% So what is the issue here?

% Title: Trash as a Medium.

% Subsection: the purpose of this study
% Subsection: overview of chapters

% This question “what the future of photography would be like” is the main motivator of this thesis and the visual project is based on.

% Burada önemli olan şey soru veya sorgulanan şey. Bu soru projeye yön verecek. Yazılı tez ise bunun doğrulamasını, üretilen işin kavramsal ve kuramsal çerçevesini belirleyecek. Tartışmalar ne üzerine olmalı bu durumda. 

% The project aims to what? and what needs to justify what it questions? 

% At this point, the question of what will be in the future of photography has to be asked. Bu çok önemli, bir şeyler hazırlayıp neyin sorgulanması gerektiğini sormak gerekli.


%%%%%%%%%%%%%%%%%%
% About the thesis statement and focus:
% Is it too broad?
% Is it debatable? Is is fact? (Trash is not a end point of objects, there is a life waiting for them?)
% What is my side? throwing away is required for society to go further? or omitting the tons of possibilities... 
% Supporting claims?
% What is my thesis statement? Again same topic... needs to be solved?

% Tipografi ve ideoloji arasında bir ilişki vardır. Fotoğrafın geleceği? Kelime ve imaj arasındaki ilişki, beden ve algı üzerinden incelenebilir? kelime ve imaj arasında bir problme olduğundan bahsediyor. Sanat işlerinde çöp kullanmak toplum tarafından atılan şeyi tekrar kullanılabileceğini gösterebilir. Toplum ve çöp nispeten bir problemli bir durum. Burada bir dert var. Atılan tüketilen manalar var. Sanat bunları provoke mi ediyor. Peki ya ben ne öneriyorum, yani aslında bir şey önermem mi gerekiyor? Sanat ve çöp arasında nasıl bir ilişki vardır? Çöp ile diğer nesneler arasında nasıl bir ilişki vardır? 

% Zaten hali hazırda sanatçılar bunu yapıyorlar, tez aslında bunları inceliyor olabilir ama tez aslında benim işimi inceliyor. iş neyi soruyor ise aslında tez de onu soruyor. Yapılan işlerle sorulan sorular arasında bir bağlantı var. Bu noktada aslında benim işin neyi sorguladığını bulmam gerekli. Çöp dönüştürülebilir, sanatsal bağlamda. (Çok geniş değil mi abi sanatsal bağlamda demek? Çöp de çok geniş bir konu.) Konuyu bir şeklide daraltmak gerekli. 

% çöpe yeni bir alternatif yaşamak gerekli. Aslında hepimiz bir şeyleri dönüştürüyoruz. Agnes varda aslında neyi anlatıyor: Toplayıcılık hala devam ediyor, bu toplayıcılar arasında snatçıları da geziyor, çünkü onlar da topluyorlar. Onlar da o çöplerde farklı şeyler görüyorlar. Kendi de mesela çöpe atılmış bir şeye anlam yüklüyor. Kendisininde aslında bir toplayıcı olduğunun farkına varıyor.

